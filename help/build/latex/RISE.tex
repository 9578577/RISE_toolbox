% Generated by Sphinx.
\def\sphinxdocclass{report}
\documentclass[letterpaper,10pt,english]{sphinxmanual}
\usepackage[utf8]{inputenc}
\DeclareUnicodeCharacter{00A0}{\nobreakspace}
\usepackage[T1]{fontenc}
\usepackage{babel}
\usepackage{times}
\usepackage[Bjarne]{fncychap}
\usepackage{longtable}
\usepackage{sphinx}
\usepackage{multirow}


\title{RISE Documentation}
\date{November 19, 2014}
\release{1.0.1}
\author{Junior Maih}
\newcommand{\sphinxlogo}{}
\renewcommand{\releasename}{Release}
\makeindex

\makeatletter
\def\PYG@reset{\let\PYG@it=\relax \let\PYG@bf=\relax%
    \let\PYG@ul=\relax \let\PYG@tc=\relax%
    \let\PYG@bc=\relax \let\PYG@ff=\relax}
\def\PYG@tok#1{\csname PYG@tok@#1\endcsname}
\def\PYG@toks#1+{\ifx\relax#1\empty\else%
    \PYG@tok{#1}\expandafter\PYG@toks\fi}
\def\PYG@do#1{\PYG@bc{\PYG@tc{\PYG@ul{%
    \PYG@it{\PYG@bf{\PYG@ff{#1}}}}}}}
\def\PYG#1#2{\PYG@reset\PYG@toks#1+\relax+\PYG@do{#2}}

\expandafter\def\csname PYG@tok@gd\endcsname{\def\PYG@tc##1{\textcolor[rgb]{0.63,0.00,0.00}{##1}}}
\expandafter\def\csname PYG@tok@gu\endcsname{\let\PYG@bf=\textbf\def\PYG@tc##1{\textcolor[rgb]{0.50,0.00,0.50}{##1}}}
\expandafter\def\csname PYG@tok@gt\endcsname{\def\PYG@tc##1{\textcolor[rgb]{0.00,0.27,0.87}{##1}}}
\expandafter\def\csname PYG@tok@gs\endcsname{\let\PYG@bf=\textbf}
\expandafter\def\csname PYG@tok@gr\endcsname{\def\PYG@tc##1{\textcolor[rgb]{1.00,0.00,0.00}{##1}}}
\expandafter\def\csname PYG@tok@cm\endcsname{\let\PYG@it=\textit\def\PYG@tc##1{\textcolor[rgb]{0.25,0.50,0.56}{##1}}}
\expandafter\def\csname PYG@tok@vg\endcsname{\def\PYG@tc##1{\textcolor[rgb]{0.73,0.38,0.84}{##1}}}
\expandafter\def\csname PYG@tok@m\endcsname{\def\PYG@tc##1{\textcolor[rgb]{0.13,0.50,0.31}{##1}}}
\expandafter\def\csname PYG@tok@mh\endcsname{\def\PYG@tc##1{\textcolor[rgb]{0.13,0.50,0.31}{##1}}}
\expandafter\def\csname PYG@tok@cs\endcsname{\def\PYG@tc##1{\textcolor[rgb]{0.25,0.50,0.56}{##1}}\def\PYG@bc##1{\setlength{\fboxsep}{0pt}\colorbox[rgb]{1.00,0.94,0.94}{\strut ##1}}}
\expandafter\def\csname PYG@tok@ge\endcsname{\let\PYG@it=\textit}
\expandafter\def\csname PYG@tok@vc\endcsname{\def\PYG@tc##1{\textcolor[rgb]{0.73,0.38,0.84}{##1}}}
\expandafter\def\csname PYG@tok@il\endcsname{\def\PYG@tc##1{\textcolor[rgb]{0.13,0.50,0.31}{##1}}}
\expandafter\def\csname PYG@tok@go\endcsname{\def\PYG@tc##1{\textcolor[rgb]{0.20,0.20,0.20}{##1}}}
\expandafter\def\csname PYG@tok@cp\endcsname{\def\PYG@tc##1{\textcolor[rgb]{0.00,0.44,0.13}{##1}}}
\expandafter\def\csname PYG@tok@gi\endcsname{\def\PYG@tc##1{\textcolor[rgb]{0.00,0.63,0.00}{##1}}}
\expandafter\def\csname PYG@tok@gh\endcsname{\let\PYG@bf=\textbf\def\PYG@tc##1{\textcolor[rgb]{0.00,0.00,0.50}{##1}}}
\expandafter\def\csname PYG@tok@ni\endcsname{\let\PYG@bf=\textbf\def\PYG@tc##1{\textcolor[rgb]{0.84,0.33,0.22}{##1}}}
\expandafter\def\csname PYG@tok@nl\endcsname{\let\PYG@bf=\textbf\def\PYG@tc##1{\textcolor[rgb]{0.00,0.13,0.44}{##1}}}
\expandafter\def\csname PYG@tok@nn\endcsname{\let\PYG@bf=\textbf\def\PYG@tc##1{\textcolor[rgb]{0.05,0.52,0.71}{##1}}}
\expandafter\def\csname PYG@tok@no\endcsname{\def\PYG@tc##1{\textcolor[rgb]{0.38,0.68,0.84}{##1}}}
\expandafter\def\csname PYG@tok@na\endcsname{\def\PYG@tc##1{\textcolor[rgb]{0.25,0.44,0.63}{##1}}}
\expandafter\def\csname PYG@tok@nb\endcsname{\def\PYG@tc##1{\textcolor[rgb]{0.00,0.44,0.13}{##1}}}
\expandafter\def\csname PYG@tok@nc\endcsname{\let\PYG@bf=\textbf\def\PYG@tc##1{\textcolor[rgb]{0.05,0.52,0.71}{##1}}}
\expandafter\def\csname PYG@tok@nd\endcsname{\let\PYG@bf=\textbf\def\PYG@tc##1{\textcolor[rgb]{0.33,0.33,0.33}{##1}}}
\expandafter\def\csname PYG@tok@ne\endcsname{\def\PYG@tc##1{\textcolor[rgb]{0.00,0.44,0.13}{##1}}}
\expandafter\def\csname PYG@tok@nf\endcsname{\def\PYG@tc##1{\textcolor[rgb]{0.02,0.16,0.49}{##1}}}
\expandafter\def\csname PYG@tok@si\endcsname{\let\PYG@it=\textit\def\PYG@tc##1{\textcolor[rgb]{0.44,0.63,0.82}{##1}}}
\expandafter\def\csname PYG@tok@s2\endcsname{\def\PYG@tc##1{\textcolor[rgb]{0.25,0.44,0.63}{##1}}}
\expandafter\def\csname PYG@tok@vi\endcsname{\def\PYG@tc##1{\textcolor[rgb]{0.73,0.38,0.84}{##1}}}
\expandafter\def\csname PYG@tok@nt\endcsname{\let\PYG@bf=\textbf\def\PYG@tc##1{\textcolor[rgb]{0.02,0.16,0.45}{##1}}}
\expandafter\def\csname PYG@tok@nv\endcsname{\def\PYG@tc##1{\textcolor[rgb]{0.73,0.38,0.84}{##1}}}
\expandafter\def\csname PYG@tok@s1\endcsname{\def\PYG@tc##1{\textcolor[rgb]{0.25,0.44,0.63}{##1}}}
\expandafter\def\csname PYG@tok@gp\endcsname{\let\PYG@bf=\textbf\def\PYG@tc##1{\textcolor[rgb]{0.78,0.36,0.04}{##1}}}
\expandafter\def\csname PYG@tok@sh\endcsname{\def\PYG@tc##1{\textcolor[rgb]{0.25,0.44,0.63}{##1}}}
\expandafter\def\csname PYG@tok@ow\endcsname{\let\PYG@bf=\textbf\def\PYG@tc##1{\textcolor[rgb]{0.00,0.44,0.13}{##1}}}
\expandafter\def\csname PYG@tok@sx\endcsname{\def\PYG@tc##1{\textcolor[rgb]{0.78,0.36,0.04}{##1}}}
\expandafter\def\csname PYG@tok@bp\endcsname{\def\PYG@tc##1{\textcolor[rgb]{0.00,0.44,0.13}{##1}}}
\expandafter\def\csname PYG@tok@c1\endcsname{\let\PYG@it=\textit\def\PYG@tc##1{\textcolor[rgb]{0.25,0.50,0.56}{##1}}}
\expandafter\def\csname PYG@tok@kc\endcsname{\let\PYG@bf=\textbf\def\PYG@tc##1{\textcolor[rgb]{0.00,0.44,0.13}{##1}}}
\expandafter\def\csname PYG@tok@c\endcsname{\let\PYG@it=\textit\def\PYG@tc##1{\textcolor[rgb]{0.25,0.50,0.56}{##1}}}
\expandafter\def\csname PYG@tok@mf\endcsname{\def\PYG@tc##1{\textcolor[rgb]{0.13,0.50,0.31}{##1}}}
\expandafter\def\csname PYG@tok@err\endcsname{\def\PYG@bc##1{\setlength{\fboxsep}{0pt}\fcolorbox[rgb]{1.00,0.00,0.00}{1,1,1}{\strut ##1}}}
\expandafter\def\csname PYG@tok@kd\endcsname{\let\PYG@bf=\textbf\def\PYG@tc##1{\textcolor[rgb]{0.00,0.44,0.13}{##1}}}
\expandafter\def\csname PYG@tok@ss\endcsname{\def\PYG@tc##1{\textcolor[rgb]{0.32,0.47,0.09}{##1}}}
\expandafter\def\csname PYG@tok@sr\endcsname{\def\PYG@tc##1{\textcolor[rgb]{0.14,0.33,0.53}{##1}}}
\expandafter\def\csname PYG@tok@mo\endcsname{\def\PYG@tc##1{\textcolor[rgb]{0.13,0.50,0.31}{##1}}}
\expandafter\def\csname PYG@tok@mi\endcsname{\def\PYG@tc##1{\textcolor[rgb]{0.13,0.50,0.31}{##1}}}
\expandafter\def\csname PYG@tok@kn\endcsname{\let\PYG@bf=\textbf\def\PYG@tc##1{\textcolor[rgb]{0.00,0.44,0.13}{##1}}}
\expandafter\def\csname PYG@tok@o\endcsname{\def\PYG@tc##1{\textcolor[rgb]{0.40,0.40,0.40}{##1}}}
\expandafter\def\csname PYG@tok@kr\endcsname{\let\PYG@bf=\textbf\def\PYG@tc##1{\textcolor[rgb]{0.00,0.44,0.13}{##1}}}
\expandafter\def\csname PYG@tok@s\endcsname{\def\PYG@tc##1{\textcolor[rgb]{0.25,0.44,0.63}{##1}}}
\expandafter\def\csname PYG@tok@kp\endcsname{\def\PYG@tc##1{\textcolor[rgb]{0.00,0.44,0.13}{##1}}}
\expandafter\def\csname PYG@tok@w\endcsname{\def\PYG@tc##1{\textcolor[rgb]{0.73,0.73,0.73}{##1}}}
\expandafter\def\csname PYG@tok@kt\endcsname{\def\PYG@tc##1{\textcolor[rgb]{0.56,0.13,0.00}{##1}}}
\expandafter\def\csname PYG@tok@sc\endcsname{\def\PYG@tc##1{\textcolor[rgb]{0.25,0.44,0.63}{##1}}}
\expandafter\def\csname PYG@tok@sb\endcsname{\def\PYG@tc##1{\textcolor[rgb]{0.25,0.44,0.63}{##1}}}
\expandafter\def\csname PYG@tok@k\endcsname{\let\PYG@bf=\textbf\def\PYG@tc##1{\textcolor[rgb]{0.00,0.44,0.13}{##1}}}
\expandafter\def\csname PYG@tok@se\endcsname{\let\PYG@bf=\textbf\def\PYG@tc##1{\textcolor[rgb]{0.25,0.44,0.63}{##1}}}
\expandafter\def\csname PYG@tok@sd\endcsname{\let\PYG@it=\textit\def\PYG@tc##1{\textcolor[rgb]{0.25,0.44,0.63}{##1}}}

\def\PYGZbs{\char`\\}
\def\PYGZus{\char`\_}
\def\PYGZob{\char`\{}
\def\PYGZcb{\char`\}}
\def\PYGZca{\char`\^}
\def\PYGZam{\char`\&}
\def\PYGZlt{\char`\<}
\def\PYGZgt{\char`\>}
\def\PYGZsh{\char`\#}
\def\PYGZpc{\char`\%}
\def\PYGZdl{\char`\$}
\def\PYGZhy{\char`\-}
\def\PYGZsq{\char`\'}
\def\PYGZdq{\char`\"}
\def\PYGZti{\char`\~}
% for compatibility with earlier versions
\def\PYGZat{@}
\def\PYGZlb{[}
\def\PYGZrb{]}
\makeatother

\begin{document}

\maketitle
\tableofcontents
\phantomsection\label{master_doc::doc}



\chapter{Introduction}
\label{introduction:introduction}\label{introduction::doc}\label{introduction:welcome-to-rise-s-documentation}

\section{RISE at a Glance}
\label{intro_folder/rise_at_a_glance::doc}\label{intro_folder/rise_at_a_glance:rise-at-a-glance}

\subsection{What is RISE?}
\label{intro_folder/rise_at_a_glance:what-is-rise}
RISE is the acronym for \textbf{R}ationality \textbf{I}n \textbf{S}witching \textbf{E}nvironments.

It is an object-oriented Matlab toolbox primarily designed for solving and estimating nonlinear
dynamic stochastic general equilibirium (\textbf{DSGE}) or more generally
Rational Expectations(\textbf{RE}) models with \textbf{switching parameters}.

Leading references in the field include various papers by \href{http://www.tzha.net/articles}{Roger Farmer, Dan Waggoner and Tao Zha}
and \href{http://php.indiana.edu/~eleeper/\#Papers}{Eric Leeper} among others.

RISE uses perturbation to approximate the nonlinear Markov Switching Rational
Expectations (\textbf{MSRE}) model and solves it using efficient algorithms.

RISE also implements special cases of the general Switching MSRE model. This includes
\begin{itemize}
\item {} 
\textbf{VAR}s with and without switching parameters

\item {} 
\textbf{SVAR}s with and without switching paramters

\item {} 
\textbf{Time-varying parameter VAR}s

\item {} 
etc.

\end{itemize}


\subsection{Motivation for RISE development}
\label{intro_folder/rise_at_a_glance:motivation-for-rise-development}\begin{itemize}
\item {} 
The world is not constant, it is switching

\end{itemize}


\section{Capabilities of RISE}
\label{intro_folder/rise_capabilities::doc}\label{intro_folder/rise_capabilities:capabilities-of-rise}

\subsection{DSGE modeling}
\label{intro_folder/rise_capabilities:dsge-modeling}\begin{itemize}
\item {} 
constant parameters

\item {} \begin{description}
\item[{switching parameters}] \leavevmode\begin{itemize}
\item {} 
exogenous switching

\item {} 
endogenous switching

\end{itemize}

\end{description}

\item {} \begin{description}
\item[{optimal policy (with and without switching)}] \leavevmode\begin{itemize}
\item {} 
discretion

\item {} 
commitment

\item {} 
loose commitment

\item {} 
optimized simple rules

\end{itemize}

\end{description}

\item {} 
Deterministic simulation

\item {} 
Stochastic simulation

\item {} 
higher-order perturbations

\end{itemize}


\subsection{VAR modeling}
\label{intro_folder/rise_capabilities:var-modeling}\begin{itemize}
\item {} \begin{description}
\item[{constant parameters}] \leavevmode\begin{itemize}
\item {} 
zero restrictions

\item {} 
sign restrictions

\item {} 
restrictions on lag structure

\item {} 
linear restrictions

\end{itemize}

\end{description}

\item {} \begin{description}
\item[{switching parameters}] \leavevmode\begin{itemize}
\item {} 
linear restrictions

\end{itemize}

\end{description}

\end{itemize}


\subsection{SVAR modeling}
\label{intro_folder/rise_capabilities:svar-modeling}\begin{itemize}
\item {} 
constant parameters

\item {} \begin{description}
\item[{switching parameters}] \leavevmode\begin{itemize}
\item {} 
linear restrictions

\end{itemize}

\end{description}

\end{itemize}


\subsection{Time-Varying parameter VAR modeling}
\label{intro_folder/rise_capabilities:time-varying-parameter-var-modeling}
Under implementation


\subsection{Smooth transition VAR modeling}
\label{intro_folder/rise_capabilities:smooth-transition-var-modeling}
Not yet implemented


\subsection{Forecasting and Conditional Forecasting}
\label{intro_folder/rise_capabilities:forecasting-and-conditional-forecasting}

\subsection{Global sensitivity analysis}
\label{intro_folder/rise_capabilities:global-sensitivity-analysis}\begin{itemize}
\item {} 
Monte carlo filtering

\item {} 
High dimensional model representation

\end{itemize}


\subsection{Maximum Likelihood and Bayesian Estimation}
\label{intro_folder/rise_capabilities:maximum-likelihood-and-bayesian-estimation}\begin{itemize}
\item {} 
linear restrictions

\item {} 
nonlinear restrictions

\end{itemize}


\subsection{Time series}
\label{intro_folder/rise_capabilities:time-series}

\subsection{Reporting}
\label{intro_folder/rise_capabilities:reporting}

\section{How RISE works}
\label{intro_folder/how_rise_works:how-rise-works}\label{intro_folder/how_rise_works::doc}

\subsection{Object orientation}
\label{intro_folder/how_rise_works:object-orientation}

\subsection{Basic principles}
\label{intro_folder/how_rise_works:basic-principles}\begin{itemize}
\item {} 
you can pass different options at any time

\end{itemize}


\section{Background and mathematical formulations}
\label{intro_folder/background::doc}\label{intro_folder/background:background-and-mathematical-formulations}

\section{Using this documentation}
\label{intro_folder/using_this_doc:using-this-documentation}\label{intro_folder/using_this_doc::doc}

\subsection{how to find help}
\label{intro_folder/using_this_doc:how-to-find-help}

\subsection{Road map}
\label{intro_folder/using_this_doc:road-map}

\section{Citing RISE in your research}
\label{intro_folder/citing_rise:citing-rise-in-your-research}\label{intro_folder/citing_rise::doc}

\chapter{Getting started with RISE}
\label{getting_started:getting-started-with-rise}\label{getting_started::doc}

\section{Installation guide}
\label{getting_started_folder/installation_configuration::doc}\label{getting_started_folder/installation_configuration:installation-guide}

\subsection{Software requirements}
\label{getting_started_folder/installation_configuration:software-requirements}
I order to use RISE, the following software will need to be installed:
\begin{itemize}
\item {} 
Matlab version ? or higher

\item {} 
MikTex (Windows users) MacTex (mac users)

\end{itemize}


\subsection{How to obtain RISE}
\label{getting_started_folder/installation_configuration:how-to-obtain-rise}
There are (at least) two ways to acquire RISE:


\subsubsection{The zip file option}
\label{getting_started_folder/installation_configuration:the-zip-file-option}\begin{enumerate}
\item {} 
Go online to \href{https://github.com/jmaih/RISE\_toolbox}{https://github.com/jmaih/RISE\_toolbox}

\item {} 
download the zip file and unzip it in some directory on your
computer.

\end{enumerate}

This option is not recommended but is convenient for people
who are not allowed to install new software on their
machines/laptop.


\subsubsection{Github for the bleeding-edge installation (highly recommended)}
\label{getting_started_folder/installation_configuration:github-for-the-bleeding-edge-installation-highly-recommended}\begin{enumerate}
\item {} 
Go to \href{http://windows.github.com}{http://windows.github.com} if you are a windows user or
to \href{http://mac.github.com}{http://mac.github.com} if you are a mac user

\item {} 
Create an account online through the website and download
the Github program

\item {} 
Sign in both online and on the github on your machine. It is
obvious online, but on your machine, just go to
Github\textgreater{}Preference\textgreater{}Account

\item {} 
Go online to \href{https://github.com/jmaih/RISE\_toolbox}{https://github.com/jmaih/RISE\_toolbox}

\item {} 
Look for an icon with title ’Clone in Desktop’ (or possibly
clone in mac). There are options to locate where the
repository will reside

\end{enumerate}

The reason why this option is recommended is that you don't need to
re-download the whole toolbox every time a marginal update is made.
With one click and within seconds you can have the version of the toolbox
on your computer updated.


\subsubsection{The git option (never tested!!!)}
\label{getting_started_folder/installation_configuration:the-git-option-never-tested}
The following has never been tested and so the syntax might be wrong:

\begin{Verbatim}[commandchars=\\\{\}]
git clone https://github.com/jmaih/RISE\_toolbox.git
\end{Verbatim}


\subsubsection{Testing your installation}
\label{getting_started_folder/installation_configuration:testing-your-installation}
More on this later...


\subsection{Loading and starting RISE}
\label{getting_started_folder/installation_configuration:loading-and-starting-rise}\begin{enumerate}
\item {} 
Locate the RISE\_toolbox directory and add its path to matlab
in the command window as

\begin{Verbatim}[commandchars=\\\{\}]
addpath(’C:/Users/JMaih/GithubRepositories/RISE\_toolbox’)
\end{Verbatim}

\item {} 
You will need to adapt this path to conform with the location
of the toolbox on your machine.

\item {} 
run rise\_startup()

\end{enumerate}


\subsection{Updating RISE}
\label{getting_started_folder/installation_configuration:updating-rise}
New features are constantly added, efficiency is improved, users sometimes report bugs that are corrected.
All this makes it necessary to update RISE every now and then in order to keep abreast of the
latest changes and developments.

However, updating RISE depends on precisely how you installed it in the first place:
\begin{itemize}
\item {} 
If you downloaded a zip file, you will have to redownload a zip file even if the recent change was just an added comma.

\item {} 
if instead you invested in opening a github account, with one click you will be able to update just the changes you don't have.

\item {} 
with git, you would just execute the command

\begin{Verbatim}[commandchars=\\\{\}]
git pull
\end{Verbatim}

\end{itemize}


\section{Troubleshooting}
\label{getting_started_folder/troubleshooting::doc}\label{getting_started_folder/troubleshooting:troubleshooting}

\section{RISE basics/basic principles}
\label{getting_started_folder/basic_principles:rise-basics-basic-principles}\label{getting_started_folder/basic_principles::doc}\begin{enumerate}
\item {} 
create an empty RISE object e.g.

\begin{Verbatim}[commandchars=\\\{\}]
\PYG{n}{tao}\PYG{o}{=}\PYG{n}{rise}\PYG{o}{.}\PYG{n}{empty}\PYG{p}{(}\PYG{l+m+mi}{0}\PYG{p}{)}\PYG{p}{;}
\end{Verbatim}

\item {} 
run methods(rise) or methods(tao) to see the
functions/methods that can be applied to a RISE object

\item {} 
run those methods on r''. e.g. ``irf(r)'', simulate(r)'', solve(r)'',
etc. this will give you the default options of each method and
tell you how you can modify the behavior of the method

\end{enumerate}


\section{Tutorial: A toy example}
\label{getting_started_folder/tutorial:tutorial-a-toy-example}\label{getting_started_folder/tutorial::doc}

\subsection{Foerster, Rubio-Ramirez, Waggoner and Zha (2014)}
\label{getting_started_folder/tutorial:foerster-rubio-ramirez-waggoner-and-zha-2014}
They consider the following model:
\begin{gather}
\begin{split}E_{t}\left[
\begin{array}{c}
1-\beta\frac{\left( 1-\frac{\kappa}{2}\left( \Pi_{t}-1\right) ^{2}\right)
Y_{t}}{\left( 1-\frac{\kappa}{2}\left( \Pi_{t+1}-1\right) ^{2}\right) Y_{t+1}%
}\frac{1}{e^{\mu_{t+1}}}\frac{R_{t}}{\Pi_{t+1}} \\
\left( 1-\eta\right) +\eta\left( 1-\frac{\kappa}{2}\left( \Pi _{t}-1\right)
^{2}\right) Y_{t}+\beta\kappa\frac{\left( 1-\frac{\kappa}{2}\left(
\Pi_{t}-1\right) ^{2}\right) }{\left( 1-\frac{\kappa}{2}\left(
\Pi_{t+1}-1\right) ^{2}\right) }\left( \Pi_{t+1}-1\right)
\Pi_{t+1}-\kappa\left( \Pi_{t}-1\right) \Pi_{t} \\
\left( \frac{R_{t-1}}{R_{ss}}\right) ^{\rho}\Pi_{t}^{\left( 1-\rho\right)
\psi}\exp\left( \sigma\varepsilon_{t}\right) -\frac{R_{t}}{R_{ss}}%
\end{array}
\right] =0\end{split}\notag\\\begin{split}with\end{split}\notag\\\begin{split}\mu_{t+1}=\bar{\mu}+\sigma\hat{\mu}_{t+1}.\end{split}\notag
\end{gather}
The first equation is an Euler equation, the second equation a Phillips
curve and the third equation a nonlinear Taylor rule.

The switching parameters are $\mu$ and  $\psi$.


\subsection{The RISE code}
\label{getting_started_folder/tutorial:the-rise-code}
The RISE code with parameterization is given by

\begin{Verbatim}[commandchars=\\\{\}]
endogenous PAI,Y,R

exogenous EPS\_R

parameters a\_tp\_1\_2, a\_tp\_2\_1, betta, eta, kappa, mu, mu\_bar, psi, rhor, sigr
parameters(a,2) mu, psi

model
        1-betta*(1-.5*kappa*(PAI-1)\textasciicircum{}2)*Y*R/((1-.5*kappa*(PAI(+1)-1)\textasciicircum{}2)*Y(+1)*exp(mu)*PAI(+1));

        1-eta+eta*(1-.5*kappa*(PAI-1)\textasciicircum{}2)*Y+betta*kappa*(1-.5*kappa*(PAI-1)\textasciicircum{}2)*(PAI(+1)-1)*PAI(+1)/(1-.5*kappa*(PAI(+1)-1)\textasciicircum{}2)
        -kappa*(PAI-1)*PAI;

        (R(-1)/steady\_state(R))\textasciicircum{}rhor*(PAI/steady\_state(PAI))\textasciicircum{}((1-rhor)*psi)*exp(sigr*EPS\_R)-R/steady\_state(R);


steady\_state\_model(unique,imposed)
    PAI=1;
    Y=(eta-1)/eta;
    R=exp(mu\_bar)/betta*PAI;


parameterization
        a\_tp\_1\_2,1-.9;
        a\_tp\_2\_1,1-.9;
        betta, .99;
        kappa, 161;
        eta, 10;
        rhor, .8;
        sigr, 0.0025;
        mu\_bar,0.02;
        mu(a,1), 0.03;
        mu(a,2), 0.01;
        psi(a,1), 3.1;
        psi(a,2), 0.9;
\end{Verbatim}


\subsection{Running the example}
\label{getting_started_folder/tutorial:running-the-example}
Assume this example is saved in a file named frwz\_nk.rs . The to run this example in Matlab, we run the following commands:

\begin{Verbatim}[commandchars=\\\{\}]
frwz=rise('frwz\_nk'); \% load the model and its parameterization

frwz=solve(frwz); \% Solving the model

print\_solution(frwz) \% print the solution
\end{Verbatim}


\section{How to find help?}
\label{getting_started_folder/howto_find_doc:how-to-find-help}\label{getting_started_folder/howto_find_doc::doc}

\section{Where to go from here}
\label{getting_started_folder/where_to_go_now::doc}\label{getting_started_folder/where_to_go_now:where-to-go-from-here}

\chapter{RISE Capabilities}
\label{capabilities:rise-capabilities}\label{capabilities::doc}

\section{Overview}
\label{capabilities:overview}

\section{Markov switching DSGE modeling}
\label{capabilities:markov-switching-dsge-modeling}

\section{Markov switching SVAR modeling}
\label{capabilities:markov-switching-svar-modeling}

\section{Markov switching VAR modeling}
\label{capabilities:markov-switching-var-modeling}

\section{Smooth transition VAR modeling}
\label{capabilities:smooth-transition-var-modeling}

\section{Time-varying parameter modeling}
\label{capabilities:time-varying-parameter-modeling}

\section{Maximum Likelihood and Bayesian Estimation}
\label{capabilities:maximum-likelihood-and-bayesian-estimation}

\section{Differentiation}
\label{capabilities:differentiation}

\subsection{numerical differentiation}
\label{capabilities:numerical-differentiation}

\subsection{Symbolic differentiation}
\label{capabilities:symbolic-differentiation}

\subsection{Automatic/Algorithmic differentiation}
\label{capabilities:automatic-algorithmic-differentiation}

\section{Time series}
\label{capabilities:time-series}

\section{Reporting}
\label{capabilities:reporting}

\section{Derivative-free optimization}
\label{capabilities:derivative-free-optimization}

\section{Global sensitivity analysis}
\label{capabilities:global-sensitivity-analysis}

\subsection{Monte Carlo filtering}
\label{capabilities:monte-carlo-filtering}

\subsection{High dimensional model representation}
\label{capabilities:high-dimensional-model-representation}

\chapter{The Markov switching DSGE interface}
\label{dsge_interface:the-markov-switching-dsge-interface}\label{dsge_interface::doc}

\section{The general framework}
\label{dsge_interface:the-general-framework}
The general form of the models is:
\begin{gather}
\begin{split}E_{t}\sum_{r_{t+1}=1}^{h}\pi _{r_{t},r_{t+1}}\left( I_{t}\right) \tilde{d}_{r_{t}}\left(b_{t+1}
\left( r_{t+1}\right),b_{t}\left( r_{t}\right),b_{t-1},\varepsilon _{t}, \theta _{r_{t+1}}\right) =0\end{split}\notag
\end{gather}\begin{itemize}
\item {} 
The switching of the parameters is governed by Markov processes and can be endogenous.

\item {} 
\href{http://www.kansascityfed.org/publicat/events/research/2010CenBankForecasting/Maih\_paper.pdf}{Agents can have information about future events}

\end{itemize}


\section{The model file}
\label{dsge_interface:the-model-file}

\subsection{Conventions}
\label{dsge_interface:conventions}

\subsection{Variable declarations}
\label{dsge_interface:variable-declarations}

\subsection{Expressions}
\label{dsge_interface:expressions}\begin{itemize}
\item {} \begin{description}
\item[{parameters and variables}] \leavevmode\begin{itemize}
\item {} 
inside the model

\item {} 
outside the model

\end{itemize}

\end{description}

\item {} 
operators

\item {} \begin{description}
\item[{functions}] \leavevmode\begin{itemize}
\item {} 
built-in functions

\item {} 
external/user-defined functions

\end{itemize}

\end{description}

\end{itemize}


\subsection{model declaration}
\label{dsge_interface:model-declaration}\begin{itemize}
\item {} 
model equations

\item {} 
endogenous transition probabilities

\item {} 
auxiliary parameters/variables

\item {} 
inequality restrictions

\end{itemize}


\subsection{auxiliary variables}
\label{dsge_interface:auxiliary-variables}

\subsection{initial and terminal conditions}
\label{dsge_interface:initial-and-terminal-conditions}

\subsection{shocks on exogenous variables}
\label{dsge_interface:shocks-on-exogenous-variables}

\subsection{other general declarations}
\label{dsge_interface:other-general-declarations}

\section{steady state}
\label{dsge_interface:steady-state}\begin{itemize}
\item {} 
finding the steady state with the RISE nonlinear solver

\item {} 
using a steady state file

\item {} 
using the steady state model

\end{itemize}


\section{getting information about the model}
\label{dsge_interface:getting-information-about-the-model}

\section{deterministic simulation}
\label{dsge_interface:deterministic-simulation}

\section{stochastic solution and simulation}
\label{dsge_interface:stochastic-solution-and-simulation}\begin{itemize}
\item {} 
computing the stochastic solution

\item {} 
typology and ordering of variables

\item {} 
first-order approximation

\item {} 
second-order approximation

\item {} 
third-order approximation

\item {} 
fourth-order approximation

\item {} 
fifth-order approximation

\end{itemize}


\section{Estimation}
\label{dsge_interface:estimation}

\section{Forecasting and conditional forecasting}
\label{dsge_interface:forecasting-and-conditional-forecasting}

\section{Optimal policy}
\label{dsge_interface:optimal-policy}\begin{itemize}
\item {} 
optimal simple rules

\item {} 
Commitment, discretion and loose commitment

\end{itemize}


\chapter{Markov Switching Dynamic Stochastic General Equilibrium Modeling}
\label{classes/models/@dsge/dsge:markov-switching-dynamic-stochastic-general-equilibrium-modeling}\label{classes/models/@dsge/dsge::doc}

\section{methods}
\label{classes/models/@dsge/dsge:methods}\begin{itemize}
\item {} 
{[} {\hyperref[classes/models/@dsge/dsge:check-derivatives]{check\_derivatives}} {]}(dsge/check\_derivatives)

\item {} 
{[} {\hyperref[classes/models/@dsge/dsge:check-optimum]{check\_optimum}} {]}(dsge/check\_optimum)

\item {} 
{[} {\hyperref[classes/models/@dsge/dsge:compute-steady-state]{compute\_steady\_state}} {]}(dsge/compute\_steady\_state)

\item {} 
{[} {\hyperref[classes/models/@dsge/dsge:create-estimation-blocks]{create\_estimation\_blocks}} {]}(dsge/create\_estimation\_blocks)

\item {} 
{[} {\hyperref[classes/models/@dsge/dsge:create-state-list]{create\_state\_list}} {]}(dsge/create\_state\_list)

\item {} 
{[} {\hyperref[classes/models/@dsge/dsge:draw-parameter]{draw\_parameter}} {]}(dsge/draw\_parameter)

\item {} 
{[} {\hyperref[classes/models/@dsge/dsge:dsge]{dsge}} {]}(dsge/dsge)

\item {} 
{[} {\hyperref[classes/models/@dsge/dsge:estimate]{estimate}} {]}(dsge/estimate)

\item {} 
{[} {\hyperref[classes/models/@dsge/dsge:filter]{filter}} {]}(dsge/filter)

\item {} 
{[} {\hyperref[classes/models/@dsge/dsge:forecast]{forecast}} {]}(dsge/forecast)

\item {} 
{[} {\hyperref[classes/models/@dsge/dsge:forecast-real-time]{forecast\_real\_time}} {]}(dsge/forecast\_real\_time)

\item {} 
{[} {\hyperref[classes/models/@dsge/dsge:get]{get}} {]}(dsge/get)

\item {} 
{[} {\hyperref[classes/models/@dsge/dsge:historical-decomposition]{historical\_decomposition}} {]}(dsge/historical\_decomposition)

\item {} 
{[} {\hyperref[classes/models/@dsge/dsge:irf]{irf}} {]}(dsge/irf)

\item {} 
{[} {\hyperref[classes/models/@dsge/dsge:is-stable-system]{is\_stable\_system}} {]}(dsge/is\_stable\_system)

\item {} 
{[} {\hyperref[classes/models/@dsge/dsge:isnan]{isnan}} {]}(dsge/isnan)

\item {} 
{[} {\hyperref[classes/models/@dsge/dsge:load-parameters]{load\_parameters}} {]}(dsge/load\_parameters)

\item {} 
{[} {\hyperref[classes/models/@dsge/dsge:log-marginal-data-density]{log\_marginal\_data\_density}} {]}(dsge/log\_marginal\_data\_density)

\item {} 
{[} {\hyperref[classes/models/@dsge/dsge:log-posterior-kernel]{log\_posterior\_kernel}} {]}(dsge/log\_posterior\_kernel)

\item {} 
{[} {\hyperref[classes/models/@dsge/dsge:log-prior-density]{log\_prior\_density}} {]}(dsge/log\_prior\_density)

\item {} 
{[} {\hyperref[classes/models/@dsge/dsge:monte-carlo-filtering]{monte\_carlo\_filtering}} {]}(dsge/monte\_carlo\_filtering)

\item {} 
{[} {\hyperref[classes/models/@dsge/dsge:posterior-marginal-and-prior-densities]{posterior\_marginal\_and\_prior\_densities}} {]}(dsge/posterior\_marginal\_and\_prior\_densities)

\item {} 
{[} {\hyperref[classes/models/@dsge/dsge:posterior-simulator]{posterior\_simulator}} {]}(dsge/posterior\_simulator)

\item {} 
{[} {\hyperref[classes/models/@dsge/dsge:print-estimation-results]{print\_estimation\_results}} {]}(dsge/print\_estimation\_results)

\item {} 
{[} {\hyperref[classes/models/@dsge/dsge:print-solution]{print\_solution}} {]}(dsge/print\_solution)

\item {} 
{[} {\hyperref[classes/models/@dsge/dsge:prior-plots]{prior\_plots}} {]}(dsge/prior\_plots)

\item {} 
{[} {\hyperref[classes/models/@dsge/dsge:report]{report}} {]}(dsge/report)

\item {} 
{[} {\hyperref[classes/models/@dsge/dsge:resid]{resid}} {]}(dsge/resid)

\item {} 
{[} {\hyperref[classes/models/@dsge/dsge:set]{set}} {]}(dsge/set)

\item {} 
{[} {\hyperref[classes/models/@dsge/dsge:set-solution-to-companion]{set\_solution\_to\_companion}} {]}(dsge/set\_solution\_to\_companion)

\item {} 
{[} {\hyperref[classes/models/@dsge/dsge:simulate]{simulate}} {]}(dsge/simulate)

\item {} 
{[} {\hyperref[classes/models/@dsge/dsge:simulate-nonlinear]{simulate\_nonlinear}} {]}(dsge/simulate\_nonlinear)

\item {} 
{[} {\hyperref[classes/models/@dsge/dsge:simulation-diagnostics]{simulation\_diagnostics}} {]}(dsge/simulation\_diagnostics)

\item {} 
{[} {\hyperref[classes/models/@dsge/dsge:solve]{solve}} {]}(dsge/solve)

\item {} 
{[} {\hyperref[classes/models/@dsge/dsge:solve-alternatives]{solve\_alternatives}} {]}(dsge/solve\_alternatives)

\item {} 
{[} {\hyperref[classes/models/@dsge/dsge:stoch-simul]{stoch\_simul}} {]}(dsge/stoch\_simul)

\item {} 
{[} {\hyperref[classes/models/@dsge/dsge:theoretical-autocorrelations]{theoretical\_autocorrelations}} {]}(dsge/theoretical\_autocorrelations)

\item {} 
{[} {\hyperref[classes/models/@dsge/dsge:theoretical-autocovariances]{theoretical\_autocovariances}} {]}(dsge/theoretical\_autocovariances)

\item {} 
{[} {\hyperref[classes/models/@dsge/dsge:variance-decomposition]{variance\_decomposition}} {]}(dsge/variance\_decomposition)

\end{itemize}


\section{properties}
\label{classes/models/@dsge/dsge:properties}\begin{itemize}
\item {} 
{[}definitions{]} -

\item {} 
{[}equations{]} -

\item {} 
{[}folders\_paths{]} -

\item {} 
{[}dsge\_var{]} -

\item {} 
{[}filename{]} -

\item {} 
{[}legend{]} -

\item {} 
{[}endogenous{]} -

\item {} 
{[}exogenous{]} -

\item {} 
{[}parameters{]} -

\item {} 
{[}observables{]} -

\item {} 
{[}markov\_chains{]} -

\item {} 
{[}options{]} -

\item {} 
{[}estimation{]} -

\item {} 
{[}solution{]} -

\item {} 
{[}filtering{]} -

\end{itemize}


\bigskip\hrule{}\bigskip



\bigskip\hrule{}\bigskip



\section{check\_derivatives}
\label{classes/models/@dsge/dsge:check-derivatives}\label{classes/models/@dsge/dsge:id1}
\textbf{check\_derivatives} - compares the derivatives and the solutions from various differentiation techniques


\subsection{Syntax}
\label{classes/models/@dsge/dsge:syntax}
\begin{Verbatim}[commandchars=\\\{\}]
\PYG{n}{check\PYGZus{}derivatives}\PYG{p}{(}\PYG{n}{obj}\PYG{p}{)}
\PYG{n}{retcode}\PYG{o}{=}\PYG{n}{check\PYGZus{}derivatives}\PYG{p}{(}\PYG{n}{obj}\PYG{p}{)}
\end{Verbatim}


\subsection{Inputs}
\label{classes/models/@dsge/dsge:inputs}\begin{itemize}
\item {} 
\textbf{obj} {[}rise\textbar{}dsge{]}: model object or vectors of model objects

\end{itemize}


\subsection{Outputs}
\label{classes/models/@dsge/dsge:outputs}\begin{itemize}
\item {} 
\textbf{retcode} {[}numeric{]}: 0 if no problem is encountered during the
comparisons. Else the meaning of recode can be found by running
decipher(retcode)

\end{itemize}


\subsection{More About}
\label{classes/models/@dsge/dsge:more-about}\begin{itemize}
\item {} 
The derivatives computed are `automatic', `symbolic' or `numerical'

\item {} 
The comparisons are done relative to automatic derivatives, which are
assumed to be the most accurate.

\end{itemize}


\subsection{Examples}
\label{classes/models/@dsge/dsge:examples}
See also:


\bigskip\hrule{}\bigskip



\section{check\_optimum}
\label{classes/models/@dsge/dsge:check-optimum}\label{classes/models/@dsge/dsge:id2}
H1 line


\subsection{Syntax}
\label{classes/models/@dsge/dsge:id3}

\subsection{Inputs}
\label{classes/models/@dsge/dsge:id4}

\subsection{Outputs}
\label{classes/models/@dsge/dsge:id5}

\subsection{More About}
\label{classes/models/@dsge/dsge:id6}

\subsection{Examples}
\label{classes/models/@dsge/dsge:id7}
See also:

Help for dsge/check\_optimum is inherited from superclass RISE\_GENERIC


\bigskip\hrule{}\bigskip



\section{compute\_steady\_state}
\label{classes/models/@dsge/dsge:id8}\label{classes/models/@dsge/dsge:compute-steady-state}
H1 line


\subsection{Syntax}
\label{classes/models/@dsge/dsge:id9}

\subsection{Inputs}
\label{classes/models/@dsge/dsge:id10}

\subsection{Outputs}
\label{classes/models/@dsge/dsge:id11}

\subsection{More About}
\label{classes/models/@dsge/dsge:id12}

\subsection{Examples}
\label{classes/models/@dsge/dsge:id13}
See also:


\bigskip\hrule{}\bigskip



\section{create\_estimation\_blocks}
\label{classes/models/@dsge/dsge:create-estimation-blocks}\label{classes/models/@dsge/dsge:id14}
H1 line


\subsection{Syntax}
\label{classes/models/@dsge/dsge:id15}

\subsection{Inputs}
\label{classes/models/@dsge/dsge:id16}

\subsection{Outputs}
\label{classes/models/@dsge/dsge:id17}

\subsection{More About}
\label{classes/models/@dsge/dsge:id18}

\subsection{Examples}
\label{classes/models/@dsge/dsge:id19}
See also:


\bigskip\hrule{}\bigskip



\section{create\_state\_list}
\label{classes/models/@dsge/dsge:create-state-list}\label{classes/models/@dsge/dsge:id20}
\textbf{create\_state\_list} creates the list of the state variables in the solution


\subsection{Syntax}
\label{classes/models/@dsge/dsge:id21}
\begin{Verbatim}[commandchars=\\\{\}]
\PYG{n}{final\PYGZus{}list}\PYG{o}{=}\PYG{n}{create\PYGZus{}state\PYGZus{}list}\PYG{p}{(}\PYG{n}{m}\PYG{p}{)}
\PYG{n}{final\PYGZus{}list}\PYG{o}{=}\PYG{n}{create\PYGZus{}state\PYGZus{}list}\PYG{p}{(}\PYG{n}{m}\PYG{p}{,}\PYG{n}{orders}\PYG{p}{)}
\end{Verbatim}


\subsection{Inputs}
\label{classes/models/@dsge/dsge:id22}\begin{itemize}
\item {} 
\textbf{m} {[}dsge\textbar{}rise{]} : model object

\item {} 
\textbf{orders} {[}integer array\textbar{}\{1:m.options.solve\_order\}{]} : approximation
orders

\item {} 
\textbf{compact\_form} {[}true\textbar{}\{false\}{]} : if true, only unique combinations
will be returned. Else, all combinations will be returned.

\end{itemize}


\subsection{Outputs}
\label{classes/models/@dsge/dsge:id23}\begin{itemize}
\item {} 
\textbf{final\_list} {[}cellstr{]} : list of the state variables

\item {} 
\textbf{kept} {[}vector{]} : location of kept state variables (computed only if
compact\_form is set to true)

\end{itemize}


\subsection{More About}
\label{classes/models/@dsge/dsge:id24}

\subsection{Examples}
\label{classes/models/@dsge/dsge:id25}
See also:


\bigskip\hrule{}\bigskip



\section{draw\_parameter}
\label{classes/models/@dsge/dsge:draw-parameter}\label{classes/models/@dsge/dsge:id26}
H1 line


\subsection{Syntax}
\label{classes/models/@dsge/dsge:id27}

\subsection{Inputs}
\label{classes/models/@dsge/dsge:id28}

\subsection{Outputs}
\label{classes/models/@dsge/dsge:id29}

\subsection{More About}
\label{classes/models/@dsge/dsge:id30}

\subsection{Examples}
\label{classes/models/@dsge/dsge:id31}
See also:

Help for dsge/draw\_parameter is inherited from superclass RISE\_GENERIC


\bigskip\hrule{}\bigskip



\section{dsge}
\label{classes/models/@dsge/dsge:id32}\label{classes/models/@dsge/dsge:dsge}\begin{quote}

\textasciitilde{}\textasciitilde{} no help found
\end{quote}


\bigskip\hrule{}\bigskip



\section{estimate}
\label{classes/models/@dsge/dsge:estimate}\label{classes/models/@dsge/dsge:id33}
\textbf{estimate} - estimates the parameters of a RISE model


\subsection{Syntax}
\label{classes/models/@dsge/dsge:id34}
\begin{Verbatim}[commandchars=\\\{\}]
\PYG{n}{obj}\PYG{o}{=}\PYG{n}{estimate}\PYG{p}{(}\PYG{n}{obj}\PYG{p}{)}
\PYG{n}{obj}\PYG{o}{=}\PYG{n}{estimate}\PYG{p}{(}\PYG{n}{obj}\PYG{p}{,}\PYG{n}{varargin}\PYG{p}{)}
\end{Verbatim}


\subsection{Inputs}
\label{classes/models/@dsge/dsge:id35}\begin{itemize}
\item {} 
\textbf{obj} {[}rise\textbar{}dsge\textbar{}rfvar\textbar{}svar{]}: model object

\item {} 
\textbf{varargin} additional optional inputs among which the most relevant
for estimation are:

\item {} 
\textbf{estim\_parallel} {[}integer\textbar{}\{1\}{]}: Number of starting values

\item {} 
\textbf{estim\_start\_from\_mode} {[}true\textbar{}false\textbar{}\{{[}{]}\}{]}: when empty, the user is
prompted to answer the question as to whether to start estimation from
a previously found mode or not. If true or false, no question is asked.

\item {} 
\textbf{estim\_start\_date} {[}numeric\textbar{}char\textbar{}serial date{]}: date of the first
observation to use in the dataset provided for estimation

\item {} 
\textbf{estim\_end\_date} {[}numeric\textbar{}char\textbar{}serial date{]}: date of the last
observation to use in the dataset provided for estimation

\item {} 
\textbf{estim\_max\_trials} {[}integer\textbar{}\{500\}{]}: When the initial value of the
log-likelihood is too low, RISE uniformly draws from the prior support
in search for a better starting point. It will try this for a maximum
number of \textbf{estim\_max\_trials} times before squeaking with an error.

\item {} 
\textbf{estim\_start\_vals} {[}\{{[}{]}\}\textbar{}struct{]}: when not empty, the parameters
whose names are fields of the structure will see their start values
updated or overriden by the information in \textbf{estim\_start\_vals}. There
is no need to provide values to update the start values for the
estimated parameters.

\item {} 
\textbf{estim\_general\_restrictions} {[}\{{[}{]}\}\textbar{}function handle{]}: when not empty,
the argument should be a function handle that takes as input a
parameterized RISE object and returns a scalar or vector of numbers
representing the strength of the violation of the nonlinear
constraints. Those constraints will be added to the constraints already
included in a rise/dsge file before being presented to the optimization
function.

\item {} 
\textbf{estim\_linear\_restrictions} {[}\{{[}{]}\}\textbar{}cell{]}: This is most often used in
the estimation of rfvar or svar models either to impose block
exogeneity or to impose other forms of linear restrictions. When not
empty, \textbf{estim\_linear\_restrictions} must be a 2-column cell:
- Each row of the first column represents a particular linear
combination of the estimated parameters. Those linear combinations are
constructed using the \textbf{coef} class. Check help for coef.coef for more
details.
- Each row of the second column holds the value of the linear
combination.

\item {} 
\textbf{estim\_blocks} {[}\{{[}{]}\}\textbar{}cell{]}: When not empty, this triggers blockwise
optimization. For further information on how to set blocks, see help
for dsge.create\_estimation\_blocks

\item {} 
\textbf{estim\_priors} {[}\{{[}{]}\}\textbar{}struct{]}: This provides an alternative to
setting priors inside the rise/dsge model file. Each field of the
structure must be the name of an estimated parameter. Each field will
hold a cell array whose structure is described in help
rise\_generic.setup\_priors.

\item {} 
\textbf{estim\_penalty} {[}numeric\textbar{}\{1e+8\}{]}: value of the objective function
when a problem occurs. Possible problems include:
- no solution found
- very low likelihood
- stochastic singularity
- problems computing the initial covariance matrix
- non-positive definite covariance matrices
- etc.

\item {} 
\textbf{estim\_cond\_vars} {[}char\textbar{}cellstr{]}: list of the variables to condition
on during estimation of a dsge model. This then assumes that the data
provided for estimation have several pages. The first page is the
actual data, while the subsequent pages are the expectations data.

\item {} 
\textbf{optimset} {[}struct{]}: identical to matlab's optimset

\item {} 
\textbf{optimizer} {[}char\textbar{}function handle\textbar{}cell{]}: This can be the name of a
standard matlab optimizer or RISE optimization routine or a
user-defined optimization procedure available of the matlab search
path. If the optimzer is provided as a cell, then the first element of
the cell is the name of the optimizer or its handle and the remaining
entries in the cell are additional input arguments to the user-defined
optimization routine. A user-defined optimization function should have
the following syntax

\begin{Verbatim}[commandchars=\\\{\}]
\PYG{p}{[}\PYG{n}{xfinal}\PYG{p}{,}\PYG{n}{ffinal}\PYG{p}{,}\PYG{n}{exitflag}\PYG{p}{]}\PYG{o}{=}\PYG{n}{optimizer}\PYG{p}{(}\PYG{n}{fh}\PYG{p}{,}\PYG{n}{x0}\PYG{p}{,}\PYG{n}{lb}\PYG{p}{,}\PYG{n}{ub}\PYG{p}{,}\PYG{n}{options}\PYG{p}{,}\PYG{n}{varargin}\PYG{p}{)}\PYG{p}{;}
\end{Verbatim}
\begin{description}
\item[{That is, it accepts as inputs:}] \leavevmode\begin{itemize}
\item {} 
\textbf{fh}: the function to optimize

\item {} 
\textbf{x0}: a vector column of initial values of the parameters

\item {} 
\textbf{lb}: a vector column of lower bounds

\item {} 
\textbf{ub}: a vector column of upper bounds

\item {} \begin{description}
\item[{\textbf{options}: a structure of options whose fields will be similar}] \leavevmode
to matlab's optimset

\end{description}

\item {} \begin{description}
\item[{\textbf{varargin}: additional arguments to the user-defined}] \leavevmode
optimization procedure

\end{description}

\end{itemize}

\item[{That is, it provides as outputs:}] \leavevmode\begin{itemize}
\item {} 
\textbf{xfinal}: the vector of final values

\item {} 
\textbf{ffinal}: the value of \textbf{fh} at \textbf{xfinal}

\item {} 
\textbf{exitflag}: a flag similar to the ones provided by matlab's

\end{itemize}

optimization functions.

\end{description}

\item {} 
\textbf{hessian\_type} {[}\{`fd'\}\textbar{}'opg'{]}: The hessian is either computed by
finite differences (fd) or by outer-product-gradient (opg)

\item {} 
\textbf{hessian\_repair} {[}\{false\}\textbar{}true{]}: If the Hessian is not positive
definite, it nevertheless can be repaired and prepared for a potential
mcmc simulation.

\end{itemize}


\subsection{Outputs}
\label{classes/models/@dsge/dsge:id36}\begin{itemize}
\item {} 
\textbf{obj} {[}rise\textbar{}dsge\textbar{}rfvar\textbar{}svar{]}: model object parameterized with the
mode found and holding additional estimation results and statistics
that can be found under obj.estimation

\end{itemize}


\subsection{More About}
\label{classes/models/@dsge/dsge:id37}\begin{itemize}
\item {} 
recursive estimation may be done easily by passing a different
estim\_end\_date at the beginning of each estimation run.

\end{itemize}


\subsection{Examples}
\label{classes/models/@dsge/dsge:id38}
See also:

Help for dsge/estimate is inherited from superclass RISE\_GENERIC


\bigskip\hrule{}\bigskip



\section{filter}
\label{classes/models/@dsge/dsge:filter}\label{classes/models/@dsge/dsge:id39}
H1 line


\subsection{Syntax}
\label{classes/models/@dsge/dsge:id40}

\subsection{Inputs}
\label{classes/models/@dsge/dsge:id41}

\subsection{Outputs}
\label{classes/models/@dsge/dsge:id42}

\subsection{More About}
\label{classes/models/@dsge/dsge:id43}

\subsection{Examples}
\label{classes/models/@dsge/dsge:id44}
See also:


\bigskip\hrule{}\bigskip



\section{forecast}
\label{classes/models/@dsge/dsge:id45}\label{classes/models/@dsge/dsge:forecast}
\textbf{forecast} - computes forecasts for rise\textbar{}dsge\textbar{}svar\textbar{}rfvar models


\subsection{Syntax}
\label{classes/models/@dsge/dsge:id46}
\begin{Verbatim}[commandchars=\\\{\}]
\PYG{n}{cond\PYGZus{}fkst\PYGZus{}db}\PYG{o}{=}\PYG{n}{forecast}\PYG{p}{(}\PYG{n}{obj}\PYG{p}{,}\PYG{n}{varargin}\PYG{p}{)}
\end{Verbatim}


\subsection{Inputs}
\label{classes/models/@dsge/dsge:id47}\begin{itemize}
\item {} 
\textbf{obj} {[}rise\textbar{}dsge\textbar{}svar\textbar{}rfvar{]}: model object

\item {} 
\textbf{varargin} : additional inputs coming in pairs. These include but are
not restricted to:
- \textbf{forecast\_to\_time\_series} {[}\{true\}\textbar{}false{]}: sets the output to time
\begin{quote}

series format or not
\end{quote}
\begin{itemize}
\item {} 
\textbf{forecast\_nsteps} {[}integer\textbar{}\{12\}{]}: number of forecasting steps

\item {} \begin{description}
\item[{\textbf{forecast\_start\_date} {[}char\textbar{}numeric\textbar{}serial date{]}: date when the}] \leavevmode
forecasts start (end of history + 1)

\end{description}

\item {} \begin{description}
\item[{\textbf{forecast\_conditional\_hypothesis} {[}\{jma\}\textbar{}ncp\textbar{}nas{]}: in dsge models in}] \leavevmode
which agents have information beyond the current period, this
option determines the number of periods of shocks need to match the
restrictions:
- Hypothesis \textbf{jma} assumes that irrespective of how
\begin{quote}

many periods of conditioning information are remaining, agents
always receive information on the same number of shocks.
\end{quote}
\begin{itemize}
\item {} \begin{description}
\item[{Hypothesis \textbf{ncp} assumes there are as many shocks periods as}] \leavevmode
the number of the number of conditioning periods

\end{description}

\item {} \begin{description}
\item[{Hypothesis \textbf{nas} assumes there are as many shocks periods as}] \leavevmode
the number of anticipated steps

\end{description}

\end{itemize}

\end{description}

\end{itemize}

\end{itemize}


\subsection{Outputs}
\label{classes/models/@dsge/dsge:id48}\begin{itemize}
\item {} 
\textbf{cond\_fkst\_db} {[}struct\textbar{}matrix{]}: depending on the value of
\textbf{forecast\_to\_time\_series} the returned output is a structure with
time series or a cell containing a matrix and the information to
reconstruct the time series.

\end{itemize}


\subsection{More About}
\label{classes/models/@dsge/dsge:id49}\begin{itemize}
\item {} 
the historical information as well as the conditioning information come
from the same database. The time series must be organized such that for
each series, the first page represents the actual data and all
subsequent pages represent conditional information. If a particular
condition is ``nan'', that location is not constrained

\item {} 
Conditional forecasting for nonlinear models is also supported.
However, the solving of the implied nonlinear problem may fail if the
model displays instability

\item {} 
Both HARD CONDITIONS and SOFT CONDITIONS are implemented but the latter
are currently disabled in expectation of a better user interface.

\item {} 
The data may also contain time series for a variable with name
\textbf{regime} in that case, the forecast/simulation paths are computed
following the information therein. \textbf{regime} must be a member of 1:h,
where h is the maximum number of regimes.

\end{itemize}


\subsection{Examples}
\label{classes/models/@dsge/dsge:id50}
See also: simulate

Help for dsge/forecast is inherited from superclass RISE\_GENERIC


\bigskip\hrule{}\bigskip



\section{forecast\_real\_time}
\label{classes/models/@dsge/dsge:forecast-real-time}\label{classes/models/@dsge/dsge:id51}
\textbf{forecast\_real\_time} - forecast from each point in time


\subsection{Syntax}
\label{classes/models/@dsge/dsge:id52}
\begin{Verbatim}[commandchars=\\\{\}]
\PYG{o}{\PYGZhy{}} \PYG{p}{[}\PYG{n}{ts\PYGZus{}fkst}\PYG{p}{,}\PYG{n}{ts\PYGZus{}rmse}\PYG{p}{,}\PYG{n}{rmse}\PYG{p}{,}\PYG{n}{Updates}\PYG{p}{]}\PYG{o}{=}\PYG{n}{forecast\PYGZus{}real\PYGZus{}time}\PYG{p}{(}\PYG{n}{obj}\PYG{p}{)}
\PYG{o}{\PYGZhy{}} \PYG{p}{[}\PYG{n}{ts\PYGZus{}fkst}\PYG{p}{,}\PYG{n}{ts\PYGZus{}rmse}\PYG{p}{,}\PYG{n}{rmse}\PYG{p}{,}\PYG{n}{Updates}\PYG{p}{]}\PYG{o}{=}\PYG{n}{forecast\PYGZus{}real\PYGZus{}time}\PYG{p}{(}\PYG{n}{obj}\PYG{p}{,}\PYG{n}{varargin}\PYG{p}{)}
\end{Verbatim}


\subsection{Inputs}
\label{classes/models/@dsge/dsge:id53}\begin{itemize}
\item {} 
\textbf{obj} {[}dsge\textbar{}svar\textbar{}rfvar{]} : model object

\item {} 
\textbf{varargin} : valid optional inputs coming in pairs. The main inputs
of interest for changing the default behavior are:
- \textbf{fkst\_rt\_nahead} {[}integer{]} : number of periods ahead

\end{itemize}


\subsection{Outputs}
\label{classes/models/@dsge/dsge:id54}\begin{itemize}
\item {} 
\textbf{ts\_fkst} {[}struct{]} : fields are forecasts in the form of ts objects
for the different endogenous variables

\item {} 
\textbf{ts\_rmse} {[}struct{]} : fields are RMSEs in the form of ts objects
for the different endogenous variables

\item {} 
\textbf{rmse} {[}matrix{]} : RMSEs for the different endogenous variables

\item {} 
\textbf{Updates} {[}struct{]} : fields are the updated (in a filtering sense) in
the form of ts objects for the different endogenous variables

\end{itemize}


\subsection{More About}
\label{classes/models/@dsge/dsge:id55}

\subsection{Examples}
\label{classes/models/@dsge/dsge:id56}
See also: plot\_real\_time


\bigskip\hrule{}\bigskip



\section{get}
\label{classes/models/@dsge/dsge:id57}\label{classes/models/@dsge/dsge:get}
H1 line


\subsection{Syntax}
\label{classes/models/@dsge/dsge:id58}

\subsection{Inputs}
\label{classes/models/@dsge/dsge:id59}

\subsection{Outputs}
\label{classes/models/@dsge/dsge:id60}

\subsection{More About}
\label{classes/models/@dsge/dsge:id61}

\subsection{Examples}
\label{classes/models/@dsge/dsge:id62}
See also:

Help for dsge/get is inherited from superclass RISE\_GENERIC


\bigskip\hrule{}\bigskip



\section{historical\_decomposition}
\label{classes/models/@dsge/dsge:id63}\label{classes/models/@dsge/dsge:historical-decomposition}
\textbf{historical\_decomposition} Computes historical decompositions of a DSGE model


\subsection{Syntax}
\label{classes/models/@dsge/dsge:id64}
\begin{Verbatim}[commandchars=\\\{\}]
\PYG{p}{[}\PYG{n}{Histdec}\PYG{p}{,}\PYG{n}{obj}\PYG{p}{]}\PYG{o}{=}\PYG{n}{history\PYGZus{}dec}\PYG{p}{(}\PYG{n}{obj}\PYG{p}{)}
\PYG{p}{[}\PYG{n}{Histdec}\PYG{p}{,}\PYG{n}{obj}\PYG{p}{]}\PYG{o}{=}\PYG{n}{history\PYGZus{}dec}\PYG{p}{(}\PYG{n}{obj}\PYG{p}{,}\PYG{n}{varargin}\PYG{p}{)}
\end{Verbatim}


\subsection{Inputs}
\label{classes/models/@dsge/dsge:id65}\begin{itemize}
\item {} 
obj : {[}rise\textbar{}dsge\textbar{}rfvar\textbar{}svar{]} model(s) for which to compute the
decomposition. obj could be a vector of models

\item {} 
varargin : standard optional inputs \textbf{coming in pairs}. Among which:
- \textbf{histdec\_start\_date} : {[}char\textbar{}numeric\textbar{}\{`'\}{]} : date at which the
\begin{quote}

decomposition starts. If empty, the decomposition starts at he
beginning of the history of the dataset
\end{quote}

\end{itemize}


\subsection{Outputs}
\label{classes/models/@dsge/dsge:id66}\begin{itemize}
\item {} 
Histdec : {[}struct\textbar{}cell array{]} structure or cell array of structures
with the decompositions in each model. The decompositions are given in
terms of:
- the exogenous variables
- \textbf{InitialConditions} : the effect of initial conditions
- \textbf{risk} : measure of the effect of non-certainty equivalence
- \textbf{switch} : the effect of switching (which is also a shock!!!)
- \textbf{steady\_state} : the contribution of the steady state

\end{itemize}


\subsection{Remarks}
\label{classes/models/@dsge/dsge:remarks}\begin{itemize}
\item {} 
the elements that do not contribute to any of the variables are
automatically discarded.

\item {} 
\textbf{N.B} : a switching model is inherently nonlinear and so, strictly
speaking, the type of decomposition we do for linear/linearized
constant-parameter models is not feasible. RISE takes an approximation
in which the variables, shocks and states matrices across states are
averaged. The averaging weights are the smoothed probabilities.

\end{itemize}


\subsection{Examples}
\label{classes/models/@dsge/dsge:id67}
See also:

Help for dsge/historical\_decomposition is inherited from superclass RISE\_GENERIC


\bigskip\hrule{}\bigskip



\section{irf}
\label{classes/models/@dsge/dsge:id68}\label{classes/models/@dsge/dsge:irf}
H1 line


\subsection{Syntax}
\label{classes/models/@dsge/dsge:id69}

\subsection{Inputs}
\label{classes/models/@dsge/dsge:id70}

\subsection{Outputs}
\label{classes/models/@dsge/dsge:id71}

\subsection{More About}
\label{classes/models/@dsge/dsge:id72}

\subsection{Examples}
\label{classes/models/@dsge/dsge:id73}
See also:


\bigskip\hrule{}\bigskip



\section{is\_stable\_system}
\label{classes/models/@dsge/dsge:is-stable-system}\label{classes/models/@dsge/dsge:id74}
H1 line


\subsection{Syntax}
\label{classes/models/@dsge/dsge:id75}

\subsection{Inputs}
\label{classes/models/@dsge/dsge:id76}

\subsection{Outputs}
\label{classes/models/@dsge/dsge:id77}

\subsection{More About}
\label{classes/models/@dsge/dsge:id78}

\subsection{Examples}
\label{classes/models/@dsge/dsge:id79}
See also:


\bigskip\hrule{}\bigskip



\section{isnan}
\label{classes/models/@dsge/dsge:isnan}\label{classes/models/@dsge/dsge:id80}
H1 line


\subsection{Syntax}
\label{classes/models/@dsge/dsge:id81}

\subsection{Inputs}
\label{classes/models/@dsge/dsge:id82}

\subsection{Outputs}
\label{classes/models/@dsge/dsge:id83}

\subsection{More About}
\label{classes/models/@dsge/dsge:id84}

\subsection{Examples}
\label{classes/models/@dsge/dsge:id85}
See also:

Help for dsge/isnan is inherited from superclass RISE\_GENERIC


\bigskip\hrule{}\bigskip



\section{load\_parameters}
\label{classes/models/@dsge/dsge:id86}\label{classes/models/@dsge/dsge:load-parameters}
H1 line


\subsection{Syntax}
\label{classes/models/@dsge/dsge:id87}

\subsection{Inputs}
\label{classes/models/@dsge/dsge:id88}

\subsection{Outputs}
\label{classes/models/@dsge/dsge:id89}

\subsection{More About}
\label{classes/models/@dsge/dsge:id90}

\subsection{Examples}
\label{classes/models/@dsge/dsge:id91}
See also:

Help for dsge/load\_parameters is inherited from superclass RISE\_GENERIC


\bigskip\hrule{}\bigskip



\section{log\_marginal\_data\_density}
\label{classes/models/@dsge/dsge:id92}\label{classes/models/@dsge/dsge:log-marginal-data-density}
H1 line


\subsection{Syntax}
\label{classes/models/@dsge/dsge:id93}

\subsection{Inputs}
\label{classes/models/@dsge/dsge:id94}

\subsection{Outputs}
\label{classes/models/@dsge/dsge:id95}

\subsection{More About}
\label{classes/models/@dsge/dsge:id96}

\subsection{Examples}
\label{classes/models/@dsge/dsge:id97}
See also:

Help for dsge/log\_marginal\_data\_density is inherited from superclass RISE\_GENERIC


\bigskip\hrule{}\bigskip



\section{log\_posterior\_kernel}
\label{classes/models/@dsge/dsge:log-posterior-kernel}\label{classes/models/@dsge/dsge:id98}
H1 line


\subsection{Syntax}
\label{classes/models/@dsge/dsge:id99}

\subsection{Inputs}
\label{classes/models/@dsge/dsge:id100}

\subsection{Outputs}
\label{classes/models/@dsge/dsge:id101}

\subsection{More About}
\label{classes/models/@dsge/dsge:id102}

\subsection{Examples}
\label{classes/models/@dsge/dsge:id103}
See also:

Help for dsge/log\_posterior\_kernel is inherited from superclass RISE\_GENERIC


\bigskip\hrule{}\bigskip



\section{log\_prior\_density}
\label{classes/models/@dsge/dsge:id104}\label{classes/models/@dsge/dsge:log-prior-density}
H1 line


\subsection{Syntax}
\label{classes/models/@dsge/dsge:id105}

\subsection{Inputs}
\label{classes/models/@dsge/dsge:id106}

\subsection{Outputs}
\label{classes/models/@dsge/dsge:id107}

\subsection{More About}
\label{classes/models/@dsge/dsge:id108}

\subsection{Examples}
\label{classes/models/@dsge/dsge:id109}
See also:

Help for dsge/log\_prior\_density is inherited from superclass RISE\_GENERIC


\bigskip\hrule{}\bigskip



\section{monte\_carlo\_filtering}
\label{classes/models/@dsge/dsge:monte-carlo-filtering}\label{classes/models/@dsge/dsge:id110}
H1 line


\subsection{Syntax}
\label{classes/models/@dsge/dsge:id111}

\subsection{Inputs}
\label{classes/models/@dsge/dsge:id112}

\subsection{Outputs}
\label{classes/models/@dsge/dsge:id113}

\subsection{More About}
\label{classes/models/@dsge/dsge:id114}

\subsection{Examples}
\label{classes/models/@dsge/dsge:id115}
See also:


\bigskip\hrule{}\bigskip



\section{posterior\_marginal\_and\_prior\_densities}
\label{classes/models/@dsge/dsge:id116}\label{classes/models/@dsge/dsge:posterior-marginal-and-prior-densities}
H1 line


\subsection{Syntax}
\label{classes/models/@dsge/dsge:id117}

\subsection{Inputs}
\label{classes/models/@dsge/dsge:id118}

\subsection{Outputs}
\label{classes/models/@dsge/dsge:id119}

\subsection{More About}
\label{classes/models/@dsge/dsge:id120}

\subsection{Examples}
\label{classes/models/@dsge/dsge:id121}
See also:

Help for dsge/posterior\_marginal\_and\_prior\_densities is inherited from superclass RISE\_GENERIC


\bigskip\hrule{}\bigskip



\section{posterior\_simulator}
\label{classes/models/@dsge/dsge:posterior-simulator}\label{classes/models/@dsge/dsge:id122}
H1 line


\subsection{Syntax}
\label{classes/models/@dsge/dsge:id123}

\subsection{Inputs}
\label{classes/models/@dsge/dsge:id124}

\subsection{Outputs}
\label{classes/models/@dsge/dsge:id125}

\subsection{More About}
\label{classes/models/@dsge/dsge:id126}

\subsection{Examples}
\label{classes/models/@dsge/dsge:id127}
See also:

Help for dsge/posterior\_simulator is inherited from superclass RISE\_GENERIC


\bigskip\hrule{}\bigskip



\section{print\_estimation\_results}
\label{classes/models/@dsge/dsge:print-estimation-results}\label{classes/models/@dsge/dsge:id128}
H1 line


\subsection{Syntax}
\label{classes/models/@dsge/dsge:id129}

\subsection{Inputs}
\label{classes/models/@dsge/dsge:id130}

\subsection{Outputs}
\label{classes/models/@dsge/dsge:id131}

\subsection{More About}
\label{classes/models/@dsge/dsge:id132}

\subsection{Examples}
\label{classes/models/@dsge/dsge:id133}
See also:

Help for dsge/print\_estimation\_results is inherited from superclass RISE\_GENERIC


\bigskip\hrule{}\bigskip



\section{print\_solution}
\label{classes/models/@dsge/dsge:print-solution}\label{classes/models/@dsge/dsge:id134}
\textbf{print\_solution} - print the solution of a model or vector of models


\subsection{Syntax}
\label{classes/models/@dsge/dsge:id135}
\begin{Verbatim}[commandchars=\\\{\}]
\PYG{n}{print\PYGZus{}solution}\PYG{p}{(}\PYG{n}{obj}\PYG{p}{)}
\PYG{n}{print\PYGZus{}solution}\PYG{p}{(}\PYG{n}{obj}\PYG{p}{,}\PYG{n}{varlist}\PYG{p}{)}
\PYG{n}{print\PYGZus{}solution}\PYG{p}{(}\PYG{n}{obj}\PYG{p}{,}\PYG{n}{varlist}\PYG{p}{,}\PYG{n}{orders}\PYG{p}{)}
\PYG{n}{print\PYGZus{}solution}\PYG{p}{(}\PYG{n}{obj}\PYG{p}{,}\PYG{n}{varlist}\PYG{p}{,}\PYG{n}{orders}\PYG{p}{,}\PYG{n}{compact\PYGZus{}form}\PYG{p}{)}
\PYG{n}{print\PYGZus{}solution}\PYG{p}{(}\PYG{n}{obj}\PYG{p}{,}\PYG{n}{varlist}\PYG{p}{,}\PYG{n}{orders}\PYG{p}{,}\PYG{n}{compact\PYGZus{}form}\PYG{p}{,}\PYG{n}{precision}\PYG{p}{)}
\PYG{n}{print\PYGZus{}solution}\PYG{p}{(}\PYG{n}{obj}\PYG{p}{,}\PYG{n}{varlist}\PYG{p}{,}\PYG{n}{orders}\PYG{p}{,}\PYG{n}{compact\PYGZus{}form}\PYG{p}{,}\PYG{n}{precision}\PYG{p}{,}\PYG{n}{equation\PYGZus{}format}\PYG{p}{)}
\PYG{n}{print\PYGZus{}solution}\PYG{p}{(}\PYG{n}{obj}\PYG{p}{,}\PYG{n}{varlist}\PYG{p}{,}\PYG{n}{orders}\PYG{p}{,}\PYG{n}{compact\PYGZus{}form}\PYG{p}{,}\PYG{n}{precision}\PYG{p}{,}\PYG{n}{equation\PYGZus{}format}\PYG{p}{,}\PYG{n}{file2save2}\PYG{p}{)}
\PYG{n}{outcell}\PYG{o}{=}\PYG{n}{print\PYGZus{}solution}\PYG{p}{(}\PYG{n}{obj}\PYG{p}{,}\PYG{o}{.}\PYG{o}{.}\PYG{o}{.}\PYG{p}{)}
\end{Verbatim}


\subsection{Inputs}
\label{classes/models/@dsge/dsge:id136}\begin{itemize}
\item {} 
\textbf{obj} {[}rise\textbar{}dsge{]} : model object or vector of model objects

\item {} 
\textbf{varlist} {[}char\textbar{}cellstr\textbar{}\{{[}{]}\}{]} : list of variables of interest

\item {} 
\textbf{orders} {[}numeric\textbar{}\{{[}1:solve\_order{]}\}{]} : orders for which we want to
see the solution

\item {} 
\textbf{compact\_form} {[}\{true\}\textbar{}false{]} : if true, only the solution of unique
tuples (i,j,k) such that i\textless{}=j\textless{}=k is presented. If false, the solution
of all combinations is presented. i.e.
(i,j,k)(i,k,j)(j,i,k)(j,k,i)(k,i,j)(k,j,i)

\item {} 
\textbf{precision} {[}char\textbar{}\{`\%8.6f'\}{]} : precision of the numbers printed

\item {} 
\textbf{equation\_format} {[}true\textbar{}\{false\}{]} : if true, the solution is presented
in the form of equations for each endogenous variable (not recommended)

\item {} 
\textbf{file2save2} {[}char\textbar{}\{`'\}{]} : if not empty, the solution is written to a
file rather than printed on screen. For this to happen, print\_solution
has to be called without ouput arguments

\end{itemize}


\subsection{Outputs}
\label{classes/models/@dsge/dsge:id137}\begin{itemize}
\item {} 
\textbf{outcell} {[}cellstr{]} : If an output is requested, the solution is not
printed on screen or to a file.

\end{itemize}


\subsection{More About}
\label{classes/models/@dsge/dsge:id138}
If a model is solved, say, up to 3rd order, one may still want to see the
first-order solution or the solution up to second-order only or any
combination of orders.


\subsection{Examples}
\label{classes/models/@dsge/dsge:id139}
See also:


\bigskip\hrule{}\bigskip



\section{prior\_plots}
\label{classes/models/@dsge/dsge:prior-plots}\label{classes/models/@dsge/dsge:id140}
H1 line


\subsection{Syntax}
\label{classes/models/@dsge/dsge:id141}

\subsection{Inputs}
\label{classes/models/@dsge/dsge:id142}

\subsection{Outputs}
\label{classes/models/@dsge/dsge:id143}

\subsection{More About}
\label{classes/models/@dsge/dsge:id144}

\subsection{Examples}
\label{classes/models/@dsge/dsge:id145}
See also:

Help for dsge/prior\_plots is inherited from superclass RISE\_GENERIC


\bigskip\hrule{}\bigskip



\section{report}
\label{classes/models/@dsge/dsge:report}\label{classes/models/@dsge/dsge:id146}
\textbf{REPORT} assigns the elements of interest to a rise\_report.report object


\subsection{Syntax}
\label{classes/models/@dsge/dsge:id147}\begin{description}
\item[{::}] \leavevmode\begin{itemize}
\item {} 
REPORT(rise.empty(0)) : displays the default inputs

\item {} 
REPORT(obj,destination\_root,rep\_items) : assigns the reported
elements in rep\_items to destination\_root

\item {} 
REPORT(obj,destination\_root,rep\_items,varargin) : assigns varargin to
obj before doing the rest

\end{itemize}

\end{description}


\subsection{Inputs}
\label{classes/models/@dsge/dsge:id148}\begin{itemize}
\item {} 
obj : {[}rise\textbar{}dsge{]}

\item {} 
destination\_root : {[}rise\_report.report{]} : handle for the actual report

\item {} 
rep\_items : {[}char\textbar{}cellstr{]} : list of desired items to report. This list
can only include : `endogenous', `exogenous', `observables',
`parameters', `solution', `estimation', `estimation\_statistics',
`equations', `code'

\end{itemize}


\subsection{Outputs}
\label{classes/models/@dsge/dsge:id149}
none


\subsection{More About}
\label{classes/models/@dsge/dsge:id150}

\subsection{Examples}
\label{classes/models/@dsge/dsge:id151}
See also:

Help for dsge/report is inherited from superclass RISE\_GENERIC


\bigskip\hrule{}\bigskip



\section{resid}
\label{classes/models/@dsge/dsge:id152}\label{classes/models/@dsge/dsge:resid}
H1 line


\subsection{Syntax}
\label{classes/models/@dsge/dsge:id153}

\subsection{Inputs}
\label{classes/models/@dsge/dsge:id154}

\subsection{Outputs}
\label{classes/models/@dsge/dsge:id155}

\subsection{More About}
\label{classes/models/@dsge/dsge:id156}

\subsection{Examples}
\label{classes/models/@dsge/dsge:id157}
See also:


\bigskip\hrule{}\bigskip



\section{set}
\label{classes/models/@dsge/dsge:set}\label{classes/models/@dsge/dsge:id158}
\textbf{set} - sets options for dsge\textbar{}rise models


\subsection{Syntax}
\label{classes/models/@dsge/dsge:id159}
\begin{Verbatim}[commandchars=\\\{\}]
\PYG{n}{obj}\PYG{o}{=}\PYG{n+nb}{set}\PYG{p}{(}\PYG{n}{obj}\PYG{p}{,}\PYG{n}{varargin}\PYG{p}{)}
\end{Verbatim}


\subsection{Inputs}
\label{classes/models/@dsge/dsge:id160}\begin{itemize}
\item {} 
\textbf{obj} {[}rise\textbar{}dsge{]}: model object

\item {} 
\textbf{varargin} : valid input arguments coming in pairs. Notable fields to
that can be set include and are not restricted to:
- \textbf{solve\_shock\_horizon} {[}integer\textbar{}struct\textbar{}cell{]}
\begin{itemize}
\item {} 
for the integer case, all shocks are set to the same integer

\item {} \begin{description}
\item[{for the struct case, the input must be a structure with shock}] \leavevmode
names as fields. Only the shock names whose value is to change
have to be listed. In this case, different shocks can have
different horizons k. The default is k=0 i.e. agents don't
see into the future

\end{description}

\item {} \begin{description}
\item[{for the cell case, the cell should have two columns. The first}] \leavevmode
column includes the names of the shocks whose horizon is to
change. The second column includes the horizon for each shock
name on the left.

\end{description}

\end{itemize}
\begin{itemize}
\item {} \begin{description}
\item[{\textbf{solve\_function\_mode} {[}\{explicit/amateur\}\textbar{}vectorized/professional\textbar{}disc{]}}] \leavevmode\begin{itemize}
\item {} \begin{description}
\item[{in the \textbf{amateur} or \textbf{explicit} mode the functions are kept in}] \leavevmode
cell arrays of anonymous functions and evaluated using for
loops

\end{description}

\item {} \begin{description}
\item[{in the \textbf{vectorized} or \textbf{professional} mode the functions are}] \leavevmode
compacted into one long and unreadable function.

\end{description}

\item {} \begin{description}
\item[{in the \textbf{disc} mode the functions are written to disc in a}] \leavevmode
subdirectory called routines.

\end{description}

\end{itemize}

\end{description}

\end{itemize}

\end{itemize}


\subsection{Outputs}
\label{classes/models/@dsge/dsge:id161}\begin{itemize}
\item {} 
\textbf{obj} {[}rise\textbar{}dsge{]}: model object

\end{itemize}


\subsection{More About}
\label{classes/models/@dsge/dsge:id162}

\subsection{Examples}
\label{classes/models/@dsge/dsge:id163}
obj=set(obj,'solve\_shock\_horizon',struct(`shock1',2,'shock3',4))
obj=set(obj,'solve\_shock\_horizon',5)

See also: rise\_generic.set


\bigskip\hrule{}\bigskip



\section{set\_solution\_to\_companion}
\label{classes/models/@dsge/dsge:set-solution-to-companion}\label{classes/models/@dsge/dsge:id164}
H1 line


\subsection{Syntax}
\label{classes/models/@dsge/dsge:id165}

\subsection{Inputs}
\label{classes/models/@dsge/dsge:id166}

\subsection{Outputs}
\label{classes/models/@dsge/dsge:id167}

\subsection{More About}
\label{classes/models/@dsge/dsge:id168}

\subsection{Examples}
\label{classes/models/@dsge/dsge:id169}
See also:


\bigskip\hrule{}\bigskip



\section{simulate}
\label{classes/models/@dsge/dsge:id170}\label{classes/models/@dsge/dsge:simulate}
\textbf{simulate} - simulates a RISE model


\subsection{Syntax}
\label{classes/models/@dsge/dsge:id171}
\begin{Verbatim}[commandchars=\\\{\}]
\PYG{p}{[}\PYG{n}{db}\PYG{p}{,}\PYG{n}{states}\PYG{p}{,}\PYG{n}{retcode}\PYG{p}{]} \PYG{o}{=} \PYG{n}{simulate}\PYG{p}{(}\PYG{n}{obj}\PYG{p}{,}\PYG{n}{varargin}\PYG{p}{)}
\end{Verbatim}


\subsection{Inputs}
\label{classes/models/@dsge/dsge:id172}\begin{itemize}
\item {} 
\textbf{obj} {[}rfvar\textbar{}dsge\textbar{}rise\textbar{}svar{]}: model object

\item {} 
\textbf{varargin} : additional arguments including but not restricted to
\begin{itemize}
\item {} 
\textbf{simul\_periods} {[}integer\textbar{}\{100\}{]}: number of simulation periods

\item {} 
\textbf{simul\_burn} {[}integer\textbar{}\{100\}{]}: number of burn-in periods

\item {} \begin{description}
\item[{\textbf{simul\_algo} {[}{[}\{mt19937ar\}\textbar{} mcg16807\textbar{}mlfg6331\_64\textbar{}mrg32k3a\textbar{}}] \leavevmode
shr3cong\textbar{}swb2712{]}{]}: matlab's seeding algorithms

\end{description}

\item {} 
\textbf{simul\_seed} {[}numeric\textbar{}\{0\}{]}: seed of the computations

\item {} \begin{description}
\item[{\textbf{simul\_historical\_data} {[}ts\textbar{}struct\textbar{}\{`'\}{]}: historical data from}] \leavevmode
which the simulations are based. If empty, the simulations start at
the steady state.

\end{description}

\item {} \begin{description}
\item[{\textbf{simul\_history\_end\_date} {[}char\textbar{}integer\textbar{}serial date{]}: last date of}] \leavevmode
history

\end{description}

\item {} \begin{description}
\item[{\textbf{simul\_regime} {[}integer\textbar{}vector\textbar{}\{{[}{]}\}{]}: regimes for which the model}] \leavevmode
is simulated

\end{description}

\item {} \begin{description}
\item[{\textbf{simul\_update\_shocks\_handle} {[}function handle{]}: we may want to}] \leavevmode
update the shocks if some condition on the state of the economy is
satisfied. For instance, shock monetary policy to keep the interest
rate at the floor for an extented period of time if we already are
at the ZLB/ZIF. simul\_update\_shocks\_handle takes as inputs the
current shocks and the state vector (all the endogenous variables)
and returns the updated shocks. But for all this to be put into
motion, the user also has to turn on \textbf{simul\_do\_update\_shocks} by
setting it to true.

\end{description}

\item {} \begin{description}
\item[{\textbf{simul\_do\_update\_shocks} {[}true\textbar{}\{false\}{]}: update the shocks based on}] \leavevmode
\textbf{simul\_update\_shocks\_handle} or not.

\end{description}

\item {} \begin{description}
\item[{\textbf{simul\_to\_time\_series} {[}\{true\}\textbar{}false{]}: if true, the output is a}] \leavevmode
time series, else a cell array with a matrix and information on
elements that help reconstruct the time series.

\end{description}

\end{itemize}

\end{itemize}


\subsection{Outputs}
\label{classes/models/@dsge/dsge:id173}\begin{itemize}
\item {} 
\textbf{db} {[}struct\textbar{}cell array{]}: if \textbf{simul\_to\_time\_series} is true, the
output is a time series, else a cell array with a matrix and
information on elements that help reconstruct the time series.

\item {} 
\textbf{states} {[}vector{]}: history of the regimes over the forecast horizon

\item {} 
\textbf{retcode} {[}integer{]}: if 0, the simulation went fine. Else something
got wrong. In that case one can understand the problem by running
decipher(retcode)

\end{itemize}


\subsection{More About}
\label{classes/models/@dsge/dsge:id174}\begin{itemize}
\item {} 
\textbf{simul\_historical\_data} contains the historical data as well as
conditional information over the forecast horizon. It may also include
as an alternative to \textbf{simul\_regime}, a time series with name
\textbf{regime}, which indicates the regimes over the forecast horizon.

\end{itemize}


\subsection{Examples}
\label{classes/models/@dsge/dsge:id175}
See also:

Help for dsge/simulate is inherited from superclass RISE\_GENERIC


\bigskip\hrule{}\bigskip



\section{simulate\_nonlinear}
\label{classes/models/@dsge/dsge:id176}\label{classes/models/@dsge/dsge:simulate-nonlinear}
H1 line


\subsection{Syntax}
\label{classes/models/@dsge/dsge:id177}

\subsection{Inputs}
\label{classes/models/@dsge/dsge:id178}

\subsection{Outputs}
\label{classes/models/@dsge/dsge:id179}

\subsection{More About}
\label{classes/models/@dsge/dsge:id180}

\subsection{Examples}
\label{classes/models/@dsge/dsge:id181}
See also:


\bigskip\hrule{}\bigskip



\section{simulation\_diagnostics}
\label{classes/models/@dsge/dsge:id182}\label{classes/models/@dsge/dsge:simulation-diagnostics}
H1 line


\subsection{Syntax}
\label{classes/models/@dsge/dsge:id183}

\subsection{Inputs}
\label{classes/models/@dsge/dsge:id184}

\subsection{Outputs}
\label{classes/models/@dsge/dsge:id185}

\subsection{More About}
\label{classes/models/@dsge/dsge:id186}

\subsection{Examples}
\label{classes/models/@dsge/dsge:id187}
See also:

Help for dsge/simulation\_diagnostics is inherited from superclass RISE\_GENERIC


\bigskip\hrule{}\bigskip



\section{solve}
\label{classes/models/@dsge/dsge:id188}\label{classes/models/@dsge/dsge:solve}
H1 line


\subsection{Syntax}
\label{classes/models/@dsge/dsge:id189}

\subsection{Inputs}
\label{classes/models/@dsge/dsge:id190}

\subsection{Outputs}
\label{classes/models/@dsge/dsge:id191}

\subsection{More About}
\label{classes/models/@dsge/dsge:id192}

\subsection{Examples}
\label{classes/models/@dsge/dsge:id193}
obj=solve(obj,'solve\_shock\_horizon',struct(`shock1',2,'shock3',4))
obj=solve(obj,'solve\_shock\_horizon',5)

See also:


\bigskip\hrule{}\bigskip



\section{solve\_alternatives}
\label{classes/models/@dsge/dsge:solve-alternatives}\label{classes/models/@dsge/dsge:id194}
H1 line


\subsection{Syntax}
\label{classes/models/@dsge/dsge:id195}

\subsection{Inputs}
\label{classes/models/@dsge/dsge:id196}

\subsection{Outputs}
\label{classes/models/@dsge/dsge:id197}

\subsection{More About}
\label{classes/models/@dsge/dsge:id198}

\subsection{Examples}
\label{classes/models/@dsge/dsge:id199}
See also:


\bigskip\hrule{}\bigskip



\section{stoch\_simul}
\label{classes/models/@dsge/dsge:id200}\label{classes/models/@dsge/dsge:stoch-simul}
H1 line


\subsection{Syntax}
\label{classes/models/@dsge/dsge:id201}

\subsection{Inputs}
\label{classes/models/@dsge/dsge:id202}

\subsection{Outputs}
\label{classes/models/@dsge/dsge:id203}

\subsection{More About}
\label{classes/models/@dsge/dsge:id204}

\subsection{Examples}
\label{classes/models/@dsge/dsge:id205}
See also:

Help for dsge/stoch\_simul is inherited from superclass RISE\_GENERIC


\bigskip\hrule{}\bigskip



\section{theoretical\_autocorrelations}
\label{classes/models/@dsge/dsge:theoretical-autocorrelations}\label{classes/models/@dsge/dsge:id206}
H1 line


\subsection{Syntax}
\label{classes/models/@dsge/dsge:id207}

\subsection{Inputs}
\label{classes/models/@dsge/dsge:id208}

\subsection{Outputs}
\label{classes/models/@dsge/dsge:id209}

\subsection{More About}
\label{classes/models/@dsge/dsge:id210}

\subsection{Examples}
\label{classes/models/@dsge/dsge:id211}
See also:

Help for dsge/theoretical\_autocorrelations is inherited from superclass RISE\_GENERIC


\bigskip\hrule{}\bigskip



\section{theoretical\_autocovariances}
\label{classes/models/@dsge/dsge:theoretical-autocovariances}\label{classes/models/@dsge/dsge:id212}
H1 line


\subsection{Syntax}
\label{classes/models/@dsge/dsge:id213}

\subsection{Inputs}
\label{classes/models/@dsge/dsge:id214}

\subsection{Outputs}
\label{classes/models/@dsge/dsge:id215}

\subsection{More About}
\label{classes/models/@dsge/dsge:id216}

\subsection{Examples}
\label{classes/models/@dsge/dsge:id217}
See also:

Help for dsge/theoretical\_autocovariances is inherited from superclass RISE\_GENERIC


\bigskip\hrule{}\bigskip



\section{variance\_decomposition}
\label{classes/models/@dsge/dsge:id218}\label{classes/models/@dsge/dsge:variance-decomposition}
H1 line


\subsection{Syntax}
\label{classes/models/@dsge/dsge:id219}

\subsection{Inputs}
\label{classes/models/@dsge/dsge:id220}

\subsection{Outputs}
\label{classes/models/@dsge/dsge:id221}

\subsection{More About}
\label{classes/models/@dsge/dsge:id222}

\subsection{Examples}
\label{classes/models/@dsge/dsge:id223}
See also:

Help for dsge/variance\_decomposition is inherited from superclass RISE\_GENERIC


\chapter{Reduced-form VAR modeling}
\label{classes/models/@rfvar/rfvar::doc}\label{classes/models/@rfvar/rfvar:reduced-form-var-modeling}

\section{methods}
\label{classes/models/@rfvar/rfvar:methods}\begin{itemize}
\item {} 
{[} {\hyperref[classes/models/@rfvar/rfvar:check-identification]{check\_identification}} {]}(rfvar/check\_identification)

\item {} 
{[} {\hyperref[classes/models/@rfvar/rfvar:check-optimum]{check\_optimum}} {]}(rfvar/check\_optimum)

\item {} 
{[} {\hyperref[classes/models/@rfvar/rfvar:draw-parameter]{draw\_parameter}} {]}(rfvar/draw\_parameter)

\item {} 
{[} {\hyperref[classes/models/@rfvar/rfvar:estimate]{estimate}} {]}(rfvar/estimate)

\item {} 
{[} {\hyperref[classes/models/@rfvar/rfvar:forecast]{forecast}} {]}(rfvar/forecast)

\item {} 
{[} {\hyperref[classes/models/@rfvar/rfvar:get]{get}} {]}(rfvar/get)

\item {} 
{[} {\hyperref[classes/models/@rfvar/rfvar:historical-decomposition]{historical\_decomposition}} {]}(rfvar/historical\_decomposition)

\item {} 
{[} {\hyperref[classes/models/@rfvar/rfvar:irf]{irf}} {]}(rfvar/irf)

\item {} 
{[} {\hyperref[classes/models/@rfvar/rfvar:isnan]{isnan}} {]}(rfvar/isnan)

\item {} 
{[} {\hyperref[classes/models/@rfvar/rfvar:load-parameters]{load\_parameters}} {]}(rfvar/load\_parameters)

\item {} 
{[} {\hyperref[classes/models/@rfvar/rfvar:log-marginal-data-density]{log\_marginal\_data\_density}} {]}(rfvar/log\_marginal\_data\_density)

\item {} 
{[} {\hyperref[classes/models/@rfvar/rfvar:log-posterior-kernel]{log\_posterior\_kernel}} {]}(rfvar/log\_posterior\_kernel)

\item {} 
{[} {\hyperref[classes/models/@rfvar/rfvar:log-prior-density]{log\_prior\_density}} {]}(rfvar/log\_prior\_density)

\item {} 
{[} {\hyperref[classes/models/@rfvar/rfvar:msvar-priors]{msvar\_priors}} {]}(rfvar/msvar\_priors)

\item {} 
{[} {\hyperref[classes/models/@rfvar/rfvar:posterior-marginal-and-prior-densities]{posterior\_marginal\_and\_prior\_densities}} {]}(rfvar/posterior\_marginal\_and\_prior\_densities)

\item {} 
{[} {\hyperref[classes/models/@rfvar/rfvar:posterior-simulator]{posterior\_simulator}} {]}(rfvar/posterior\_simulator)

\item {} 
{[} {\hyperref[classes/models/@rfvar/rfvar:print-estimation-results]{print\_estimation\_results}} {]}(rfvar/print\_estimation\_results)

\item {} 
{[} {\hyperref[classes/models/@rfvar/rfvar:prior-plots]{prior\_plots}} {]}(rfvar/prior\_plots)

\item {} 
{[} {\hyperref[classes/models/@rfvar/rfvar:report]{report}} {]}(rfvar/report)

\item {} 
{[} {\hyperref[classes/models/@rfvar/rfvar:rfvar]{rfvar}} {]}(rfvar/rfvar)

\item {} 
{[} {\hyperref[classes/models/@rfvar/rfvar:set]{set}} {]}(rfvar/set)

\item {} 
{[} {\hyperref[classes/models/@rfvar/rfvar:set-solution-to-companion]{set\_solution\_to\_companion}} {]}(rfvar/set\_solution\_to\_companion)

\item {} 
{[} {\hyperref[classes/models/@rfvar/rfvar:simulate]{simulate}} {]}(rfvar/simulate)

\item {} 
{[} {\hyperref[classes/models/@rfvar/rfvar:simulation-diagnostics]{simulation\_diagnostics}} {]}(rfvar/simulation\_diagnostics)

\item {} 
{[} {\hyperref[classes/models/@rfvar/rfvar:solve]{solve}} {]}(rfvar/solve)

\item {} 
{[} {\hyperref[classes/models/@rfvar/rfvar:stoch-simul]{stoch\_simul}} {]}(rfvar/stoch\_simul)

\item {} 
{[} {\hyperref[classes/models/@rfvar/rfvar:structural-form]{structural\_form}} {]}(rfvar/structural\_form)

\item {} 
{[} {\hyperref[classes/models/@rfvar/rfvar:template]{template}} {]}(rfvar/template)

\item {} 
{[} {\hyperref[classes/models/@rfvar/rfvar:theoretical-autocorrelations]{theoretical\_autocorrelations}} {]}(rfvar/theoretical\_autocorrelations)

\item {} 
{[} {\hyperref[classes/models/@rfvar/rfvar:theoretical-autocovariances]{theoretical\_autocovariances}} {]}(rfvar/theoretical\_autocovariances)

\item {} 
{[} {\hyperref[classes/models/@rfvar/rfvar:variance-decomposition]{variance\_decomposition}} {]}(rfvar/variance\_decomposition)

\end{itemize}


\section{properties}
\label{classes/models/@rfvar/rfvar:properties}\begin{itemize}
\item {} 
{[}identification{]} -

\item {} 
{[}structural\_shocks{]} -

\item {} 
{[}nonlinear\_restrictions{]} -

\item {} 
{[}constant{]} -

\item {} 
{[}nlags{]} -

\item {} 
{[}legend{]} -

\item {} 
{[}endogenous{]} -

\item {} 
{[}exogenous{]} -

\item {} 
{[}parameters{]} -

\item {} 
{[}observables{]} -

\item {} 
{[}markov\_chains{]} -

\item {} 
{[}options{]} -

\item {} 
{[}estimation{]} -

\item {} 
{[}solution{]} -

\item {} 
{[}filtering{]} -

\end{itemize}


\bigskip\hrule{}\bigskip



\bigskip\hrule{}\bigskip



\section{check\_identification}
\label{classes/models/@rfvar/rfvar:id1}\label{classes/models/@rfvar/rfvar:check-identification}
H1 line


\subsection{Syntax}
\label{classes/models/@rfvar/rfvar:syntax}

\subsection{Inputs}
\label{classes/models/@rfvar/rfvar:inputs}

\subsection{Outputs}
\label{classes/models/@rfvar/rfvar:outputs}

\subsection{More About}
\label{classes/models/@rfvar/rfvar:more-about}

\subsection{Examples}
\label{classes/models/@rfvar/rfvar:examples}
See also:


\bigskip\hrule{}\bigskip



\section{check\_optimum}
\label{classes/models/@rfvar/rfvar:check-optimum}\label{classes/models/@rfvar/rfvar:id2}
H1 line


\subsection{Syntax}
\label{classes/models/@rfvar/rfvar:id3}

\subsection{Inputs}
\label{classes/models/@rfvar/rfvar:id4}

\subsection{Outputs}
\label{classes/models/@rfvar/rfvar:id5}

\subsection{More About}
\label{classes/models/@rfvar/rfvar:id6}

\subsection{Examples}
\label{classes/models/@rfvar/rfvar:id7}
See also:

Help for rfvar/check\_optimum is inherited from superclass RISE\_GENERIC


\bigskip\hrule{}\bigskip



\section{draw\_parameter}
\label{classes/models/@rfvar/rfvar:id8}\label{classes/models/@rfvar/rfvar:draw-parameter}
H1 line


\subsection{Syntax}
\label{classes/models/@rfvar/rfvar:id9}

\subsection{Inputs}
\label{classes/models/@rfvar/rfvar:id10}

\subsection{Outputs}
\label{classes/models/@rfvar/rfvar:id11}

\subsection{More About}
\label{classes/models/@rfvar/rfvar:id12}

\subsection{Examples}
\label{classes/models/@rfvar/rfvar:id13}
See also:

Help for rfvar/draw\_parameter is inherited from superclass RISE\_GENERIC


\bigskip\hrule{}\bigskip



\section{estimate}
\label{classes/models/@rfvar/rfvar:estimate}\label{classes/models/@rfvar/rfvar:id14}
\textbf{estimate} - estimates the parameters of a RISE model


\subsection{Syntax}
\label{classes/models/@rfvar/rfvar:id15}
\begin{Verbatim}[commandchars=\\\{\}]
\PYG{n}{obj}\PYG{o}{=}\PYG{n}{estimate}\PYG{p}{(}\PYG{n}{obj}\PYG{p}{)}
\PYG{n}{obj}\PYG{o}{=}\PYG{n}{estimate}\PYG{p}{(}\PYG{n}{obj}\PYG{p}{,}\PYG{n}{varargin}\PYG{p}{)}
\end{Verbatim}


\subsection{Inputs}
\label{classes/models/@rfvar/rfvar:id16}\begin{itemize}
\item {} 
\textbf{obj} {[}rise\textbar{}dsge\textbar{}rfvar\textbar{}svar{]}: model object

\item {} 
\textbf{varargin} additional optional inputs among which the most relevant
for estimation are:

\item {} 
\textbf{estim\_parallel} {[}integer\textbar{}\{1\}{]}: Number of starting values

\item {} 
\textbf{estim\_start\_from\_mode} {[}true\textbar{}false\textbar{}\{{[}{]}\}{]}: when empty, the user is
prompted to answer the question as to whether to start estimation from
a previously found mode or not. If true or false, no question is asked.

\item {} 
\textbf{estim\_start\_date} {[}numeric\textbar{}char\textbar{}serial date{]}: date of the first
observation to use in the dataset provided for estimation

\item {} 
\textbf{estim\_end\_date} {[}numeric\textbar{}char\textbar{}serial date{]}: date of the last
observation to use in the dataset provided for estimation

\item {} 
\textbf{estim\_max\_trials} {[}integer\textbar{}\{500\}{]}: When the initial value of the
log-likelihood is too low, RISE uniformly draws from the prior support
in search for a better starting point. It will try this for a maximum
number of \textbf{estim\_max\_trials} times before squeaking with an error.

\item {} 
\textbf{estim\_start\_vals} {[}\{{[}{]}\}\textbar{}struct{]}: when not empty, the parameters
whose names are fields of the structure will see their start values
updated or overriden by the information in \textbf{estim\_start\_vals}. There
is no need to provide values to update the start values for the
estimated parameters.

\item {} 
\textbf{estim\_general\_restrictions} {[}\{{[}{]}\}\textbar{}function handle{]}: when not empty,
the argument should be a function handle that takes as input a
parameterized RISE object and returns a scalar or vector of numbers
representing the strength of the violation of the nonlinear
constraints. Those constraints will be added to the constraints already
included in a rise/dsge file before being presented to the optimization
function.

\item {} 
\textbf{estim\_linear\_restrictions} {[}\{{[}{]}\}\textbar{}cell{]}: This is most often used in
the estimation of rfvar or svar models either to impose block
exogeneity or to impose other forms of linear restrictions. When not
empty, \textbf{estim\_linear\_restrictions} must be a 2-column cell:
- Each row of the first column represents a particular linear
combination of the estimated parameters. Those linear combinations are
constructed using the \textbf{coef} class. Check help for coef.coef for more
details.
- Each row of the second column holds the value of the linear
combination.

\item {} 
\textbf{estim\_blocks} {[}\{{[}{]}\}\textbar{}cell{]}: When not empty, this triggers blockwise
optimization. For further information on how to set blocks, see help
for dsge.create\_estimation\_blocks

\item {} 
\textbf{estim\_priors} {[}\{{[}{]}\}\textbar{}struct{]}: This provides an alternative to
setting priors inside the rise/dsge model file. Each field of the
structure must be the name of an estimated parameter. Each field will
hold a cell array whose structure is described in help
rise\_generic.setup\_priors.

\item {} 
\textbf{estim\_penalty} {[}numeric\textbar{}\{1e+8\}{]}: value of the objective function
when a problem occurs. Possible problems include:
- no solution found
- very low likelihood
- stochastic singularity
- problems computing the initial covariance matrix
- non-positive definite covariance matrices
- etc.

\item {} 
\textbf{estim\_cond\_vars} {[}char\textbar{}cellstr{]}: list of the variables to condition
on during estimation of a dsge model. This then assumes that the data
provided for estimation have several pages. The first page is the
actual data, while the subsequent pages are the expectations data.

\item {} 
\textbf{optimset} {[}struct{]}: identical to matlab's optimset

\item {} 
\textbf{optimizer} {[}char\textbar{}function handle\textbar{}cell{]}: This can be the name of a
standard matlab optimizer or RISE optimization routine or a
user-defined optimization procedure available of the matlab search
path. If the optimzer is provided as a cell, then the first element of
the cell is the name of the optimizer or its handle and the remaining
entries in the cell are additional input arguments to the user-defined
optimization routine. A user-defined optimization function should have
the following syntax

\begin{Verbatim}[commandchars=\\\{\}]
\PYG{p}{[}\PYG{n}{xfinal}\PYG{p}{,}\PYG{n}{ffinal}\PYG{p}{,}\PYG{n}{exitflag}\PYG{p}{]}\PYG{o}{=}\PYG{n}{optimizer}\PYG{p}{(}\PYG{n}{fh}\PYG{p}{,}\PYG{n}{x0}\PYG{p}{,}\PYG{n}{lb}\PYG{p}{,}\PYG{n}{ub}\PYG{p}{,}\PYG{n}{options}\PYG{p}{,}\PYG{n}{varargin}\PYG{p}{)}\PYG{p}{;}
\end{Verbatim}
\begin{description}
\item[{That is, it accepts as inputs:}] \leavevmode\begin{itemize}
\item {} 
\textbf{fh}: the function to optimize

\item {} 
\textbf{x0}: a vector column of initial values of the parameters

\item {} 
\textbf{lb}: a vector column of lower bounds

\item {} 
\textbf{ub}: a vector column of upper bounds

\item {} \begin{description}
\item[{\textbf{options}: a structure of options whose fields will be similar}] \leavevmode
to matlab's optimset

\end{description}

\item {} \begin{description}
\item[{\textbf{varargin}: additional arguments to the user-defined}] \leavevmode
optimization procedure

\end{description}

\end{itemize}

\item[{That is, it provides as outputs:}] \leavevmode\begin{itemize}
\item {} 
\textbf{xfinal}: the vector of final values

\item {} 
\textbf{ffinal}: the value of \textbf{fh} at \textbf{xfinal}

\item {} 
\textbf{exitflag}: a flag similar to the ones provided by matlab's

\end{itemize}

optimization functions.

\end{description}

\item {} 
\textbf{hessian\_type} {[}\{`fd'\}\textbar{}'opg'{]}: The hessian is either computed by
finite differences (fd) or by outer-product-gradient (opg)

\item {} 
\textbf{hessian\_repair} {[}\{false\}\textbar{}true{]}: If the Hessian is not positive
definite, it nevertheless can be repaired and prepared for a potential
mcmc simulation.

\end{itemize}


\subsection{Outputs}
\label{classes/models/@rfvar/rfvar:id17}\begin{itemize}
\item {} 
\textbf{obj} {[}rise\textbar{}dsge\textbar{}rfvar\textbar{}svar{]}: model object parameterized with the
mode found and holding additional estimation results and statistics
that can be found under obj.estimation

\end{itemize}


\subsection{More About}
\label{classes/models/@rfvar/rfvar:id18}\begin{itemize}
\item {} 
recursive estimation may be done easily by passing a different
estim\_end\_date at the beginning of each estimation run.

\end{itemize}


\subsection{Examples}
\label{classes/models/@rfvar/rfvar:id19}
See also:

Help for rfvar/estimate is inherited from superclass RISE\_GENERIC


\bigskip\hrule{}\bigskip



\section{forecast}
\label{classes/models/@rfvar/rfvar:id20}\label{classes/models/@rfvar/rfvar:forecast}
\textbf{forecast} - computes forecasts for rise\textbar{}dsge\textbar{}svar\textbar{}rfvar models


\subsection{Syntax}
\label{classes/models/@rfvar/rfvar:id21}
\begin{Verbatim}[commandchars=\\\{\}]
\PYG{n}{cond\PYGZus{}fkst\PYGZus{}db}\PYG{o}{=}\PYG{n}{forecast}\PYG{p}{(}\PYG{n}{obj}\PYG{p}{,}\PYG{n}{varargin}\PYG{p}{)}
\end{Verbatim}


\subsection{Inputs}
\label{classes/models/@rfvar/rfvar:id22}\begin{itemize}
\item {} 
\textbf{obj} {[}rise\textbar{}dsge\textbar{}svar\textbar{}rfvar{]}: model object

\item {} 
\textbf{varargin} : additional inputs coming in pairs. These include but are
not restricted to:
- \textbf{forecast\_to\_time\_series} {[}\{true\}\textbar{}false{]}: sets the output to time
\begin{quote}

series format or not
\end{quote}
\begin{itemize}
\item {} 
\textbf{forecast\_nsteps} {[}integer\textbar{}\{12\}{]}: number of forecasting steps

\item {} \begin{description}
\item[{\textbf{forecast\_start\_date} {[}char\textbar{}numeric\textbar{}serial date{]}: date when the}] \leavevmode
forecasts start (end of history + 1)

\end{description}

\item {} \begin{description}
\item[{\textbf{forecast\_conditional\_hypothesis} {[}\{jma\}\textbar{}ncp\textbar{}nas{]}: in dsge models in}] \leavevmode
which agents have information beyond the current period, this
option determines the number of periods of shocks need to match the
restrictions:
- Hypothesis \textbf{jma} assumes that irrespective of how
\begin{quote}

many periods of conditioning information are remaining, agents
always receive information on the same number of shocks.
\end{quote}
\begin{itemize}
\item {} \begin{description}
\item[{Hypothesis \textbf{ncp} assumes there are as many shocks periods as}] \leavevmode
the number of the number of conditioning periods

\end{description}

\item {} \begin{description}
\item[{Hypothesis \textbf{nas} assumes there are as many shocks periods as}] \leavevmode
the number of anticipated steps

\end{description}

\end{itemize}

\end{description}

\end{itemize}

\end{itemize}


\subsection{Outputs}
\label{classes/models/@rfvar/rfvar:id23}\begin{itemize}
\item {} 
\textbf{cond\_fkst\_db} {[}struct\textbar{}matrix{]}: depending on the value of
\textbf{forecast\_to\_time\_series} the returned output is a structure with
time series or a cell containing a matrix and the information to
reconstruct the time series.

\end{itemize}


\subsection{More About}
\label{classes/models/@rfvar/rfvar:id24}\begin{itemize}
\item {} 
the historical information as well as the conditioning information come
from the same database. The time series must be organized such that for
each series, the first page represents the actual data and all
subsequent pages represent conditional information. If a particular
condition is ``nan'', that location is not constrained

\item {} 
Conditional forecasting for nonlinear models is also supported.
However, the solving of the implied nonlinear problem may fail if the
model displays instability

\item {} 
Both HARD CONDITIONS and SOFT CONDITIONS are implemented but the latter
are currently disabled in expectation of a better user interface.

\item {} 
The data may also contain time series for a variable with name
\textbf{regime} in that case, the forecast/simulation paths are computed
following the information therein. \textbf{regime} must be a member of 1:h,
where h is the maximum number of regimes.

\end{itemize}


\subsection{Examples}
\label{classes/models/@rfvar/rfvar:id25}
See also: simulate

Help for rfvar/forecast is inherited from superclass RISE\_GENERIC


\bigskip\hrule{}\bigskip



\section{get}
\label{classes/models/@rfvar/rfvar:id26}\label{classes/models/@rfvar/rfvar:get}
H1 line


\subsection{Syntax}
\label{classes/models/@rfvar/rfvar:id27}

\subsection{Inputs}
\label{classes/models/@rfvar/rfvar:id28}

\subsection{Outputs}
\label{classes/models/@rfvar/rfvar:id29}

\subsection{More About}
\label{classes/models/@rfvar/rfvar:id30}

\subsection{Examples}
\label{classes/models/@rfvar/rfvar:id31}
See also:

Help for rfvar/get is inherited from superclass RISE\_GENERIC


\bigskip\hrule{}\bigskip



\section{historical\_decomposition}
\label{classes/models/@rfvar/rfvar:id32}\label{classes/models/@rfvar/rfvar:historical-decomposition}
\textbf{historical\_decomposition} Computes historical decompositions of a DSGE model


\subsection{Syntax}
\label{classes/models/@rfvar/rfvar:id33}
\begin{Verbatim}[commandchars=\\\{\}]
\PYG{p}{[}\PYG{n}{Histdec}\PYG{p}{,}\PYG{n}{obj}\PYG{p}{]}\PYG{o}{=}\PYG{n}{history\PYGZus{}dec}\PYG{p}{(}\PYG{n}{obj}\PYG{p}{)}
\PYG{p}{[}\PYG{n}{Histdec}\PYG{p}{,}\PYG{n}{obj}\PYG{p}{]}\PYG{o}{=}\PYG{n}{history\PYGZus{}dec}\PYG{p}{(}\PYG{n}{obj}\PYG{p}{,}\PYG{n}{varargin}\PYG{p}{)}
\end{Verbatim}


\subsection{Inputs}
\label{classes/models/@rfvar/rfvar:id34}\begin{itemize}
\item {} 
obj : {[}rise\textbar{}dsge\textbar{}rfvar\textbar{}svar{]} model(s) for which to compute the
decomposition. obj could be a vector of models

\item {} 
varargin : standard optional inputs \textbf{coming in pairs}. Among which:
- \textbf{histdec\_start\_date} : {[}char\textbar{}numeric\textbar{}\{`'\}{]} : date at which the
\begin{quote}

decomposition starts. If empty, the decomposition starts at he
beginning of the history of the dataset
\end{quote}

\end{itemize}


\subsection{Outputs}
\label{classes/models/@rfvar/rfvar:id35}\begin{itemize}
\item {} 
Histdec : {[}struct\textbar{}cell array{]} structure or cell array of structures
with the decompositions in each model. The decompositions are given in
terms of:
- the exogenous variables
- \textbf{InitialConditions} : the effect of initial conditions
- \textbf{risk} : measure of the effect of non-certainty equivalence
- \textbf{switch} : the effect of switching (which is also a shock!!!)
- \textbf{steady\_state} : the contribution of the steady state

\end{itemize}


\subsection{Remarks}
\label{classes/models/@rfvar/rfvar:remarks}\begin{itemize}
\item {} 
the elements that do not contribute to any of the variables are
automatically discarded.

\item {} 
\textbf{N.B} : a switching model is inherently nonlinear and so, strictly
speaking, the type of decomposition we do for linear/linearized
constant-parameter models is not feasible. RISE takes an approximation
in which the variables, shocks and states matrices across states are
averaged. The averaging weights are the smoothed probabilities.

\end{itemize}


\subsection{Examples}
\label{classes/models/@rfvar/rfvar:id36}
See also:

Help for rfvar/historical\_decomposition is inherited from superclass RISE\_GENERIC


\bigskip\hrule{}\bigskip



\section{irf}
\label{classes/models/@rfvar/rfvar:id37}\label{classes/models/@rfvar/rfvar:irf}
\textbf{irf} - computes impulse responses for a RISE model


\subsection{Syntax}
\label{classes/models/@rfvar/rfvar:id38}
\begin{Verbatim}[commandchars=\\\{\}]
\PYG{n}{myirfs}\PYG{o}{=}\PYG{n}{irf}\PYG{p}{(}\PYG{n}{obj}\PYG{p}{)}

\PYG{n}{myirfs}\PYG{o}{=}\PYG{n}{irf}\PYG{p}{(}\PYG{n}{obj}\PYG{p}{,}\PYG{n}{varargin}\PYG{p}{)}
\end{Verbatim}


\subsection{Inputs}
\label{classes/models/@rfvar/rfvar:id39}\begin{itemize}
\item {} 
\textbf{obj} {[}rise\textbar{}dsge\textbar{}rfvar\textbar{}svar{]}: single or vector of RISE models

\item {} 
\textbf{varargin} : optional options coming in pairs. The notable ones that
will influence the behavior of the impulse responses are:

\item {} 
\textbf{irf\_shock\_list} {[}char\textbar{}cellstr\textbar{}\{`'\}{]}: list of shocks for which we
want to compute impulse responses

\item {} 
\textbf{irf\_var\_list} {[}char\textbar{}cellstr\textbar{}\{`'\}{]}: list of the endogenous variables
we want to report

\item {} 
\textbf{irf\_periods} {[}integer\textbar{}\{40\}{]}: length of the irfs

\item {} 
\textbf{irf\_shock\_sign} {[}numeric\textbar{}-1\textbar{}\{1\}{]}: sign or scale of the original
impulse. If \textbf{irf\_shock\_sign} \textgreater{}0, we get impulse responses to a
positive shock. If \textbf{irf\_shock\_sign} \textless{}0, the responses are negative.
If If \textbf{irf\_shock\_sign} =0, all the responses are 0.

\item {} 
\textbf{irf\_draws} {[}integer\textbar{}\{50\}{]}: number of draws used in the simulation
impulse responses in a nonlinear model. A nonlinear model is defined as
a model that satisfies at least one of the following criteria
- solved at an order \textgreater{}1
- has more than one regime and option \textbf{irf\_regime\_specific} below is
\begin{quote}

set to false
\end{quote}

\item {} 
\textbf{irf\_type} {[}\{irf\}\textbar{}girf{]}: type of irfs. If the type is irf, the
impulse responses are computed directly exploiting the fact that the
model is linear. If the type is girf, the formula for the generalized
impulse responses is used: the irf is defined as the expectation of the
difference of two simulation paths. In the first path the initial
impulse for the shock of interest is nonzero while it is zero for the
second path. All other shocks are the same for both paths in a given
simulation.

\item {} 
\textbf{irf\_regime\_specific} {[}\{true\}\textbar{}false{]}: In a switching model, we may or
may not want to compute impulse responses specific to each regime.

\item {} 
\textbf{irf\_use\_historical\_data} {[}\{false\}\textbar{}true{]}: if true, the data stored in
option \textbf{simul\_historical\_data} are used as initial conditions. But
the model has to be nonlinear otherwise the initial conditions are set
to zero. This option gives the flexibility to set the initial
conditions for the impulse responses.

\item {} 
\textbf{irf\_to\_time\_series} {[}\{true\}\textbar{}false{]}: If true, the output is in the
form of time series. Else it is in the form of a cell containing the
information needed to reconstruct the time series.

\end{itemize}


\subsection{Outputs}
\label{classes/models/@rfvar/rfvar:id40}\begin{itemize}
\item {} 
\textbf{myirfs} {[}\{struct\}\textbar{}cell{]}: Impulse response data

\end{itemize}


\subsection{More About}
\label{classes/models/@rfvar/rfvar:id41}\begin{itemize}
\item {} 
for linear models or models solved up to first order, the initial
conditions as well as the steady states are set to 0 in the computation
of the impulse responses.

\item {} 
for nonlinear models, the initial conditions is the ergodic mean

\end{itemize}


\subsection{Examples}
\label{classes/models/@rfvar/rfvar:id42}
See also:

Help for rfvar/irf is inherited from superclass RISE\_GENERIC


\bigskip\hrule{}\bigskip



\section{isnan}
\label{classes/models/@rfvar/rfvar:isnan}\label{classes/models/@rfvar/rfvar:id43}
H1 line


\subsection{Syntax}
\label{classes/models/@rfvar/rfvar:id44}

\subsection{Inputs}
\label{classes/models/@rfvar/rfvar:id45}

\subsection{Outputs}
\label{classes/models/@rfvar/rfvar:id46}

\subsection{More About}
\label{classes/models/@rfvar/rfvar:id47}

\subsection{Examples}
\label{classes/models/@rfvar/rfvar:id48}
See also:

Help for rfvar/isnan is inherited from superclass RISE\_GENERIC


\bigskip\hrule{}\bigskip



\section{load\_parameters}
\label{classes/models/@rfvar/rfvar:id49}\label{classes/models/@rfvar/rfvar:load-parameters}
H1 line


\subsection{Syntax}
\label{classes/models/@rfvar/rfvar:id50}

\subsection{Inputs}
\label{classes/models/@rfvar/rfvar:id51}

\subsection{Outputs}
\label{classes/models/@rfvar/rfvar:id52}

\subsection{More About}
\label{classes/models/@rfvar/rfvar:id53}

\subsection{Examples}
\label{classes/models/@rfvar/rfvar:id54}
See also:

Help for rfvar/load\_parameters is inherited from superclass RISE\_GENERIC


\bigskip\hrule{}\bigskip



\section{log\_marginal\_data\_density}
\label{classes/models/@rfvar/rfvar:id55}\label{classes/models/@rfvar/rfvar:log-marginal-data-density}
H1 line


\subsection{Syntax}
\label{classes/models/@rfvar/rfvar:id56}

\subsection{Inputs}
\label{classes/models/@rfvar/rfvar:id57}

\subsection{Outputs}
\label{classes/models/@rfvar/rfvar:id58}

\subsection{More About}
\label{classes/models/@rfvar/rfvar:id59}

\subsection{Examples}
\label{classes/models/@rfvar/rfvar:id60}
See also:

Help for rfvar/log\_marginal\_data\_density is inherited from superclass RISE\_GENERIC


\bigskip\hrule{}\bigskip



\section{log\_posterior\_kernel}
\label{classes/models/@rfvar/rfvar:log-posterior-kernel}\label{classes/models/@rfvar/rfvar:id61}
H1 line


\subsection{Syntax}
\label{classes/models/@rfvar/rfvar:id62}

\subsection{Inputs}
\label{classes/models/@rfvar/rfvar:id63}

\subsection{Outputs}
\label{classes/models/@rfvar/rfvar:id64}

\subsection{More About}
\label{classes/models/@rfvar/rfvar:id65}

\subsection{Examples}
\label{classes/models/@rfvar/rfvar:id66}
See also:

Help for rfvar/log\_posterior\_kernel is inherited from superclass RISE\_GENERIC


\bigskip\hrule{}\bigskip



\section{log\_prior\_density}
\label{classes/models/@rfvar/rfvar:id67}\label{classes/models/@rfvar/rfvar:log-prior-density}
H1 line


\subsection{Syntax}
\label{classes/models/@rfvar/rfvar:id68}

\subsection{Inputs}
\label{classes/models/@rfvar/rfvar:id69}

\subsection{Outputs}
\label{classes/models/@rfvar/rfvar:id70}

\subsection{More About}
\label{classes/models/@rfvar/rfvar:id71}

\subsection{Examples}
\label{classes/models/@rfvar/rfvar:id72}
See also:

Help for rfvar/log\_prior\_density is inherited from superclass RISE\_GENERIC


\bigskip\hrule{}\bigskip



\section{msvar\_priors}
\label{classes/models/@rfvar/rfvar:id73}\label{classes/models/@rfvar/rfvar:msvar-priors}
H1 line


\subsection{Syntax}
\label{classes/models/@rfvar/rfvar:id74}

\subsection{Inputs}
\label{classes/models/@rfvar/rfvar:id75}

\subsection{Outputs}
\label{classes/models/@rfvar/rfvar:id76}

\subsection{More About}
\label{classes/models/@rfvar/rfvar:id77}

\subsection{Examples}
\label{classes/models/@rfvar/rfvar:id78}
See also:

Help for rfvar/msvar\_priors is inherited from superclass SVAR


\bigskip\hrule{}\bigskip



\section{posterior\_marginal\_and\_prior\_densities}
\label{classes/models/@rfvar/rfvar:id79}\label{classes/models/@rfvar/rfvar:posterior-marginal-and-prior-densities}
H1 line


\subsection{Syntax}
\label{classes/models/@rfvar/rfvar:id80}

\subsection{Inputs}
\label{classes/models/@rfvar/rfvar:id81}

\subsection{Outputs}
\label{classes/models/@rfvar/rfvar:id82}

\subsection{More About}
\label{classes/models/@rfvar/rfvar:id83}

\subsection{Examples}
\label{classes/models/@rfvar/rfvar:id84}
See also:

Help for rfvar/posterior\_marginal\_and\_prior\_densities is inherited from superclass RISE\_GENERIC


\bigskip\hrule{}\bigskip



\section{posterior\_simulator}
\label{classes/models/@rfvar/rfvar:posterior-simulator}\label{classes/models/@rfvar/rfvar:id85}
H1 line


\subsection{Syntax}
\label{classes/models/@rfvar/rfvar:id86}

\subsection{Inputs}
\label{classes/models/@rfvar/rfvar:id87}

\subsection{Outputs}
\label{classes/models/@rfvar/rfvar:id88}

\subsection{More About}
\label{classes/models/@rfvar/rfvar:id89}

\subsection{Examples}
\label{classes/models/@rfvar/rfvar:id90}
See also:

Help for rfvar/posterior\_simulator is inherited from superclass RISE\_GENERIC


\bigskip\hrule{}\bigskip



\section{print\_estimation\_results}
\label{classes/models/@rfvar/rfvar:id91}\label{classes/models/@rfvar/rfvar:print-estimation-results}
H1 line


\subsection{Syntax}
\label{classes/models/@rfvar/rfvar:id92}

\subsection{Inputs}
\label{classes/models/@rfvar/rfvar:id93}

\subsection{Outputs}
\label{classes/models/@rfvar/rfvar:id94}

\subsection{More About}
\label{classes/models/@rfvar/rfvar:id95}

\subsection{Examples}
\label{classes/models/@rfvar/rfvar:id96}
See also:

Help for rfvar/print\_estimation\_results is inherited from superclass RISE\_GENERIC


\bigskip\hrule{}\bigskip



\section{prior\_plots}
\label{classes/models/@rfvar/rfvar:prior-plots}\label{classes/models/@rfvar/rfvar:id97}
H1 line


\subsection{Syntax}
\label{classes/models/@rfvar/rfvar:id98}

\subsection{Inputs}
\label{classes/models/@rfvar/rfvar:id99}

\subsection{Outputs}
\label{classes/models/@rfvar/rfvar:id100}

\subsection{More About}
\label{classes/models/@rfvar/rfvar:id101}

\subsection{Examples}
\label{classes/models/@rfvar/rfvar:id102}
See also:

Help for rfvar/prior\_plots is inherited from superclass RISE\_GENERIC


\bigskip\hrule{}\bigskip



\section{report}
\label{classes/models/@rfvar/rfvar:report}\label{classes/models/@rfvar/rfvar:id103}
\textbf{REPORT} assigns the elements of interest to a rise\_report.report object


\subsection{Syntax}
\label{classes/models/@rfvar/rfvar:id104}\begin{description}
\item[{::}] \leavevmode\begin{itemize}
\item {} 
REPORT(rise.empty(0)) : displays the default inputs

\item {} 
REPORT(obj,destination\_root,rep\_items) : assigns the reported
elements in rep\_items to destination\_root

\item {} 
REPORT(obj,destination\_root,rep\_items,varargin) : assigns varargin to
obj before doing the rest

\end{itemize}

\end{description}


\subsection{Inputs}
\label{classes/models/@rfvar/rfvar:id105}\begin{itemize}
\item {} 
obj : {[}rise\textbar{}dsge{]}

\item {} 
destination\_root : {[}rise\_report.report{]} : handle for the actual report

\item {} 
rep\_items : {[}char\textbar{}cellstr{]} : list of desired items to report. This list
can only include : `endogenous', `exogenous', `observables',
`parameters', `solution', `estimation', `estimation\_statistics',
`equations', `code'

\end{itemize}


\subsection{Outputs}
\label{classes/models/@rfvar/rfvar:id106}
none


\subsection{More About}
\label{classes/models/@rfvar/rfvar:id107}

\subsection{Examples}
\label{classes/models/@rfvar/rfvar:id108}
See also:

Help for rfvar/report is inherited from superclass RISE\_GENERIC


\bigskip\hrule{}\bigskip



\section{rfvar}
\label{classes/models/@rfvar/rfvar:id109}\label{classes/models/@rfvar/rfvar:rfvar}\begin{quote}

\textasciitilde{}\textasciitilde{} no help found
\end{quote}


\bigskip\hrule{}\bigskip



\section{set}
\label{classes/models/@rfvar/rfvar:id110}\label{classes/models/@rfvar/rfvar:set}
\textbf{set} - sets options for RISE models


\subsection{Syntax}
\label{classes/models/@rfvar/rfvar:id111}
\begin{Verbatim}[commandchars=\\\{\}]
\PYG{n}{obj}\PYG{o}{=}\PYG{n+nb}{set}\PYG{p}{(}\PYG{n}{obj}\PYG{p}{,}\PYG{n}{varargin}\PYG{p}{)}
\end{Verbatim}


\subsection{Inputs}
\label{classes/models/@rfvar/rfvar:id112}\begin{itemize}
\item {} 
\textbf{obj} {[}rise\textbar{}dsge\textbar{}svar\textbar{}rfvar{]}: model object

\item {} 
\textbf{varargin} : valid input arguments coming in pairs.

\end{itemize}


\subsection{Outputs}
\label{classes/models/@rfvar/rfvar:id113}\begin{itemize}
\item {} 
\textbf{obj} {[}rise\textbar{}dsge\textbar{}svar\textbar{}rfvar{]}: model object

\end{itemize}


\subsection{More About}
\label{classes/models/@rfvar/rfvar:id114}\begin{itemize}
\item {} 
one can force a new field into the options by prefixing it with a `+'
sign. Let's say yourfield is not part of the options and you would like
to force it to be in the options because it is going to be used in some
function or algorithm down the road. Then you can run
m=set(m,'+yourfield',value). then m will be part of the new options.

\end{itemize}


\subsection{Examples}
\label{classes/models/@rfvar/rfvar:id115}
See also:

Help for rfvar/set is inherited from superclass RISE\_GENERIC


\bigskip\hrule{}\bigskip



\section{set\_solution\_to\_companion}
\label{classes/models/@rfvar/rfvar:id116}\label{classes/models/@rfvar/rfvar:set-solution-to-companion}
H1 line


\subsection{Syntax}
\label{classes/models/@rfvar/rfvar:id117}

\subsection{Inputs}
\label{classes/models/@rfvar/rfvar:id118}

\subsection{Outputs}
\label{classes/models/@rfvar/rfvar:id119}

\subsection{More About}
\label{classes/models/@rfvar/rfvar:id120}

\subsection{Examples}
\label{classes/models/@rfvar/rfvar:id121}
See also:

Help for rfvar/set\_solution\_to\_companion is inherited from superclass SVAR


\bigskip\hrule{}\bigskip



\section{simulate}
\label{classes/models/@rfvar/rfvar:id122}\label{classes/models/@rfvar/rfvar:simulate}
\textbf{simulate} - simulates a RISE model


\subsection{Syntax}
\label{classes/models/@rfvar/rfvar:id123}
\begin{Verbatim}[commandchars=\\\{\}]
\PYG{p}{[}\PYG{n}{db}\PYG{p}{,}\PYG{n}{states}\PYG{p}{,}\PYG{n}{retcode}\PYG{p}{]} \PYG{o}{=} \PYG{n}{simulate}\PYG{p}{(}\PYG{n}{obj}\PYG{p}{,}\PYG{n}{varargin}\PYG{p}{)}
\end{Verbatim}


\subsection{Inputs}
\label{classes/models/@rfvar/rfvar:id124}\begin{itemize}
\item {} 
\textbf{obj} {[}rfvar\textbar{}dsge\textbar{}rise\textbar{}svar{]}: model object

\item {} 
\textbf{varargin} : additional arguments including but not restricted to
\begin{itemize}
\item {} 
\textbf{simul\_periods} {[}integer\textbar{}\{100\}{]}: number of simulation periods

\item {} 
\textbf{simul\_burn} {[}integer\textbar{}\{100\}{]}: number of burn-in periods

\item {} \begin{description}
\item[{\textbf{simul\_algo} {[}{[}\{mt19937ar\}\textbar{} mcg16807\textbar{}mlfg6331\_64\textbar{}mrg32k3a\textbar{}}] \leavevmode
shr3cong\textbar{}swb2712{]}{]}: matlab's seeding algorithms

\end{description}

\item {} 
\textbf{simul\_seed} {[}numeric\textbar{}\{0\}{]}: seed of the computations

\item {} \begin{description}
\item[{\textbf{simul\_historical\_data} {[}ts\textbar{}struct\textbar{}\{`'\}{]}: historical data from}] \leavevmode
which the simulations are based. If empty, the simulations start at
the steady state.

\end{description}

\item {} \begin{description}
\item[{\textbf{simul\_history\_end\_date} {[}char\textbar{}integer\textbar{}serial date{]}: last date of}] \leavevmode
history

\end{description}

\item {} \begin{description}
\item[{\textbf{simul\_regime} {[}integer\textbar{}vector\textbar{}\{{[}{]}\}{]}: regimes for which the model}] \leavevmode
is simulated

\end{description}

\item {} \begin{description}
\item[{\textbf{simul\_update\_shocks\_handle} {[}function handle{]}: we may want to}] \leavevmode
update the shocks if some condition on the state of the economy is
satisfied. For instance, shock monetary policy to keep the interest
rate at the floor for an extented period of time if we already are
at the ZLB/ZIF. simul\_update\_shocks\_handle takes as inputs the
current shocks and the state vector (all the endogenous variables)
and returns the updated shocks. But for all this to be put into
motion, the user also has to turn on \textbf{simul\_do\_update\_shocks} by
setting it to true.

\end{description}

\item {} \begin{description}
\item[{\textbf{simul\_do\_update\_shocks} {[}true\textbar{}\{false\}{]}: update the shocks based on}] \leavevmode
\textbf{simul\_update\_shocks\_handle} or not.

\end{description}

\item {} \begin{description}
\item[{\textbf{simul\_to\_time\_series} {[}\{true\}\textbar{}false{]}: if true, the output is a}] \leavevmode
time series, else a cell array with a matrix and information on
elements that help reconstruct the time series.

\end{description}

\end{itemize}

\end{itemize}


\subsection{Outputs}
\label{classes/models/@rfvar/rfvar:id125}\begin{itemize}
\item {} 
\textbf{db} {[}struct\textbar{}cell array{]}: if \textbf{simul\_to\_time\_series} is true, the
output is a time series, else a cell array with a matrix and
information on elements that help reconstruct the time series.

\item {} 
\textbf{states} {[}vector{]}: history of the regimes over the forecast horizon

\item {} 
\textbf{retcode} {[}integer{]}: if 0, the simulation went fine. Else something
got wrong. In that case one can understand the problem by running
decipher(retcode)

\end{itemize}


\subsection{More About}
\label{classes/models/@rfvar/rfvar:id126}\begin{itemize}
\item {} 
\textbf{simul\_historical\_data} contains the historical data as well as
conditional information over the forecast horizon. It may also include
as an alternative to \textbf{simul\_regime}, a time series with name
\textbf{regime}, which indicates the regimes over the forecast horizon.

\end{itemize}


\subsection{Examples}
\label{classes/models/@rfvar/rfvar:id127}
See also:

Help for rfvar/simulate is inherited from superclass RISE\_GENERIC


\bigskip\hrule{}\bigskip



\section{simulation\_diagnostics}
\label{classes/models/@rfvar/rfvar:simulation-diagnostics}\label{classes/models/@rfvar/rfvar:id128}
H1 line


\subsection{Syntax}
\label{classes/models/@rfvar/rfvar:id129}

\subsection{Inputs}
\label{classes/models/@rfvar/rfvar:id130}

\subsection{Outputs}
\label{classes/models/@rfvar/rfvar:id131}

\subsection{More About}
\label{classes/models/@rfvar/rfvar:id132}

\subsection{Examples}
\label{classes/models/@rfvar/rfvar:id133}
See also:

Help for rfvar/simulation\_diagnostics is inherited from superclass RISE\_GENERIC


\bigskip\hrule{}\bigskip



\section{solve}
\label{classes/models/@rfvar/rfvar:id134}\label{classes/models/@rfvar/rfvar:solve}
H1 line


\subsection{Syntax}
\label{classes/models/@rfvar/rfvar:id135}

\subsection{Inputs}
\label{classes/models/@rfvar/rfvar:id136}

\subsection{Outputs}
\label{classes/models/@rfvar/rfvar:id137}

\subsection{More About}
\label{classes/models/@rfvar/rfvar:id138}

\subsection{Examples}
\label{classes/models/@rfvar/rfvar:id139}
See also:


\bigskip\hrule{}\bigskip



\section{stoch\_simul}
\label{classes/models/@rfvar/rfvar:stoch-simul}\label{classes/models/@rfvar/rfvar:id140}
H1 line


\subsection{Syntax}
\label{classes/models/@rfvar/rfvar:id141}

\subsection{Inputs}
\label{classes/models/@rfvar/rfvar:id142}

\subsection{Outputs}
\label{classes/models/@rfvar/rfvar:id143}

\subsection{More About}
\label{classes/models/@rfvar/rfvar:id144}

\subsection{Examples}
\label{classes/models/@rfvar/rfvar:id145}
See also:

Help for rfvar/stoch\_simul is inherited from superclass RISE\_GENERIC


\bigskip\hrule{}\bigskip



\section{structural\_form}
\label{classes/models/@rfvar/rfvar:id146}\label{classes/models/@rfvar/rfvar:structural-form}
\textbf{structural\_form} finds A structural form given the imposed restrictions


\subsection{Syntax}
\label{classes/models/@rfvar/rfvar:id147}
\begin{Verbatim}[commandchars=\\\{\}]
\PYG{n}{newobj}\PYG{o}{=}\PYG{n}{structural\PYGZus{}form}\PYG{p}{(}\PYG{n}{obj}\PYG{p}{)}
\PYG{n}{newobj}\PYG{o}{=}\PYG{n}{structural\PYGZus{}form}\PYG{p}{(}\PYG{n}{obj}\PYG{p}{,}\PYG{n}{varargin}\PYG{p}{)}
\end{Verbatim}


\subsection{Inputs}
\label{classes/models/@rfvar/rfvar:id148}\begin{itemize}
\item {} 
\textbf{obj} : {[}rfvar{]} : reduced form VAR object

\item {} 
varargin : standard optional inputs \textbf{coming in pairs}. Among which:
\begin{itemize}
\item {} \begin{description}
\item[{\textbf{restrict\_lags} : {[}cell array\textbar{}\{`'\}{]} : restrictions on the lag}] \leavevmode
structure. There are two equivalent syntaxes for this:
\begin{itemize}
\item {} 
\href{mailto:\{'var\_name1@var\_name2\{lag}{\{`var\_name1@var\_name2\{lag}\}'\}

\item {} 
\{`alag(var\_name1,var\_name2)'\} : here alag should be understood as
a-lag, where lag is the ``lag'' e.g. a1(infl,unemp) means unemp
does not enter the infl equation at lag 1.

\end{itemize}

\end{description}

\item {} \begin{description}
\item[{\textbf{restrict\_irf\_sign} : {[}cell array\textbar{}\{`'\}{]} : sign restrictions on the}] \leavevmode
impulse responses. The general syntax is
\href{mailto:\{'var\_name\{period\}@shock\_name}{\{`var\_name\{period\}@shock\_name}`,'sign'\} and the default period is
``0'' (for contemporaneous). That means
\href{mailto:\{'var\_name\{0\}@shock\_name}{\{`var\_name\{0\}@shock\_name}`,'+'\} and \href{mailto:\{'var\_name@shock\_name}{\{`var\_name@shock\_name}`,'+'\}
are equivalent

\end{description}

\item {} \begin{description}
\item[{\textbf{restrict\_irf\_zero} : {[}cell array\textbar{}\{`'\}{]} : zero restrictions on the}] \leavevmode
impulse responses. The general syntax is
\href{mailto:\{'var\_name\{period\}@shock\_name}{\{`var\_name\{period\}@shock\_name}`\} and the default period is
``0'' (for contemporaneous). That means
\href{mailto:\{'var\_name\{0\}@shock\_name}{\{`var\_name\{0\}@shock\_name}`\} and \href{mailto:\{'var\_name@shock\_name}{\{`var\_name@shock\_name}`\}
are equivalent

\end{description}

\item {} \begin{description}
\item[{\textbf{structural\_shocks} : {[}cell array\textbar{}\{`'\}{]} : List of structural}] \leavevmode
shocks. The shock names can be entered with or without their
description. For instance :
- \{`E\_PAI','E\_U','E\_MP'\}
- \{`E\_PAI',```inflation shock''','E\_U',```unempl shock''','E\_MP'\}

\end{description}

\item {} \begin{description}
\item[{\textbf{irf\_sample\_max}}] \leavevmode{[}{[}numeric\textbar{}\{10000\}{]}{]}{[}maximum number of trials in{]}
the drawing of rotation matrices

\end{description}

\end{itemize}

\end{itemize}


\subsection{Outputs}
\label{classes/models/@rfvar/rfvar:id149}\begin{itemize}
\item {} 
newobj : {[}rfvar{]}: new rfvar object with the drawn structural form

\end{itemize}


\subsection{More About}
\label{classes/models/@rfvar/rfvar:id150}\begin{itemize}
\item {} 
RISE automatically orders the endogenous variables alphabetically and
tags each equation with one of the endogenous variables. This may be
useful for understanding the behavior of \textbf{restrict\_lags} above.

\item {} 
The Choleski identification scheme is not implemented per se. The user
has to explicitly enter the zeros in the right places. This gives the
flexibility in implementing the restrictions. For instance, one could
imagine a scheme in which choleski restrictions hold only in the long
run.

\item {} 
With only zero restrictions, one cannot expect the impulse responses to
automatically have the correct sign. The rotation imposes zero
restrictions but not the sign. If you would like to have
correctly-signed impulse responses there are two choices:
- explicitly add sign restrictions
- multiply the impulse responses for the wrongly-signed shock with
minus.

\item {} 
If the signs are not explicitly enforced under zeros restrictions,
in an exercise in which one draws many rotations, some will have
one sign and some others a diffferent sign. Here, perhaps more than
elsewhere, it is important to add some sign restrictions to have
consistent results throughout.

\item {} 
Many periods can be entered simultaneously. For instance
`var\_name\{0,3,5,10:20,inf\}@shock\_name'

\item {} 
long-run restrictions are denoted by ``inf''. For instance
\href{mailto:'var\_name\{inf\}@shock\_name}{`var\_name\{inf\}@shock\_name}`

\item {} 
Identification for Markov switching VARs is not implemented/supported.

\end{itemize}


\subsection{Examples}
\label{classes/models/@rfvar/rfvar:id151}
See also:


\bigskip\hrule{}\bigskip



\section{template}
\label{classes/models/@rfvar/rfvar:id152}\label{classes/models/@rfvar/rfvar:template}\begin{quote}

\textasciitilde{}\textasciitilde{} no help found
\end{quote}


\bigskip\hrule{}\bigskip



\section{theoretical\_autocorrelations}
\label{classes/models/@rfvar/rfvar:theoretical-autocorrelations}\label{classes/models/@rfvar/rfvar:id153}
H1 line


\subsection{Syntax}
\label{classes/models/@rfvar/rfvar:id154}

\subsection{Inputs}
\label{classes/models/@rfvar/rfvar:id155}

\subsection{Outputs}
\label{classes/models/@rfvar/rfvar:id156}

\subsection{More About}
\label{classes/models/@rfvar/rfvar:id157}

\subsection{Examples}
\label{classes/models/@rfvar/rfvar:id158}
See also:

Help for rfvar/theoretical\_autocorrelations is inherited from superclass RISE\_GENERIC


\bigskip\hrule{}\bigskip



\section{theoretical\_autocovariances}
\label{classes/models/@rfvar/rfvar:theoretical-autocovariances}\label{classes/models/@rfvar/rfvar:id159}
H1 line


\subsection{Syntax}
\label{classes/models/@rfvar/rfvar:id160}

\subsection{Inputs}
\label{classes/models/@rfvar/rfvar:id161}

\subsection{Outputs}
\label{classes/models/@rfvar/rfvar:id162}

\subsection{More About}
\label{classes/models/@rfvar/rfvar:id163}

\subsection{Examples}
\label{classes/models/@rfvar/rfvar:id164}
See also:

Help for rfvar/theoretical\_autocovariances is inherited from superclass RISE\_GENERIC


\bigskip\hrule{}\bigskip



\section{variance\_decomposition}
\label{classes/models/@rfvar/rfvar:variance-decomposition}\label{classes/models/@rfvar/rfvar:id165}
H1 line


\subsection{Syntax}
\label{classes/models/@rfvar/rfvar:id166}

\subsection{Inputs}
\label{classes/models/@rfvar/rfvar:id167}

\subsection{Outputs}
\label{classes/models/@rfvar/rfvar:id168}

\subsection{More About}
\label{classes/models/@rfvar/rfvar:id169}

\subsection{Examples}
\label{classes/models/@rfvar/rfvar:id170}
See also:

Help for rfvar/variance\_decomposition is inherited from superclass RISE\_GENERIC


\chapter{Structural VAR modeling}
\label{classes/models/@svar/svar:structural-var-modeling}\label{classes/models/@svar/svar::doc}

\section{methods}
\label{classes/models/@svar/svar:methods}\begin{itemize}
\item {} 
{[} {\hyperref[classes/models/@svar/svar:check-optimum]{check\_optimum}} {]}(svar/check\_optimum)

\item {} 
{[} {\hyperref[classes/models/@svar/svar:draw-parameter]{draw\_parameter}} {]}(svar/draw\_parameter)

\item {} 
{[} {\hyperref[classes/models/@svar/svar:estimate]{estimate}} {]}(svar/estimate)

\item {} 
{[} {\hyperref[classes/models/@svar/svar:forecast]{forecast}} {]}(svar/forecast)

\item {} 
{[} {\hyperref[classes/models/@svar/svar:get]{get}} {]}(svar/get)

\item {} 
{[} {\hyperref[classes/models/@svar/svar:historical-decomposition]{historical\_decomposition}} {]}(svar/historical\_decomposition)

\item {} 
{[} {\hyperref[classes/models/@svar/svar:irf]{irf}} {]}(svar/irf)

\item {} 
{[} {\hyperref[classes/models/@svar/svar:isnan]{isnan}} {]}(svar/isnan)

\item {} 
{[} {\hyperref[classes/models/@svar/svar:load-parameters]{load\_parameters}} {]}(svar/load\_parameters)

\item {} 
{[} {\hyperref[classes/models/@svar/svar:log-marginal-data-density]{log\_marginal\_data\_density}} {]}(svar/log\_marginal\_data\_density)

\item {} 
{[} {\hyperref[classes/models/@svar/svar:log-posterior-kernel]{log\_posterior\_kernel}} {]}(svar/log\_posterior\_kernel)

\item {} 
{[} {\hyperref[classes/models/@svar/svar:log-prior-density]{log\_prior\_density}} {]}(svar/log\_prior\_density)

\item {} 
{[} {\hyperref[classes/models/@svar/svar:msvar-priors]{msvar\_priors}} {]}(svar/msvar\_priors)

\item {} 
{[} {\hyperref[classes/models/@svar/svar:posterior-marginal-and-prior-densities]{posterior\_marginal\_and\_prior\_densities}} {]}(svar/posterior\_marginal\_and\_prior\_densities)

\item {} 
{[} {\hyperref[classes/models/@svar/svar:posterior-simulator]{posterior\_simulator}} {]}(svar/posterior\_simulator)

\item {} 
{[} {\hyperref[classes/models/@svar/svar:print-estimation-results]{print\_estimation\_results}} {]}(svar/print\_estimation\_results)

\item {} 
{[} {\hyperref[classes/models/@svar/svar:prior-plots]{prior\_plots}} {]}(svar/prior\_plots)

\item {} 
{[} {\hyperref[classes/models/@svar/svar:report]{report}} {]}(svar/report)

\item {} 
{[} {\hyperref[classes/models/@svar/svar:set]{set}} {]}(svar/set)

\item {} 
{[} {\hyperref[classes/models/@svar/svar:set-solution-to-companion]{set\_solution\_to\_companion}} {]}(svar/set\_solution\_to\_companion)

\item {} 
{[} {\hyperref[classes/models/@svar/svar:simulate]{simulate}} {]}(svar/simulate)

\item {} 
{[} {\hyperref[classes/models/@svar/svar:simulation-diagnostics]{simulation\_diagnostics}} {]}(svar/simulation\_diagnostics)

\item {} 
{[} {\hyperref[classes/models/@svar/svar:solve]{solve}} {]}(svar/solve)

\item {} 
{[} {\hyperref[classes/models/@svar/svar:stoch-simul]{stoch\_simul}} {]}(svar/stoch\_simul)

\item {} 
{[} {\hyperref[classes/models/@svar/svar:svar]{svar}} {]}(svar/svar)

\item {} 
{[} {\hyperref[classes/models/@svar/svar:template]{template}} {]}(svar/template)

\item {} 
{[} {\hyperref[classes/models/@svar/svar:theoretical-autocorrelations]{theoretical\_autocorrelations}} {]}(svar/theoretical\_autocorrelations)

\item {} 
{[} {\hyperref[classes/models/@svar/svar:theoretical-autocovariances]{theoretical\_autocovariances}} {]}(svar/theoretical\_autocovariances)

\item {} 
{[} {\hyperref[classes/models/@svar/svar:variance-decomposition]{variance\_decomposition}} {]}(svar/variance\_decomposition)

\end{itemize}


\section{properties}
\label{classes/models/@svar/svar:properties}\begin{itemize}
\item {} 
{[}constant{]} -

\item {} 
{[}nlags{]} -

\item {} 
{[}legend{]} -

\item {} 
{[}endogenous{]} -

\item {} 
{[}exogenous{]} -

\item {} 
{[}parameters{]} -

\item {} 
{[}observables{]} -

\item {} 
{[}markov\_chains{]} -

\item {} 
{[}options{]} -

\item {} 
{[}estimation{]} -

\item {} 
{[}solution{]} -

\item {} 
{[}filtering{]} -

\end{itemize}


\bigskip\hrule{}\bigskip



\bigskip\hrule{}\bigskip



\section{check\_optimum}
\label{classes/models/@svar/svar:check-optimum}\label{classes/models/@svar/svar:id1}
H1 line


\subsection{Syntax}
\label{classes/models/@svar/svar:syntax}

\subsection{Inputs}
\label{classes/models/@svar/svar:inputs}

\subsection{Outputs}
\label{classes/models/@svar/svar:outputs}

\subsection{More About}
\label{classes/models/@svar/svar:more-about}

\subsection{Examples}
\label{classes/models/@svar/svar:examples}
See also:

Help for svar/check\_optimum is inherited from superclass RISE\_GENERIC


\bigskip\hrule{}\bigskip



\section{draw\_parameter}
\label{classes/models/@svar/svar:id2}\label{classes/models/@svar/svar:draw-parameter}
H1 line


\subsection{Syntax}
\label{classes/models/@svar/svar:id3}

\subsection{Inputs}
\label{classes/models/@svar/svar:id4}

\subsection{Outputs}
\label{classes/models/@svar/svar:id5}

\subsection{More About}
\label{classes/models/@svar/svar:id6}

\subsection{Examples}
\label{classes/models/@svar/svar:id7}
See also:

Help for svar/draw\_parameter is inherited from superclass RISE\_GENERIC


\bigskip\hrule{}\bigskip



\section{estimate}
\label{classes/models/@svar/svar:estimate}\label{classes/models/@svar/svar:id8}
\textbf{estimate} - estimates the parameters of a RISE model


\subsection{Syntax}
\label{classes/models/@svar/svar:id9}
\begin{Verbatim}[commandchars=\\\{\}]
\PYG{n}{obj}\PYG{o}{=}\PYG{n}{estimate}\PYG{p}{(}\PYG{n}{obj}\PYG{p}{)}
\PYG{n}{obj}\PYG{o}{=}\PYG{n}{estimate}\PYG{p}{(}\PYG{n}{obj}\PYG{p}{,}\PYG{n}{varargin}\PYG{p}{)}
\end{Verbatim}


\subsection{Inputs}
\label{classes/models/@svar/svar:id10}\begin{itemize}
\item {} 
\textbf{obj} {[}rise\textbar{}dsge\textbar{}rfvar\textbar{}svar{]}: model object

\item {} 
\textbf{varargin} additional optional inputs among which the most relevant
for estimation are:

\item {} 
\textbf{estim\_parallel} {[}integer\textbar{}\{1\}{]}: Number of starting values

\item {} 
\textbf{estim\_start\_from\_mode} {[}true\textbar{}false\textbar{}\{{[}{]}\}{]}: when empty, the user is
prompted to answer the question as to whether to start estimation from
a previously found mode or not. If true or false, no question is asked.

\item {} 
\textbf{estim\_start\_date} {[}numeric\textbar{}char\textbar{}serial date{]}: date of the first
observation to use in the dataset provided for estimation

\item {} 
\textbf{estim\_end\_date} {[}numeric\textbar{}char\textbar{}serial date{]}: date of the last
observation to use in the dataset provided for estimation

\item {} 
\textbf{estim\_max\_trials} {[}integer\textbar{}\{500\}{]}: When the initial value of the
log-likelihood is too low, RISE uniformly draws from the prior support
in search for a better starting point. It will try this for a maximum
number of \textbf{estim\_max\_trials} times before squeaking with an error.

\item {} 
\textbf{estim\_start\_vals} {[}\{{[}{]}\}\textbar{}struct{]}: when not empty, the parameters
whose names are fields of the structure will see their start values
updated or overriden by the information in \textbf{estim\_start\_vals}. There
is no need to provide values to update the start values for the
estimated parameters.

\item {} 
\textbf{estim\_general\_restrictions} {[}\{{[}{]}\}\textbar{}function handle{]}: when not empty,
the argument should be a function handle that takes as input a
parameterized RISE object and returns a scalar or vector of numbers
representing the strength of the violation of the nonlinear
constraints. Those constraints will be added to the constraints already
included in a rise/dsge file before being presented to the optimization
function.

\item {} 
\textbf{estim\_linear\_restrictions} {[}\{{[}{]}\}\textbar{}cell{]}: This is most often used in
the estimation of rfvar or svar models either to impose block
exogeneity or to impose other forms of linear restrictions. When not
empty, \textbf{estim\_linear\_restrictions} must be a 2-column cell:
- Each row of the first column represents a particular linear
combination of the estimated parameters. Those linear combinations are
constructed using the \textbf{coef} class. Check help for coef.coef for more
details.
- Each row of the second column holds the value of the linear
combination.

\item {} 
\textbf{estim\_blocks} {[}\{{[}{]}\}\textbar{}cell{]}: When not empty, this triggers blockwise
optimization. For further information on how to set blocks, see help
for dsge.create\_estimation\_blocks

\item {} 
\textbf{estim\_priors} {[}\{{[}{]}\}\textbar{}struct{]}: This provides an alternative to
setting priors inside the rise/dsge model file. Each field of the
structure must be the name of an estimated parameter. Each field will
hold a cell array whose structure is described in help
rise\_generic.setup\_priors.

\item {} 
\textbf{estim\_penalty} {[}numeric\textbar{}\{1e+8\}{]}: value of the objective function
when a problem occurs. Possible problems include:
- no solution found
- very low likelihood
- stochastic singularity
- problems computing the initial covariance matrix
- non-positive definite covariance matrices
- etc.

\item {} 
\textbf{estim\_cond\_vars} {[}char\textbar{}cellstr{]}: list of the variables to condition
on during estimation of a dsge model. This then assumes that the data
provided for estimation have several pages. The first page is the
actual data, while the subsequent pages are the expectations data.

\item {} 
\textbf{optimset} {[}struct{]}: identical to matlab's optimset

\item {} 
\textbf{optimizer} {[}char\textbar{}function handle\textbar{}cell{]}: This can be the name of a
standard matlab optimizer or RISE optimization routine or a
user-defined optimization procedure available of the matlab search
path. If the optimzer is provided as a cell, then the first element of
the cell is the name of the optimizer or its handle and the remaining
entries in the cell are additional input arguments to the user-defined
optimization routine. A user-defined optimization function should have
the following syntax

\begin{Verbatim}[commandchars=\\\{\}]
\PYG{p}{[}\PYG{n}{xfinal}\PYG{p}{,}\PYG{n}{ffinal}\PYG{p}{,}\PYG{n}{exitflag}\PYG{p}{]}\PYG{o}{=}\PYG{n}{optimizer}\PYG{p}{(}\PYG{n}{fh}\PYG{p}{,}\PYG{n}{x0}\PYG{p}{,}\PYG{n}{lb}\PYG{p}{,}\PYG{n}{ub}\PYG{p}{,}\PYG{n}{options}\PYG{p}{,}\PYG{n}{varargin}\PYG{p}{)}\PYG{p}{;}
\end{Verbatim}
\begin{description}
\item[{That is, it accepts as inputs:}] \leavevmode\begin{itemize}
\item {} 
\textbf{fh}: the function to optimize

\item {} 
\textbf{x0}: a vector column of initial values of the parameters

\item {} 
\textbf{lb}: a vector column of lower bounds

\item {} 
\textbf{ub}: a vector column of upper bounds

\item {} \begin{description}
\item[{\textbf{options}: a structure of options whose fields will be similar}] \leavevmode
to matlab's optimset

\end{description}

\item {} \begin{description}
\item[{\textbf{varargin}: additional arguments to the user-defined}] \leavevmode
optimization procedure

\end{description}

\end{itemize}

\item[{That is, it provides as outputs:}] \leavevmode\begin{itemize}
\item {} 
\textbf{xfinal}: the vector of final values

\item {} 
\textbf{ffinal}: the value of \textbf{fh} at \textbf{xfinal}

\item {} 
\textbf{exitflag}: a flag similar to the ones provided by matlab's

\end{itemize}

optimization functions.

\end{description}

\item {} 
\textbf{hessian\_type} {[}\{`fd'\}\textbar{}'opg'{]}: The hessian is either computed by
finite differences (fd) or by outer-product-gradient (opg)

\item {} 
\textbf{hessian\_repair} {[}\{false\}\textbar{}true{]}: If the Hessian is not positive
definite, it nevertheless can be repaired and prepared for a potential
mcmc simulation.

\end{itemize}


\subsection{Outputs}
\label{classes/models/@svar/svar:id11}\begin{itemize}
\item {} 
\textbf{obj} {[}rise\textbar{}dsge\textbar{}rfvar\textbar{}svar{]}: model object parameterized with the
mode found and holding additional estimation results and statistics
that can be found under obj.estimation

\end{itemize}


\subsection{More About}
\label{classes/models/@svar/svar:id12}\begin{itemize}
\item {} 
recursive estimation may be done easily by passing a different
estim\_end\_date at the beginning of each estimation run.

\end{itemize}


\subsection{Examples}
\label{classes/models/@svar/svar:id13}
See also:

Help for svar/estimate is inherited from superclass RISE\_GENERIC


\bigskip\hrule{}\bigskip



\section{forecast}
\label{classes/models/@svar/svar:id14}\label{classes/models/@svar/svar:forecast}
\textbf{forecast} - computes forecasts for rise\textbar{}dsge\textbar{}svar\textbar{}rfvar models


\subsection{Syntax}
\label{classes/models/@svar/svar:id15}
\begin{Verbatim}[commandchars=\\\{\}]
\PYG{n}{cond\PYGZus{}fkst\PYGZus{}db}\PYG{o}{=}\PYG{n}{forecast}\PYG{p}{(}\PYG{n}{obj}\PYG{p}{,}\PYG{n}{varargin}\PYG{p}{)}
\end{Verbatim}


\subsection{Inputs}
\label{classes/models/@svar/svar:id16}\begin{itemize}
\item {} 
\textbf{obj} {[}rise\textbar{}dsge\textbar{}svar\textbar{}rfvar{]}: model object

\item {} 
\textbf{varargin} : additional inputs coming in pairs. These include but are
not restricted to:
- \textbf{forecast\_to\_time\_series} {[}\{true\}\textbar{}false{]}: sets the output to time
\begin{quote}

series format or not
\end{quote}
\begin{itemize}
\item {} 
\textbf{forecast\_nsteps} {[}integer\textbar{}\{12\}{]}: number of forecasting steps

\item {} \begin{description}
\item[{\textbf{forecast\_start\_date} {[}char\textbar{}numeric\textbar{}serial date{]}: date when the}] \leavevmode
forecasts start (end of history + 1)

\end{description}

\item {} \begin{description}
\item[{\textbf{forecast\_conditional\_hypothesis} {[}\{jma\}\textbar{}ncp\textbar{}nas{]}: in dsge models in}] \leavevmode
which agents have information beyond the current period, this
option determines the number of periods of shocks need to match the
restrictions:
- Hypothesis \textbf{jma} assumes that irrespective of how
\begin{quote}

many periods of conditioning information are remaining, agents
always receive information on the same number of shocks.
\end{quote}
\begin{itemize}
\item {} \begin{description}
\item[{Hypothesis \textbf{ncp} assumes there are as many shocks periods as}] \leavevmode
the number of the number of conditioning periods

\end{description}

\item {} \begin{description}
\item[{Hypothesis \textbf{nas} assumes there are as many shocks periods as}] \leavevmode
the number of anticipated steps

\end{description}

\end{itemize}

\end{description}

\end{itemize}

\end{itemize}


\subsection{Outputs}
\label{classes/models/@svar/svar:id17}\begin{itemize}
\item {} 
\textbf{cond\_fkst\_db} {[}struct\textbar{}matrix{]}: depending on the value of
\textbf{forecast\_to\_time\_series} the returned output is a structure with
time series or a cell containing a matrix and the information to
reconstruct the time series.

\end{itemize}


\subsection{More About}
\label{classes/models/@svar/svar:id18}\begin{itemize}
\item {} 
the historical information as well as the conditioning information come
from the same database. The time series must be organized such that for
each series, the first page represents the actual data and all
subsequent pages represent conditional information. If a particular
condition is ``nan'', that location is not constrained

\item {} 
Conditional forecasting for nonlinear models is also supported.
However, the solving of the implied nonlinear problem may fail if the
model displays instability

\item {} 
Both HARD CONDITIONS and SOFT CONDITIONS are implemented but the latter
are currently disabled in expectation of a better user interface.

\item {} 
The data may also contain time series for a variable with name
\textbf{regime} in that case, the forecast/simulation paths are computed
following the information therein. \textbf{regime} must be a member of 1:h,
where h is the maximum number of regimes.

\end{itemize}


\subsection{Examples}
\label{classes/models/@svar/svar:id19}
See also: simulate

Help for svar/forecast is inherited from superclass RISE\_GENERIC


\bigskip\hrule{}\bigskip



\section{get}
\label{classes/models/@svar/svar:id20}\label{classes/models/@svar/svar:get}
H1 line


\subsection{Syntax}
\label{classes/models/@svar/svar:id21}

\subsection{Inputs}
\label{classes/models/@svar/svar:id22}

\subsection{Outputs}
\label{classes/models/@svar/svar:id23}

\subsection{More About}
\label{classes/models/@svar/svar:id24}

\subsection{Examples}
\label{classes/models/@svar/svar:id25}
See also:

Help for svar/get is inherited from superclass RISE\_GENERIC


\bigskip\hrule{}\bigskip



\section{historical\_decomposition}
\label{classes/models/@svar/svar:historical-decomposition}\label{classes/models/@svar/svar:id26}
\textbf{historical\_decomposition} Computes historical decompositions of a DSGE model


\subsection{Syntax}
\label{classes/models/@svar/svar:id27}
\begin{Verbatim}[commandchars=\\\{\}]
\PYG{p}{[}\PYG{n}{Histdec}\PYG{p}{,}\PYG{n}{obj}\PYG{p}{]}\PYG{o}{=}\PYG{n}{history\PYGZus{}dec}\PYG{p}{(}\PYG{n}{obj}\PYG{p}{)}
\PYG{p}{[}\PYG{n}{Histdec}\PYG{p}{,}\PYG{n}{obj}\PYG{p}{]}\PYG{o}{=}\PYG{n}{history\PYGZus{}dec}\PYG{p}{(}\PYG{n}{obj}\PYG{p}{,}\PYG{n}{varargin}\PYG{p}{)}
\end{Verbatim}


\subsection{Inputs}
\label{classes/models/@svar/svar:id28}\begin{itemize}
\item {} 
obj : {[}rise\textbar{}dsge\textbar{}rfvar\textbar{}svar{]} model(s) for which to compute the
decomposition. obj could be a vector of models

\item {} 
varargin : standard optional inputs \textbf{coming in pairs}. Among which:
- \textbf{histdec\_start\_date} : {[}char\textbar{}numeric\textbar{}\{`'\}{]} : date at which the
\begin{quote}

decomposition starts. If empty, the decomposition starts at he
beginning of the history of the dataset
\end{quote}

\end{itemize}


\subsection{Outputs}
\label{classes/models/@svar/svar:id29}\begin{itemize}
\item {} 
Histdec : {[}struct\textbar{}cell array{]} structure or cell array of structures
with the decompositions in each model. The decompositions are given in
terms of:
- the exogenous variables
- \textbf{InitialConditions} : the effect of initial conditions
- \textbf{risk} : measure of the effect of non-certainty equivalence
- \textbf{switch} : the effect of switching (which is also a shock!!!)
- \textbf{steady\_state} : the contribution of the steady state

\end{itemize}


\subsection{Remarks}
\label{classes/models/@svar/svar:remarks}\begin{itemize}
\item {} 
the elements that do not contribute to any of the variables are
automatically discarded.

\item {} 
\textbf{N.B} : a switching model is inherently nonlinear and so, strictly
speaking, the type of decomposition we do for linear/linearized
constant-parameter models is not feasible. RISE takes an approximation
in which the variables, shocks and states matrices across states are
averaged. The averaging weights are the smoothed probabilities.

\end{itemize}


\subsection{Examples}
\label{classes/models/@svar/svar:id30}
See also:

Help for svar/historical\_decomposition is inherited from superclass RISE\_GENERIC


\bigskip\hrule{}\bigskip



\section{irf}
\label{classes/models/@svar/svar:irf}\label{classes/models/@svar/svar:id31}
\textbf{irf} - computes impulse responses for a RISE model


\subsection{Syntax}
\label{classes/models/@svar/svar:id32}
\begin{Verbatim}[commandchars=\\\{\}]
\PYG{n}{myirfs}\PYG{o}{=}\PYG{n}{irf}\PYG{p}{(}\PYG{n}{obj}\PYG{p}{)}

\PYG{n}{myirfs}\PYG{o}{=}\PYG{n}{irf}\PYG{p}{(}\PYG{n}{obj}\PYG{p}{,}\PYG{n}{varargin}\PYG{p}{)}
\end{Verbatim}


\subsection{Inputs}
\label{classes/models/@svar/svar:id33}\begin{itemize}
\item {} 
\textbf{obj} {[}rise\textbar{}dsge\textbar{}rfvar\textbar{}svar{]}: single or vector of RISE models

\item {} 
\textbf{varargin} : optional options coming in pairs. The notable ones that
will influence the behavior of the impulse responses are:

\item {} 
\textbf{irf\_shock\_list} {[}char\textbar{}cellstr\textbar{}\{`'\}{]}: list of shocks for which we
want to compute impulse responses

\item {} 
\textbf{irf\_var\_list} {[}char\textbar{}cellstr\textbar{}\{`'\}{]}: list of the endogenous variables
we want to report

\item {} 
\textbf{irf\_periods} {[}integer\textbar{}\{40\}{]}: length of the irfs

\item {} 
\textbf{irf\_shock\_sign} {[}numeric\textbar{}-1\textbar{}\{1\}{]}: sign or scale of the original
impulse. If \textbf{irf\_shock\_sign} \textgreater{}0, we get impulse responses to a
positive shock. If \textbf{irf\_shock\_sign} \textless{}0, the responses are negative.
If If \textbf{irf\_shock\_sign} =0, all the responses are 0.

\item {} 
\textbf{irf\_draws} {[}integer\textbar{}\{50\}{]}: number of draws used in the simulation
impulse responses in a nonlinear model. A nonlinear model is defined as
a model that satisfies at least one of the following criteria
- solved at an order \textgreater{}1
- has more than one regime and option \textbf{irf\_regime\_specific} below is
\begin{quote}

set to false
\end{quote}

\item {} 
\textbf{irf\_type} {[}\{irf\}\textbar{}girf{]}: type of irfs. If the type is irf, the
impulse responses are computed directly exploiting the fact that the
model is linear. If the type is girf, the formula for the generalized
impulse responses is used: the irf is defined as the expectation of the
difference of two simulation paths. In the first path the initial
impulse for the shock of interest is nonzero while it is zero for the
second path. All other shocks are the same for both paths in a given
simulation.

\item {} 
\textbf{irf\_regime\_specific} {[}\{true\}\textbar{}false{]}: In a switching model, we may or
may not want to compute impulse responses specific to each regime.

\item {} 
\textbf{irf\_use\_historical\_data} {[}\{false\}\textbar{}true{]}: if true, the data stored in
option \textbf{simul\_historical\_data} are used as initial conditions. But
the model has to be nonlinear otherwise the initial conditions are set
to zero. This option gives the flexibility to set the initial
conditions for the impulse responses.

\item {} 
\textbf{irf\_to\_time\_series} {[}\{true\}\textbar{}false{]}: If true, the output is in the
form of time series. Else it is in the form of a cell containing the
information needed to reconstruct the time series.

\end{itemize}


\subsection{Outputs}
\label{classes/models/@svar/svar:id34}\begin{itemize}
\item {} 
\textbf{myirfs} {[}\{struct\}\textbar{}cell{]}: Impulse response data

\end{itemize}


\subsection{More About}
\label{classes/models/@svar/svar:id35}\begin{itemize}
\item {} 
for linear models or models solved up to first order, the initial
conditions as well as the steady states are set to 0 in the computation
of the impulse responses.

\item {} 
for nonlinear models, the initial conditions is the ergodic mean

\end{itemize}


\subsection{Examples}
\label{classes/models/@svar/svar:id36}
See also:

Help for svar/irf is inherited from superclass RISE\_GENERIC


\bigskip\hrule{}\bigskip



\section{isnan}
\label{classes/models/@svar/svar:id37}\label{classes/models/@svar/svar:isnan}
H1 line


\subsection{Syntax}
\label{classes/models/@svar/svar:id38}

\subsection{Inputs}
\label{classes/models/@svar/svar:id39}

\subsection{Outputs}
\label{classes/models/@svar/svar:id40}

\subsection{More About}
\label{classes/models/@svar/svar:id41}

\subsection{Examples}
\label{classes/models/@svar/svar:id42}
See also:

Help for svar/isnan is inherited from superclass RISE\_GENERIC


\bigskip\hrule{}\bigskip



\section{load\_parameters}
\label{classes/models/@svar/svar:id43}\label{classes/models/@svar/svar:load-parameters}
H1 line


\subsection{Syntax}
\label{classes/models/@svar/svar:id44}

\subsection{Inputs}
\label{classes/models/@svar/svar:id45}

\subsection{Outputs}
\label{classes/models/@svar/svar:id46}

\subsection{More About}
\label{classes/models/@svar/svar:id47}

\subsection{Examples}
\label{classes/models/@svar/svar:id48}
See also:

Help for svar/load\_parameters is inherited from superclass RISE\_GENERIC


\bigskip\hrule{}\bigskip



\section{log\_marginal\_data\_density}
\label{classes/models/@svar/svar:log-marginal-data-density}\label{classes/models/@svar/svar:id49}
H1 line


\subsection{Syntax}
\label{classes/models/@svar/svar:id50}

\subsection{Inputs}
\label{classes/models/@svar/svar:id51}

\subsection{Outputs}
\label{classes/models/@svar/svar:id52}

\subsection{More About}
\label{classes/models/@svar/svar:id53}

\subsection{Examples}
\label{classes/models/@svar/svar:id54}
See also:

Help for svar/log\_marginal\_data\_density is inherited from superclass RISE\_GENERIC


\bigskip\hrule{}\bigskip



\section{log\_posterior\_kernel}
\label{classes/models/@svar/svar:log-posterior-kernel}\label{classes/models/@svar/svar:id55}
H1 line


\subsection{Syntax}
\label{classes/models/@svar/svar:id56}

\subsection{Inputs}
\label{classes/models/@svar/svar:id57}

\subsection{Outputs}
\label{classes/models/@svar/svar:id58}

\subsection{More About}
\label{classes/models/@svar/svar:id59}

\subsection{Examples}
\label{classes/models/@svar/svar:id60}
See also:

Help for svar/log\_posterior\_kernel is inherited from superclass RISE\_GENERIC


\bigskip\hrule{}\bigskip



\section{log\_prior\_density}
\label{classes/models/@svar/svar:id61}\label{classes/models/@svar/svar:log-prior-density}
H1 line


\subsection{Syntax}
\label{classes/models/@svar/svar:id62}

\subsection{Inputs}
\label{classes/models/@svar/svar:id63}

\subsection{Outputs}
\label{classes/models/@svar/svar:id64}

\subsection{More About}
\label{classes/models/@svar/svar:id65}

\subsection{Examples}
\label{classes/models/@svar/svar:id66}
See also:

Help for svar/log\_prior\_density is inherited from superclass RISE\_GENERIC


\bigskip\hrule{}\bigskip



\section{msvar\_priors}
\label{classes/models/@svar/svar:id67}\label{classes/models/@svar/svar:msvar-priors}
H1 line


\subsection{Syntax}
\label{classes/models/@svar/svar:id68}

\subsection{Inputs}
\label{classes/models/@svar/svar:id69}

\subsection{Outputs}
\label{classes/models/@svar/svar:id70}

\subsection{More About}
\label{classes/models/@svar/svar:id71}

\subsection{Examples}
\label{classes/models/@svar/svar:id72}
See also:


\bigskip\hrule{}\bigskip



\section{posterior\_marginal\_and\_prior\_densities}
\label{classes/models/@svar/svar:id73}\label{classes/models/@svar/svar:posterior-marginal-and-prior-densities}
H1 line


\subsection{Syntax}
\label{classes/models/@svar/svar:id74}

\subsection{Inputs}
\label{classes/models/@svar/svar:id75}

\subsection{Outputs}
\label{classes/models/@svar/svar:id76}

\subsection{More About}
\label{classes/models/@svar/svar:id77}

\subsection{Examples}
\label{classes/models/@svar/svar:id78}
See also:

Help for svar/posterior\_marginal\_and\_prior\_densities is inherited from superclass RISE\_GENERIC


\bigskip\hrule{}\bigskip



\section{posterior\_simulator}
\label{classes/models/@svar/svar:posterior-simulator}\label{classes/models/@svar/svar:id79}
H1 line


\subsection{Syntax}
\label{classes/models/@svar/svar:id80}

\subsection{Inputs}
\label{classes/models/@svar/svar:id81}

\subsection{Outputs}
\label{classes/models/@svar/svar:id82}

\subsection{More About}
\label{classes/models/@svar/svar:id83}

\subsection{Examples}
\label{classes/models/@svar/svar:id84}
See also:

Help for svar/posterior\_simulator is inherited from superclass RISE\_GENERIC


\bigskip\hrule{}\bigskip



\section{print\_estimation\_results}
\label{classes/models/@svar/svar:print-estimation-results}\label{classes/models/@svar/svar:id85}
H1 line


\subsection{Syntax}
\label{classes/models/@svar/svar:id86}

\subsection{Inputs}
\label{classes/models/@svar/svar:id87}

\subsection{Outputs}
\label{classes/models/@svar/svar:id88}

\subsection{More About}
\label{classes/models/@svar/svar:id89}

\subsection{Examples}
\label{classes/models/@svar/svar:id90}
See also:

Help for svar/print\_estimation\_results is inherited from superclass RISE\_GENERIC


\bigskip\hrule{}\bigskip



\section{prior\_plots}
\label{classes/models/@svar/svar:id91}\label{classes/models/@svar/svar:prior-plots}
H1 line


\subsection{Syntax}
\label{classes/models/@svar/svar:id92}

\subsection{Inputs}
\label{classes/models/@svar/svar:id93}

\subsection{Outputs}
\label{classes/models/@svar/svar:id94}

\subsection{More About}
\label{classes/models/@svar/svar:id95}

\subsection{Examples}
\label{classes/models/@svar/svar:id96}
See also:

Help for svar/prior\_plots is inherited from superclass RISE\_GENERIC


\bigskip\hrule{}\bigskip



\section{report}
\label{classes/models/@svar/svar:report}\label{classes/models/@svar/svar:id97}
\textbf{REPORT} assigns the elements of interest to a rise\_report.report object


\subsection{Syntax}
\label{classes/models/@svar/svar:id98}\begin{description}
\item[{::}] \leavevmode\begin{itemize}
\item {} 
REPORT(rise.empty(0)) : displays the default inputs

\item {} 
REPORT(obj,destination\_root,rep\_items) : assigns the reported
elements in rep\_items to destination\_root

\item {} 
REPORT(obj,destination\_root,rep\_items,varargin) : assigns varargin to
obj before doing the rest

\end{itemize}

\end{description}


\subsection{Inputs}
\label{classes/models/@svar/svar:id99}\begin{itemize}
\item {} 
obj : {[}rise\textbar{}dsge{]}

\item {} 
destination\_root : {[}rise\_report.report{]} : handle for the actual report

\item {} 
rep\_items : {[}char\textbar{}cellstr{]} : list of desired items to report. This list
can only include : `endogenous', `exogenous', `observables',
`parameters', `solution', `estimation', `estimation\_statistics',
`equations', `code'

\end{itemize}


\subsection{Outputs}
\label{classes/models/@svar/svar:id100}
none


\subsection{More About}
\label{classes/models/@svar/svar:id101}

\subsection{Examples}
\label{classes/models/@svar/svar:id102}
See also:

Help for svar/report is inherited from superclass RISE\_GENERIC


\bigskip\hrule{}\bigskip



\section{set}
\label{classes/models/@svar/svar:id103}\label{classes/models/@svar/svar:set}
\textbf{set} - sets options for RISE models


\subsection{Syntax}
\label{classes/models/@svar/svar:id104}
\begin{Verbatim}[commandchars=\\\{\}]
\PYG{n}{obj}\PYG{o}{=}\PYG{n+nb}{set}\PYG{p}{(}\PYG{n}{obj}\PYG{p}{,}\PYG{n}{varargin}\PYG{p}{)}
\end{Verbatim}


\subsection{Inputs}
\label{classes/models/@svar/svar:id105}\begin{itemize}
\item {} 
\textbf{obj} {[}rise\textbar{}dsge\textbar{}svar\textbar{}rfvar{]}: model object

\item {} 
\textbf{varargin} : valid input arguments coming in pairs.

\end{itemize}


\subsection{Outputs}
\label{classes/models/@svar/svar:id106}\begin{itemize}
\item {} 
\textbf{obj} {[}rise\textbar{}dsge\textbar{}svar\textbar{}rfvar{]}: model object

\end{itemize}


\subsection{More About}
\label{classes/models/@svar/svar:id107}\begin{itemize}
\item {} 
one can force a new field into the options by prefixing it with a `+'
sign. Let's say yourfield is not part of the options and you would like
to force it to be in the options because it is going to be used in some
function or algorithm down the road. Then you can run
m=set(m,'+yourfield',value). then m will be part of the new options.

\end{itemize}


\subsection{Examples}
\label{classes/models/@svar/svar:id108}
See also:

Help for svar/set is inherited from superclass RISE\_GENERIC


\bigskip\hrule{}\bigskip



\section{set\_solution\_to\_companion}
\label{classes/models/@svar/svar:id109}\label{classes/models/@svar/svar:set-solution-to-companion}
H1 line


\subsection{Syntax}
\label{classes/models/@svar/svar:id110}

\subsection{Inputs}
\label{classes/models/@svar/svar:id111}

\subsection{Outputs}
\label{classes/models/@svar/svar:id112}

\subsection{More About}
\label{classes/models/@svar/svar:id113}

\subsection{Examples}
\label{classes/models/@svar/svar:id114}
See also:


\bigskip\hrule{}\bigskip



\section{simulate}
\label{classes/models/@svar/svar:simulate}\label{classes/models/@svar/svar:id115}
\textbf{simulate} - simulates a RISE model


\subsection{Syntax}
\label{classes/models/@svar/svar:id116}
\begin{Verbatim}[commandchars=\\\{\}]
\PYG{p}{[}\PYG{n}{db}\PYG{p}{,}\PYG{n}{states}\PYG{p}{,}\PYG{n}{retcode}\PYG{p}{]} \PYG{o}{=} \PYG{n}{simulate}\PYG{p}{(}\PYG{n}{obj}\PYG{p}{,}\PYG{n}{varargin}\PYG{p}{)}
\end{Verbatim}


\subsection{Inputs}
\label{classes/models/@svar/svar:id117}\begin{itemize}
\item {} 
\textbf{obj} {[}rfvar\textbar{}dsge\textbar{}rise\textbar{}svar{]}: model object

\item {} 
\textbf{varargin} : additional arguments including but not restricted to
\begin{itemize}
\item {} 
\textbf{simul\_periods} {[}integer\textbar{}\{100\}{]}: number of simulation periods

\item {} 
\textbf{simul\_burn} {[}integer\textbar{}\{100\}{]}: number of burn-in periods

\item {} \begin{description}
\item[{\textbf{simul\_algo} {[}{[}\{mt19937ar\}\textbar{} mcg16807\textbar{}mlfg6331\_64\textbar{}mrg32k3a\textbar{}}] \leavevmode
shr3cong\textbar{}swb2712{]}{]}: matlab's seeding algorithms

\end{description}

\item {} 
\textbf{simul\_seed} {[}numeric\textbar{}\{0\}{]}: seed of the computations

\item {} \begin{description}
\item[{\textbf{simul\_historical\_data} {[}ts\textbar{}struct\textbar{}\{`'\}{]}: historical data from}] \leavevmode
which the simulations are based. If empty, the simulations start at
the steady state.

\end{description}

\item {} \begin{description}
\item[{\textbf{simul\_history\_end\_date} {[}char\textbar{}integer\textbar{}serial date{]}: last date of}] \leavevmode
history

\end{description}

\item {} \begin{description}
\item[{\textbf{simul\_regime} {[}integer\textbar{}vector\textbar{}\{{[}{]}\}{]}: regimes for which the model}] \leavevmode
is simulated

\end{description}

\item {} \begin{description}
\item[{\textbf{simul\_update\_shocks\_handle} {[}function handle{]}: we may want to}] \leavevmode
update the shocks if some condition on the state of the economy is
satisfied. For instance, shock monetary policy to keep the interest
rate at the floor for an extented period of time if we already are
at the ZLB/ZIF. simul\_update\_shocks\_handle takes as inputs the
current shocks and the state vector (all the endogenous variables)
and returns the updated shocks. But for all this to be put into
motion, the user also has to turn on \textbf{simul\_do\_update\_shocks} by
setting it to true.

\end{description}

\item {} \begin{description}
\item[{\textbf{simul\_do\_update\_shocks} {[}true\textbar{}\{false\}{]}: update the shocks based on}] \leavevmode
\textbf{simul\_update\_shocks\_handle} or not.

\end{description}

\item {} \begin{description}
\item[{\textbf{simul\_to\_time\_series} {[}\{true\}\textbar{}false{]}: if true, the output is a}] \leavevmode
time series, else a cell array with a matrix and information on
elements that help reconstruct the time series.

\end{description}

\end{itemize}

\end{itemize}


\subsection{Outputs}
\label{classes/models/@svar/svar:id118}\begin{itemize}
\item {} 
\textbf{db} {[}struct\textbar{}cell array{]}: if \textbf{simul\_to\_time\_series} is true, the
output is a time series, else a cell array with a matrix and
information on elements that help reconstruct the time series.

\item {} 
\textbf{states} {[}vector{]}: history of the regimes over the forecast horizon

\item {} 
\textbf{retcode} {[}integer{]}: if 0, the simulation went fine. Else something
got wrong. In that case one can understand the problem by running
decipher(retcode)

\end{itemize}


\subsection{More About}
\label{classes/models/@svar/svar:id119}\begin{itemize}
\item {} 
\textbf{simul\_historical\_data} contains the historical data as well as
conditional information over the forecast horizon. It may also include
as an alternative to \textbf{simul\_regime}, a time series with name
\textbf{regime}, which indicates the regimes over the forecast horizon.

\end{itemize}


\subsection{Examples}
\label{classes/models/@svar/svar:id120}
See also:

Help for svar/simulate is inherited from superclass RISE\_GENERIC


\bigskip\hrule{}\bigskip



\section{simulation\_diagnostics}
\label{classes/models/@svar/svar:simulation-diagnostics}\label{classes/models/@svar/svar:id121}
H1 line


\subsection{Syntax}
\label{classes/models/@svar/svar:id122}

\subsection{Inputs}
\label{classes/models/@svar/svar:id123}

\subsection{Outputs}
\label{classes/models/@svar/svar:id124}

\subsection{More About}
\label{classes/models/@svar/svar:id125}

\subsection{Examples}
\label{classes/models/@svar/svar:id126}
See also:

Help for svar/simulation\_diagnostics is inherited from superclass RISE\_GENERIC


\bigskip\hrule{}\bigskip



\section{solve}
\label{classes/models/@svar/svar:id127}\label{classes/models/@svar/svar:solve}
H1 line


\subsection{Syntax}
\label{classes/models/@svar/svar:id128}

\subsection{Inputs}
\label{classes/models/@svar/svar:id129}

\subsection{Outputs}
\label{classes/models/@svar/svar:id130}

\subsection{More About}
\label{classes/models/@svar/svar:id131}

\subsection{Examples}
\label{classes/models/@svar/svar:id132}
See also:


\bigskip\hrule{}\bigskip



\section{stoch\_simul}
\label{classes/models/@svar/svar:id133}\label{classes/models/@svar/svar:stoch-simul}
H1 line


\subsection{Syntax}
\label{classes/models/@svar/svar:id134}

\subsection{Inputs}
\label{classes/models/@svar/svar:id135}

\subsection{Outputs}
\label{classes/models/@svar/svar:id136}

\subsection{More About}
\label{classes/models/@svar/svar:id137}

\subsection{Examples}
\label{classes/models/@svar/svar:id138}
See also:

Help for svar/stoch\_simul is inherited from superclass RISE\_GENERIC


\bigskip\hrule{}\bigskip



\section{svar}
\label{classes/models/@svar/svar:id139}\label{classes/models/@svar/svar:svar}\begin{quote}

\textasciitilde{}\textasciitilde{} no help found
\end{quote}


\bigskip\hrule{}\bigskip



\section{template}
\label{classes/models/@svar/svar:id140}\label{classes/models/@svar/svar:template}\begin{quote}

\textasciitilde{}\textasciitilde{} no help found
\end{quote}


\bigskip\hrule{}\bigskip



\section{theoretical\_autocorrelations}
\label{classes/models/@svar/svar:theoretical-autocorrelations}\label{classes/models/@svar/svar:id141}
H1 line


\subsection{Syntax}
\label{classes/models/@svar/svar:id142}

\subsection{Inputs}
\label{classes/models/@svar/svar:id143}

\subsection{Outputs}
\label{classes/models/@svar/svar:id144}

\subsection{More About}
\label{classes/models/@svar/svar:id145}

\subsection{Examples}
\label{classes/models/@svar/svar:id146}
See also:

Help for svar/theoretical\_autocorrelations is inherited from superclass RISE\_GENERIC


\bigskip\hrule{}\bigskip



\section{theoretical\_autocovariances}
\label{classes/models/@svar/svar:id147}\label{classes/models/@svar/svar:theoretical-autocovariances}
H1 line


\subsection{Syntax}
\label{classes/models/@svar/svar:id148}

\subsection{Inputs}
\label{classes/models/@svar/svar:id149}

\subsection{Outputs}
\label{classes/models/@svar/svar:id150}

\subsection{More About}
\label{classes/models/@svar/svar:id151}

\subsection{Examples}
\label{classes/models/@svar/svar:id152}
See also:

Help for svar/theoretical\_autocovariances is inherited from superclass RISE\_GENERIC


\bigskip\hrule{}\bigskip



\section{variance\_decomposition}
\label{classes/models/@svar/svar:variance-decomposition}\label{classes/models/@svar/svar:id153}
H1 line


\subsection{Syntax}
\label{classes/models/@svar/svar:id154}

\subsection{Inputs}
\label{classes/models/@svar/svar:id155}

\subsection{Outputs}
\label{classes/models/@svar/svar:id156}

\subsection{More About}
\label{classes/models/@svar/svar:id157}

\subsection{Examples}
\label{classes/models/@svar/svar:id158}
See also:

Help for svar/variance\_decomposition is inherited from superclass RISE\_GENERIC


\chapter{Time series}
\label{classes/time_series/@ts/ts:time-series}\label{classes/time_series/@ts/ts::doc}
ts Methods:

acos -   H1 line
acosh -   H1 line
acot -   H1 line
acoth -   H1 line
aggregate -   H1 line
allmean -   H1 line
and -   H1 line
apply -   H1 line
asin -   H1 line
asinh -   H1 line
atan -   H1 line
atanh -   H1 line
automatic\_model\_selection -   H1 line
bar -   H1 line
barh -   H1 line
boxplot -   H1 line
bsxfun -   H1 line
cat - concatenates time series along the specified dimension
collect -   H1 line
corr -   H1 line
corrcoef -   H1 line
cos -   H1 line
cosh -   H1 line
cot -   H1 line
coth -   H1 line
cov -   H1 line
ctranspose -   H1 line
cumprod -   H1 line
cumsum -   H1 line
decompose\_series -   H1 line
describe -   H1 line
display -   H1 line
double -   H1 line
drop -   H1 line
dummy -   H1 line
eq -   H1 line
exp -   H1 line
expanding -   H1 line
fanchart -   H1 line
ge -   H1 line
get -   H1 line
gt -   H1 line
head -   H1 line
hist -   H1 line
horzcat -   H1 line
hpfilter -   H1 line
index -   H1 line
interpolate -   H1 line
intersect -   H1 line
isfinite -   H1 line
isinf -   H1 line
isnan -   H1 line
jbtest -   H1 line
kurtosis -   H1 line
le -   H1 line
log -   H1 line
lt -   H1 line
max -   H1 line
mean -   H1 line
median -   H1 line
min -   H1 line
minus -   H1 line
mode -   H1 line
mpower -   H1 line
mrdivide -   H1 line
mtimes -   H1 line
nan -   H1 line
ne -   H1 line
numel -   H1 line
ones - overloads ones for ts objects
pages2struct -   H1 line
plot -   H1 line
plotyy -   H1 line
plus -   H1 line
power -   H1 line
prctile - Percentiles of a time series (ts)
quantile -   H1 line
rand -   H1 line
randn -   H1 line
range -   H1 line
rdivide -   H1 line
regress -   H1 line
reset\_start\_date -   H1 line
rolling -   H1 line
sin -   H1 line
sinh -   H1 line
skewness -   H1 line
sort -   H1 line
spectrum -   H1 line
std -   H1 line
step\_dummy -   H1 line
subsasgn -   H1 line
subsref -   H1 line
sum -   H1 line
tail -   H1 line
times -   H1 line
transform -   H1 line
transpose -   H1 line
ts - Methods:
uminus -   H1 line
values -   H1 line
var -   H1 line
zeros -   H1 line

ts  Properties:

varnames -   names of the variables in the database
start - time of the time series
finish -   end time of the time series
frequency - of the time series
NumberOfObservations -   number of observations in the time series
NumberOfPages -   number of pages (third dimension) of the time series
NumberOfVariables -   number of variables in the time series


\chapter{Markov Chain Monte Carlo for Bayesian Estimation}
\label{mcmc::doc}\label{mcmc:markov-chain-monte-carlo-for-bayesian-estimation}

\section{Metropolis Hastings}
\label{mcmc:metropolis-hastings}

\section{Gibbs sampling}
\label{mcmc:gibbs-sampling}

\section{Marginal data density}
\label{mcmc:marginal-data-density}

\subsection{Laplace approximation}
\label{mcmc:laplace-approximation}

\subsection{Modified harmonic mean}
\label{mcmc:modified-harmonic-mean}

\subsection{Waggoner and Zha (2008)}
\label{mcmc:waggoner-and-zha-2008}

\subsection{Mueller}
\label{mcmc:mueller}

\subsection{Chib and Jeliazkov}
\label{mcmc:chib-and-jeliazkov}

\chapter{Derivative-free optimization}
\label{derivative_free_optimization:derivative-free-optimization}\label{derivative_free_optimization::doc}\begin{itemize}
\item {} 
differential evolution

\item {} 
bee algorithm

\item {} 
biogeography

\item {} 
studga

\item {} 
ants

\end{itemize}


\chapter{Monte Carlo Filtering}
\label{classes/utils/@mcf/mcf:monte-carlo-filtering}\label{classes/utils/@mcf/mcf::doc}

\section{methods}
\label{classes/utils/@mcf/mcf:methods}\begin{itemize}
\item {} 
{[} {\hyperref[classes/utils/@mcf/mcf:addlistener]{addlistener}} {]}(mcf/addlistener)

\item {} 
{[} {\hyperref[classes/utils/@mcf/mcf:cdf]{cdf}} {]}(mcf/cdf)

\item {} 
{[} {\hyperref[classes/utils/@mcf/mcf:cdf-plot]{cdf\_plot}} {]}(mcf/cdf\_plot)

\item {} 
{[} {\hyperref[classes/utils/@mcf/mcf:correlation-patterns-plot]{correlation\_patterns\_plot}} {]}(mcf/correlation\_patterns\_plot)

\item {} 
{[} {\hyperref[classes/utils/@mcf/mcf:delete]{delete}} {]}(mcf/delete)

\item {} 
{[} {\hyperref[classes/utils/@mcf/mcf:eq]{eq}} {]}(mcf/eq)

\item {} 
{[} {\hyperref[classes/utils/@mcf/mcf:findobj]{findobj}} {]}(mcf/findobj)

\item {} 
{[} {\hyperref[classes/utils/@mcf/mcf:findprop]{findprop}} {]}(mcf/findprop)

\item {} 
{[} {\hyperref[classes/utils/@mcf/mcf:ge]{ge}} {]}(mcf/ge)

\item {} 
{[} {\hyperref[classes/utils/@mcf/mcf:gt]{gt}} {]}(mcf/gt)

\item {} 
{[} {\hyperref[classes/utils/@mcf/mcf:isvalid]{isvalid}} {]}(mcf/isvalid)

\item {} 
{[} {\hyperref[classes/utils/@mcf/mcf:kolmogorov-smirnov-test]{kolmogorov\_smirnov\_test}} {]}(mcf/kolmogorov\_smirnov\_test)

\item {} 
{[} {\hyperref[classes/utils/@mcf/mcf:le]{le}} {]}(mcf/le)

\item {} 
{[} {\hyperref[classes/utils/@mcf/mcf:lt]{lt}} {]}(mcf/lt)

\item {} 
{[} {\hyperref[classes/utils/@mcf/mcf:mcf]{mcf}} {]}(mcf/mcf)

\item {} 
{[} {\hyperref[classes/utils/@mcf/mcf:ne]{ne}} {]}(mcf/ne)

\item {} 
{[} {\hyperref[classes/utils/@mcf/mcf:notify]{notify}} {]}(mcf/notify)

\item {} 
{[} {\hyperref[classes/utils/@mcf/mcf:scatter]{scatter}} {]}(mcf/scatter)

\end{itemize}


\section{properties}
\label{classes/utils/@mcf/mcf:properties}\begin{itemize}
\item {} 
{[}lb{]} -

\item {} 
{[}ub{]} -

\item {} 
{[}nsim{]} -

\item {} 
{[}procedure{]} -

\item {} 
{[}parameter\_names{]} -

\item {} 
{[}samples{]} -

\item {} 
{[}is\_behaved{]} -

\item {} 
{[}nparam{]} -

\item {} 
{[}is\_sampled{]} -

\item {} 
{[}check\_behavior{]} -

\item {} 
{[}number\_of\_outputs{]} -

\item {} 
{[}user\_outputs{]} -

\item {} 
{[}known\_procedures{]} -

\end{itemize}


\bigskip\hrule{}\bigskip



\bigskip\hrule{}\bigskip



\section{addlistener}
\label{classes/utils/@mcf/mcf:addlistener}\label{classes/utils/@mcf/mcf:id1}\begin{quote}
\begin{description}
\item[{\textbf{ADDLISTENER}   Add listener for event.}] \leavevmode
el = ADDLISTENER(hSource, `Eventname', Callback) creates a listener
for the event named Eventname, the source of which is handle object
hSource.  If hSource is an array of source handles, the listener
responds to the named event on any handle in the array.  The Callback
is a function handle that is invoked when the event is triggered.

el = ADDLISTENER(hSource, PropName, `Eventname', Callback) adds a
listener for a property event.  Eventname must be one of the strings
`PreGet', `PostGet', `PreSet', and `PostSet'.  PropName must be either
a single property name or cell array of property names, or a single
meta.property or array of meta.property objects.  The properties must
belong to the class of hSource.  If hSource is scalar, PropName can
include dynamic properties.

For all forms, addlistener returns an event.listener.  To remove a
listener, delete the object returned by addlistener.  For example,
delete(el) calls the handle class delete method to remove the listener
and delete it from the workspace.

See also MCF, NOTIFY, DELETE, EVENT.LISTENER, META.PROPERTY, EVENTS,
DYNAMICPROPS

\end{description}
\end{quote}
\begin{description}
\item[{Help for mcf/addlistener is inherited from superclass HANDLE}] \leavevmode\begin{description}
\item[{Reference page in Help browser}] \leavevmode
doc mcf/addlistener

\end{description}

\end{description}


\bigskip\hrule{}\bigskip



\section{cdf}
\label{classes/utils/@mcf/mcf:cdf}\label{classes/utils/@mcf/mcf:id2}\begin{quote}

\textasciitilde{}\textasciitilde{} no help found
\end{quote}


\bigskip\hrule{}\bigskip



\section{cdf\_plot}
\label{classes/utils/@mcf/mcf:id3}\label{classes/utils/@mcf/mcf:cdf-plot}\begin{quote}

\textasciitilde{}\textasciitilde{} no help found
\end{quote}


\bigskip\hrule{}\bigskip



\section{correlation\_patterns\_plot}
\label{classes/utils/@mcf/mcf:correlation-patterns-plot}\label{classes/utils/@mcf/mcf:id4}\begin{quote}

\textasciitilde{}\textasciitilde{} no help found
\end{quote}


\bigskip\hrule{}\bigskip



\section{delete}
\label{classes/utils/@mcf/mcf:id5}\label{classes/utils/@mcf/mcf:delete}\begin{quote}
\begin{description}
\item[{\textbf{DELETE}   Delete a handle object.}] \leavevmode
The DELETE method deletes a handle object but does not clear the handle
from the workspace.  A deleted handle is no longer valid.

DELETE(H) deletes the handle object H, where H is a scalar handle.

See also MCF, MCF/ISVALID, CLEAR

\end{description}
\end{quote}
\begin{description}
\item[{Help for mcf/delete is inherited from superclass HANDLE}] \leavevmode\begin{description}
\item[{Reference page in Help browser}] \leavevmode
doc mcf/delete

\end{description}

\end{description}


\bigskip\hrule{}\bigskip



\section{eq}
\label{classes/utils/@mcf/mcf:id6}\label{classes/utils/@mcf/mcf:eq}\begin{description}
\item[{== (EQ)   Test handle equality.}] \leavevmode
Handles are equal if they are handles for the same object.

H1 == H2 performs element-wise comparisons between handle arrays H1 and
H2.  H1 and H2 must be of the same dimensions unless one is a scalar.
The result is a logical array of the same dimensions, where each
element is an element-wise equality result.

If one of H1 or H2 is scalar, scalar expansion is performed and the
result will match the dimensions of the array that is not scalar.

TF = EQ(H1, H2) stores the result in a logical array of the same
dimensions.

See also MCF, MCF/GE, MCF/GT, MCF/LE, MCF/LT, MCF/NE

\end{description}

Help for mcf/eq is inherited from superclass HANDLE


\bigskip\hrule{}\bigskip



\section{findobj}
\label{classes/utils/@mcf/mcf:id7}\label{classes/utils/@mcf/mcf:findobj}\begin{quote}
\begin{description}
\item[{\textbf{FINDOBJ}   Find objects matching specified conditions.}] \leavevmode
The FINDOBJ method of the HANDLE class follows the same syntax as the
MATLAB FINDOBJ command, except that the first argument must be an array
of handles to objects.

HM = FINDOBJ(H, \textless{}conditions\textgreater{}) searches the handle object array H and
returns an array of handle objects matching the specified conditions.
Only the public members of the objects of H are considered when
evaluating the conditions.

See also FINDOBJ, MCF

\end{description}
\end{quote}
\begin{description}
\item[{Help for mcf/findobj is inherited from superclass HANDLE}] \leavevmode\begin{description}
\item[{Reference page in Help browser}] \leavevmode
doc mcf/findobj

\end{description}

\end{description}


\bigskip\hrule{}\bigskip



\section{findprop}
\label{classes/utils/@mcf/mcf:findprop}\label{classes/utils/@mcf/mcf:id8}\begin{quote}
\begin{description}
\item[{\textbf{FINDPROP}   Find property of MATLAB handle object.}] \leavevmode
p = FINDPROP(H,'PROPNAME') finds and returns the META.PROPERTY object
associated with property name PROPNAME of scalar handle object H.
PROPNAME must be a string.  It can be the name of a property defined
by the class of H or a dynamic property added to scalar object H.

If no property named PROPNAME exists for object H, an empty
META.PROPERTY array is returned.

See also MCF, MCF/FINDOBJ, DYNAMICPROPS, META.PROPERTY

\end{description}
\end{quote}
\begin{description}
\item[{Help for mcf/findprop is inherited from superclass HANDLE}] \leavevmode\begin{description}
\item[{Reference page in Help browser}] \leavevmode
doc mcf/findprop

\end{description}

\end{description}


\bigskip\hrule{}\bigskip



\section{ge}
\label{classes/utils/@mcf/mcf:ge}\label{classes/utils/@mcf/mcf:id9}\begin{description}
\item[{\textgreater{}= (GE)   Greater than or equal relation for handles.}] \leavevmode
H1 \textgreater{}= H2 performs element-wise comparisons between handle arrays H1 and
H2.  H1 and H2 must be of the same dimensions unless one is a scalar.
The result is a logical array of the same dimensions, where each
element is an element-wise \textgreater{}= result.

If one of H1 or H2 is scalar, scalar expansion is performed and the
result will match the dimensions of the array that is not scalar.

TF = GE(H1, H2) stores the result in a logical array of the same
dimensions.

See also MCF, MCF/EQ, MCF/GT, MCF/LE, MCF/LT, MCF/NE

\end{description}

Help for mcf/ge is inherited from superclass HANDLE


\bigskip\hrule{}\bigskip



\section{gt}
\label{classes/utils/@mcf/mcf:id10}\label{classes/utils/@mcf/mcf:gt}\begin{description}
\item[{\textgreater{} (GT)   Greater than relation for handles.}] \leavevmode
H1 \textgreater{} H2 performs element-wise comparisons between handle arrays H1 and
H2.  H1 and H2 must be of the same dimensions unless one is a scalar.
The result is a logical array of the same dimensions, where each
element is an element-wise \textgreater{} result.

If one of H1 or H2 is scalar, scalar expansion is performed and the
result will match the dimensions of the array that is not scalar.

TF = GT(H1, H2) stores the result in a logical array of the same
dimensions.

See also MCF, MCF/EQ, MCF/GE, MCF/LE, MCF/LT, MCF/NE

\end{description}

Help for mcf/gt is inherited from superclass HANDLE


\bigskip\hrule{}\bigskip



\section{isvalid}
\label{classes/utils/@mcf/mcf:isvalid}\label{classes/utils/@mcf/mcf:id11}\begin{quote}
\begin{description}
\item[{\textbf{ISVALID}   Test handle validity.}] \leavevmode
TF = ISVALID(H) performs an element-wise check for validity on the
handle elements of H.  The result is a logical array of the same
dimensions as H, where each element is the element-wise validity
result.

A handle is invalid if it has been deleted or if it is an element
of a handle array and has not yet been initialized.

See also MCF, MCF/DELETE

\end{description}
\end{quote}
\begin{description}
\item[{Help for mcf/isvalid is inherited from superclass HANDLE}] \leavevmode\begin{description}
\item[{Reference page in Help browser}] \leavevmode
doc mcf/isvalid

\end{description}

\end{description}


\bigskip\hrule{}\bigskip



\section{kolmogorov\_smirnov\_test}
\label{classes/utils/@mcf/mcf:id12}\label{classes/utils/@mcf/mcf:kolmogorov-smirnov-test}
tests the equality of two distributions using their CDFs


\bigskip\hrule{}\bigskip



\section{le}
\label{classes/utils/@mcf/mcf:id13}\label{classes/utils/@mcf/mcf:le}\begin{description}
\item[{\textless{}= (LE)   Less than or equal relation for handles.}] \leavevmode
Handles are equal if they are handles for the same object.  All
comparisons use a number associated with each handle object.  Nothing
can be assumed about the result of a handle comparison except that the
repeated comparison of two handles in the same MATLAB session will
yield the same result.  The order of handle values is purely arbitrary
and has no connection to the state of the handle objects being
compared.

H1 \textless{}= H2 performs element-wise comparisons between handle arrays H1 and
H2.  H1 and H2 must be of the same dimensions unless one is a scalar.
The result is a logical array of the same dimensions, where each
element is an element-wise \textgreater{}= result.

If one of H1 or H2 is scalar, scalar expansion is performed and the
result will match the dimensions of the array that is not scalar.

TF = LE(H1, H2) stores the result in a logical array of the same
dimensions.

See also MCF, MCF/EQ, MCF/GE, MCF/GT, MCF/LT, MCF/NE

\end{description}

Help for mcf/le is inherited from superclass HANDLE


\bigskip\hrule{}\bigskip



\section{lt}
\label{classes/utils/@mcf/mcf:lt}\label{classes/utils/@mcf/mcf:id14}\begin{description}
\item[{\textless{} (LT)   Less than relation for handles.}] \leavevmode
H1 \textless{} H2 performs element-wise comparisons between handle arrays H1 and
H2.  H1 and H2 must be of the same dimensions unless one is a scalar.
The result is a logical array of the same dimensions, where each
element is an element-wise \textless{} result.

If one of H1 or H2 is scalar, scalar expansion is performed and the
result will match the dimensions of the array that is not scalar.

TF = LT(H1, H2) stores the result in a logical array of the same
dimensions.

See also MCF, MCF/EQ, MCF/GE, MCF/GT, MCF/LE, MCF/NE

\end{description}

Help for mcf/lt is inherited from superclass HANDLE


\bigskip\hrule{}\bigskip



\section{mcf}
\label{classes/utils/@mcf/mcf:mcf}\label{classes/utils/@mcf/mcf:id15}\begin{quote}

\textasciitilde{}\textasciitilde{} no help found
\end{quote}


\bigskip\hrule{}\bigskip



\section{ne}
\label{classes/utils/@mcf/mcf:ne}\label{classes/utils/@mcf/mcf:id16}\begin{description}
\item[{\textasciitilde{}= (NE)   Not equal relation for handles.}] \leavevmode
Handles are equal if they are handles for the same object and are
unequal otherwise.

H1 \textasciitilde{}= H2 performs element-wise comparisons between handle arrays H1
and H2.  H1 and H2 must be of the same dimensions unless one is a
scalar.  The result is a logical array of the same dimensions, where
each element is an element-wise equality result.

If one of H1 or H2 is scalar, scalar expansion is performed and the
result will match the dimensions of the array that is not scalar.

TF = NE(H1, H2) stores the result in a logical array of the same
dimensions.

See also MCF, MCF/EQ, MCF/GE, MCF/GT, MCF/LE, MCF/LT

\end{description}

Help for mcf/ne is inherited from superclass HANDLE


\bigskip\hrule{}\bigskip



\section{notify}
\label{classes/utils/@mcf/mcf:notify}\label{classes/utils/@mcf/mcf:id17}\begin{quote}
\begin{description}
\item[{\textbf{NOTIFY}   Notify listeners of event.}] \leavevmode
NOTIFY(H,'EVENTNAME') notifies listeners added to the event named
EVENTNAME on handle object array H that the event is taking place.
H is the array of handles to objects triggering the event, and
EVENTNAME must be a string.

NOTIFY(H,'EVENTNAME',DATA) provides a way of encapsulating information
about an event which can then be accessed by each registered listener.
DATA must belong to the EVENT.EVENTDATA class.

See also MCF, MCF/ADDLISTENER, EVENT.EVENTDATA, EVENTS

\end{description}
\end{quote}
\begin{description}
\item[{Help for mcf/notify is inherited from superclass HANDLE}] \leavevmode\begin{description}
\item[{Reference page in Help browser}] \leavevmode
doc mcf/notify

\end{description}

\end{description}


\bigskip\hrule{}\bigskip



\section{scatter}
\label{classes/utils/@mcf/mcf:id18}\label{classes/utils/@mcf/mcf:scatter}\begin{quote}

\textasciitilde{}\textasciitilde{} no help found
\end{quote}


\chapter{High dimensional model representation}
\label{classes/utils/@hdmr/hdmr::doc}\label{classes/utils/@hdmr/hdmr:high-dimensional-model-representation}

\section{methods}
\label{classes/utils/@hdmr/hdmr:methods}\begin{itemize}
\item {} 
{[} {\hyperref[classes/utils/@hdmr/hdmr:estimate]{estimate}} {]}(hdmr/estimate)

\item {} 
{[} {\hyperref[classes/utils/@hdmr/hdmr:first-order-effect]{first\_order\_effect}} {]}(hdmr/first\_order\_effect)

\item {} 
{[} {\hyperref[classes/utils/@hdmr/hdmr:hdmr]{hdmr}} {]}(hdmr/hdmr)

\item {} 
{[} {\hyperref[classes/utils/@hdmr/hdmr:metamodel]{metamodel}} {]}(hdmr/metamodel)

\item {} 
{[} {\hyperref[classes/utils/@hdmr/hdmr:plot-fit]{plot\_fit}} {]}(hdmr/plot\_fit)

\item {} 
{[} {\hyperref[classes/utils/@hdmr/hdmr:polynomial-evaluation]{polynomial\_evaluation}} {]}(hdmr/polynomial\_evaluation)

\item {} 
{[} {\hyperref[classes/utils/@hdmr/hdmr:polynomial-integration]{polynomial\_integration}} {]}(hdmr/polynomial\_integration)

\item {} 
{[} {\hyperref[classes/utils/@hdmr/hdmr:polynomial-multiplication]{polynomial\_multiplication}} {]}(hdmr/polynomial\_multiplication)

\end{itemize}


\section{properties}
\label{classes/utils/@hdmr/hdmr:properties}\begin{itemize}
\item {} 
{[}N{]} -

\item {} 
{[}Nobs{]} -

\item {} 
{[}n{]} -

\item {} 
{[}output\_nbr{]} -

\item {} 
{[}theta{]} -

\item {} 
{[}theta\_low{]} -

\item {} 
{[}theta\_high{]} -

\item {} 
{[}g{]} -

\item {} 
{[}x{]} -

\item {} 
{[}expansion\_order{]} -

\item {} 
{[}pol\_max\_order{]} -

\item {} 
{[}poly\_coefs{]} -

\item {} 
{[}Indices{]} -

\item {} 
{[}coefficients{]} -

\item {} 
{[}aggregate{]} -

\item {} 
{[}f0{]} -

\item {} 
{[}D{]} -

\item {} 
{[}sample\_percentage{]} -

\item {} 
{[}optimal{]} -

\item {} 
{[}param\_names{]} -

\end{itemize}


\bigskip\hrule{}\bigskip



\bigskip\hrule{}\bigskip



\section{estimate}
\label{classes/utils/@hdmr/hdmr:estimate}\label{classes/utils/@hdmr/hdmr:id1}\begin{quote}

\textasciitilde{}\textasciitilde{} no help found
\end{quote}


\bigskip\hrule{}\bigskip



\section{first\_order\_effect}
\label{classes/utils/@hdmr/hdmr:id2}\label{classes/utils/@hdmr/hdmr:first-order-effect}\begin{quote}

\textasciitilde{}\textasciitilde{} no help found
\end{quote}


\bigskip\hrule{}\bigskip



\section{hdmr}
\label{classes/utils/@hdmr/hdmr:hdmr}\label{classes/utils/@hdmr/hdmr:id3}\begin{quote}

\textasciitilde{}\textasciitilde{} no help found
\end{quote}


\bigskip\hrule{}\bigskip



\section{metamodel}
\label{classes/utils/@hdmr/hdmr:id4}\label{classes/utils/@hdmr/hdmr:metamodel}\begin{quote}

\textasciitilde{}\textasciitilde{} no help found
\end{quote}


\bigskip\hrule{}\bigskip



\section{plot\_fit}
\label{classes/utils/@hdmr/hdmr:id5}\label{classes/utils/@hdmr/hdmr:plot-fit}\begin{quote}

\textasciitilde{}\textasciitilde{} no help found
\end{quote}


\bigskip\hrule{}\bigskip



\section{polynomial\_evaluation}
\label{classes/utils/@hdmr/hdmr:id6}\label{classes/utils/@hdmr/hdmr:polynomial-evaluation}
later on, the function that normalizes could come in here so that the
normalization is done according to the hdmr\_type of polynomial chosen.


\bigskip\hrule{}\bigskip



\section{polynomial\_integration}
\label{classes/utils/@hdmr/hdmr:id7}\label{classes/utils/@hdmr/hdmr:polynomial-integration}
polynomial is of the form a0+a1*x+...+ar*x\textasciicircum{}r
the integral is then a0*x+a1/2*x\textasciicircum{}2+...+ar/(r+1)*x\textasciicircum{}(r+1)


\bigskip\hrule{}\bigskip



\section{polynomial\_multiplication}
\label{classes/utils/@hdmr/hdmr:polynomial-multiplication}\label{classes/utils/@hdmr/hdmr:id8}
each polynomial is of the form a0+a1*x+...+ar*x\textasciicircum{}r


\chapter{Contributing to RISE}
\label{contributing:contributing-to-rise}\label{contributing::doc}

\section{contributing new code}
\label{contributing:contributing-new-code}

\section{contributing by helping maintain existing code}
\label{contributing:contributing-by-helping-maintain-existing-code}

\section{other ways to contribute}
\label{contributing:other-ways-to-contribute}

\section{recommended development setup}
\label{contributing:recommended-development-setup}

\section{RISE structure}
\label{contributing:rise-structure}

\section{useful links, FAQ, checklist}
\label{contributing:useful-links-faq-checklist}

\chapter{Acknowledgements}
\label{acknowledgements:acknowledgements}\label{acknowledgements::doc}
Many people have, oftentimes unknowingly, provided help in the form of reporting bugs, making suggestions, asking challenging questions, etc.
I would like to single out a few of them but the list is far from exhaustive:
\begin{itemize}
\item {} 
Dan Waggoner

\item {} 
Doug Laxton

\item {} 
Eric Leeper

\item {} 
Jesper Linde

\item {} 
Jim Nason

\item {} 
Kjetil Olsen

\item {} 
Kostas Theodoridis

\item {} 
Leif Brubakk

\item {} 
Marco Ratto

\item {} 
Michel Juillard

\item {} 
Pablo Winnant (dolo)

\item {} 
Pelin Ilbas

\item {} 
Raf Wouters

\item {} 
Tao Zha

\end{itemize}


\chapter{Bibliography}
\label{bibliography:bibliography}\label{bibliography::doc}

\chapter{Indices and tables}
\label{master_doc:indices-and-tables}\begin{itemize}
\item {} 
\emph{genindex}

\item {} 
\emph{modindex}

\item {} 
\emph{search}

\end{itemize}



\renewcommand{\indexname}{Index}
\printindex
\end{document}
