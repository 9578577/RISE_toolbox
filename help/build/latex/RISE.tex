% Generated by Sphinx.
\def\sphinxdocclass{report}
\documentclass[letterpaper,10pt,english]{sphinxmanual}
\usepackage[utf8]{inputenc}
\DeclareUnicodeCharacter{00A0}{\nobreakspace}
\usepackage[T1]{fontenc}
\usepackage{babel}
\usepackage{times}
\usepackage[Bjarne]{fncychap}
\usepackage{longtable}
\usepackage{sphinx}
\usepackage{multirow}


\title{RISE Documentation}
\date{October 12, 2014}
\release{1.0.0}
\author{Junior Maih}
\newcommand{\sphinxlogo}{}
\renewcommand{\releasename}{Release}
\makeindex

\makeatletter
\def\PYG@reset{\let\PYG@it=\relax \let\PYG@bf=\relax%
    \let\PYG@ul=\relax \let\PYG@tc=\relax%
    \let\PYG@bc=\relax \let\PYG@ff=\relax}
\def\PYG@tok#1{\csname PYG@tok@#1\endcsname}
\def\PYG@toks#1+{\ifx\relax#1\empty\else%
    \PYG@tok{#1}\expandafter\PYG@toks\fi}
\def\PYG@do#1{\PYG@bc{\PYG@tc{\PYG@ul{%
    \PYG@it{\PYG@bf{\PYG@ff{#1}}}}}}}
\def\PYG#1#2{\PYG@reset\PYG@toks#1+\relax+\PYG@do{#2}}

\expandafter\def\csname PYG@tok@gd\endcsname{\def\PYG@tc##1{\textcolor[rgb]{0.63,0.00,0.00}{##1}}}
\expandafter\def\csname PYG@tok@gu\endcsname{\let\PYG@bf=\textbf\def\PYG@tc##1{\textcolor[rgb]{0.50,0.00,0.50}{##1}}}
\expandafter\def\csname PYG@tok@gt\endcsname{\def\PYG@tc##1{\textcolor[rgb]{0.00,0.27,0.87}{##1}}}
\expandafter\def\csname PYG@tok@gs\endcsname{\let\PYG@bf=\textbf}
\expandafter\def\csname PYG@tok@gr\endcsname{\def\PYG@tc##1{\textcolor[rgb]{1.00,0.00,0.00}{##1}}}
\expandafter\def\csname PYG@tok@cm\endcsname{\let\PYG@it=\textit\def\PYG@tc##1{\textcolor[rgb]{0.25,0.50,0.56}{##1}}}
\expandafter\def\csname PYG@tok@vg\endcsname{\def\PYG@tc##1{\textcolor[rgb]{0.73,0.38,0.84}{##1}}}
\expandafter\def\csname PYG@tok@m\endcsname{\def\PYG@tc##1{\textcolor[rgb]{0.13,0.50,0.31}{##1}}}
\expandafter\def\csname PYG@tok@mh\endcsname{\def\PYG@tc##1{\textcolor[rgb]{0.13,0.50,0.31}{##1}}}
\expandafter\def\csname PYG@tok@cs\endcsname{\def\PYG@tc##1{\textcolor[rgb]{0.25,0.50,0.56}{##1}}\def\PYG@bc##1{\setlength{\fboxsep}{0pt}\colorbox[rgb]{1.00,0.94,0.94}{\strut ##1}}}
\expandafter\def\csname PYG@tok@ge\endcsname{\let\PYG@it=\textit}
\expandafter\def\csname PYG@tok@vc\endcsname{\def\PYG@tc##1{\textcolor[rgb]{0.73,0.38,0.84}{##1}}}
\expandafter\def\csname PYG@tok@il\endcsname{\def\PYG@tc##1{\textcolor[rgb]{0.13,0.50,0.31}{##1}}}
\expandafter\def\csname PYG@tok@go\endcsname{\def\PYG@tc##1{\textcolor[rgb]{0.20,0.20,0.20}{##1}}}
\expandafter\def\csname PYG@tok@cp\endcsname{\def\PYG@tc##1{\textcolor[rgb]{0.00,0.44,0.13}{##1}}}
\expandafter\def\csname PYG@tok@gi\endcsname{\def\PYG@tc##1{\textcolor[rgb]{0.00,0.63,0.00}{##1}}}
\expandafter\def\csname PYG@tok@gh\endcsname{\let\PYG@bf=\textbf\def\PYG@tc##1{\textcolor[rgb]{0.00,0.00,0.50}{##1}}}
\expandafter\def\csname PYG@tok@ni\endcsname{\let\PYG@bf=\textbf\def\PYG@tc##1{\textcolor[rgb]{0.84,0.33,0.22}{##1}}}
\expandafter\def\csname PYG@tok@nl\endcsname{\let\PYG@bf=\textbf\def\PYG@tc##1{\textcolor[rgb]{0.00,0.13,0.44}{##1}}}
\expandafter\def\csname PYG@tok@nn\endcsname{\let\PYG@bf=\textbf\def\PYG@tc##1{\textcolor[rgb]{0.05,0.52,0.71}{##1}}}
\expandafter\def\csname PYG@tok@no\endcsname{\def\PYG@tc##1{\textcolor[rgb]{0.38,0.68,0.84}{##1}}}
\expandafter\def\csname PYG@tok@na\endcsname{\def\PYG@tc##1{\textcolor[rgb]{0.25,0.44,0.63}{##1}}}
\expandafter\def\csname PYG@tok@nb\endcsname{\def\PYG@tc##1{\textcolor[rgb]{0.00,0.44,0.13}{##1}}}
\expandafter\def\csname PYG@tok@nc\endcsname{\let\PYG@bf=\textbf\def\PYG@tc##1{\textcolor[rgb]{0.05,0.52,0.71}{##1}}}
\expandafter\def\csname PYG@tok@nd\endcsname{\let\PYG@bf=\textbf\def\PYG@tc##1{\textcolor[rgb]{0.33,0.33,0.33}{##1}}}
\expandafter\def\csname PYG@tok@ne\endcsname{\def\PYG@tc##1{\textcolor[rgb]{0.00,0.44,0.13}{##1}}}
\expandafter\def\csname PYG@tok@nf\endcsname{\def\PYG@tc##1{\textcolor[rgb]{0.02,0.16,0.49}{##1}}}
\expandafter\def\csname PYG@tok@si\endcsname{\let\PYG@it=\textit\def\PYG@tc##1{\textcolor[rgb]{0.44,0.63,0.82}{##1}}}
\expandafter\def\csname PYG@tok@s2\endcsname{\def\PYG@tc##1{\textcolor[rgb]{0.25,0.44,0.63}{##1}}}
\expandafter\def\csname PYG@tok@vi\endcsname{\def\PYG@tc##1{\textcolor[rgb]{0.73,0.38,0.84}{##1}}}
\expandafter\def\csname PYG@tok@nt\endcsname{\let\PYG@bf=\textbf\def\PYG@tc##1{\textcolor[rgb]{0.02,0.16,0.45}{##1}}}
\expandafter\def\csname PYG@tok@nv\endcsname{\def\PYG@tc##1{\textcolor[rgb]{0.73,0.38,0.84}{##1}}}
\expandafter\def\csname PYG@tok@s1\endcsname{\def\PYG@tc##1{\textcolor[rgb]{0.25,0.44,0.63}{##1}}}
\expandafter\def\csname PYG@tok@gp\endcsname{\let\PYG@bf=\textbf\def\PYG@tc##1{\textcolor[rgb]{0.78,0.36,0.04}{##1}}}
\expandafter\def\csname PYG@tok@sh\endcsname{\def\PYG@tc##1{\textcolor[rgb]{0.25,0.44,0.63}{##1}}}
\expandafter\def\csname PYG@tok@ow\endcsname{\let\PYG@bf=\textbf\def\PYG@tc##1{\textcolor[rgb]{0.00,0.44,0.13}{##1}}}
\expandafter\def\csname PYG@tok@sx\endcsname{\def\PYG@tc##1{\textcolor[rgb]{0.78,0.36,0.04}{##1}}}
\expandafter\def\csname PYG@tok@bp\endcsname{\def\PYG@tc##1{\textcolor[rgb]{0.00,0.44,0.13}{##1}}}
\expandafter\def\csname PYG@tok@c1\endcsname{\let\PYG@it=\textit\def\PYG@tc##1{\textcolor[rgb]{0.25,0.50,0.56}{##1}}}
\expandafter\def\csname PYG@tok@kc\endcsname{\let\PYG@bf=\textbf\def\PYG@tc##1{\textcolor[rgb]{0.00,0.44,0.13}{##1}}}
\expandafter\def\csname PYG@tok@c\endcsname{\let\PYG@it=\textit\def\PYG@tc##1{\textcolor[rgb]{0.25,0.50,0.56}{##1}}}
\expandafter\def\csname PYG@tok@mf\endcsname{\def\PYG@tc##1{\textcolor[rgb]{0.13,0.50,0.31}{##1}}}
\expandafter\def\csname PYG@tok@err\endcsname{\def\PYG@bc##1{\setlength{\fboxsep}{0pt}\fcolorbox[rgb]{1.00,0.00,0.00}{1,1,1}{\strut ##1}}}
\expandafter\def\csname PYG@tok@kd\endcsname{\let\PYG@bf=\textbf\def\PYG@tc##1{\textcolor[rgb]{0.00,0.44,0.13}{##1}}}
\expandafter\def\csname PYG@tok@ss\endcsname{\def\PYG@tc##1{\textcolor[rgb]{0.32,0.47,0.09}{##1}}}
\expandafter\def\csname PYG@tok@sr\endcsname{\def\PYG@tc##1{\textcolor[rgb]{0.14,0.33,0.53}{##1}}}
\expandafter\def\csname PYG@tok@mo\endcsname{\def\PYG@tc##1{\textcolor[rgb]{0.13,0.50,0.31}{##1}}}
\expandafter\def\csname PYG@tok@mi\endcsname{\def\PYG@tc##1{\textcolor[rgb]{0.13,0.50,0.31}{##1}}}
\expandafter\def\csname PYG@tok@kn\endcsname{\let\PYG@bf=\textbf\def\PYG@tc##1{\textcolor[rgb]{0.00,0.44,0.13}{##1}}}
\expandafter\def\csname PYG@tok@o\endcsname{\def\PYG@tc##1{\textcolor[rgb]{0.40,0.40,0.40}{##1}}}
\expandafter\def\csname PYG@tok@kr\endcsname{\let\PYG@bf=\textbf\def\PYG@tc##1{\textcolor[rgb]{0.00,0.44,0.13}{##1}}}
\expandafter\def\csname PYG@tok@s\endcsname{\def\PYG@tc##1{\textcolor[rgb]{0.25,0.44,0.63}{##1}}}
\expandafter\def\csname PYG@tok@kp\endcsname{\def\PYG@tc##1{\textcolor[rgb]{0.00,0.44,0.13}{##1}}}
\expandafter\def\csname PYG@tok@w\endcsname{\def\PYG@tc##1{\textcolor[rgb]{0.73,0.73,0.73}{##1}}}
\expandafter\def\csname PYG@tok@kt\endcsname{\def\PYG@tc##1{\textcolor[rgb]{0.56,0.13,0.00}{##1}}}
\expandafter\def\csname PYG@tok@sc\endcsname{\def\PYG@tc##1{\textcolor[rgb]{0.25,0.44,0.63}{##1}}}
\expandafter\def\csname PYG@tok@sb\endcsname{\def\PYG@tc##1{\textcolor[rgb]{0.25,0.44,0.63}{##1}}}
\expandafter\def\csname PYG@tok@k\endcsname{\let\PYG@bf=\textbf\def\PYG@tc##1{\textcolor[rgb]{0.00,0.44,0.13}{##1}}}
\expandafter\def\csname PYG@tok@se\endcsname{\let\PYG@bf=\textbf\def\PYG@tc##1{\textcolor[rgb]{0.25,0.44,0.63}{##1}}}
\expandafter\def\csname PYG@tok@sd\endcsname{\let\PYG@it=\textit\def\PYG@tc##1{\textcolor[rgb]{0.25,0.44,0.63}{##1}}}

\def\PYGZbs{\char`\\}
\def\PYGZus{\char`\_}
\def\PYGZob{\char`\{}
\def\PYGZcb{\char`\}}
\def\PYGZca{\char`\^}
\def\PYGZam{\char`\&}
\def\PYGZlt{\char`\<}
\def\PYGZgt{\char`\>}
\def\PYGZsh{\char`\#}
\def\PYGZpc{\char`\%}
\def\PYGZdl{\char`\$}
\def\PYGZhy{\char`\-}
\def\PYGZsq{\char`\'}
\def\PYGZdq{\char`\"}
\def\PYGZti{\char`\~}
% for compatibility with earlier versions
\def\PYGZat{@}
\def\PYGZlb{[}
\def\PYGZrb{]}
\makeatother

\begin{document}

\maketitle
\tableofcontents
\phantomsection\label{master_doc::doc}



\chapter{Introduction}
\label{introduction:introduction}\label{introduction::doc}\label{introduction:welcome-to-rise-s-documentation}

\section{RISE at a Glance}
\label{intro_folder/rise_at_a_glance::doc}\label{intro_folder/rise_at_a_glance:rise-at-a-glance}

\subsection{What is RISE?}
\label{intro_folder/rise_at_a_glance:what-is-rise}
RISE is the acronym for \textbf{R}ationality \textbf{I}n \textbf{S}witching \textbf{E}nvironments.

It is an object-oriented Matlab toolbox primarily designed for solving and estimating nonlinear
dynamic stochastic general equilibirium (\textbf{DSGE}) or more generally
Rational Expectations(\textbf{RE}) models with \textbf{switching parameters}.

Leading references in the field include various papers by \href{http://www.tzha.net/articles}{Roger Farmer, Dan Waggoner and Tao Zha}
and \href{http://php.indiana.edu/~eleeper/\#Papers}{Eric Leeper} among others.

RISE uses perturbation to approximate the nonlinear Markov Switching Rational
Expectations (\textbf{MSRE}) model and solves it using efficient algorithms.

RISE also implements special cases of the general Switching MSRE model. This includes
\begin{itemize}
\item {} 
\textbf{VAR}s with and without switching parameters

\item {} 
\textbf{SVAR}s with and without switching paramters

\item {} 
\textbf{Time-varying parameter VAR}s

\item {} 
etc.

\end{itemize}


\subsection{Motivation for RISE development}
\label{intro_folder/rise_at_a_glance:motivation-for-rise-development}\begin{itemize}
\item {} 
The world is not constant, it is switching

\end{itemize}


\section{Capabilities of RISE}
\label{intro_folder/rise_capabilities::doc}\label{intro_folder/rise_capabilities:capabilities-of-rise}

\subsection{DSGE modeling}
\label{intro_folder/rise_capabilities:dsge-modeling}\begin{itemize}
\item {} 
constant parameters

\item {} \begin{description}
\item[{switching parameters}] \leavevmode\begin{itemize}
\item {} 
exogenous switching

\item {} 
endogenous switching

\end{itemize}

\end{description}

\item {} \begin{description}
\item[{optimal policy (with and without switching)}] \leavevmode\begin{itemize}
\item {} 
discretion

\item {} 
commitment

\item {} 
loose commitment

\item {} 
optimized simple rules

\end{itemize}

\end{description}

\item {} 
Deterministic simulation

\item {} 
Stochastic simulation

\item {} 
higher-order perturbations

\end{itemize}


\subsection{VAR modeling}
\label{intro_folder/rise_capabilities:var-modeling}\begin{itemize}
\item {} \begin{description}
\item[{constant parameters}] \leavevmode\begin{itemize}
\item {} 
zero restrictions

\item {} 
sign restrictions

\item {} 
restrictions on lag structure

\item {} 
linear restrictions

\end{itemize}

\end{description}

\item {} \begin{description}
\item[{switching parameters}] \leavevmode\begin{itemize}
\item {} 
linear restrictions

\end{itemize}

\end{description}

\end{itemize}


\subsection{SVAR modeling}
\label{intro_folder/rise_capabilities:svar-modeling}\begin{itemize}
\item {} 
constant parameters

\item {} \begin{description}
\item[{switching parameters}] \leavevmode\begin{itemize}
\item {} 
linear restrictions

\end{itemize}

\end{description}

\end{itemize}


\subsection{Time-Varying parameter VAR modeling}
\label{intro_folder/rise_capabilities:time-varying-parameter-var-modeling}
Under implementation


\subsection{Smooth transition VAR modeling}
\label{intro_folder/rise_capabilities:smooth-transition-var-modeling}
Not yet implemented


\subsection{Forecasting and Conditional Forecasting}
\label{intro_folder/rise_capabilities:forecasting-and-conditional-forecasting}

\subsection{Global sensitivity analysis}
\label{intro_folder/rise_capabilities:global-sensitivity-analysis}\begin{itemize}
\item {} 
Monte carlo filtering

\item {} 
High dimensional model representation

\end{itemize}


\subsection{Maximum Likelihood and Bayesian Estimation}
\label{intro_folder/rise_capabilities:maximum-likelihood-and-bayesian-estimation}\begin{itemize}
\item {} 
linear restrictions

\item {} 
nonlinear restrictions

\end{itemize}


\subsection{Time series}
\label{intro_folder/rise_capabilities:time-series}

\subsection{Reporting}
\label{intro_folder/rise_capabilities:reporting}

\section{How RISE works}
\label{intro_folder/how_rise_works:how-rise-works}\label{intro_folder/how_rise_works::doc}

\subsection{Object orientation}
\label{intro_folder/how_rise_works:object-orientation}

\subsection{Basic principles}
\label{intro_folder/how_rise_works:basic-principles}\begin{itemize}
\item {} 
you can pass different options at any time

\end{itemize}


\section{Background and mathematical formulations}
\label{intro_folder/background::doc}\label{intro_folder/background:background-and-mathematical-formulations}

\section{Using this documentation}
\label{intro_folder/using_this_doc:using-this-documentation}\label{intro_folder/using_this_doc::doc}

\subsection{how to find help}
\label{intro_folder/using_this_doc:how-to-find-help}

\subsection{Road map}
\label{intro_folder/using_this_doc:road-map}

\section{Citing RISE in your research}
\label{intro_folder/citing_rise:citing-rise-in-your-research}\label{intro_folder/citing_rise::doc}

\chapter{Getting started with RISE}
\label{getting_started:getting-started-with-rise}\label{getting_started::doc}

\section{Installation guide}
\label{getting_started_folder/installation_configuration::doc}\label{getting_started_folder/installation_configuration:installation-guide}

\subsection{Software requirements}
\label{getting_started_folder/installation_configuration:software-requirements}
I order to use RISE, the following software will need to be installed:
\begin{itemize}
\item {} 
Matlab version ? or higher

\item {} 
MikTex (Windows users) MacTex (mac users)

\end{itemize}


\subsection{How to obtain RISE}
\label{getting_started_folder/installation_configuration:how-to-obtain-rise}
There are (at least) two ways to acquire RISE:


\subsubsection{The zip file option}
\label{getting_started_folder/installation_configuration:the-zip-file-option}\begin{enumerate}
\item {} 
Go online to \href{https://github.com/jmaih/RISE\_toolbox}{https://github.com/jmaih/RISE\_toolbox}

\item {} 
download the zip file and unzip it in some directory on your
computer.

\end{enumerate}

This option is not recommended but is convenient for people
who are not allowed to install new software on their
machines/laptop.


\subsubsection{Github for the bleeding-edge installation (highly recommended)}
\label{getting_started_folder/installation_configuration:github-for-the-bleeding-edge-installation-highly-recommended}\begin{enumerate}
\item {} 
Go to \href{http://windows.github.com}{http://windows.github.com} if you are a windows user or
to \href{http://mac.github.com}{http://mac.github.com} if you are a mac user

\item {} 
Create an account online through the website and download
the Github program

\item {} 
Sign in both online and on the github on your machine. It is
obvious online, but on your machine, just go to
Github\textgreater{}Preference\textgreater{}Account

\item {} 
Go online to \href{https://github.com/jmaih/RISE\_toolbox}{https://github.com/jmaih/RISE\_toolbox}

\item {} 
Look for an icon with title ’Clone in Desktop’ (or possibly
clone in mac). There are options to locate where the
repository will reside

\end{enumerate}

The reason why this option is recommended is that you don't need to
re-download the whole toolbox every time a marginal update is made.
With one click and within seconds you can have the version of the toolbox
on your computer updated.


\subsubsection{The git option (never tested!!!)}
\label{getting_started_folder/installation_configuration:the-git-option-never-tested}
The following has never been tested and so the syntax might be wrong:

\begin{Verbatim}[commandchars=\\\{\}]
git clone https://github.com/jmaih/RISE\_toolbox.git
\end{Verbatim}


\subsubsection{Testing your installation}
\label{getting_started_folder/installation_configuration:testing-your-installation}
More on this later...


\subsection{Loading and starting RISE}
\label{getting_started_folder/installation_configuration:loading-and-starting-rise}\begin{enumerate}
\item {} 
Locate the RISE\_toolbox directory and add its path to matlab
in the command window as

\begin{Verbatim}[commandchars=\\\{\}]
addpath(’C:/Users/JMaih/GithubRepositories/RISE\_toolbox’)
\end{Verbatim}

\item {} 
You will need to adapt this path to conform with the location
of the toolbox on your machine.

\item {} 
run rise\_startup()

\end{enumerate}


\subsection{Updating RISE}
\label{getting_started_folder/installation_configuration:updating-rise}
New features are constantly added, efficiency is improved, users sometimes report bugs that are corrected.
All this makes it necessary to update RISE every now and then in order to keep abreast of the
latest changes and developments.

However, updating RISE depends on precisely how you installed it in the first place:
\begin{itemize}
\item {} 
If you downloaded a zip file, you will have to redownload a zip file even if the recent change was just an added comma.

\item {} 
if instead you invested in opening a github account, with one click you will be able to update just the changes you don't have.

\item {} 
with git, you would just execute the command

\begin{Verbatim}[commandchars=\\\{\}]
git pull
\end{Verbatim}

\end{itemize}


\section{Troubleshooting}
\label{getting_started_folder/troubleshooting::doc}\label{getting_started_folder/troubleshooting:troubleshooting}

\section{RISE basics/basic principles}
\label{getting_started_folder/basic_principles:rise-basics-basic-principles}\label{getting_started_folder/basic_principles::doc}\begin{enumerate}
\item {} 
create an empty RISE object e.g.

\begin{Verbatim}[commandchars=\\\{\}]
\PYG{n}{tao}\PYG{o}{=}\PYG{n}{rise}\PYG{o}{.}\PYG{n}{empty}\PYG{p}{(}\PYG{l+m+mi}{0}\PYG{p}{)}\PYG{p}{;}
\end{Verbatim}

\item {} 
run methods(rise) or methods(tao) to see the
functions/methods that can be applied to a RISE object

\item {} 
run those methods on r''. e.g. ``irf(r)'', simulate(r)'', solve(r)'',
etc. this will give you the default options of each method and
tell you how you can modify the behavior of the method

\end{enumerate}


\section{Tutorial: A toy example}
\label{getting_started_folder/tutorial:tutorial-a-toy-example}\label{getting_started_folder/tutorial::doc}

\subsection{Foerster, Rubio-Ramirez, Waggoner and Zha (2014)}
\label{getting_started_folder/tutorial:foerster-rubio-ramirez-waggoner-and-zha-2014}
They consider the following model:
\begin{gather}
\begin{split}E_{t}\left[
\begin{array}{c}
1-\beta\frac{\left( 1-\frac{\kappa}{2}\left( \Pi_{t}-1\right) ^{2}\right)
Y_{t}}{\left( 1-\frac{\kappa}{2}\left( \Pi_{t+1}-1\right) ^{2}\right) Y_{t+1}%
}\frac{1}{e^{\mu_{t+1}}}\frac{R_{t}}{\Pi_{t+1}} \\
\left( 1-\eta\right) +\eta\left( 1-\frac{\kappa}{2}\left( \Pi _{t}-1\right)
^{2}\right) Y_{t}+\beta\kappa\frac{\left( 1-\frac{\kappa}{2}\left(
\Pi_{t}-1\right) ^{2}\right) }{\left( 1-\frac{\kappa}{2}\left(
\Pi_{t+1}-1\right) ^{2}\right) }\left( \Pi_{t+1}-1\right)
\Pi_{t+1}-\kappa\left( \Pi_{t}-1\right) \Pi_{t} \\
\left( \frac{R_{t-1}}{R_{ss}}\right) ^{\rho}\Pi_{t}^{\left( 1-\rho\right)
\psi}\exp\left( \sigma\varepsilon_{t}\right) -\frac{R_{t}}{R_{ss}}%
\end{array}
\right] =0\end{split}\notag\\\begin{split}with\end{split}\notag\\\begin{split}\mu_{t+1}=\bar{\mu}+\sigma\hat{\mu}_{t+1}.\end{split}\notag
\end{gather}
The first equation is an Euler equation, the second equation a Phillips
curve and the third equation a nonlinear Taylor rule.

The switching parameters are $\mu$ and  $\psi$.


\subsection{The RISE code}
\label{getting_started_folder/tutorial:the-rise-code}
The RISE code with parameterization is given by

\begin{Verbatim}[commandchars=\\\{\}]
endogenous PAI,Y,R

exogenous EPS\_R

parameters a\_tp\_1\_2, a\_tp\_2\_1, betta, eta, kappa, mu, mu\_bar, psi, rhor, sigr
parameters(a,2) mu, psi

model
        1-betta*(1-.5*kappa*(PAI-1)\textasciicircum{}2)*Y*R/((1-.5*kappa*(PAI(+1)-1)\textasciicircum{}2)*Y(+1)*exp(mu)*PAI(+1));

        1-eta+eta*(1-.5*kappa*(PAI-1)\textasciicircum{}2)*Y+betta*kappa*(1-.5*kappa*(PAI-1)\textasciicircum{}2)*(PAI(+1)-1)*PAI(+1)/(1-.5*kappa*(PAI(+1)-1)\textasciicircum{}2)
        -kappa*(PAI-1)*PAI;

        (R(-1)/steady\_state(R))\textasciicircum{}rhor*(PAI/steady\_state(PAI))\textasciicircum{}((1-rhor)*psi)*exp(sigr*EPS\_R)-R/steady\_state(R);


steady\_state\_model(unique,imposed)
    PAI=1;
    Y=(eta-1)/eta;
    R=exp(mu\_bar)/betta*PAI;


parameterization
        a\_tp\_1\_2,1-.9;
        a\_tp\_2\_1,1-.9;
        betta, .99;
        kappa, 161;
        eta, 10;
        rhor, .8;
        sigr, 0.0025;
        mu\_bar,0.02;
        mu(a,1), 0.03;
        mu(a,2), 0.01;
        psi(a,1), 3.1;
        psi(a,2), 0.9;
\end{Verbatim}


\subsection{Running the example}
\label{getting_started_folder/tutorial:running-the-example}
Assume this example is saved in a file named frwz\_nk.rs . The to run this example in Matlab, we run the following commands:

\begin{Verbatim}[commandchars=\\\{\}]
frwz=rise('frwz\_nk'); \% load the model and its parameterization

frwz=solve(frwz); \% Solving the model

print\_solution(frwz) \% print the solution
\end{Verbatim}


\section{How to find help?}
\label{getting_started_folder/howto_find_doc:how-to-find-help}\label{getting_started_folder/howto_find_doc::doc}

\section{Where to go from here}
\label{getting_started_folder/where_to_go_now::doc}\label{getting_started_folder/where_to_go_now:where-to-go-from-here}

\chapter{RISE Capabilities}
\label{capabilities:rise-capabilities}\label{capabilities::doc}

\section{Overview}
\label{capabilities:overview}

\section{Markov switching DSGE modeling}
\label{capabilities:markov-switching-dsge-modeling}

\section{Markov switching SVAR modeling}
\label{capabilities:markov-switching-svar-modeling}

\section{Markov switching VAR modeling}
\label{capabilities:markov-switching-var-modeling}

\section{Smooth transition VAR modeling}
\label{capabilities:smooth-transition-var-modeling}

\section{Time-varying parameter modeling}
\label{capabilities:time-varying-parameter-modeling}

\section{Maximum Likelihood and Bayesian Estimation}
\label{capabilities:maximum-likelihood-and-bayesian-estimation}

\section{Differentiation}
\label{capabilities:differentiation}

\subsection{numerical differentiation}
\label{capabilities:numerical-differentiation}

\subsection{Symbolic differentiation}
\label{capabilities:symbolic-differentiation}

\subsection{Automatic/Algorithmic differentiation}
\label{capabilities:automatic-algorithmic-differentiation}

\section{Time series}
\label{capabilities:time-series}

\section{Reporting}
\label{capabilities:reporting}

\section{Derivative-free optimization}
\label{capabilities:derivative-free-optimization}

\section{Global sensitivity analysis}
\label{capabilities:global-sensitivity-analysis}

\subsection{Monte Carlo filtering}
\label{capabilities:monte-carlo-filtering}

\subsection{High dimensional model representation}
\label{capabilities:high-dimensional-model-representation}

\chapter{The Markov switching DSGE interface}
\label{dsge_interface:the-markov-switching-dsge-interface}\label{dsge_interface::doc}

\section{The general framework}
\label{dsge_interface:the-general-framework}
The general form of the models is:
\begin{gather}
\begin{split}E_{t}\sum_{r_{t+1}=1}^{h}\pi _{r_{t},r_{t+1}}\left( I_{t}\right) \tilde{d}_{r_{t}}\left(b_{t+1}
\left( r_{t+1}\right),b_{t}\left( r_{t}\right),b_{t-1},\varepsilon _{t}, \theta _{r_{t+1}}\right) =0\end{split}\notag
\end{gather}\begin{itemize}
\item {} 
The switching of the parameters is governed by Markov processes and can be endogenous.

\item {} 
\href{http://www.kansascityfed.org/publicat/events/research/2010CenBankForecasting/Maih\_paper.pdf}{Agents can have information about future events}

\end{itemize}


\section{The model file}
\label{dsge_interface:the-model-file}

\subsection{Conventions}
\label{dsge_interface:conventions}

\subsection{Variable declarations}
\label{dsge_interface:variable-declarations}

\subsection{Expressions}
\label{dsge_interface:expressions}\begin{itemize}
\item {} \begin{description}
\item[{parameters and variables}] \leavevmode\begin{itemize}
\item {} 
inside the model

\item {} 
outside the model

\end{itemize}

\end{description}

\item {} 
operators

\item {} \begin{description}
\item[{functions}] \leavevmode\begin{itemize}
\item {} 
built-in functions

\item {} 
external/user-defined functions

\end{itemize}

\end{description}

\end{itemize}


\subsection{model declaration}
\label{dsge_interface:model-declaration}\begin{itemize}
\item {} 
model equations

\item {} 
endogenous transition probabilities

\item {} 
auxiliary parameters/variables

\item {} 
inequality restrictions

\end{itemize}


\subsection{auxiliary variables}
\label{dsge_interface:auxiliary-variables}

\subsection{initial and terminal conditions}
\label{dsge_interface:initial-and-terminal-conditions}

\subsection{shocks on exogenous variables}
\label{dsge_interface:shocks-on-exogenous-variables}

\subsection{other general declarations}
\label{dsge_interface:other-general-declarations}

\section{steady state}
\label{dsge_interface:steady-state}\begin{itemize}
\item {} 
finding the steady state with the RISE nonlinear solver

\item {} 
using a steady state file

\item {} 
using the steady state model

\end{itemize}


\section{getting information about the model}
\label{dsge_interface:getting-information-about-the-model}

\section{deterministic simulation}
\label{dsge_interface:deterministic-simulation}

\section{stochastic solution and simulation}
\label{dsge_interface:stochastic-solution-and-simulation}\begin{itemize}
\item {} 
computing the stochastic solution

\item {} 
typology and ordering of variables

\item {} 
first-order approximation

\item {} 
second-order approximation

\item {} 
third-order approximation

\item {} 
fourth-order approximation

\item {} 
fifth-order approximation

\end{itemize}


\section{Estimation}
\label{dsge_interface:estimation}

\section{Forecasting and conditional forecasting}
\label{dsge_interface:forecasting-and-conditional-forecasting}

\section{Optimal policy}
\label{dsge_interface:optimal-policy}\begin{itemize}
\item {} 
optimal simple rules

\item {} 
Commitment, discretion and loose commitment

\end{itemize}


\chapter{Markov Switching Dynamic Stochastic General Equilibrium Modeling}
\label{classes/models/@dsge/dsge:markov-switching-dynamic-stochastic-general-equilibrium-modeling}\label{classes/models/@dsge/dsge::doc}

\section{methods}
\label{classes/models/@dsge/dsge:methods}\begin{itemize}
\item {} 
{[} {\hyperref[classes/models/@dsge/dsge:check-derivatives]{check\_derivatives}} {]}(dsge/check\_derivatives)

\item {} 
{[} {\hyperref[classes/models/@dsge/dsge:check-optimum]{check\_optimum}} {]}(dsge/check\_optimum)

\item {} 
{[} {\hyperref[classes/models/@dsge/dsge:compute-steady-state]{compute\_steady\_state}} {]}(dsge/compute\_steady\_state)

\item {} 
{[} {\hyperref[classes/models/@dsge/dsge:create-estimation-blocks]{create\_estimation\_blocks}} {]}(dsge/create\_estimation\_blocks)

\item {} 
{[} {\hyperref[classes/models/@dsge/dsge:draw-parameter]{draw\_parameter}} {]}(dsge/draw\_parameter)

\item {} 
{[} {\hyperref[classes/models/@dsge/dsge:dsge]{dsge}} {]}(dsge/dsge)

\item {} 
{[} {\hyperref[classes/models/@dsge/dsge:estimate]{estimate}} {]}(dsge/estimate)

\item {} 
{[} {\hyperref[classes/models/@dsge/dsge:filter]{filter}} {]}(dsge/filter)

\item {} 
{[} {\hyperref[classes/models/@dsge/dsge:forecast]{forecast}} {]}(dsge/forecast)

\item {} 
{[} {\hyperref[classes/models/@dsge/dsge:forecast-real-time]{forecast\_real\_time}} {]}(dsge/forecast\_real\_time)

\item {} 
{[} {\hyperref[classes/models/@dsge/dsge:get]{get}} {]}(dsge/get)

\item {} 
{[} {\hyperref[classes/models/@dsge/dsge:historical-decomposition]{historical\_decomposition}} {]}(dsge/historical\_decomposition)

\item {} 
{[} {\hyperref[classes/models/@dsge/dsge:irf]{irf}} {]}(dsge/irf)

\item {} 
{[} {\hyperref[classes/models/@dsge/dsge:is-stable-system]{is\_stable\_system}} {]}(dsge/is\_stable\_system)

\item {} 
{[} {\hyperref[classes/models/@dsge/dsge:isnan]{isnan}} {]}(dsge/isnan)

\item {} 
{[} {\hyperref[classes/models/@dsge/dsge:load-parameters]{load\_parameters}} {]}(dsge/load\_parameters)

\item {} 
{[} {\hyperref[classes/models/@dsge/dsge:log-marginal-data-density]{log\_marginal\_data\_density}} {]}(dsge/log\_marginal\_data\_density)

\item {} 
{[} {\hyperref[classes/models/@dsge/dsge:log-posterior-kernel]{log\_posterior\_kernel}} {]}(dsge/log\_posterior\_kernel)

\item {} 
{[} {\hyperref[classes/models/@dsge/dsge:log-prior-density]{log\_prior\_density}} {]}(dsge/log\_prior\_density)

\item {} 
{[} {\hyperref[classes/models/@dsge/dsge:monte-carlo-filtering]{monte\_carlo\_filtering}} {]}(dsge/monte\_carlo\_filtering)

\item {} 
{[} {\hyperref[classes/models/@dsge/dsge:posterior-marginal-and-prior-densities]{posterior\_marginal\_and\_prior\_densities}} {]}(dsge/posterior\_marginal\_and\_prior\_densities)

\item {} 
{[} {\hyperref[classes/models/@dsge/dsge:posterior-simulator]{posterior\_simulator}} {]}(dsge/posterior\_simulator)

\item {} 
{[} {\hyperref[classes/models/@dsge/dsge:print-estimation-results]{print\_estimation\_results}} {]}(dsge/print\_estimation\_results)

\item {} 
{[} {\hyperref[classes/models/@dsge/dsge:print-solution]{print\_solution}} {]}(dsge/print\_solution)

\item {} 
{[} {\hyperref[classes/models/@dsge/dsge:prior-plots]{prior\_plots}} {]}(dsge/prior\_plots)

\item {} 
{[} {\hyperref[classes/models/@dsge/dsge:report]{report}} {]}(dsge/report)

\item {} 
{[} {\hyperref[classes/models/@dsge/dsge:resid]{resid}} {]}(dsge/resid)

\item {} 
{[} {\hyperref[classes/models/@dsge/dsge:set]{set}} {]}(dsge/set)

\item {} 
{[} {\hyperref[classes/models/@dsge/dsge:set-solution-to-companion]{set\_solution\_to\_companion}} {]}(dsge/set\_solution\_to\_companion)

\item {} 
{[} {\hyperref[classes/models/@dsge/dsge:simulate]{simulate}} {]}(dsge/simulate)

\item {} 
{[} {\hyperref[classes/models/@dsge/dsge:simulate-nonlinear]{simulate\_nonlinear}} {]}(dsge/simulate\_nonlinear)

\item {} 
{[} {\hyperref[classes/models/@dsge/dsge:simulation-diagnostics]{simulation\_diagnostics}} {]}(dsge/simulation\_diagnostics)

\item {} 
{[} {\hyperref[classes/models/@dsge/dsge:solve]{solve}} {]}(dsge/solve)

\item {} 
{[} {\hyperref[classes/models/@dsge/dsge:solve-alternatives]{solve\_alternatives}} {]}(dsge/solve\_alternatives)

\item {} 
{[} {\hyperref[classes/models/@dsge/dsge:stoch-simul]{stoch\_simul}} {]}(dsge/stoch\_simul)

\item {} 
{[} {\hyperref[classes/models/@dsge/dsge:theoretical-autocorrelations]{theoretical\_autocorrelations}} {]}(dsge/theoretical\_autocorrelations)

\item {} 
{[} {\hyperref[classes/models/@dsge/dsge:theoretical-autocovariances]{theoretical\_autocovariances}} {]}(dsge/theoretical\_autocovariances)

\item {} 
{[} {\hyperref[classes/models/@dsge/dsge:variance-decomposition]{variance\_decomposition}} {]}(dsge/variance\_decomposition)

\end{itemize}


\section{properties}
\label{classes/models/@dsge/dsge:properties}\begin{itemize}
\item {} 
{[}definitions{]} -

\item {} 
{[}equations{]} -

\item {} 
{[}folders\_paths{]} -

\item {} 
{[}dsge\_var{]} -

\item {} 
{[}filename{]} -

\item {} 
{[}legend{]} -

\item {} 
{[}endogenous{]} -

\item {} 
{[}exogenous{]} -

\item {} 
{[}parameters{]} -

\item {} 
{[}observables{]} -

\item {} 
{[}markov\_chains{]} -

\item {} 
{[}options{]} -

\item {} 
{[}estimation{]} -

\item {} 
{[}solution{]} -

\item {} 
{[}filtering{]} -

\end{itemize}


\section{Synopsis and description on methods}
\label{classes/models/@dsge/dsge:synopsis-and-description-on-methods}

\bigskip\hrule{}\bigskip



\subsection{check\_derivatives}
\label{classes/models/@dsge/dsge:check-derivatives}\label{classes/models/@dsge/dsge:id1}
H1 line


\subsubsection{Syntax}
\label{classes/models/@dsge/dsge:syntax}

\subsubsection{Inputs}
\label{classes/models/@dsge/dsge:inputs}

\subsubsection{Outputs}
\label{classes/models/@dsge/dsge:outputs}

\subsubsection{Description}
\label{classes/models/@dsge/dsge:description}

\subsubsection{Examples}
\label{classes/models/@dsge/dsge:examples}
See also:


\bigskip\hrule{}\bigskip



\subsection{check\_optimum}
\label{classes/models/@dsge/dsge:check-optimum}\label{classes/models/@dsge/dsge:id2}
H1 line


\subsubsection{Syntax}
\label{classes/models/@dsge/dsge:id3}

\subsubsection{Inputs}
\label{classes/models/@dsge/dsge:id4}

\subsubsection{Outputs}
\label{classes/models/@dsge/dsge:id5}

\subsubsection{Description}
\label{classes/models/@dsge/dsge:id6}

\subsubsection{Examples}
\label{classes/models/@dsge/dsge:id7}
See also:

Help for dsge/check\_optimum is inherited from superclass RISE\_GENERIC


\bigskip\hrule{}\bigskip



\subsection{compute\_steady\_state}
\label{classes/models/@dsge/dsge:id8}\label{classes/models/@dsge/dsge:compute-steady-state}
H1 line


\subsubsection{Syntax}
\label{classes/models/@dsge/dsge:id9}

\subsubsection{Inputs}
\label{classes/models/@dsge/dsge:id10}

\subsubsection{Outputs}
\label{classes/models/@dsge/dsge:id11}

\subsubsection{Description}
\label{classes/models/@dsge/dsge:id12}

\subsubsection{Examples}
\label{classes/models/@dsge/dsge:id13}
See also:


\bigskip\hrule{}\bigskip



\subsection{create\_estimation\_blocks}
\label{classes/models/@dsge/dsge:create-estimation-blocks}\label{classes/models/@dsge/dsge:id14}
H1 line


\subsubsection{Syntax}
\label{classes/models/@dsge/dsge:id15}

\subsubsection{Inputs}
\label{classes/models/@dsge/dsge:id16}

\subsubsection{Outputs}
\label{classes/models/@dsge/dsge:id17}

\subsubsection{Description}
\label{classes/models/@dsge/dsge:id18}

\subsubsection{Examples}
\label{classes/models/@dsge/dsge:id19}
See also:


\bigskip\hrule{}\bigskip



\subsection{draw\_parameter}
\label{classes/models/@dsge/dsge:id20}\label{classes/models/@dsge/dsge:draw-parameter}
H1 line


\subsubsection{Syntax}
\label{classes/models/@dsge/dsge:id21}

\subsubsection{Inputs}
\label{classes/models/@dsge/dsge:id22}

\subsubsection{Outputs}
\label{classes/models/@dsge/dsge:id23}

\subsubsection{Description}
\label{classes/models/@dsge/dsge:id24}

\subsubsection{Examples}
\label{classes/models/@dsge/dsge:id25}
See also:

Help for dsge/draw\_parameter is inherited from superclass RISE\_GENERIC


\bigskip\hrule{}\bigskip

\phantomsection\label{classes/models/@dsge/dsge:dsge}
\textbf{dsge}
\begin{quote}

-- no help found
\end{quote}


\bigskip\hrule{}\bigskip



\subsection{estimate}
\label{classes/models/@dsge/dsge:estimate}\label{classes/models/@dsge/dsge:id26}
H1 line


\subsubsection{Syntax}
\label{classes/models/@dsge/dsge:id27}

\subsubsection{Inputs}
\label{classes/models/@dsge/dsge:id28}

\subsubsection{Outputs}
\label{classes/models/@dsge/dsge:id29}

\subsubsection{Description}
\label{classes/models/@dsge/dsge:id30}

\subsubsection{Examples}
\label{classes/models/@dsge/dsge:id31}
See also:

Help for dsge/estimate is inherited from superclass RISE\_GENERIC


\bigskip\hrule{}\bigskip



\subsection{filter}
\label{classes/models/@dsge/dsge:filter}\label{classes/models/@dsge/dsge:id32}
H1 line


\subsubsection{Syntax}
\label{classes/models/@dsge/dsge:id33}

\subsubsection{Inputs}
\label{classes/models/@dsge/dsge:id34}

\subsubsection{Outputs}
\label{classes/models/@dsge/dsge:id35}

\subsubsection{Description}
\label{classes/models/@dsge/dsge:id36}

\subsubsection{Examples}
\label{classes/models/@dsge/dsge:id37}
See also:


\bigskip\hrule{}\bigskip



\subsection{forecast}
\label{classes/models/@dsge/dsge:id38}\label{classes/models/@dsge/dsge:forecast}
H1 line


\subsubsection{Syntax}
\label{classes/models/@dsge/dsge:id39}

\subsubsection{Inputs}
\label{classes/models/@dsge/dsge:id40}

\subsubsection{Outputs}
\label{classes/models/@dsge/dsge:id41}

\subsubsection{Description}
\label{classes/models/@dsge/dsge:id42}

\subsubsection{Examples}
\label{classes/models/@dsge/dsge:id43}
See also:

Help for dsge/forecast is inherited from superclass RISE\_GENERIC


\bigskip\hrule{}\bigskip



\subsection{forecast\_real\_time}
\label{classes/models/@dsge/dsge:id44}\label{classes/models/@dsge/dsge:forecast-real-time}
H1 line


\subsubsection{Syntax}
\label{classes/models/@dsge/dsge:id45}

\subsubsection{Inputs}
\label{classes/models/@dsge/dsge:id46}

\subsubsection{Outputs}
\label{classes/models/@dsge/dsge:id47}

\subsubsection{Description}
\label{classes/models/@dsge/dsge:id48}

\subsubsection{Examples}
\label{classes/models/@dsge/dsge:id49}
See also:


\bigskip\hrule{}\bigskip



\subsection{get}
\label{classes/models/@dsge/dsge:id50}\label{classes/models/@dsge/dsge:get}
H1 line


\subsubsection{Syntax}
\label{classes/models/@dsge/dsge:id51}

\subsubsection{Inputs}
\label{classes/models/@dsge/dsge:id52}

\subsubsection{Outputs}
\label{classes/models/@dsge/dsge:id53}

\subsubsection{Description}
\label{classes/models/@dsge/dsge:id54}

\subsubsection{Examples}
\label{classes/models/@dsge/dsge:id55}
See also:

Help for dsge/get is inherited from superclass RISE\_GENERIC


\bigskip\hrule{}\bigskip



\subsection{historical\_decomposition}
\label{classes/models/@dsge/dsge:id56}\label{classes/models/@dsge/dsge:historical-decomposition}
H1 line


\subsubsection{Syntax}
\label{classes/models/@dsge/dsge:id57}

\subsubsection{Inputs}
\label{classes/models/@dsge/dsge:id58}

\subsubsection{Outputs}
\label{classes/models/@dsge/dsge:id59}

\subsubsection{Description}
\label{classes/models/@dsge/dsge:id60}

\subsubsection{Examples}
\label{classes/models/@dsge/dsge:id61}
See also:

Help for dsge/historical\_decomposition is inherited from superclass RISE\_GENERIC


\bigskip\hrule{}\bigskip



\subsection{irf}
\label{classes/models/@dsge/dsge:irf}\label{classes/models/@dsge/dsge:id62}
H1 line


\subsubsection{Syntax}
\label{classes/models/@dsge/dsge:id63}

\subsubsection{Inputs}
\label{classes/models/@dsge/dsge:id64}

\subsubsection{Outputs}
\label{classes/models/@dsge/dsge:id65}

\subsubsection{Description}
\label{classes/models/@dsge/dsge:id66}

\subsubsection{Examples}
\label{classes/models/@dsge/dsge:id67}
See also:


\bigskip\hrule{}\bigskip



\subsection{is\_stable\_system}
\label{classes/models/@dsge/dsge:is-stable-system}\label{classes/models/@dsge/dsge:id68}
H1 line


\subsubsection{Syntax}
\label{classes/models/@dsge/dsge:id69}

\subsubsection{Inputs}
\label{classes/models/@dsge/dsge:id70}

\subsubsection{Outputs}
\label{classes/models/@dsge/dsge:id71}

\subsubsection{Description}
\label{classes/models/@dsge/dsge:id72}

\subsubsection{Examples}
\label{classes/models/@dsge/dsge:id73}
See also:


\bigskip\hrule{}\bigskip



\subsection{isnan}
\label{classes/models/@dsge/dsge:isnan}\label{classes/models/@dsge/dsge:id74}
H1 line


\subsubsection{Syntax}
\label{classes/models/@dsge/dsge:id75}

\subsubsection{Inputs}
\label{classes/models/@dsge/dsge:id76}

\subsubsection{Outputs}
\label{classes/models/@dsge/dsge:id77}

\subsubsection{Description}
\label{classes/models/@dsge/dsge:id78}

\subsubsection{Examples}
\label{classes/models/@dsge/dsge:id79}
See also:

Help for dsge/isnan is inherited from superclass RISE\_GENERIC


\bigskip\hrule{}\bigskip



\subsection{load\_parameters}
\label{classes/models/@dsge/dsge:id80}\label{classes/models/@dsge/dsge:load-parameters}
H1 line


\subsubsection{Syntax}
\label{classes/models/@dsge/dsge:id81}

\subsubsection{Inputs}
\label{classes/models/@dsge/dsge:id82}

\subsubsection{Outputs}
\label{classes/models/@dsge/dsge:id83}

\subsubsection{Description}
\label{classes/models/@dsge/dsge:id84}

\subsubsection{Examples}
\label{classes/models/@dsge/dsge:id85}
See also:

Help for dsge/load\_parameters is inherited from superclass RISE\_GENERIC


\bigskip\hrule{}\bigskip



\subsection{log\_marginal\_data\_density}
\label{classes/models/@dsge/dsge:log-marginal-data-density}\label{classes/models/@dsge/dsge:id86}
H1 line


\subsubsection{Syntax}
\label{classes/models/@dsge/dsge:id87}

\subsubsection{Inputs}
\label{classes/models/@dsge/dsge:id88}

\subsubsection{Outputs}
\label{classes/models/@dsge/dsge:id89}

\subsubsection{Description}
\label{classes/models/@dsge/dsge:id90}

\subsubsection{Examples}
\label{classes/models/@dsge/dsge:id91}
See also:

Help for dsge/log\_marginal\_data\_density is inherited from superclass RISE\_GENERIC


\bigskip\hrule{}\bigskip



\subsection{log\_posterior\_kernel}
\label{classes/models/@dsge/dsge:log-posterior-kernel}\label{classes/models/@dsge/dsge:id92}
H1 line


\subsubsection{Syntax}
\label{classes/models/@dsge/dsge:id93}

\subsubsection{Inputs}
\label{classes/models/@dsge/dsge:id94}

\subsubsection{Outputs}
\label{classes/models/@dsge/dsge:id95}

\subsubsection{Description}
\label{classes/models/@dsge/dsge:id96}

\subsubsection{Examples}
\label{classes/models/@dsge/dsge:id97}
See also:

Help for dsge/log\_posterior\_kernel is inherited from superclass RISE\_GENERIC


\bigskip\hrule{}\bigskip



\subsection{log\_prior\_density}
\label{classes/models/@dsge/dsge:id98}\label{classes/models/@dsge/dsge:log-prior-density}
H1 line


\subsubsection{Syntax}
\label{classes/models/@dsge/dsge:id99}

\subsubsection{Inputs}
\label{classes/models/@dsge/dsge:id100}

\subsubsection{Outputs}
\label{classes/models/@dsge/dsge:id101}

\subsubsection{Description}
\label{classes/models/@dsge/dsge:id102}

\subsubsection{Examples}
\label{classes/models/@dsge/dsge:id103}
See also:

Help for dsge/log\_prior\_density is inherited from superclass RISE\_GENERIC


\bigskip\hrule{}\bigskip



\subsection{monte\_carlo\_filtering}
\label{classes/models/@dsge/dsge:monte-carlo-filtering}\label{classes/models/@dsge/dsge:id104}
H1 line


\subsubsection{Syntax}
\label{classes/models/@dsge/dsge:id105}

\subsubsection{Inputs}
\label{classes/models/@dsge/dsge:id106}

\subsubsection{Outputs}
\label{classes/models/@dsge/dsge:id107}

\subsubsection{Description}
\label{classes/models/@dsge/dsge:id108}

\subsubsection{Examples}
\label{classes/models/@dsge/dsge:id109}
See also:


\bigskip\hrule{}\bigskip



\subsection{posterior\_marginal\_and\_prior\_densities}
\label{classes/models/@dsge/dsge:id110}\label{classes/models/@dsge/dsge:posterior-marginal-and-prior-densities}
H1 line


\subsubsection{Syntax}
\label{classes/models/@dsge/dsge:id111}

\subsubsection{Inputs}
\label{classes/models/@dsge/dsge:id112}

\subsubsection{Outputs}
\label{classes/models/@dsge/dsge:id113}

\subsubsection{Description}
\label{classes/models/@dsge/dsge:id114}

\subsubsection{Examples}
\label{classes/models/@dsge/dsge:id115}
See also:

Help for dsge/posterior\_marginal\_and\_prior\_densities is inherited from superclass RISE\_GENERIC


\bigskip\hrule{}\bigskip



\subsection{posterior\_simulator}
\label{classes/models/@dsge/dsge:posterior-simulator}\label{classes/models/@dsge/dsge:id116}
H1 line


\subsubsection{Syntax}
\label{classes/models/@dsge/dsge:id117}

\subsubsection{Inputs}
\label{classes/models/@dsge/dsge:id118}

\subsubsection{Outputs}
\label{classes/models/@dsge/dsge:id119}

\subsubsection{Description}
\label{classes/models/@dsge/dsge:id120}

\subsubsection{Examples}
\label{classes/models/@dsge/dsge:id121}
See also:

Help for dsge/posterior\_simulator is inherited from superclass RISE\_GENERIC


\bigskip\hrule{}\bigskip



\subsection{print\_estimation\_results}
\label{classes/models/@dsge/dsge:print-estimation-results}\label{classes/models/@dsge/dsge:id122}
H1 line


\subsubsection{Syntax}
\label{classes/models/@dsge/dsge:id123}

\subsubsection{Inputs}
\label{classes/models/@dsge/dsge:id124}

\subsubsection{Outputs}
\label{classes/models/@dsge/dsge:id125}

\subsubsection{Description}
\label{classes/models/@dsge/dsge:id126}

\subsubsection{Examples}
\label{classes/models/@dsge/dsge:id127}
See also:

Help for dsge/print\_estimation\_results is inherited from superclass RISE\_GENERIC


\bigskip\hrule{}\bigskip



\subsection{print\_solution}
\label{classes/models/@dsge/dsge:print-solution}\label{classes/models/@dsge/dsge:id128}
H1 line


\subsubsection{Syntax}
\label{classes/models/@dsge/dsge:id129}

\subsubsection{Inputs}
\label{classes/models/@dsge/dsge:id130}

\subsubsection{Outputs}
\label{classes/models/@dsge/dsge:id131}

\subsubsection{Description}
\label{classes/models/@dsge/dsge:id132}

\subsubsection{Examples}
\label{classes/models/@dsge/dsge:id133}
See also:


\bigskip\hrule{}\bigskip



\subsection{prior\_plots}
\label{classes/models/@dsge/dsge:id134}\label{classes/models/@dsge/dsge:prior-plots}
H1 line


\subsubsection{Syntax}
\label{classes/models/@dsge/dsge:id135}

\subsubsection{Inputs}
\label{classes/models/@dsge/dsge:id136}

\subsubsection{Outputs}
\label{classes/models/@dsge/dsge:id137}

\subsubsection{Description}
\label{classes/models/@dsge/dsge:id138}

\subsubsection{Examples}
\label{classes/models/@dsge/dsge:id139}
See also:

Help for dsge/prior\_plots is inherited from superclass RISE\_GENERIC


\bigskip\hrule{}\bigskip

\phantomsection\label{classes/models/@dsge/dsge:report}
\textbf{REPORT} assigns the elements of interest to a rise\_report.report object


\section{Syntax}
\label{classes/models/@dsge/dsge:id140}\begin{description}
\item[{::}] \leavevmode\begin{itemize}
\item {} 
REPORT(rise.empty(0)) : displays the default inputs

\item {} 
REPORT(obj,destination\_root,rep\_items) : assigns the reported
elements in rep\_items to destination\_root

\item {} 
REPORT(obj,destination\_root,rep\_items,varargin) : assigns varargin to
obj before doing the rest

\end{itemize}

\end{description}


\section{Inputs}
\label{classes/models/@dsge/dsge:id141}\begin{itemize}
\item {} 
obj : {[}rise\textbar{}dsge{]}

\item {} 
destination\_root : {[}rise\_report.report{]} : handle for the actual report

\item {} 
rep\_items : {[}char\textbar{}cellstr{]} : list of desired items to report. This list
can only include : `endogenous', `exogenous', `observables',
`parameters', `solution', `estimation', `estimation\_statistics',
`equations', `code'

\end{itemize}


\section{Outputs}
\label{classes/models/@dsge/dsge:id142}
none


\section{Description}
\label{classes/models/@dsge/dsge:id143}

\section{Examples}
\label{classes/models/@dsge/dsge:id144}
See also:

Help for dsge/report is inherited from superclass RISE\_GENERIC


\bigskip\hrule{}\bigskip



\subsection{resid}
\label{classes/models/@dsge/dsge:id145}\label{classes/models/@dsge/dsge:resid}
H1 line


\subsubsection{Syntax}
\label{classes/models/@dsge/dsge:id146}

\subsubsection{Inputs}
\label{classes/models/@dsge/dsge:id147}

\subsubsection{Outputs}
\label{classes/models/@dsge/dsge:id148}

\subsubsection{Description}
\label{classes/models/@dsge/dsge:id149}

\subsubsection{Examples}
\label{classes/models/@dsge/dsge:id150}
See also:


\bigskip\hrule{}\bigskip



\subsection{set}
\label{classes/models/@dsge/dsge:set}\label{classes/models/@dsge/dsge:id151}
H1 line


\subsubsection{Syntax}
\label{classes/models/@dsge/dsge:id152}

\subsubsection{Inputs}
\label{classes/models/@dsge/dsge:id153}

\subsubsection{Outputs}
\label{classes/models/@dsge/dsge:id154}

\subsubsection{Description}
\label{classes/models/@dsge/dsge:id155}

\subsubsection{Examples}
\label{classes/models/@dsge/dsge:id156}
See also:


\bigskip\hrule{}\bigskip



\subsection{set\_solution\_to\_companion}
\label{classes/models/@dsge/dsge:id157}\label{classes/models/@dsge/dsge:set-solution-to-companion}
H1 line


\subsubsection{Syntax}
\label{classes/models/@dsge/dsge:id158}

\subsubsection{Inputs}
\label{classes/models/@dsge/dsge:id159}

\subsubsection{Outputs}
\label{classes/models/@dsge/dsge:id160}

\subsubsection{Description}
\label{classes/models/@dsge/dsge:id161}

\subsubsection{Examples}
\label{classes/models/@dsge/dsge:id162}
See also:


\bigskip\hrule{}\bigskip



\subsection{simulate}
\label{classes/models/@dsge/dsge:id163}\label{classes/models/@dsge/dsge:simulate}
H1 line


\subsubsection{Syntax}
\label{classes/models/@dsge/dsge:id164}

\subsubsection{Inputs}
\label{classes/models/@dsge/dsge:id165}

\subsubsection{Outputs}
\label{classes/models/@dsge/dsge:id166}

\subsubsection{Description}
\label{classes/models/@dsge/dsge:id167}

\subsubsection{Examples}
\label{classes/models/@dsge/dsge:id168}
See also:


\bigskip\hrule{}\bigskip



\subsection{simulate\_nonlinear}
\label{classes/models/@dsge/dsge:id169}\label{classes/models/@dsge/dsge:simulate-nonlinear}
H1 line


\subsubsection{Syntax}
\label{classes/models/@dsge/dsge:id170}

\subsubsection{Inputs}
\label{classes/models/@dsge/dsge:id171}

\subsubsection{Outputs}
\label{classes/models/@dsge/dsge:id172}

\subsubsection{Description}
\label{classes/models/@dsge/dsge:id173}

\subsubsection{Examples}
\label{classes/models/@dsge/dsge:id174}
See also:


\bigskip\hrule{}\bigskip



\subsection{simulation\_diagnostics}
\label{classes/models/@dsge/dsge:id175}\label{classes/models/@dsge/dsge:simulation-diagnostics}
H1 line


\subsubsection{Syntax}
\label{classes/models/@dsge/dsge:id176}

\subsubsection{Inputs}
\label{classes/models/@dsge/dsge:id177}

\subsubsection{Outputs}
\label{classes/models/@dsge/dsge:id178}

\subsubsection{Description}
\label{classes/models/@dsge/dsge:id179}

\subsubsection{Examples}
\label{classes/models/@dsge/dsge:id180}
See also:

Help for dsge/simulation\_diagnostics is inherited from superclass RISE\_GENERIC


\bigskip\hrule{}\bigskip



\subsection{solve}
\label{classes/models/@dsge/dsge:id181}\label{classes/models/@dsge/dsge:solve}
H1 line


\subsubsection{Syntax}
\label{classes/models/@dsge/dsge:id182}

\subsubsection{Inputs}
\label{classes/models/@dsge/dsge:id183}

\subsubsection{Outputs}
\label{classes/models/@dsge/dsge:id184}

\subsubsection{Description}
\label{classes/models/@dsge/dsge:id185}

\subsubsection{Examples}
\label{classes/models/@dsge/dsge:id186}
See also:


\bigskip\hrule{}\bigskip



\subsection{solve\_alternatives}
\label{classes/models/@dsge/dsge:solve-alternatives}\label{classes/models/@dsge/dsge:id187}
H1 line


\subsubsection{Syntax}
\label{classes/models/@dsge/dsge:id188}

\subsubsection{Inputs}
\label{classes/models/@dsge/dsge:id189}

\subsubsection{Outputs}
\label{classes/models/@dsge/dsge:id190}

\subsubsection{Description}
\label{classes/models/@dsge/dsge:id191}

\subsubsection{Examples}
\label{classes/models/@dsge/dsge:id192}
See also:


\bigskip\hrule{}\bigskip



\subsection{stoch\_simul}
\label{classes/models/@dsge/dsge:id193}\label{classes/models/@dsge/dsge:stoch-simul}
H1 line


\subsubsection{Syntax}
\label{classes/models/@dsge/dsge:id194}

\subsubsection{Inputs}
\label{classes/models/@dsge/dsge:id195}

\subsubsection{Outputs}
\label{classes/models/@dsge/dsge:id196}

\subsubsection{Description}
\label{classes/models/@dsge/dsge:id197}

\subsubsection{Examples}
\label{classes/models/@dsge/dsge:id198}
See also:

Help for dsge/stoch\_simul is inherited from superclass RISE\_GENERIC


\bigskip\hrule{}\bigskip



\subsection{theoretical\_autocorrelations}
\label{classes/models/@dsge/dsge:theoretical-autocorrelations}\label{classes/models/@dsge/dsge:id199}
H1 line


\subsubsection{Syntax}
\label{classes/models/@dsge/dsge:id200}

\subsubsection{Inputs}
\label{classes/models/@dsge/dsge:id201}

\subsubsection{Outputs}
\label{classes/models/@dsge/dsge:id202}

\subsubsection{Description}
\label{classes/models/@dsge/dsge:id203}

\subsubsection{Examples}
\label{classes/models/@dsge/dsge:id204}
See also:

Help for dsge/theoretical\_autocorrelations is inherited from superclass RISE\_GENERIC


\bigskip\hrule{}\bigskip



\subsection{theoretical\_autocovariances}
\label{classes/models/@dsge/dsge:theoretical-autocovariances}\label{classes/models/@dsge/dsge:id205}
H1 line


\subsubsection{Syntax}
\label{classes/models/@dsge/dsge:id206}

\subsubsection{Inputs}
\label{classes/models/@dsge/dsge:id207}

\subsubsection{Outputs}
\label{classes/models/@dsge/dsge:id208}

\subsubsection{Description}
\label{classes/models/@dsge/dsge:id209}

\subsubsection{Examples}
\label{classes/models/@dsge/dsge:id210}
See also:

Help for dsge/theoretical\_autocovariances is inherited from superclass RISE\_GENERIC


\bigskip\hrule{}\bigskip



\subsection{variance\_decomposition}
\label{classes/models/@dsge/dsge:id211}\label{classes/models/@dsge/dsge:variance-decomposition}
H1 line


\subsubsection{Syntax}
\label{classes/models/@dsge/dsge:id212}

\subsubsection{Inputs}
\label{classes/models/@dsge/dsge:id213}

\subsubsection{Outputs}
\label{classes/models/@dsge/dsge:id214}

\subsubsection{Description}
\label{classes/models/@dsge/dsge:id215}

\subsubsection{Examples}
\label{classes/models/@dsge/dsge:id216}
See also:

Help for dsge/variance\_decomposition is inherited from superclass RISE\_GENERIC


\chapter{Reduced-form VAR modeling}
\label{classes/models/@rfvar/rfvar::doc}\label{classes/models/@rfvar/rfvar:reduced-form-var-modeling}

\section{methods}
\label{classes/models/@rfvar/rfvar:methods}\begin{itemize}
\item {} 
{[} {\hyperref[classes/models/@rfvar/rfvar:check-identification]{check\_identification}} {]}(rfvar/check\_identification)

\item {} 
{[} {\hyperref[classes/models/@rfvar/rfvar:check-optimum]{check\_optimum}} {]}(rfvar/check\_optimum)

\item {} 
{[} {\hyperref[classes/models/@rfvar/rfvar:draw-parameter]{draw\_parameter}} {]}(rfvar/draw\_parameter)

\item {} 
{[} {\hyperref[classes/models/@rfvar/rfvar:estimate]{estimate}} {]}(rfvar/estimate)

\item {} 
{[} {\hyperref[classes/models/@rfvar/rfvar:forecast]{forecast}} {]}(rfvar/forecast)

\item {} 
{[} {\hyperref[classes/models/@rfvar/rfvar:get]{get}} {]}(rfvar/get)

\item {} 
{[} {\hyperref[classes/models/@rfvar/rfvar:historical-decomposition]{historical\_decomposition}} {]}(rfvar/historical\_decomposition)

\item {} 
{[} {\hyperref[classes/models/@rfvar/rfvar:irf]{irf}} {]}(rfvar/irf)

\item {} 
{[} {\hyperref[classes/models/@rfvar/rfvar:isnan]{isnan}} {]}(rfvar/isnan)

\item {} 
{[} {\hyperref[classes/models/@rfvar/rfvar:load-parameters]{load\_parameters}} {]}(rfvar/load\_parameters)

\item {} 
{[} {\hyperref[classes/models/@rfvar/rfvar:log-marginal-data-density]{log\_marginal\_data\_density}} {]}(rfvar/log\_marginal\_data\_density)

\item {} 
{[} {\hyperref[classes/models/@rfvar/rfvar:log-posterior-kernel]{log\_posterior\_kernel}} {]}(rfvar/log\_posterior\_kernel)

\item {} 
{[} {\hyperref[classes/models/@rfvar/rfvar:log-prior-density]{log\_prior\_density}} {]}(rfvar/log\_prior\_density)

\item {} 
{[} {\hyperref[classes/models/@rfvar/rfvar:msvar-priors]{msvar\_priors}} {]}(rfvar/msvar\_priors)

\item {} 
{[} {\hyperref[classes/models/@rfvar/rfvar:posterior-marginal-and-prior-densities]{posterior\_marginal\_and\_prior\_densities}} {]}(rfvar/posterior\_marginal\_and\_prior\_densities)

\item {} 
{[} {\hyperref[classes/models/@rfvar/rfvar:posterior-simulator]{posterior\_simulator}} {]}(rfvar/posterior\_simulator)

\item {} 
{[} {\hyperref[classes/models/@rfvar/rfvar:print-estimation-results]{print\_estimation\_results}} {]}(rfvar/print\_estimation\_results)

\item {} 
{[} {\hyperref[classes/models/@rfvar/rfvar:prior-plots]{prior\_plots}} {]}(rfvar/prior\_plots)

\item {} 
{[} {\hyperref[classes/models/@rfvar/rfvar:report]{report}} {]}(rfvar/report)

\item {} 
{[} {\hyperref[classes/models/@rfvar/rfvar:rfvar]{rfvar}} {]}(rfvar/rfvar)

\item {} 
{[} {\hyperref[classes/models/@rfvar/rfvar:set]{set}} {]}(rfvar/set)

\item {} 
{[} {\hyperref[classes/models/@rfvar/rfvar:set-solution-to-companion]{set\_solution\_to\_companion}} {]}(rfvar/set\_solution\_to\_companion)

\item {} 
{[} {\hyperref[classes/models/@rfvar/rfvar:simulate]{simulate}} {]}(rfvar/simulate)

\item {} 
{[} {\hyperref[classes/models/@rfvar/rfvar:simulation-diagnostics]{simulation\_diagnostics}} {]}(rfvar/simulation\_diagnostics)

\item {} 
{[} {\hyperref[classes/models/@rfvar/rfvar:solve]{solve}} {]}(rfvar/solve)

\item {} 
{[} {\hyperref[classes/models/@rfvar/rfvar:stoch-simul]{stoch\_simul}} {]}(rfvar/stoch\_simul)

\item {} 
{[} {\hyperref[classes/models/@rfvar/rfvar:structural-form]{structural\_form}} {]}(rfvar/structural\_form)

\item {} 
{[} {\hyperref[classes/models/@rfvar/rfvar:template]{template}} {]}(rfvar/template)

\item {} 
{[} {\hyperref[classes/models/@rfvar/rfvar:theoretical-autocorrelations]{theoretical\_autocorrelations}} {]}(rfvar/theoretical\_autocorrelations)

\item {} 
{[} {\hyperref[classes/models/@rfvar/rfvar:theoretical-autocovariances]{theoretical\_autocovariances}} {]}(rfvar/theoretical\_autocovariances)

\item {} 
{[} {\hyperref[classes/models/@rfvar/rfvar:variance-decomposition]{variance\_decomposition}} {]}(rfvar/variance\_decomposition)

\end{itemize}


\section{properties}
\label{classes/models/@rfvar/rfvar:properties}\begin{itemize}
\item {} 
{[}identification{]} -

\item {} 
{[}structural\_shocks{]} -

\item {} 
{[}nonlinear\_restrictions{]} -

\item {} 
{[}constant{]} -

\item {} 
{[}nlags{]} -

\item {} 
{[}legend{]} -

\item {} 
{[}endogenous{]} -

\item {} 
{[}exogenous{]} -

\item {} 
{[}parameters{]} -

\item {} 
{[}observables{]} -

\item {} 
{[}markov\_chains{]} -

\item {} 
{[}options{]} -

\item {} 
{[}estimation{]} -

\item {} 
{[}solution{]} -

\item {} 
{[}filtering{]} -

\end{itemize}


\section{Synopsis and description on methods}
\label{classes/models/@rfvar/rfvar:synopsis-and-description-on-methods}

\bigskip\hrule{}\bigskip



\subsection{check\_identification}
\label{classes/models/@rfvar/rfvar:id1}\label{classes/models/@rfvar/rfvar:check-identification}
H1 line


\subsubsection{Syntax}
\label{classes/models/@rfvar/rfvar:syntax}

\subsubsection{Inputs}
\label{classes/models/@rfvar/rfvar:inputs}

\subsubsection{Outputs}
\label{classes/models/@rfvar/rfvar:outputs}

\subsubsection{Description}
\label{classes/models/@rfvar/rfvar:description}

\subsubsection{Examples}
\label{classes/models/@rfvar/rfvar:examples}
See also:


\bigskip\hrule{}\bigskip



\subsection{check\_optimum}
\label{classes/models/@rfvar/rfvar:check-optimum}\label{classes/models/@rfvar/rfvar:id2}
H1 line


\subsubsection{Syntax}
\label{classes/models/@rfvar/rfvar:id3}

\subsubsection{Inputs}
\label{classes/models/@rfvar/rfvar:id4}

\subsubsection{Outputs}
\label{classes/models/@rfvar/rfvar:id5}

\subsubsection{Description}
\label{classes/models/@rfvar/rfvar:id6}

\subsubsection{Examples}
\label{classes/models/@rfvar/rfvar:id7}
See also:

Help for rfvar/check\_optimum is inherited from superclass RISE\_GENERIC


\bigskip\hrule{}\bigskip



\subsection{draw\_parameter}
\label{classes/models/@rfvar/rfvar:id8}\label{classes/models/@rfvar/rfvar:draw-parameter}
H1 line


\subsubsection{Syntax}
\label{classes/models/@rfvar/rfvar:id9}

\subsubsection{Inputs}
\label{classes/models/@rfvar/rfvar:id10}

\subsubsection{Outputs}
\label{classes/models/@rfvar/rfvar:id11}

\subsubsection{Description}
\label{classes/models/@rfvar/rfvar:id12}

\subsubsection{Examples}
\label{classes/models/@rfvar/rfvar:id13}
See also:

Help for rfvar/draw\_parameter is inherited from superclass RISE\_GENERIC


\bigskip\hrule{}\bigskip



\subsection{estimate}
\label{classes/models/@rfvar/rfvar:estimate}\label{classes/models/@rfvar/rfvar:id14}
H1 line


\subsubsection{Syntax}
\label{classes/models/@rfvar/rfvar:id15}

\subsubsection{Inputs}
\label{classes/models/@rfvar/rfvar:id16}

\subsubsection{Outputs}
\label{classes/models/@rfvar/rfvar:id17}

\subsubsection{Description}
\label{classes/models/@rfvar/rfvar:id18}

\subsubsection{Examples}
\label{classes/models/@rfvar/rfvar:id19}
See also:

Help for rfvar/estimate is inherited from superclass RISE\_GENERIC


\bigskip\hrule{}\bigskip



\subsection{forecast}
\label{classes/models/@rfvar/rfvar:id20}\label{classes/models/@rfvar/rfvar:forecast}
H1 line


\subsubsection{Syntax}
\label{classes/models/@rfvar/rfvar:id21}

\subsubsection{Inputs}
\label{classes/models/@rfvar/rfvar:id22}

\subsubsection{Outputs}
\label{classes/models/@rfvar/rfvar:id23}

\subsubsection{Description}
\label{classes/models/@rfvar/rfvar:id24}

\subsubsection{Examples}
\label{classes/models/@rfvar/rfvar:id25}
See also:

Help for rfvar/forecast is inherited from superclass RISE\_GENERIC


\bigskip\hrule{}\bigskip



\subsection{get}
\label{classes/models/@rfvar/rfvar:id26}\label{classes/models/@rfvar/rfvar:get}
H1 line


\subsubsection{Syntax}
\label{classes/models/@rfvar/rfvar:id27}

\subsubsection{Inputs}
\label{classes/models/@rfvar/rfvar:id28}

\subsubsection{Outputs}
\label{classes/models/@rfvar/rfvar:id29}

\subsubsection{Description}
\label{classes/models/@rfvar/rfvar:id30}

\subsubsection{Examples}
\label{classes/models/@rfvar/rfvar:id31}
See also:

Help for rfvar/get is inherited from superclass RISE\_GENERIC


\bigskip\hrule{}\bigskip



\subsection{historical\_decomposition}
\label{classes/models/@rfvar/rfvar:id32}\label{classes/models/@rfvar/rfvar:historical-decomposition}
H1 line


\subsubsection{Syntax}
\label{classes/models/@rfvar/rfvar:id33}

\subsubsection{Inputs}
\label{classes/models/@rfvar/rfvar:id34}

\subsubsection{Outputs}
\label{classes/models/@rfvar/rfvar:id35}

\subsubsection{Description}
\label{classes/models/@rfvar/rfvar:id36}

\subsubsection{Examples}
\label{classes/models/@rfvar/rfvar:id37}
See also:

Help for rfvar/historical\_decomposition is inherited from superclass RISE\_GENERIC


\bigskip\hrule{}\bigskip



\subsection{irf}
\label{classes/models/@rfvar/rfvar:id38}\label{classes/models/@rfvar/rfvar:irf}
H1 line


\subsubsection{Syntax}
\label{classes/models/@rfvar/rfvar:id39}

\subsubsection{Inputs}
\label{classes/models/@rfvar/rfvar:id40}

\subsubsection{Outputs}
\label{classes/models/@rfvar/rfvar:id41}

\subsubsection{Description}
\label{classes/models/@rfvar/rfvar:id42}

\subsubsection{Examples}
\label{classes/models/@rfvar/rfvar:id43}
See also:

Help for rfvar/irf is inherited from superclass RISE\_GENERIC


\bigskip\hrule{}\bigskip



\subsection{isnan}
\label{classes/models/@rfvar/rfvar:isnan}\label{classes/models/@rfvar/rfvar:id44}
H1 line


\subsubsection{Syntax}
\label{classes/models/@rfvar/rfvar:id45}

\subsubsection{Inputs}
\label{classes/models/@rfvar/rfvar:id46}

\subsubsection{Outputs}
\label{classes/models/@rfvar/rfvar:id47}

\subsubsection{Description}
\label{classes/models/@rfvar/rfvar:id48}

\subsubsection{Examples}
\label{classes/models/@rfvar/rfvar:id49}
See also:

Help for rfvar/isnan is inherited from superclass RISE\_GENERIC


\bigskip\hrule{}\bigskip



\subsection{load\_parameters}
\label{classes/models/@rfvar/rfvar:id50}\label{classes/models/@rfvar/rfvar:load-parameters}
H1 line


\subsubsection{Syntax}
\label{classes/models/@rfvar/rfvar:id51}

\subsubsection{Inputs}
\label{classes/models/@rfvar/rfvar:id52}

\subsubsection{Outputs}
\label{classes/models/@rfvar/rfvar:id53}

\subsubsection{Description}
\label{classes/models/@rfvar/rfvar:id54}

\subsubsection{Examples}
\label{classes/models/@rfvar/rfvar:id55}
See also:

Help for rfvar/load\_parameters is inherited from superclass RISE\_GENERIC


\bigskip\hrule{}\bigskip



\subsection{log\_marginal\_data\_density}
\label{classes/models/@rfvar/rfvar:id56}\label{classes/models/@rfvar/rfvar:log-marginal-data-density}
H1 line


\subsubsection{Syntax}
\label{classes/models/@rfvar/rfvar:id57}

\subsubsection{Inputs}
\label{classes/models/@rfvar/rfvar:id58}

\subsubsection{Outputs}
\label{classes/models/@rfvar/rfvar:id59}

\subsubsection{Description}
\label{classes/models/@rfvar/rfvar:id60}

\subsubsection{Examples}
\label{classes/models/@rfvar/rfvar:id61}
See also:

Help for rfvar/log\_marginal\_data\_density is inherited from superclass RISE\_GENERIC


\bigskip\hrule{}\bigskip



\subsection{log\_posterior\_kernel}
\label{classes/models/@rfvar/rfvar:log-posterior-kernel}\label{classes/models/@rfvar/rfvar:id62}
H1 line


\subsubsection{Syntax}
\label{classes/models/@rfvar/rfvar:id63}

\subsubsection{Inputs}
\label{classes/models/@rfvar/rfvar:id64}

\subsubsection{Outputs}
\label{classes/models/@rfvar/rfvar:id65}

\subsubsection{Description}
\label{classes/models/@rfvar/rfvar:id66}

\subsubsection{Examples}
\label{classes/models/@rfvar/rfvar:id67}
See also:

Help for rfvar/log\_posterior\_kernel is inherited from superclass RISE\_GENERIC


\bigskip\hrule{}\bigskip



\subsection{log\_prior\_density}
\label{classes/models/@rfvar/rfvar:id68}\label{classes/models/@rfvar/rfvar:log-prior-density}
H1 line


\subsubsection{Syntax}
\label{classes/models/@rfvar/rfvar:id69}

\subsubsection{Inputs}
\label{classes/models/@rfvar/rfvar:id70}

\subsubsection{Outputs}
\label{classes/models/@rfvar/rfvar:id71}

\subsubsection{Description}
\label{classes/models/@rfvar/rfvar:id72}

\subsubsection{Examples}
\label{classes/models/@rfvar/rfvar:id73}
See also:

Help for rfvar/log\_prior\_density is inherited from superclass RISE\_GENERIC


\bigskip\hrule{}\bigskip



\subsection{msvar\_priors}
\label{classes/models/@rfvar/rfvar:msvar-priors}\label{classes/models/@rfvar/rfvar:id74}
H1 line


\subsubsection{Syntax}
\label{classes/models/@rfvar/rfvar:id75}

\subsubsection{Inputs}
\label{classes/models/@rfvar/rfvar:id76}

\subsubsection{Outputs}
\label{classes/models/@rfvar/rfvar:id77}

\subsubsection{Description}
\label{classes/models/@rfvar/rfvar:id78}

\subsubsection{Examples}
\label{classes/models/@rfvar/rfvar:id79}
See also:

Help for rfvar/msvar\_priors is inherited from superclass SVAR


\bigskip\hrule{}\bigskip



\subsection{posterior\_marginal\_and\_prior\_densities}
\label{classes/models/@rfvar/rfvar:id80}\label{classes/models/@rfvar/rfvar:posterior-marginal-and-prior-densities}
H1 line


\subsubsection{Syntax}
\label{classes/models/@rfvar/rfvar:id81}

\subsubsection{Inputs}
\label{classes/models/@rfvar/rfvar:id82}

\subsubsection{Outputs}
\label{classes/models/@rfvar/rfvar:id83}

\subsubsection{Description}
\label{classes/models/@rfvar/rfvar:id84}

\subsubsection{Examples}
\label{classes/models/@rfvar/rfvar:id85}
See also:

Help for rfvar/posterior\_marginal\_and\_prior\_densities is inherited from superclass RISE\_GENERIC


\bigskip\hrule{}\bigskip



\subsection{posterior\_simulator}
\label{classes/models/@rfvar/rfvar:posterior-simulator}\label{classes/models/@rfvar/rfvar:id86}
H1 line


\subsubsection{Syntax}
\label{classes/models/@rfvar/rfvar:id87}

\subsubsection{Inputs}
\label{classes/models/@rfvar/rfvar:id88}

\subsubsection{Outputs}
\label{classes/models/@rfvar/rfvar:id89}

\subsubsection{Description}
\label{classes/models/@rfvar/rfvar:id90}

\subsubsection{Examples}
\label{classes/models/@rfvar/rfvar:id91}
See also:

Help for rfvar/posterior\_simulator is inherited from superclass RISE\_GENERIC


\bigskip\hrule{}\bigskip



\subsection{print\_estimation\_results}
\label{classes/models/@rfvar/rfvar:print-estimation-results}\label{classes/models/@rfvar/rfvar:id92}
H1 line


\subsubsection{Syntax}
\label{classes/models/@rfvar/rfvar:id93}

\subsubsection{Inputs}
\label{classes/models/@rfvar/rfvar:id94}

\subsubsection{Outputs}
\label{classes/models/@rfvar/rfvar:id95}

\subsubsection{Description}
\label{classes/models/@rfvar/rfvar:id96}

\subsubsection{Examples}
\label{classes/models/@rfvar/rfvar:id97}
See also:

Help for rfvar/print\_estimation\_results is inherited from superclass RISE\_GENERIC


\bigskip\hrule{}\bigskip



\subsection{prior\_plots}
\label{classes/models/@rfvar/rfvar:id98}\label{classes/models/@rfvar/rfvar:prior-plots}
H1 line


\subsubsection{Syntax}
\label{classes/models/@rfvar/rfvar:id99}

\subsubsection{Inputs}
\label{classes/models/@rfvar/rfvar:id100}

\subsubsection{Outputs}
\label{classes/models/@rfvar/rfvar:id101}

\subsubsection{Description}
\label{classes/models/@rfvar/rfvar:id102}

\subsubsection{Examples}
\label{classes/models/@rfvar/rfvar:id103}
See also:

Help for rfvar/prior\_plots is inherited from superclass RISE\_GENERIC


\bigskip\hrule{}\bigskip

\phantomsection\label{classes/models/@rfvar/rfvar:report}
\textbf{REPORT} assigns the elements of interest to a rise\_report.report object


\section{Syntax}
\label{classes/models/@rfvar/rfvar:id104}\begin{description}
\item[{::}] \leavevmode\begin{itemize}
\item {} 
REPORT(rise.empty(0)) : displays the default inputs

\item {} 
REPORT(obj,destination\_root,rep\_items) : assigns the reported
elements in rep\_items to destination\_root

\item {} 
REPORT(obj,destination\_root,rep\_items,varargin) : assigns varargin to
obj before doing the rest

\end{itemize}

\end{description}


\section{Inputs}
\label{classes/models/@rfvar/rfvar:id105}\begin{itemize}
\item {} 
obj : {[}rise\textbar{}dsge{]}

\item {} 
destination\_root : {[}rise\_report.report{]} : handle for the actual report

\item {} 
rep\_items : {[}char\textbar{}cellstr{]} : list of desired items to report. This list
can only include : `endogenous', `exogenous', `observables',
`parameters', `solution', `estimation', `estimation\_statistics',
`equations', `code'

\end{itemize}


\section{Outputs}
\label{classes/models/@rfvar/rfvar:id106}
none


\section{Description}
\label{classes/models/@rfvar/rfvar:id107}

\section{Examples}
\label{classes/models/@rfvar/rfvar:id108}
See also:

Help for rfvar/report is inherited from superclass RISE\_GENERIC


\bigskip\hrule{}\bigskip

\phantomsection\label{classes/models/@rfvar/rfvar:rfvar}
\textbf{rfvar}
\begin{quote}

-- no help found
\end{quote}


\bigskip\hrule{}\bigskip



\subsection{set}
\label{classes/models/@rfvar/rfvar:id109}\label{classes/models/@rfvar/rfvar:set}
H1 line


\subsubsection{Syntax}
\label{classes/models/@rfvar/rfvar:id110}

\subsubsection{Inputs}
\label{classes/models/@rfvar/rfvar:id111}

\subsubsection{Outputs}
\label{classes/models/@rfvar/rfvar:id112}

\subsubsection{Description}
\label{classes/models/@rfvar/rfvar:id113}

\subsubsection{Examples}
\label{classes/models/@rfvar/rfvar:id114}
See also:

Help for rfvar/set is inherited from superclass RISE\_GENERIC


\bigskip\hrule{}\bigskip



\subsection{set\_solution\_to\_companion}
\label{classes/models/@rfvar/rfvar:set-solution-to-companion}\label{classes/models/@rfvar/rfvar:id115}
H1 line


\subsubsection{Syntax}
\label{classes/models/@rfvar/rfvar:id116}

\subsubsection{Inputs}
\label{classes/models/@rfvar/rfvar:id117}

\subsubsection{Outputs}
\label{classes/models/@rfvar/rfvar:id118}

\subsubsection{Description}
\label{classes/models/@rfvar/rfvar:id119}

\subsubsection{Examples}
\label{classes/models/@rfvar/rfvar:id120}
See also:

Help for rfvar/set\_solution\_to\_companion is inherited from superclass SVAR


\bigskip\hrule{}\bigskip



\subsection{simulate}
\label{classes/models/@rfvar/rfvar:id121}\label{classes/models/@rfvar/rfvar:simulate}
H1 line


\subsubsection{Syntax}
\label{classes/models/@rfvar/rfvar:id122}

\subsubsection{Inputs}
\label{classes/models/@rfvar/rfvar:id123}

\subsubsection{Outputs}
\label{classes/models/@rfvar/rfvar:id124}

\subsubsection{Description}
\label{classes/models/@rfvar/rfvar:id125}

\subsubsection{Examples}
\label{classes/models/@rfvar/rfvar:id126}
See also:

Help for rfvar/simulate is inherited from superclass RISE\_GENERIC


\bigskip\hrule{}\bigskip



\subsection{simulation\_diagnostics}
\label{classes/models/@rfvar/rfvar:id127}\label{classes/models/@rfvar/rfvar:simulation-diagnostics}
H1 line


\subsubsection{Syntax}
\label{classes/models/@rfvar/rfvar:id128}

\subsubsection{Inputs}
\label{classes/models/@rfvar/rfvar:id129}

\subsubsection{Outputs}
\label{classes/models/@rfvar/rfvar:id130}

\subsubsection{Description}
\label{classes/models/@rfvar/rfvar:id131}

\subsubsection{Examples}
\label{classes/models/@rfvar/rfvar:id132}
See also:

Help for rfvar/simulation\_diagnostics is inherited from superclass RISE\_GENERIC


\bigskip\hrule{}\bigskip



\subsection{solve}
\label{classes/models/@rfvar/rfvar:id133}\label{classes/models/@rfvar/rfvar:solve}
H1 line


\subsubsection{Syntax}
\label{classes/models/@rfvar/rfvar:id134}

\subsubsection{Inputs}
\label{classes/models/@rfvar/rfvar:id135}

\subsubsection{Outputs}
\label{classes/models/@rfvar/rfvar:id136}

\subsubsection{Description}
\label{classes/models/@rfvar/rfvar:id137}

\subsubsection{Examples}
\label{classes/models/@rfvar/rfvar:id138}
See also:


\bigskip\hrule{}\bigskip



\subsection{stoch\_simul}
\label{classes/models/@rfvar/rfvar:id139}\label{classes/models/@rfvar/rfvar:stoch-simul}
H1 line


\subsubsection{Syntax}
\label{classes/models/@rfvar/rfvar:id140}

\subsubsection{Inputs}
\label{classes/models/@rfvar/rfvar:id141}

\subsubsection{Outputs}
\label{classes/models/@rfvar/rfvar:id142}

\subsubsection{Description}
\label{classes/models/@rfvar/rfvar:id143}

\subsubsection{Examples}
\label{classes/models/@rfvar/rfvar:id144}
See also:

Help for rfvar/stoch\_simul is inherited from superclass RISE\_GENERIC


\bigskip\hrule{}\bigskip



\subsection{structural\_form}
\label{classes/models/@rfvar/rfvar:id145}\label{classes/models/@rfvar/rfvar:structural-form}
H1 line


\subsubsection{Syntax}
\label{classes/models/@rfvar/rfvar:id146}

\subsubsection{Inputs}
\label{classes/models/@rfvar/rfvar:id147}

\subsubsection{Outputs}
\label{classes/models/@rfvar/rfvar:id148}

\subsubsection{Description}
\label{classes/models/@rfvar/rfvar:id149}

\subsubsection{Examples}
\label{classes/models/@rfvar/rfvar:id150}
See also:


\bigskip\hrule{}\bigskip

\phantomsection\label{classes/models/@rfvar/rfvar:template}
\textbf{template}
\begin{quote}

-- no help found
\end{quote}


\bigskip\hrule{}\bigskip



\subsection{theoretical\_autocorrelations}
\label{classes/models/@rfvar/rfvar:theoretical-autocorrelations}\label{classes/models/@rfvar/rfvar:id151}
H1 line


\subsubsection{Syntax}
\label{classes/models/@rfvar/rfvar:id152}

\subsubsection{Inputs}
\label{classes/models/@rfvar/rfvar:id153}

\subsubsection{Outputs}
\label{classes/models/@rfvar/rfvar:id154}

\subsubsection{Description}
\label{classes/models/@rfvar/rfvar:id155}

\subsubsection{Examples}
\label{classes/models/@rfvar/rfvar:id156}
See also:

Help for rfvar/theoretical\_autocorrelations is inherited from superclass RISE\_GENERIC


\bigskip\hrule{}\bigskip



\subsection{theoretical\_autocovariances}
\label{classes/models/@rfvar/rfvar:id157}\label{classes/models/@rfvar/rfvar:theoretical-autocovariances}
H1 line


\subsubsection{Syntax}
\label{classes/models/@rfvar/rfvar:id158}

\subsubsection{Inputs}
\label{classes/models/@rfvar/rfvar:id159}

\subsubsection{Outputs}
\label{classes/models/@rfvar/rfvar:id160}

\subsubsection{Description}
\label{classes/models/@rfvar/rfvar:id161}

\subsubsection{Examples}
\label{classes/models/@rfvar/rfvar:id162}
See also:

Help for rfvar/theoretical\_autocovariances is inherited from superclass RISE\_GENERIC


\bigskip\hrule{}\bigskip



\subsection{variance\_decomposition}
\label{classes/models/@rfvar/rfvar:id163}\label{classes/models/@rfvar/rfvar:variance-decomposition}
H1 line


\subsubsection{Syntax}
\label{classes/models/@rfvar/rfvar:id164}

\subsubsection{Inputs}
\label{classes/models/@rfvar/rfvar:id165}

\subsubsection{Outputs}
\label{classes/models/@rfvar/rfvar:id166}

\subsubsection{Description}
\label{classes/models/@rfvar/rfvar:id167}

\subsubsection{Examples}
\label{classes/models/@rfvar/rfvar:id168}
See also:

Help for rfvar/variance\_decomposition is inherited from superclass RISE\_GENERIC


\chapter{Structural VAR modeling}
\label{classes/models/@svar/svar:structural-var-modeling}\label{classes/models/@svar/svar::doc}

\section{methods}
\label{classes/models/@svar/svar:methods}\begin{itemize}
\item {} 
{[} {\hyperref[classes/models/@svar/svar:check-optimum]{check\_optimum}} {]}(svar/check\_optimum)

\item {} 
{[} {\hyperref[classes/models/@svar/svar:draw-parameter]{draw\_parameter}} {]}(svar/draw\_parameter)

\item {} 
{[} {\hyperref[classes/models/@svar/svar:estimate]{estimate}} {]}(svar/estimate)

\item {} 
{[} {\hyperref[classes/models/@svar/svar:forecast]{forecast}} {]}(svar/forecast)

\item {} 
{[} {\hyperref[classes/models/@svar/svar:get]{get}} {]}(svar/get)

\item {} 
{[} {\hyperref[classes/models/@svar/svar:historical-decomposition]{historical\_decomposition}} {]}(svar/historical\_decomposition)

\item {} 
{[} {\hyperref[classes/models/@svar/svar:irf]{irf}} {]}(svar/irf)

\item {} 
{[} {\hyperref[classes/models/@svar/svar:isnan]{isnan}} {]}(svar/isnan)

\item {} 
{[} {\hyperref[classes/models/@svar/svar:load-parameters]{load\_parameters}} {]}(svar/load\_parameters)

\item {} 
{[} {\hyperref[classes/models/@svar/svar:log-marginal-data-density]{log\_marginal\_data\_density}} {]}(svar/log\_marginal\_data\_density)

\item {} 
{[} {\hyperref[classes/models/@svar/svar:log-posterior-kernel]{log\_posterior\_kernel}} {]}(svar/log\_posterior\_kernel)

\item {} 
{[} {\hyperref[classes/models/@svar/svar:log-prior-density]{log\_prior\_density}} {]}(svar/log\_prior\_density)

\item {} 
{[} {\hyperref[classes/models/@svar/svar:msvar-priors]{msvar\_priors}} {]}(svar/msvar\_priors)

\item {} 
{[} {\hyperref[classes/models/@svar/svar:posterior-marginal-and-prior-densities]{posterior\_marginal\_and\_prior\_densities}} {]}(svar/posterior\_marginal\_and\_prior\_densities)

\item {} 
{[} {\hyperref[classes/models/@svar/svar:posterior-simulator]{posterior\_simulator}} {]}(svar/posterior\_simulator)

\item {} 
{[} {\hyperref[classes/models/@svar/svar:print-estimation-results]{print\_estimation\_results}} {]}(svar/print\_estimation\_results)

\item {} 
{[} {\hyperref[classes/models/@svar/svar:prior-plots]{prior\_plots}} {]}(svar/prior\_plots)

\item {} 
{[} {\hyperref[classes/models/@svar/svar:report]{report}} {]}(svar/report)

\item {} 
{[} {\hyperref[classes/models/@svar/svar:set]{set}} {]}(svar/set)

\item {} 
{[} {\hyperref[classes/models/@svar/svar:set-solution-to-companion]{set\_solution\_to\_companion}} {]}(svar/set\_solution\_to\_companion)

\item {} 
{[} {\hyperref[classes/models/@svar/svar:simulate]{simulate}} {]}(svar/simulate)

\item {} 
{[} {\hyperref[classes/models/@svar/svar:simulation-diagnostics]{simulation\_diagnostics}} {]}(svar/simulation\_diagnostics)

\item {} 
{[} {\hyperref[classes/models/@svar/svar:solve]{solve}} {]}(svar/solve)

\item {} 
{[} {\hyperref[classes/models/@svar/svar:stoch-simul]{stoch\_simul}} {]}(svar/stoch\_simul)

\item {} 
{[} {\hyperref[classes/models/@svar/svar:svar]{svar}} {]}(svar/svar)

\item {} 
{[} {\hyperref[classes/models/@svar/svar:template]{template}} {]}(svar/template)

\item {} 
{[} {\hyperref[classes/models/@svar/svar:theoretical-autocorrelations]{theoretical\_autocorrelations}} {]}(svar/theoretical\_autocorrelations)

\item {} 
{[} {\hyperref[classes/models/@svar/svar:theoretical-autocovariances]{theoretical\_autocovariances}} {]}(svar/theoretical\_autocovariances)

\item {} 
{[} {\hyperref[classes/models/@svar/svar:variance-decomposition]{variance\_decomposition}} {]}(svar/variance\_decomposition)

\end{itemize}


\section{properties}
\label{classes/models/@svar/svar:properties}\begin{itemize}
\item {} 
{[}constant{]} -

\item {} 
{[}nlags{]} -

\item {} 
{[}legend{]} -

\item {} 
{[}endogenous{]} -

\item {} 
{[}exogenous{]} -

\item {} 
{[}parameters{]} -

\item {} 
{[}observables{]} -

\item {} 
{[}markov\_chains{]} -

\item {} 
{[}options{]} -

\item {} 
{[}estimation{]} -

\item {} 
{[}solution{]} -

\item {} 
{[}filtering{]} -

\end{itemize}


\section{Synopsis and description on methods}
\label{classes/models/@svar/svar:synopsis-and-description-on-methods}

\bigskip\hrule{}\bigskip



\subsection{check\_optimum}
\label{classes/models/@svar/svar:check-optimum}\label{classes/models/@svar/svar:id1}
H1 line


\subsubsection{Syntax}
\label{classes/models/@svar/svar:syntax}

\subsubsection{Inputs}
\label{classes/models/@svar/svar:inputs}

\subsubsection{Outputs}
\label{classes/models/@svar/svar:outputs}

\subsubsection{Description}
\label{classes/models/@svar/svar:description}

\subsubsection{Examples}
\label{classes/models/@svar/svar:examples}
See also:

Help for svar/check\_optimum is inherited from superclass RISE\_GENERIC


\bigskip\hrule{}\bigskip



\subsection{draw\_parameter}
\label{classes/models/@svar/svar:id2}\label{classes/models/@svar/svar:draw-parameter}
H1 line


\subsubsection{Syntax}
\label{classes/models/@svar/svar:id3}

\subsubsection{Inputs}
\label{classes/models/@svar/svar:id4}

\subsubsection{Outputs}
\label{classes/models/@svar/svar:id5}

\subsubsection{Description}
\label{classes/models/@svar/svar:id6}

\subsubsection{Examples}
\label{classes/models/@svar/svar:id7}
See also:

Help for svar/draw\_parameter is inherited from superclass RISE\_GENERIC


\bigskip\hrule{}\bigskip



\subsection{estimate}
\label{classes/models/@svar/svar:estimate}\label{classes/models/@svar/svar:id8}
H1 line


\subsubsection{Syntax}
\label{classes/models/@svar/svar:id9}

\subsubsection{Inputs}
\label{classes/models/@svar/svar:id10}

\subsubsection{Outputs}
\label{classes/models/@svar/svar:id11}

\subsubsection{Description}
\label{classes/models/@svar/svar:id12}

\subsubsection{Examples}
\label{classes/models/@svar/svar:id13}
See also:

Help for svar/estimate is inherited from superclass RISE\_GENERIC


\bigskip\hrule{}\bigskip



\subsection{forecast}
\label{classes/models/@svar/svar:id14}\label{classes/models/@svar/svar:forecast}
H1 line


\subsubsection{Syntax}
\label{classes/models/@svar/svar:id15}

\subsubsection{Inputs}
\label{classes/models/@svar/svar:id16}

\subsubsection{Outputs}
\label{classes/models/@svar/svar:id17}

\subsubsection{Description}
\label{classes/models/@svar/svar:id18}

\subsubsection{Examples}
\label{classes/models/@svar/svar:id19}
See also:

Help for svar/forecast is inherited from superclass RISE\_GENERIC


\bigskip\hrule{}\bigskip



\subsection{get}
\label{classes/models/@svar/svar:id20}\label{classes/models/@svar/svar:get}
H1 line


\subsubsection{Syntax}
\label{classes/models/@svar/svar:id21}

\subsubsection{Inputs}
\label{classes/models/@svar/svar:id22}

\subsubsection{Outputs}
\label{classes/models/@svar/svar:id23}

\subsubsection{Description}
\label{classes/models/@svar/svar:id24}

\subsubsection{Examples}
\label{classes/models/@svar/svar:id25}
See also:

Help for svar/get is inherited from superclass RISE\_GENERIC


\bigskip\hrule{}\bigskip



\subsection{historical\_decomposition}
\label{classes/models/@svar/svar:historical-decomposition}\label{classes/models/@svar/svar:id26}
H1 line


\subsubsection{Syntax}
\label{classes/models/@svar/svar:id27}

\subsubsection{Inputs}
\label{classes/models/@svar/svar:id28}

\subsubsection{Outputs}
\label{classes/models/@svar/svar:id29}

\subsubsection{Description}
\label{classes/models/@svar/svar:id30}

\subsubsection{Examples}
\label{classes/models/@svar/svar:id31}
See also:

Help for svar/historical\_decomposition is inherited from superclass RISE\_GENERIC


\bigskip\hrule{}\bigskip



\subsection{irf}
\label{classes/models/@svar/svar:id32}\label{classes/models/@svar/svar:irf}
H1 line


\subsubsection{Syntax}
\label{classes/models/@svar/svar:id33}

\subsubsection{Inputs}
\label{classes/models/@svar/svar:id34}

\subsubsection{Outputs}
\label{classes/models/@svar/svar:id35}

\subsubsection{Description}
\label{classes/models/@svar/svar:id36}

\subsubsection{Examples}
\label{classes/models/@svar/svar:id37}
See also:

Help for svar/irf is inherited from superclass RISE\_GENERIC


\bigskip\hrule{}\bigskip



\subsection{isnan}
\label{classes/models/@svar/svar:isnan}\label{classes/models/@svar/svar:id38}
H1 line


\subsubsection{Syntax}
\label{classes/models/@svar/svar:id39}

\subsubsection{Inputs}
\label{classes/models/@svar/svar:id40}

\subsubsection{Outputs}
\label{classes/models/@svar/svar:id41}

\subsubsection{Description}
\label{classes/models/@svar/svar:id42}

\subsubsection{Examples}
\label{classes/models/@svar/svar:id43}
See also:

Help for svar/isnan is inherited from superclass RISE\_GENERIC


\bigskip\hrule{}\bigskip



\subsection{load\_parameters}
\label{classes/models/@svar/svar:id44}\label{classes/models/@svar/svar:load-parameters}
H1 line


\subsubsection{Syntax}
\label{classes/models/@svar/svar:id45}

\subsubsection{Inputs}
\label{classes/models/@svar/svar:id46}

\subsubsection{Outputs}
\label{classes/models/@svar/svar:id47}

\subsubsection{Description}
\label{classes/models/@svar/svar:id48}

\subsubsection{Examples}
\label{classes/models/@svar/svar:id49}
See also:

Help for svar/load\_parameters is inherited from superclass RISE\_GENERIC


\bigskip\hrule{}\bigskip



\subsection{log\_marginal\_data\_density}
\label{classes/models/@svar/svar:id50}\label{classes/models/@svar/svar:log-marginal-data-density}
H1 line


\subsubsection{Syntax}
\label{classes/models/@svar/svar:id51}

\subsubsection{Inputs}
\label{classes/models/@svar/svar:id52}

\subsubsection{Outputs}
\label{classes/models/@svar/svar:id53}

\subsubsection{Description}
\label{classes/models/@svar/svar:id54}

\subsubsection{Examples}
\label{classes/models/@svar/svar:id55}
See also:

Help for svar/log\_marginal\_data\_density is inherited from superclass RISE\_GENERIC


\bigskip\hrule{}\bigskip



\subsection{log\_posterior\_kernel}
\label{classes/models/@svar/svar:log-posterior-kernel}\label{classes/models/@svar/svar:id56}
H1 line


\subsubsection{Syntax}
\label{classes/models/@svar/svar:id57}

\subsubsection{Inputs}
\label{classes/models/@svar/svar:id58}

\subsubsection{Outputs}
\label{classes/models/@svar/svar:id59}

\subsubsection{Description}
\label{classes/models/@svar/svar:id60}

\subsubsection{Examples}
\label{classes/models/@svar/svar:id61}
See also:

Help for svar/log\_posterior\_kernel is inherited from superclass RISE\_GENERIC


\bigskip\hrule{}\bigskip



\subsection{log\_prior\_density}
\label{classes/models/@svar/svar:id62}\label{classes/models/@svar/svar:log-prior-density}
H1 line


\subsubsection{Syntax}
\label{classes/models/@svar/svar:id63}

\subsubsection{Inputs}
\label{classes/models/@svar/svar:id64}

\subsubsection{Outputs}
\label{classes/models/@svar/svar:id65}

\subsubsection{Description}
\label{classes/models/@svar/svar:id66}

\subsubsection{Examples}
\label{classes/models/@svar/svar:id67}
See also:

Help for svar/log\_prior\_density is inherited from superclass RISE\_GENERIC


\bigskip\hrule{}\bigskip



\subsection{msvar\_priors}
\label{classes/models/@svar/svar:msvar-priors}\label{classes/models/@svar/svar:id68}
H1 line


\subsubsection{Syntax}
\label{classes/models/@svar/svar:id69}

\subsubsection{Inputs}
\label{classes/models/@svar/svar:id70}

\subsubsection{Outputs}
\label{classes/models/@svar/svar:id71}

\subsubsection{Description}
\label{classes/models/@svar/svar:id72}

\subsubsection{Examples}
\label{classes/models/@svar/svar:id73}
See also:


\bigskip\hrule{}\bigskip



\subsection{posterior\_marginal\_and\_prior\_densities}
\label{classes/models/@svar/svar:id74}\label{classes/models/@svar/svar:posterior-marginal-and-prior-densities}
H1 line


\subsubsection{Syntax}
\label{classes/models/@svar/svar:id75}

\subsubsection{Inputs}
\label{classes/models/@svar/svar:id76}

\subsubsection{Outputs}
\label{classes/models/@svar/svar:id77}

\subsubsection{Description}
\label{classes/models/@svar/svar:id78}

\subsubsection{Examples}
\label{classes/models/@svar/svar:id79}
See also:

Help for svar/posterior\_marginal\_and\_prior\_densities is inherited from superclass RISE\_GENERIC


\bigskip\hrule{}\bigskip



\subsection{posterior\_simulator}
\label{classes/models/@svar/svar:posterior-simulator}\label{classes/models/@svar/svar:id80}
H1 line


\subsubsection{Syntax}
\label{classes/models/@svar/svar:id81}

\subsubsection{Inputs}
\label{classes/models/@svar/svar:id82}

\subsubsection{Outputs}
\label{classes/models/@svar/svar:id83}

\subsubsection{Description}
\label{classes/models/@svar/svar:id84}

\subsubsection{Examples}
\label{classes/models/@svar/svar:id85}
See also:

Help for svar/posterior\_simulator is inherited from superclass RISE\_GENERIC


\bigskip\hrule{}\bigskip



\subsection{print\_estimation\_results}
\label{classes/models/@svar/svar:print-estimation-results}\label{classes/models/@svar/svar:id86}
H1 line


\subsubsection{Syntax}
\label{classes/models/@svar/svar:id87}

\subsubsection{Inputs}
\label{classes/models/@svar/svar:id88}

\subsubsection{Outputs}
\label{classes/models/@svar/svar:id89}

\subsubsection{Description}
\label{classes/models/@svar/svar:id90}

\subsubsection{Examples}
\label{classes/models/@svar/svar:id91}
See also:

Help for svar/print\_estimation\_results is inherited from superclass RISE\_GENERIC


\bigskip\hrule{}\bigskip



\subsection{prior\_plots}
\label{classes/models/@svar/svar:id92}\label{classes/models/@svar/svar:prior-plots}
H1 line


\subsubsection{Syntax}
\label{classes/models/@svar/svar:id93}

\subsubsection{Inputs}
\label{classes/models/@svar/svar:id94}

\subsubsection{Outputs}
\label{classes/models/@svar/svar:id95}

\subsubsection{Description}
\label{classes/models/@svar/svar:id96}

\subsubsection{Examples}
\label{classes/models/@svar/svar:id97}
See also:

Help for svar/prior\_plots is inherited from superclass RISE\_GENERIC


\bigskip\hrule{}\bigskip

\phantomsection\label{classes/models/@svar/svar:report}
\textbf{REPORT} assigns the elements of interest to a rise\_report.report object


\section{Syntax}
\label{classes/models/@svar/svar:id98}\begin{description}
\item[{::}] \leavevmode\begin{itemize}
\item {} 
REPORT(rise.empty(0)) : displays the default inputs

\item {} 
REPORT(obj,destination\_root,rep\_items) : assigns the reported
elements in rep\_items to destination\_root

\item {} 
REPORT(obj,destination\_root,rep\_items,varargin) : assigns varargin to
obj before doing the rest

\end{itemize}

\end{description}


\section{Inputs}
\label{classes/models/@svar/svar:id99}\begin{itemize}
\item {} 
obj : {[}rise\textbar{}dsge{]}

\item {} 
destination\_root : {[}rise\_report.report{]} : handle for the actual report

\item {} 
rep\_items : {[}char\textbar{}cellstr{]} : list of desired items to report. This list
can only include : `endogenous', `exogenous', `observables',
`parameters', `solution', `estimation', `estimation\_statistics',
`equations', `code'

\end{itemize}


\section{Outputs}
\label{classes/models/@svar/svar:id100}
none


\section{Description}
\label{classes/models/@svar/svar:id101}

\section{Examples}
\label{classes/models/@svar/svar:id102}
See also:

Help for svar/report is inherited from superclass RISE\_GENERIC


\bigskip\hrule{}\bigskip



\subsection{set}
\label{classes/models/@svar/svar:id103}\label{classes/models/@svar/svar:set}
H1 line


\subsubsection{Syntax}
\label{classes/models/@svar/svar:id104}

\subsubsection{Inputs}
\label{classes/models/@svar/svar:id105}

\subsubsection{Outputs}
\label{classes/models/@svar/svar:id106}

\subsubsection{Description}
\label{classes/models/@svar/svar:id107}

\subsubsection{Examples}
\label{classes/models/@svar/svar:id108}
See also:

Help for svar/set is inherited from superclass RISE\_GENERIC


\bigskip\hrule{}\bigskip



\subsection{set\_solution\_to\_companion}
\label{classes/models/@svar/svar:id109}\label{classes/models/@svar/svar:set-solution-to-companion}
H1 line


\subsubsection{Syntax}
\label{classes/models/@svar/svar:id110}

\subsubsection{Inputs}
\label{classes/models/@svar/svar:id111}

\subsubsection{Outputs}
\label{classes/models/@svar/svar:id112}

\subsubsection{Description}
\label{classes/models/@svar/svar:id113}

\subsubsection{Examples}
\label{classes/models/@svar/svar:id114}
See also:


\bigskip\hrule{}\bigskip



\subsection{simulate}
\label{classes/models/@svar/svar:simulate}\label{classes/models/@svar/svar:id115}
H1 line


\subsubsection{Syntax}
\label{classes/models/@svar/svar:id116}

\subsubsection{Inputs}
\label{classes/models/@svar/svar:id117}

\subsubsection{Outputs}
\label{classes/models/@svar/svar:id118}

\subsubsection{Description}
\label{classes/models/@svar/svar:id119}

\subsubsection{Examples}
\label{classes/models/@svar/svar:id120}
See also:

Help for svar/simulate is inherited from superclass RISE\_GENERIC


\bigskip\hrule{}\bigskip



\subsection{simulation\_diagnostics}
\label{classes/models/@svar/svar:simulation-diagnostics}\label{classes/models/@svar/svar:id121}
H1 line


\subsubsection{Syntax}
\label{classes/models/@svar/svar:id122}

\subsubsection{Inputs}
\label{classes/models/@svar/svar:id123}

\subsubsection{Outputs}
\label{classes/models/@svar/svar:id124}

\subsubsection{Description}
\label{classes/models/@svar/svar:id125}

\subsubsection{Examples}
\label{classes/models/@svar/svar:id126}
See also:

Help for svar/simulation\_diagnostics is inherited from superclass RISE\_GENERIC


\bigskip\hrule{}\bigskip



\subsection{solve}
\label{classes/models/@svar/svar:id127}\label{classes/models/@svar/svar:solve}
H1 line


\subsubsection{Syntax}
\label{classes/models/@svar/svar:id128}

\subsubsection{Inputs}
\label{classes/models/@svar/svar:id129}

\subsubsection{Outputs}
\label{classes/models/@svar/svar:id130}

\subsubsection{Description}
\label{classes/models/@svar/svar:id131}

\subsubsection{Examples}
\label{classes/models/@svar/svar:id132}
See also:


\bigskip\hrule{}\bigskip



\subsection{stoch\_simul}
\label{classes/models/@svar/svar:id133}\label{classes/models/@svar/svar:stoch-simul}
H1 line


\subsubsection{Syntax}
\label{classes/models/@svar/svar:id134}

\subsubsection{Inputs}
\label{classes/models/@svar/svar:id135}

\subsubsection{Outputs}
\label{classes/models/@svar/svar:id136}

\subsubsection{Description}
\label{classes/models/@svar/svar:id137}

\subsubsection{Examples}
\label{classes/models/@svar/svar:id138}
See also:

Help for svar/stoch\_simul is inherited from superclass RISE\_GENERIC


\bigskip\hrule{}\bigskip

\phantomsection\label{classes/models/@svar/svar:svar}
\textbf{svar}
\begin{quote}

-- no help found
\end{quote}


\bigskip\hrule{}\bigskip

\phantomsection\label{classes/models/@svar/svar:template}
\textbf{template}
\begin{quote}

-- no help found
\end{quote}


\bigskip\hrule{}\bigskip



\subsection{theoretical\_autocorrelations}
\label{classes/models/@svar/svar:theoretical-autocorrelations}\label{classes/models/@svar/svar:id139}
H1 line


\subsubsection{Syntax}
\label{classes/models/@svar/svar:id140}

\subsubsection{Inputs}
\label{classes/models/@svar/svar:id141}

\subsubsection{Outputs}
\label{classes/models/@svar/svar:id142}

\subsubsection{Description}
\label{classes/models/@svar/svar:id143}

\subsubsection{Examples}
\label{classes/models/@svar/svar:id144}
See also:

Help for svar/theoretical\_autocorrelations is inherited from superclass RISE\_GENERIC


\bigskip\hrule{}\bigskip



\subsection{theoretical\_autocovariances}
\label{classes/models/@svar/svar:id145}\label{classes/models/@svar/svar:theoretical-autocovariances}
H1 line


\subsubsection{Syntax}
\label{classes/models/@svar/svar:id146}

\subsubsection{Inputs}
\label{classes/models/@svar/svar:id147}

\subsubsection{Outputs}
\label{classes/models/@svar/svar:id148}

\subsubsection{Description}
\label{classes/models/@svar/svar:id149}

\subsubsection{Examples}
\label{classes/models/@svar/svar:id150}
See also:

Help for svar/theoretical\_autocovariances is inherited from superclass RISE\_GENERIC


\bigskip\hrule{}\bigskip



\subsection{variance\_decomposition}
\label{classes/models/@svar/svar:variance-decomposition}\label{classes/models/@svar/svar:id151}
H1 line


\subsubsection{Syntax}
\label{classes/models/@svar/svar:id152}

\subsubsection{Inputs}
\label{classes/models/@svar/svar:id153}

\subsubsection{Outputs}
\label{classes/models/@svar/svar:id154}

\subsubsection{Description}
\label{classes/models/@svar/svar:id155}

\subsubsection{Examples}
\label{classes/models/@svar/svar:id156}
See also:

Help for svar/variance\_decomposition is inherited from superclass RISE\_GENERIC


\chapter{Time series}
\label{classes/time_series/@ts/ts:time-series}\label{classes/time_series/@ts/ts::doc}

\section{Constructor}
\label{classes/time_series/@ts/ts:constructor}\begin{itemize}
\item {} 
{[} {\hyperref[classes/time_series/@ts/ts:ts]{ts}} {]}(ts/ts)

\end{itemize}


\section{Visualization}
\label{classes/time_series/@ts/ts:visualization}\begin{itemize}
\item {} 
{[} {\hyperref[classes/time_series/@ts/ts:head]{head}} {]}(ts/head)

\item {} 
{[} {\hyperref[classes/time_series/@ts/ts:index]{index}} {]}(ts/index)

\item {} 
{[} {\hyperref[classes/time_series/@ts/ts:describe]{describe}} {]}(ts/describe)

\item {} 
{[} {\hyperref[classes/time_series/@ts/ts:display]{display}} {]}(ts/display)

\item {} 
{[} {\hyperref[classes/time_series/@ts/ts:jbtest]{jbtest}} {]}(ts/jbtest)

\item {} 
{[} {\hyperref[classes/time_series/@ts/ts:kurtosis]{kurtosis}} {]}(ts/kurtosis)

\item {} 
{[} {\hyperref[classes/time_series/@ts/ts:isfinite]{isfinite}} {]}(ts/isfinite)

\item {} 
{[} {\hyperref[classes/time_series/@ts/ts:isinf]{isinf}} {]}(ts/isinf)

\item {} 
{[} {\hyperref[classes/time_series/@ts/ts:isnan]{isnan}} {]}(ts/isnan)

\item {} 
{[} {\hyperref[classes/time_series/@ts/ts:ge]{ge}} {]}(ts/ge)

\item {} 
{[} {\hyperref[classes/time_series/@ts/ts:get]{get}} {]}(ts/get)

\item {} 
{[} {\hyperref[classes/time_series/@ts/ts:gt]{gt}} {]}(ts/gt)

\item {} 
{[} {\hyperref[classes/time_series/@ts/ts:le]{le}} {]}(ts/le)

\item {} 
{[} {\hyperref[classes/time_series/@ts/ts:lt]{lt}} {]}(ts/lt)

\item {} 
{[} {\hyperref[classes/time_series/@ts/ts:max]{max}} {]}(ts/max)

\item {} 
{[} {\hyperref[classes/time_series/@ts/ts:mean]{mean}} {]}(ts/mean)

\item {} 
{[} {\hyperref[classes/time_series/@ts/ts:median]{median}} {]}(ts/median)

\item {} 
{[} {\hyperref[classes/time_series/@ts/ts:min]{min}} {]}(ts/min)

\item {} 
{[} {\hyperref[classes/time_series/@ts/ts:mode]{mode}} {]}(ts/mode)

\item {} 
{[} {\hyperref[classes/time_series/@ts/ts:ne]{ne}} {]}(ts/ne)

\item {} 
{[} {\hyperref[classes/time_series/@ts/ts:quantile]{quantile}} {]}(ts/quantile)

\item {} 
{[} {\hyperref[classes/time_series/@ts/ts:range]{range}} {]}(ts/range)

\item {} 
{[} {\hyperref[classes/time_series/@ts/ts:skewness]{skewness}} {]}(ts/skewness)

\item {} 
{[} {\hyperref[classes/time_series/@ts/ts:sum]{sum}} {]}(ts/sum)

\item {} 
{[} {\hyperref[classes/time_series/@ts/ts:tail]{tail}} {]}(ts/tail)

\item {} 
{[} {\hyperref[classes/time_series/@ts/ts:var]{var}} {]}(ts/var)

\item {} 
{[} {\hyperref[classes/time_series/@ts/ts:std]{std}} {]}(ts/std)

\item {} 
{[} {\hyperref[classes/time_series/@ts/ts:spectrum]{spectrum}} {]}(ts/spectrum)

\item {} 
{[} {\hyperref[classes/time_series/@ts/ts:sort]{sort}} {]}(ts/sort)

\end{itemize}


\section{Graphing}
\label{classes/time_series/@ts/ts:graphing}\begin{itemize}
\item {} 
{[} {\hyperref[classes/time_series/@ts/ts:bar]{bar}} {]}(ts/bar)

\item {} 
{[} {\hyperref[classes/time_series/@ts/ts:barh]{barh}} {]}(ts/barh)

\item {} 
{[} {\hyperref[classes/time_series/@ts/ts:boxplot]{boxplot}} {]}(ts/boxplot)

\item {} 
{[} {\hyperref[classes/time_series/@ts/ts:hist]{hist}} {]}(ts/hist)

\item {} 
{[} {\hyperref[classes/time_series/@ts/ts:plot]{plot}} {]}(ts/plot)

\item {} 
{[} {\hyperref[classes/time_series/@ts/ts:plotyy]{plotyy}} {]}(ts/plotyy)

\end{itemize}


\section{Calculus}
\label{classes/time_series/@ts/ts:calculus}\begin{itemize}
\item {} 
{[} {\hyperref[classes/time_series/@ts/ts:acos]{acos}} {]}(ts/acos)

\item {} 
{[} {\hyperref[classes/time_series/@ts/ts:acosh]{acosh}} {]}(ts/acosh)

\item {} 
{[} {\hyperref[classes/time_series/@ts/ts:acot]{acot}} {]}(ts/acot)

\item {} 
{[} {\hyperref[classes/time_series/@ts/ts:acoth]{acoth}} {]}(ts/acoth)

\item {} 
{[} {\hyperref[classes/time_series/@ts/ts:aggregate]{aggregate}} {]}(ts/aggregate)

\item {} 
{[} {\hyperref[classes/time_series/@ts/ts:allmean]{allmean}} {]}(ts/allmean)

\item {} 
{[} {\hyperref[classes/time_series/@ts/ts:apply]{apply}} {]}(ts/apply)

\item {} 
{[} {\hyperref[classes/time_series/@ts/ts:asin]{asin}} {]}(ts/asin)

\item {} 
{[} {\hyperref[classes/time_series/@ts/ts:asinh]{asinh}} {]}(ts/asinh)

\item {} 
{[} {\hyperref[classes/time_series/@ts/ts:atan]{atan}} {]}(ts/atan)

\item {} 
{[} {\hyperref[classes/time_series/@ts/ts:atanh]{atanh}} {]}(ts/atanh)

\item {} 
{[} {\hyperref[classes/time_series/@ts/ts:bsxfun]{bsxfun}} {]}(ts/bsxfun)

\item {} 
{[} {\hyperref[classes/time_series/@ts/ts:corr]{corr}} {]}(ts/corr)

\item {} 
{[} {\hyperref[classes/time_series/@ts/ts:corrcoef]{corrcoef}} {]}(ts/corrcoef)

\item {} 
{[} {\hyperref[classes/time_series/@ts/ts:cos]{cos}} {]}(ts/cos)

\item {} 
{[} {\hyperref[classes/time_series/@ts/ts:cosh]{cosh}} {]}(ts/cosh)

\item {} 
{[} {\hyperref[classes/time_series/@ts/ts:cot]{cot}} {]}(ts/cot)

\item {} 
{[} {\hyperref[classes/time_series/@ts/ts:coth]{coth}} {]}(ts/coth)

\item {} 
{[} {\hyperref[classes/time_series/@ts/ts:cov]{cov}} {]}(ts/cov)

\item {} 
{[} {\hyperref[classes/time_series/@ts/ts:cumprod]{cumprod}} {]}(ts/cumprod)

\item {} 
{[} {\hyperref[classes/time_series/@ts/ts:cumsum]{cumsum}} {]}(ts/cumsum)

\item {} 
{[} {\hyperref[classes/time_series/@ts/ts:decompose-series]{decompose\_series}} {]}(ts/decompose\_series)

\item {} 
{[} {\hyperref[classes/time_series/@ts/ts:eq]{eq}} {]}(ts/eq)

\item {} 
{[} {\hyperref[classes/time_series/@ts/ts:exp]{exp}} {]}(ts/exp)

\item {} 
{[} {\hyperref[classes/time_series/@ts/ts:hpfilter]{hpfilter}} {]}(ts/hpfilter)

\item {} 
{[} {\hyperref[classes/time_series/@ts/ts:interpolate]{interpolate}} {]}(ts/interpolate)

\item {} 
{[} {\hyperref[classes/time_series/@ts/ts:intersect]{intersect}} {]}(ts/intersect)

\item {} 
{[} {\hyperref[classes/time_series/@ts/ts:log]{log}} {]}(ts/log)

\item {} 
{[} {\hyperref[classes/time_series/@ts/ts:minus]{minus}} {]}(ts/minus)

\item {} 
{[} {\hyperref[classes/time_series/@ts/ts:mpower]{mpower}} {]}(ts/mpower)

\item {} 
{[} {\hyperref[classes/time_series/@ts/ts:mrdivide]{mrdivide}} {]}(ts/mrdivide)

\item {} 
{[} {\hyperref[classes/time_series/@ts/ts:mtimes]{mtimes}} {]}(ts/mtimes)

\item {} 
{[} {\hyperref[classes/time_series/@ts/ts:plus]{plus}} {]}(ts/plus)

\item {} 
{[} {\hyperref[classes/time_series/@ts/ts:power]{power}} {]}(ts/power)

\item {} 
{[} {\hyperref[classes/time_series/@ts/ts:rdivide]{rdivide}} {]}(ts/rdivide)

\item {} 
{[} {\hyperref[classes/time_series/@ts/ts:sin]{sin}} {]}(ts/sin)

\item {} 
{[} {\hyperref[classes/time_series/@ts/ts:sinh]{sinh}} {]}(ts/sinh)

\item {} 
{[} {\hyperref[classes/time_series/@ts/ts:transform]{transform}} {]}(ts/transform)

\item {} 
{[} {\hyperref[classes/time_series/@ts/ts:times]{times}} {]}(ts/times)

\item {} 
{[} {\hyperref[classes/time_series/@ts/ts:uminus]{uminus}} {]}(ts/uminus)

\end{itemize}


\section{Lookarounds}
\label{classes/time_series/@ts/ts:lookarounds}\begin{itemize}
\item {} 
{[} {\hyperref[classes/time_series/@ts/ts:pages2struct]{pages2struct}} {]}(ts/pages2struct)

\item {} 
{[} {\hyperref[classes/time_series/@ts/ts:subsasgn]{subsasgn}} {]}(ts/subsasgn)

\item {} 
{[} {\hyperref[classes/time_series/@ts/ts:subsref]{subsref}} {]}(ts/subsref)

\end{itemize}


\section{Utilities}
\label{classes/time_series/@ts/ts:utilities}\begin{itemize}
\item {} 
{[} {\hyperref[classes/time_series/@ts/ts:and]{and}} {]}(ts/and)

\item {} 
{[} {\hyperref[classes/time_series/@ts/ts:cat]{cat}} {]}(ts/cat)

\item {} 
{[} {\hyperref[classes/time_series/@ts/ts:collect]{collect}} {]}(ts/collect)

\item {} 
{[} {\hyperref[classes/time_series/@ts/ts:ctranspose]{ctranspose}} {]}(ts/ctranspose)

\item {} 
{[} {\hyperref[classes/time_series/@ts/ts:double]{double}} {]}(ts/double)

\item {} 
{[} {\hyperref[classes/time_series/@ts/ts:drop]{drop}} {]}(ts/drop)

\item {} 
{[} {\hyperref[classes/time_series/@ts/ts:dummy]{dummy}} {]}(ts/dummy)

\item {} 
{[} {\hyperref[classes/time_series/@ts/ts:expanding]{expanding}} {]}(ts/expanding)

\item {} 
{[} {\hyperref[classes/time_series/@ts/ts:fanchart]{fanchart}} {]}(ts/fanchart)

\item {} 
{[} {\hyperref[classes/time_series/@ts/ts:horzcat]{horzcat}} {]}(ts/horzcat)

\item {} 
{[} {\hyperref[classes/time_series/@ts/ts:nan]{nan}} {]}(ts/nan)

\item {} 
{[} {\hyperref[classes/time_series/@ts/ts:numel]{numel}} {]}(ts/numel)

\item {} 
{[} {\hyperref[classes/time_series/@ts/ts:ones]{ones}} {]}(ts/ones)

\item {} 
{[} {\hyperref[classes/time_series/@ts/ts:rand]{rand}} {]}(ts/rand)

\item {} 
{[} {\hyperref[classes/time_series/@ts/ts:randn]{randn}} {]}(ts/randn)

\item {} 
{[} {\hyperref[classes/time_series/@ts/ts:regress]{regress}} {]}(ts/regress)

\item {} 
{[} {\hyperref[classes/time_series/@ts/ts:reset-start-date]{reset\_start\_date}} {]}(ts/reset\_start\_date)

\item {} 
{[} {\hyperref[classes/time_series/@ts/ts:rolling]{rolling}} {]}(ts/rolling)

\item {} 
{[} {\hyperref[classes/time_series/@ts/ts:automatic-model-selection]{automatic\_model\_selection}} {]}(ts/automatic\_model\_selection)

\item {} 
{[} {\hyperref[classes/time_series/@ts/ts:transpose]{transpose}} {]}(ts/transpose)

\item {} 
{[} {\hyperref[classes/time_series/@ts/ts:zeros]{zeros}} {]}(ts/zeros)

\item {} 
{[} {\hyperref[classes/time_series/@ts/ts:values]{values}} {]}(ts/values)

\item {} 
{[} {\hyperref[classes/time_series/@ts/ts:step-dummy]{step\_dummy}} {]}(ts/step\_dummy)

\end{itemize}


\section{properties}
\label{classes/time_series/@ts/ts:properties}\begin{itemize}
\item {} 
{[}varnames{]} -

\item {} 
{[}start{]} -

\item {} 
{[}finish{]} -

\item {} 
{[}frequency{]} -

\item {} 
{[}NumberOfObservations{]} -

\item {} 
{[}NumberOfPages{]} -

\item {} 
{[}NumberOfVariables{]} -

\end{itemize}


\section{Synopsis and description on methods}
\label{classes/time_series/@ts/ts:synopsis-and-description-on-methods}

\bigskip\hrule{}\bigskip



\subsection{acos}
\label{classes/time_series/@ts/ts:acos}\label{classes/time_series/@ts/ts:id1}
H1 line


\subsubsection{Syntax}
\label{classes/time_series/@ts/ts:syntax}

\subsubsection{Inputs}
\label{classes/time_series/@ts/ts:inputs}

\subsubsection{Outputs}
\label{classes/time_series/@ts/ts:outputs}

\subsubsection{Description}
\label{classes/time_series/@ts/ts:description}

\subsubsection{Examples}
\label{classes/time_series/@ts/ts:examples}
See also:


\bigskip\hrule{}\bigskip



\subsection{acosh}
\label{classes/time_series/@ts/ts:id2}\label{classes/time_series/@ts/ts:acosh}
H1 line


\subsubsection{Syntax}
\label{classes/time_series/@ts/ts:id3}

\subsubsection{Inputs}
\label{classes/time_series/@ts/ts:id4}

\subsubsection{Outputs}
\label{classes/time_series/@ts/ts:id5}

\subsubsection{Description}
\label{classes/time_series/@ts/ts:id6}

\subsubsection{Examples}
\label{classes/time_series/@ts/ts:id7}
See also:


\bigskip\hrule{}\bigskip



\subsection{acot}
\label{classes/time_series/@ts/ts:acot}\label{classes/time_series/@ts/ts:id8}
H1 line


\subsubsection{Syntax}
\label{classes/time_series/@ts/ts:id9}

\subsubsection{Inputs}
\label{classes/time_series/@ts/ts:id10}

\subsubsection{Outputs}
\label{classes/time_series/@ts/ts:id11}

\subsubsection{Description}
\label{classes/time_series/@ts/ts:id12}

\subsubsection{Examples}
\label{classes/time_series/@ts/ts:id13}
See also:


\bigskip\hrule{}\bigskip



\subsection{acoth}
\label{classes/time_series/@ts/ts:id14}\label{classes/time_series/@ts/ts:acoth}
H1 line


\subsubsection{Syntax}
\label{classes/time_series/@ts/ts:id15}

\subsubsection{Inputs}
\label{classes/time_series/@ts/ts:id16}

\subsubsection{Outputs}
\label{classes/time_series/@ts/ts:id17}

\subsubsection{Description}
\label{classes/time_series/@ts/ts:id18}

\subsubsection{Examples}
\label{classes/time_series/@ts/ts:id19}
See also:


\bigskip\hrule{}\bigskip



\subsection{aggregate}
\label{classes/time_series/@ts/ts:aggregate}\label{classes/time_series/@ts/ts:id20}
H1 line


\subsubsection{Syntax}
\label{classes/time_series/@ts/ts:id21}

\subsubsection{Inputs}
\label{classes/time_series/@ts/ts:id22}

\subsubsection{Outputs}
\label{classes/time_series/@ts/ts:id23}

\subsubsection{Description}
\label{classes/time_series/@ts/ts:id24}

\subsubsection{Examples}
\label{classes/time_series/@ts/ts:id25}
See also:


\bigskip\hrule{}\bigskip



\subsection{allmean}
\label{classes/time_series/@ts/ts:allmean}\label{classes/time_series/@ts/ts:id26}
H1 line


\subsubsection{Syntax}
\label{classes/time_series/@ts/ts:id27}

\subsubsection{Inputs}
\label{classes/time_series/@ts/ts:id28}

\subsubsection{Outputs}
\label{classes/time_series/@ts/ts:id29}

\subsubsection{Description}
\label{classes/time_series/@ts/ts:id30}

\subsubsection{Examples}
\label{classes/time_series/@ts/ts:id31}
See also:


\bigskip\hrule{}\bigskip



\subsection{and}
\label{classes/time_series/@ts/ts:and}\label{classes/time_series/@ts/ts:id32}
H1 line


\subsubsection{Syntax}
\label{classes/time_series/@ts/ts:id33}

\subsubsection{Inputs}
\label{classes/time_series/@ts/ts:id34}

\subsubsection{Outputs}
\label{classes/time_series/@ts/ts:id35}

\subsubsection{Description}
\label{classes/time_series/@ts/ts:id36}

\subsubsection{Examples}
\label{classes/time_series/@ts/ts:id37}
See also:


\bigskip\hrule{}\bigskip



\subsection{apply}
\label{classes/time_series/@ts/ts:apply}\label{classes/time_series/@ts/ts:id38}
H1 line


\subsubsection{Syntax}
\label{classes/time_series/@ts/ts:id39}

\subsubsection{Inputs}
\label{classes/time_series/@ts/ts:id40}

\subsubsection{Outputs}
\label{classes/time_series/@ts/ts:id41}

\subsubsection{Description}
\label{classes/time_series/@ts/ts:id42}

\subsubsection{Examples}
\label{classes/time_series/@ts/ts:id43}
See also:


\bigskip\hrule{}\bigskip



\subsection{asin}
\label{classes/time_series/@ts/ts:asin}\label{classes/time_series/@ts/ts:id44}
H1 line


\subsubsection{Syntax}
\label{classes/time_series/@ts/ts:id45}

\subsubsection{Inputs}
\label{classes/time_series/@ts/ts:id46}

\subsubsection{Outputs}
\label{classes/time_series/@ts/ts:id47}

\subsubsection{Description}
\label{classes/time_series/@ts/ts:id48}

\subsubsection{Examples}
\label{classes/time_series/@ts/ts:id49}
See also:


\bigskip\hrule{}\bigskip



\subsection{asinh}
\label{classes/time_series/@ts/ts:asinh}\label{classes/time_series/@ts/ts:id50}
H1 line


\subsubsection{Syntax}
\label{classes/time_series/@ts/ts:id51}

\subsubsection{Inputs}
\label{classes/time_series/@ts/ts:id52}

\subsubsection{Outputs}
\label{classes/time_series/@ts/ts:id53}

\subsubsection{Description}
\label{classes/time_series/@ts/ts:id54}

\subsubsection{Examples}
\label{classes/time_series/@ts/ts:id55}
See also:


\bigskip\hrule{}\bigskip



\subsection{atan}
\label{classes/time_series/@ts/ts:id56}\label{classes/time_series/@ts/ts:atan}
H1 line


\subsubsection{Syntax}
\label{classes/time_series/@ts/ts:id57}

\subsubsection{Inputs}
\label{classes/time_series/@ts/ts:id58}

\subsubsection{Outputs}
\label{classes/time_series/@ts/ts:id59}

\subsubsection{Description}
\label{classes/time_series/@ts/ts:id60}

\subsubsection{Examples}
\label{classes/time_series/@ts/ts:id61}
See also:


\bigskip\hrule{}\bigskip



\subsection{atanh}
\label{classes/time_series/@ts/ts:atanh}\label{classes/time_series/@ts/ts:id62}
H1 line


\subsubsection{Syntax}
\label{classes/time_series/@ts/ts:id63}

\subsubsection{Inputs}
\label{classes/time_series/@ts/ts:id64}

\subsubsection{Outputs}
\label{classes/time_series/@ts/ts:id65}

\subsubsection{Description}
\label{classes/time_series/@ts/ts:id66}

\subsubsection{Examples}
\label{classes/time_series/@ts/ts:id67}
See also:


\bigskip\hrule{}\bigskip



\subsection{automatic\_model\_selection}
\label{classes/time_series/@ts/ts:automatic-model-selection}\label{classes/time_series/@ts/ts:id68}
H1 line


\subsubsection{Syntax}
\label{classes/time_series/@ts/ts:id69}

\subsubsection{Inputs}
\label{classes/time_series/@ts/ts:id70}

\subsubsection{Outputs}
\label{classes/time_series/@ts/ts:id71}

\subsubsection{Description}
\label{classes/time_series/@ts/ts:id72}

\subsubsection{Examples}
\label{classes/time_series/@ts/ts:id73}
See also:


\bigskip\hrule{}\bigskip



\subsection{bar}
\label{classes/time_series/@ts/ts:bar}\label{classes/time_series/@ts/ts:id74}
H1 line


\subsubsection{Syntax}
\label{classes/time_series/@ts/ts:id75}

\subsubsection{Inputs}
\label{classes/time_series/@ts/ts:id76}

\subsubsection{Outputs}
\label{classes/time_series/@ts/ts:id77}

\subsubsection{Description}
\label{classes/time_series/@ts/ts:id78}

\subsubsection{Examples}
\label{classes/time_series/@ts/ts:id79}
See also:


\bigskip\hrule{}\bigskip



\subsection{barh}
\label{classes/time_series/@ts/ts:barh}\label{classes/time_series/@ts/ts:id80}
H1 line


\subsubsection{Syntax}
\label{classes/time_series/@ts/ts:id81}

\subsubsection{Inputs}
\label{classes/time_series/@ts/ts:id82}

\subsubsection{Outputs}
\label{classes/time_series/@ts/ts:id83}

\subsubsection{Description}
\label{classes/time_series/@ts/ts:id84}

\subsubsection{Examples}
\label{classes/time_series/@ts/ts:id85}
See also:


\bigskip\hrule{}\bigskip



\subsection{boxplot}
\label{classes/time_series/@ts/ts:boxplot}\label{classes/time_series/@ts/ts:id86}
H1 line


\subsubsection{Syntax}
\label{classes/time_series/@ts/ts:id87}

\subsubsection{Inputs}
\label{classes/time_series/@ts/ts:id88}

\subsubsection{Outputs}
\label{classes/time_series/@ts/ts:id89}

\subsubsection{Description}
\label{classes/time_series/@ts/ts:id90}

\subsubsection{Examples}
\label{classes/time_series/@ts/ts:id91}
See also:


\bigskip\hrule{}\bigskip



\subsection{bsxfun}
\label{classes/time_series/@ts/ts:id92}\label{classes/time_series/@ts/ts:bsxfun}
H1 line


\subsubsection{Syntax}
\label{classes/time_series/@ts/ts:id93}

\subsubsection{Inputs}
\label{classes/time_series/@ts/ts:id94}

\subsubsection{Outputs}
\label{classes/time_series/@ts/ts:id95}

\subsubsection{Description}
\label{classes/time_series/@ts/ts:id96}

\subsubsection{Examples}
\label{classes/time_series/@ts/ts:id97}
See also:


\bigskip\hrule{}\bigskip



\subsection{cat}
\label{classes/time_series/@ts/ts:id98}\label{classes/time_series/@ts/ts:cat}
H1 line


\subsubsection{Syntax}
\label{classes/time_series/@ts/ts:id99}

\subsubsection{Inputs}
\label{classes/time_series/@ts/ts:id100}

\subsubsection{Outputs}
\label{classes/time_series/@ts/ts:id101}

\subsubsection{Description}
\label{classes/time_series/@ts/ts:id102}

\subsubsection{Examples}
\label{classes/time_series/@ts/ts:id103}
See also:


\bigskip\hrule{}\bigskip



\subsection{collect}
\label{classes/time_series/@ts/ts:collect}\label{classes/time_series/@ts/ts:id104}
H1 line


\subsubsection{Syntax}
\label{classes/time_series/@ts/ts:id105}

\subsubsection{Inputs}
\label{classes/time_series/@ts/ts:id106}

\subsubsection{Outputs}
\label{classes/time_series/@ts/ts:id107}

\subsubsection{Description}
\label{classes/time_series/@ts/ts:id108}

\subsubsection{Examples}
\label{classes/time_series/@ts/ts:id109}
See also:


\bigskip\hrule{}\bigskip



\subsection{corr}
\label{classes/time_series/@ts/ts:id110}\label{classes/time_series/@ts/ts:corr}
H1 line


\subsubsection{Syntax}
\label{classes/time_series/@ts/ts:id111}

\subsubsection{Inputs}
\label{classes/time_series/@ts/ts:id112}

\subsubsection{Outputs}
\label{classes/time_series/@ts/ts:id113}

\subsubsection{Description}
\label{classes/time_series/@ts/ts:id114}

\subsubsection{Examples}
\label{classes/time_series/@ts/ts:id115}
See also:


\bigskip\hrule{}\bigskip



\subsection{corrcoef}
\label{classes/time_series/@ts/ts:corrcoef}\label{classes/time_series/@ts/ts:id116}
H1 line


\subsubsection{Syntax}
\label{classes/time_series/@ts/ts:id117}

\subsubsection{Inputs}
\label{classes/time_series/@ts/ts:id118}

\subsubsection{Outputs}
\label{classes/time_series/@ts/ts:id119}

\subsubsection{Description}
\label{classes/time_series/@ts/ts:id120}

\subsubsection{Examples}
\label{classes/time_series/@ts/ts:id121}
See also:


\bigskip\hrule{}\bigskip



\subsection{cos}
\label{classes/time_series/@ts/ts:cos}\label{classes/time_series/@ts/ts:id122}
H1 line


\subsubsection{Syntax}
\label{classes/time_series/@ts/ts:id123}

\subsubsection{Inputs}
\label{classes/time_series/@ts/ts:id124}

\subsubsection{Outputs}
\label{classes/time_series/@ts/ts:id125}

\subsubsection{Description}
\label{classes/time_series/@ts/ts:id126}

\subsubsection{Examples}
\label{classes/time_series/@ts/ts:id127}
See also:


\bigskip\hrule{}\bigskip



\subsection{cosh}
\label{classes/time_series/@ts/ts:id128}\label{classes/time_series/@ts/ts:cosh}
H1 line


\subsubsection{Syntax}
\label{classes/time_series/@ts/ts:id129}

\subsubsection{Inputs}
\label{classes/time_series/@ts/ts:id130}

\subsubsection{Outputs}
\label{classes/time_series/@ts/ts:id131}

\subsubsection{Description}
\label{classes/time_series/@ts/ts:id132}

\subsubsection{Examples}
\label{classes/time_series/@ts/ts:id133}
See also:


\bigskip\hrule{}\bigskip



\subsection{cot}
\label{classes/time_series/@ts/ts:id134}\label{classes/time_series/@ts/ts:cot}
H1 line


\subsubsection{Syntax}
\label{classes/time_series/@ts/ts:id135}

\subsubsection{Inputs}
\label{classes/time_series/@ts/ts:id136}

\subsubsection{Outputs}
\label{classes/time_series/@ts/ts:id137}

\subsubsection{Description}
\label{classes/time_series/@ts/ts:id138}

\subsubsection{Examples}
\label{classes/time_series/@ts/ts:id139}
See also:


\bigskip\hrule{}\bigskip



\subsection{coth}
\label{classes/time_series/@ts/ts:id140}\label{classes/time_series/@ts/ts:coth}
H1 line


\subsubsection{Syntax}
\label{classes/time_series/@ts/ts:id141}

\subsubsection{Inputs}
\label{classes/time_series/@ts/ts:id142}

\subsubsection{Outputs}
\label{classes/time_series/@ts/ts:id143}

\subsubsection{Description}
\label{classes/time_series/@ts/ts:id144}

\subsubsection{Examples}
\label{classes/time_series/@ts/ts:id145}
See also:


\bigskip\hrule{}\bigskip



\subsection{cov}
\label{classes/time_series/@ts/ts:id146}\label{classes/time_series/@ts/ts:cov}
H1 line


\subsubsection{Syntax}
\label{classes/time_series/@ts/ts:id147}

\subsubsection{Inputs}
\label{classes/time_series/@ts/ts:id148}

\subsubsection{Outputs}
\label{classes/time_series/@ts/ts:id149}

\subsubsection{Description}
\label{classes/time_series/@ts/ts:id150}

\subsubsection{Examples}
\label{classes/time_series/@ts/ts:id151}
See also:


\bigskip\hrule{}\bigskip



\subsection{ctranspose}
\label{classes/time_series/@ts/ts:id152}\label{classes/time_series/@ts/ts:ctranspose}
H1 line


\subsubsection{Syntax}
\label{classes/time_series/@ts/ts:id153}

\subsubsection{Inputs}
\label{classes/time_series/@ts/ts:id154}

\subsubsection{Outputs}
\label{classes/time_series/@ts/ts:id155}

\subsubsection{Description}
\label{classes/time_series/@ts/ts:id156}

\subsubsection{Examples}
\label{classes/time_series/@ts/ts:id157}
See also:


\bigskip\hrule{}\bigskip



\subsection{cumprod}
\label{classes/time_series/@ts/ts:cumprod}\label{classes/time_series/@ts/ts:id158}
H1 line


\subsubsection{Syntax}
\label{classes/time_series/@ts/ts:id159}

\subsubsection{Inputs}
\label{classes/time_series/@ts/ts:id160}

\subsubsection{Outputs}
\label{classes/time_series/@ts/ts:id161}

\subsubsection{Description}
\label{classes/time_series/@ts/ts:id162}

\subsubsection{Examples}
\label{classes/time_series/@ts/ts:id163}
See also:


\bigskip\hrule{}\bigskip



\subsection{cumsum}
\label{classes/time_series/@ts/ts:cumsum}\label{classes/time_series/@ts/ts:id164}
H1 line


\subsubsection{Syntax}
\label{classes/time_series/@ts/ts:id165}

\subsubsection{Inputs}
\label{classes/time_series/@ts/ts:id166}

\subsubsection{Outputs}
\label{classes/time_series/@ts/ts:id167}

\subsubsection{Description}
\label{classes/time_series/@ts/ts:id168}

\subsubsection{Examples}
\label{classes/time_series/@ts/ts:id169}
See also:


\bigskip\hrule{}\bigskip



\subsection{decompose\_series}
\label{classes/time_series/@ts/ts:id170}\label{classes/time_series/@ts/ts:decompose-series}
H1 line


\subsubsection{Syntax}
\label{classes/time_series/@ts/ts:id171}

\subsubsection{Inputs}
\label{classes/time_series/@ts/ts:id172}

\subsubsection{Outputs}
\label{classes/time_series/@ts/ts:id173}

\subsubsection{Description}
\label{classes/time_series/@ts/ts:id174}

\subsubsection{Examples}
\label{classes/time_series/@ts/ts:id175}
See also:


\bigskip\hrule{}\bigskip



\subsection{describe}
\label{classes/time_series/@ts/ts:id176}\label{classes/time_series/@ts/ts:describe}
H1 line


\subsubsection{Syntax}
\label{classes/time_series/@ts/ts:id177}

\subsubsection{Inputs}
\label{classes/time_series/@ts/ts:id178}

\subsubsection{Outputs}
\label{classes/time_series/@ts/ts:id179}

\subsubsection{Description}
\label{classes/time_series/@ts/ts:id180}

\subsubsection{Examples}
\label{classes/time_series/@ts/ts:id181}
See also:


\bigskip\hrule{}\bigskip



\subsection{display}
\label{classes/time_series/@ts/ts:id182}\label{classes/time_series/@ts/ts:display}
H1 line


\subsubsection{Syntax}
\label{classes/time_series/@ts/ts:id183}

\subsubsection{Inputs}
\label{classes/time_series/@ts/ts:id184}

\subsubsection{Outputs}
\label{classes/time_series/@ts/ts:id185}

\subsubsection{Description}
\label{classes/time_series/@ts/ts:id186}

\subsubsection{Examples}
\label{classes/time_series/@ts/ts:id187}
See also:


\bigskip\hrule{}\bigskip



\subsection{double}
\label{classes/time_series/@ts/ts:double}\label{classes/time_series/@ts/ts:id188}
H1 line


\subsubsection{Syntax}
\label{classes/time_series/@ts/ts:id189}

\subsubsection{Inputs}
\label{classes/time_series/@ts/ts:id190}

\subsubsection{Outputs}
\label{classes/time_series/@ts/ts:id191}

\subsubsection{Description}
\label{classes/time_series/@ts/ts:id192}

\subsubsection{Examples}
\label{classes/time_series/@ts/ts:id193}
See also:


\bigskip\hrule{}\bigskip



\subsection{drop}
\label{classes/time_series/@ts/ts:drop}\label{classes/time_series/@ts/ts:id194}
H1 line


\subsubsection{Syntax}
\label{classes/time_series/@ts/ts:id195}

\subsubsection{Inputs}
\label{classes/time_series/@ts/ts:id196}

\subsubsection{Outputs}
\label{classes/time_series/@ts/ts:id197}

\subsubsection{Description}
\label{classes/time_series/@ts/ts:id198}

\subsubsection{Examples}
\label{classes/time_series/@ts/ts:id199}
See also:


\bigskip\hrule{}\bigskip



\subsection{dummy}
\label{classes/time_series/@ts/ts:dummy}\label{classes/time_series/@ts/ts:id200}
H1 line


\subsubsection{Syntax}
\label{classes/time_series/@ts/ts:id201}

\subsubsection{Inputs}
\label{classes/time_series/@ts/ts:id202}

\subsubsection{Outputs}
\label{classes/time_series/@ts/ts:id203}

\subsubsection{Description}
\label{classes/time_series/@ts/ts:id204}

\subsubsection{Examples}
\label{classes/time_series/@ts/ts:id205}
See also:


\bigskip\hrule{}\bigskip



\subsection{eq}
\label{classes/time_series/@ts/ts:id206}\label{classes/time_series/@ts/ts:eq}
H1 line


\subsubsection{Syntax}
\label{classes/time_series/@ts/ts:id207}

\subsubsection{Inputs}
\label{classes/time_series/@ts/ts:id208}

\subsubsection{Outputs}
\label{classes/time_series/@ts/ts:id209}

\subsubsection{Description}
\label{classes/time_series/@ts/ts:id210}

\subsubsection{Examples}
\label{classes/time_series/@ts/ts:id211}
See also:


\bigskip\hrule{}\bigskip



\subsection{exp}
\label{classes/time_series/@ts/ts:id212}\label{classes/time_series/@ts/ts:exp}
H1 line


\subsubsection{Syntax}
\label{classes/time_series/@ts/ts:id213}

\subsubsection{Inputs}
\label{classes/time_series/@ts/ts:id214}

\subsubsection{Outputs}
\label{classes/time_series/@ts/ts:id215}

\subsubsection{Description}
\label{classes/time_series/@ts/ts:id216}

\subsubsection{Examples}
\label{classes/time_series/@ts/ts:id217}
See also:


\bigskip\hrule{}\bigskip



\subsection{expanding}
\label{classes/time_series/@ts/ts:expanding}\label{classes/time_series/@ts/ts:id218}
H1 line


\subsubsection{Syntax}
\label{classes/time_series/@ts/ts:id219}

\subsubsection{Inputs}
\label{classes/time_series/@ts/ts:id220}

\subsubsection{Outputs}
\label{classes/time_series/@ts/ts:id221}

\subsubsection{Description}
\label{classes/time_series/@ts/ts:id222}

\subsubsection{Examples}
\label{classes/time_series/@ts/ts:id223}
See also:


\bigskip\hrule{}\bigskip



\subsection{fanchart}
\label{classes/time_series/@ts/ts:fanchart}\label{classes/time_series/@ts/ts:id224}
H1 line


\subsubsection{Syntax}
\label{classes/time_series/@ts/ts:id225}

\subsubsection{Inputs}
\label{classes/time_series/@ts/ts:id226}

\subsubsection{Outputs}
\label{classes/time_series/@ts/ts:id227}

\subsubsection{Description}
\label{classes/time_series/@ts/ts:id228}

\subsubsection{Examples}
\label{classes/time_series/@ts/ts:id229}
See also:


\bigskip\hrule{}\bigskip



\subsection{ge}
\label{classes/time_series/@ts/ts:ge}\label{classes/time_series/@ts/ts:id230}
H1 line


\subsubsection{Syntax}
\label{classes/time_series/@ts/ts:id231}

\subsubsection{Inputs}
\label{classes/time_series/@ts/ts:id232}

\subsubsection{Outputs}
\label{classes/time_series/@ts/ts:id233}

\subsubsection{Description}
\label{classes/time_series/@ts/ts:id234}

\subsubsection{Examples}
\label{classes/time_series/@ts/ts:id235}
See also:


\bigskip\hrule{}\bigskip



\subsection{get}
\label{classes/time_series/@ts/ts:id236}\label{classes/time_series/@ts/ts:get}
H1 line


\subsubsection{Syntax}
\label{classes/time_series/@ts/ts:id237}

\subsubsection{Inputs}
\label{classes/time_series/@ts/ts:id238}

\subsubsection{Outputs}
\label{classes/time_series/@ts/ts:id239}

\subsubsection{Description}
\label{classes/time_series/@ts/ts:id240}

\subsubsection{Examples}
\label{classes/time_series/@ts/ts:id241}
See also:


\bigskip\hrule{}\bigskip



\subsection{gt}
\label{classes/time_series/@ts/ts:gt}\label{classes/time_series/@ts/ts:id242}
H1 line


\subsubsection{Syntax}
\label{classes/time_series/@ts/ts:id243}

\subsubsection{Inputs}
\label{classes/time_series/@ts/ts:id244}

\subsubsection{Outputs}
\label{classes/time_series/@ts/ts:id245}

\subsubsection{Description}
\label{classes/time_series/@ts/ts:id246}

\subsubsection{Examples}
\label{classes/time_series/@ts/ts:id247}
See also:


\bigskip\hrule{}\bigskip



\subsection{head}
\label{classes/time_series/@ts/ts:head}\label{classes/time_series/@ts/ts:id248}
H1 line


\subsubsection{Syntax}
\label{classes/time_series/@ts/ts:id249}

\subsubsection{Inputs}
\label{classes/time_series/@ts/ts:id250}

\subsubsection{Outputs}
\label{classes/time_series/@ts/ts:id251}

\subsubsection{Description}
\label{classes/time_series/@ts/ts:id252}

\subsubsection{Examples}
\label{classes/time_series/@ts/ts:id253}
See also:


\bigskip\hrule{}\bigskip



\subsection{hist}
\label{classes/time_series/@ts/ts:hist}\label{classes/time_series/@ts/ts:id254}
H1 line


\subsubsection{Syntax}
\label{classes/time_series/@ts/ts:id255}

\subsubsection{Inputs}
\label{classes/time_series/@ts/ts:id256}

\subsubsection{Outputs}
\label{classes/time_series/@ts/ts:id257}

\subsubsection{Description}
\label{classes/time_series/@ts/ts:id258}

\subsubsection{Examples}
\label{classes/time_series/@ts/ts:id259}
See also:


\bigskip\hrule{}\bigskip



\subsection{horzcat}
\label{classes/time_series/@ts/ts:horzcat}\label{classes/time_series/@ts/ts:id260}
H1 line


\subsubsection{Syntax}
\label{classes/time_series/@ts/ts:id261}

\subsubsection{Inputs}
\label{classes/time_series/@ts/ts:id262}

\subsubsection{Outputs}
\label{classes/time_series/@ts/ts:id263}

\subsubsection{Description}
\label{classes/time_series/@ts/ts:id264}

\subsubsection{Examples}
\label{classes/time_series/@ts/ts:id265}
See also:


\bigskip\hrule{}\bigskip



\subsection{hpfilter}
\label{classes/time_series/@ts/ts:id266}\label{classes/time_series/@ts/ts:hpfilter}
H1 line


\subsubsection{Syntax}
\label{classes/time_series/@ts/ts:id267}

\subsubsection{Inputs}
\label{classes/time_series/@ts/ts:id268}

\subsubsection{Outputs}
\label{classes/time_series/@ts/ts:id269}

\subsubsection{Description}
\label{classes/time_series/@ts/ts:id270}

\subsubsection{Examples}
\label{classes/time_series/@ts/ts:id271}
See also:


\bigskip\hrule{}\bigskip



\subsection{index}
\label{classes/time_series/@ts/ts:index}\label{classes/time_series/@ts/ts:id272}
H1 line


\subsubsection{Syntax}
\label{classes/time_series/@ts/ts:id273}

\subsubsection{Inputs}
\label{classes/time_series/@ts/ts:id274}

\subsubsection{Outputs}
\label{classes/time_series/@ts/ts:id275}

\subsubsection{Description}
\label{classes/time_series/@ts/ts:id276}

\subsubsection{Examples}
\label{classes/time_series/@ts/ts:id277}
See also:


\bigskip\hrule{}\bigskip



\subsection{interpolate}
\label{classes/time_series/@ts/ts:id278}\label{classes/time_series/@ts/ts:interpolate}
H1 line


\subsubsection{Syntax}
\label{classes/time_series/@ts/ts:id279}

\subsubsection{Inputs}
\label{classes/time_series/@ts/ts:id280}

\subsubsection{Outputs}
\label{classes/time_series/@ts/ts:id281}

\subsubsection{Description}
\label{classes/time_series/@ts/ts:id282}

\subsubsection{Examples}
\label{classes/time_series/@ts/ts:id283}
See also:


\bigskip\hrule{}\bigskip



\subsection{intersect}
\label{classes/time_series/@ts/ts:id284}\label{classes/time_series/@ts/ts:intersect}
H1 line


\subsubsection{Syntax}
\label{classes/time_series/@ts/ts:id285}

\subsubsection{Inputs}
\label{classes/time_series/@ts/ts:id286}

\subsubsection{Outputs}
\label{classes/time_series/@ts/ts:id287}

\subsubsection{Description}
\label{classes/time_series/@ts/ts:id288}

\subsubsection{Examples}
\label{classes/time_series/@ts/ts:id289}
See also:


\bigskip\hrule{}\bigskip



\subsection{isfinite}
\label{classes/time_series/@ts/ts:id290}\label{classes/time_series/@ts/ts:isfinite}
H1 line


\subsubsection{Syntax}
\label{classes/time_series/@ts/ts:id291}

\subsubsection{Inputs}
\label{classes/time_series/@ts/ts:id292}

\subsubsection{Outputs}
\label{classes/time_series/@ts/ts:id293}

\subsubsection{Description}
\label{classes/time_series/@ts/ts:id294}

\subsubsection{Examples}
\label{classes/time_series/@ts/ts:id295}
See also:


\bigskip\hrule{}\bigskip



\subsection{isinf}
\label{classes/time_series/@ts/ts:isinf}\label{classes/time_series/@ts/ts:id296}
H1 line


\subsubsection{Syntax}
\label{classes/time_series/@ts/ts:id297}

\subsubsection{Inputs}
\label{classes/time_series/@ts/ts:id298}

\subsubsection{Outputs}
\label{classes/time_series/@ts/ts:id299}

\subsubsection{Description}
\label{classes/time_series/@ts/ts:id300}

\subsubsection{Examples}
\label{classes/time_series/@ts/ts:id301}
See also:


\bigskip\hrule{}\bigskip



\subsection{isnan}
\label{classes/time_series/@ts/ts:id302}\label{classes/time_series/@ts/ts:isnan}
H1 line


\subsubsection{Syntax}
\label{classes/time_series/@ts/ts:id303}

\subsubsection{Inputs}
\label{classes/time_series/@ts/ts:id304}

\subsubsection{Outputs}
\label{classes/time_series/@ts/ts:id305}

\subsubsection{Description}
\label{classes/time_series/@ts/ts:id306}

\subsubsection{Examples}
\label{classes/time_series/@ts/ts:id307}
See also:


\bigskip\hrule{}\bigskip



\subsection{jbtest}
\label{classes/time_series/@ts/ts:id308}\label{classes/time_series/@ts/ts:jbtest}
H1 line


\subsubsection{Syntax}
\label{classes/time_series/@ts/ts:id309}

\subsubsection{Inputs}
\label{classes/time_series/@ts/ts:id310}

\subsubsection{Outputs}
\label{classes/time_series/@ts/ts:id311}

\subsubsection{Description}
\label{classes/time_series/@ts/ts:id312}

\subsubsection{Examples}
\label{classes/time_series/@ts/ts:id313}
See also:


\bigskip\hrule{}\bigskip



\subsection{kurtosis}
\label{classes/time_series/@ts/ts:id314}\label{classes/time_series/@ts/ts:kurtosis}
H1 line


\subsubsection{Syntax}
\label{classes/time_series/@ts/ts:id315}

\subsubsection{Inputs}
\label{classes/time_series/@ts/ts:id316}

\subsubsection{Outputs}
\label{classes/time_series/@ts/ts:id317}

\subsubsection{Description}
\label{classes/time_series/@ts/ts:id318}

\subsubsection{Examples}
\label{classes/time_series/@ts/ts:id319}
See also:


\bigskip\hrule{}\bigskip



\subsection{le}
\label{classes/time_series/@ts/ts:le}\label{classes/time_series/@ts/ts:id320}
H1 line


\subsubsection{Syntax}
\label{classes/time_series/@ts/ts:id321}

\subsubsection{Inputs}
\label{classes/time_series/@ts/ts:id322}

\subsubsection{Outputs}
\label{classes/time_series/@ts/ts:id323}

\subsubsection{Description}
\label{classes/time_series/@ts/ts:id324}

\subsubsection{Examples}
\label{classes/time_series/@ts/ts:id325}
See also:


\bigskip\hrule{}\bigskip



\subsection{log}
\label{classes/time_series/@ts/ts:id326}\label{classes/time_series/@ts/ts:log}
H1 line


\subsubsection{Syntax}
\label{classes/time_series/@ts/ts:id327}

\subsubsection{Inputs}
\label{classes/time_series/@ts/ts:id328}

\subsubsection{Outputs}
\label{classes/time_series/@ts/ts:id329}

\subsubsection{Description}
\label{classes/time_series/@ts/ts:id330}

\subsubsection{Examples}
\label{classes/time_series/@ts/ts:id331}
See also:


\bigskip\hrule{}\bigskip



\subsection{lt}
\label{classes/time_series/@ts/ts:lt}\label{classes/time_series/@ts/ts:id332}
H1 line


\subsubsection{Syntax}
\label{classes/time_series/@ts/ts:id333}

\subsubsection{Inputs}
\label{classes/time_series/@ts/ts:id334}

\subsubsection{Outputs}
\label{classes/time_series/@ts/ts:id335}

\subsubsection{Description}
\label{classes/time_series/@ts/ts:id336}

\subsubsection{Examples}
\label{classes/time_series/@ts/ts:id337}
See also:


\bigskip\hrule{}\bigskip



\subsection{max}
\label{classes/time_series/@ts/ts:max}\label{classes/time_series/@ts/ts:id338}
H1 line


\subsubsection{Syntax}
\label{classes/time_series/@ts/ts:id339}

\subsubsection{Inputs}
\label{classes/time_series/@ts/ts:id340}

\subsubsection{Outputs}
\label{classes/time_series/@ts/ts:id341}

\subsubsection{Description}
\label{classes/time_series/@ts/ts:id342}

\subsubsection{Examples}
\label{classes/time_series/@ts/ts:id343}
See also:


\bigskip\hrule{}\bigskip



\subsection{mean}
\label{classes/time_series/@ts/ts:id344}\label{classes/time_series/@ts/ts:mean}
H1 line


\subsubsection{Syntax}
\label{classes/time_series/@ts/ts:id345}

\subsubsection{Inputs}
\label{classes/time_series/@ts/ts:id346}

\subsubsection{Outputs}
\label{classes/time_series/@ts/ts:id347}

\subsubsection{Description}
\label{classes/time_series/@ts/ts:id348}

\subsubsection{Examples}
\label{classes/time_series/@ts/ts:id349}
See also:


\bigskip\hrule{}\bigskip



\subsection{median}
\label{classes/time_series/@ts/ts:id350}\label{classes/time_series/@ts/ts:median}
H1 line


\subsubsection{Syntax}
\label{classes/time_series/@ts/ts:id351}

\subsubsection{Inputs}
\label{classes/time_series/@ts/ts:id352}

\subsubsection{Outputs}
\label{classes/time_series/@ts/ts:id353}

\subsubsection{Description}
\label{classes/time_series/@ts/ts:id354}

\subsubsection{Examples}
\label{classes/time_series/@ts/ts:id355}
See also:


\bigskip\hrule{}\bigskip



\subsection{min}
\label{classes/time_series/@ts/ts:id356}\label{classes/time_series/@ts/ts:min}
H1 line


\subsubsection{Syntax}
\label{classes/time_series/@ts/ts:id357}

\subsubsection{Inputs}
\label{classes/time_series/@ts/ts:id358}

\subsubsection{Outputs}
\label{classes/time_series/@ts/ts:id359}

\subsubsection{Description}
\label{classes/time_series/@ts/ts:id360}

\subsubsection{Examples}
\label{classes/time_series/@ts/ts:id361}
See also:


\bigskip\hrule{}\bigskip



\subsection{minus}
\label{classes/time_series/@ts/ts:id362}\label{classes/time_series/@ts/ts:minus}
H1 line


\subsubsection{Syntax}
\label{classes/time_series/@ts/ts:id363}

\subsubsection{Inputs}
\label{classes/time_series/@ts/ts:id364}

\subsubsection{Outputs}
\label{classes/time_series/@ts/ts:id365}

\subsubsection{Description}
\label{classes/time_series/@ts/ts:id366}

\subsubsection{Examples}
\label{classes/time_series/@ts/ts:id367}
See also:


\bigskip\hrule{}\bigskip



\subsection{mode}
\label{classes/time_series/@ts/ts:id368}\label{classes/time_series/@ts/ts:mode}
H1 line


\subsubsection{Syntax}
\label{classes/time_series/@ts/ts:id369}

\subsubsection{Inputs}
\label{classes/time_series/@ts/ts:id370}

\subsubsection{Outputs}
\label{classes/time_series/@ts/ts:id371}

\subsubsection{Description}
\label{classes/time_series/@ts/ts:id372}

\subsubsection{Examples}
\label{classes/time_series/@ts/ts:id373}
See also:


\bigskip\hrule{}\bigskip



\subsection{mpower}
\label{classes/time_series/@ts/ts:id374}\label{classes/time_series/@ts/ts:mpower}
H1 line


\subsubsection{Syntax}
\label{classes/time_series/@ts/ts:id375}

\subsubsection{Inputs}
\label{classes/time_series/@ts/ts:id376}

\subsubsection{Outputs}
\label{classes/time_series/@ts/ts:id377}

\subsubsection{Description}
\label{classes/time_series/@ts/ts:id378}

\subsubsection{Examples}
\label{classes/time_series/@ts/ts:id379}
See also:


\bigskip\hrule{}\bigskip



\subsection{mrdivide}
\label{classes/time_series/@ts/ts:mrdivide}\label{classes/time_series/@ts/ts:id380}
H1 line


\subsubsection{Syntax}
\label{classes/time_series/@ts/ts:id381}

\subsubsection{Inputs}
\label{classes/time_series/@ts/ts:id382}

\subsubsection{Outputs}
\label{classes/time_series/@ts/ts:id383}

\subsubsection{Description}
\label{classes/time_series/@ts/ts:id384}

\subsubsection{Examples}
\label{classes/time_series/@ts/ts:id385}
See also:


\bigskip\hrule{}\bigskip



\subsection{mtimes}
\label{classes/time_series/@ts/ts:mtimes}\label{classes/time_series/@ts/ts:id386}
H1 line


\subsubsection{Syntax}
\label{classes/time_series/@ts/ts:id387}

\subsubsection{Inputs}
\label{classes/time_series/@ts/ts:id388}

\subsubsection{Outputs}
\label{classes/time_series/@ts/ts:id389}

\subsubsection{Description}
\label{classes/time_series/@ts/ts:id390}

\subsubsection{Examples}
\label{classes/time_series/@ts/ts:id391}
See also:


\bigskip\hrule{}\bigskip



\subsection{nan}
\label{classes/time_series/@ts/ts:nan}\label{classes/time_series/@ts/ts:id392}
H1 line


\subsubsection{Syntax}
\label{classes/time_series/@ts/ts:id393}

\subsubsection{Inputs}
\label{classes/time_series/@ts/ts:id394}

\subsubsection{Outputs}
\label{classes/time_series/@ts/ts:id395}

\subsubsection{Description}
\label{classes/time_series/@ts/ts:id396}

\subsubsection{Examples}
\label{classes/time_series/@ts/ts:id397}
See also:


\bigskip\hrule{}\bigskip



\subsection{ne}
\label{classes/time_series/@ts/ts:id398}\label{classes/time_series/@ts/ts:ne}
H1 line


\subsubsection{Syntax}
\label{classes/time_series/@ts/ts:id399}

\subsubsection{Inputs}
\label{classes/time_series/@ts/ts:id400}

\subsubsection{Outputs}
\label{classes/time_series/@ts/ts:id401}

\subsubsection{Description}
\label{classes/time_series/@ts/ts:id402}

\subsubsection{Examples}
\label{classes/time_series/@ts/ts:id403}
See also:


\bigskip\hrule{}\bigskip



\subsection{numel}
\label{classes/time_series/@ts/ts:id404}\label{classes/time_series/@ts/ts:numel}
H1 line


\subsubsection{Syntax}
\label{classes/time_series/@ts/ts:id405}

\subsubsection{Inputs}
\label{classes/time_series/@ts/ts:id406}

\subsubsection{Outputs}
\label{classes/time_series/@ts/ts:id407}

\subsubsection{Description}
\label{classes/time_series/@ts/ts:id408}

\subsubsection{Examples}
\label{classes/time_series/@ts/ts:id409}
See also:


\bigskip\hrule{}\bigskip



\subsection{ones}
\label{classes/time_series/@ts/ts:ones}\label{classes/time_series/@ts/ts:id410}
H1 line


\subsubsection{Syntax}
\label{classes/time_series/@ts/ts:id411}

\subsubsection{Inputs}
\label{classes/time_series/@ts/ts:id412}

\subsubsection{Outputs}
\label{classes/time_series/@ts/ts:id413}

\subsubsection{Description}
\label{classes/time_series/@ts/ts:id414}

\subsubsection{Examples}
\label{classes/time_series/@ts/ts:id415}
See also:


\bigskip\hrule{}\bigskip



\subsection{pages2struct}
\label{classes/time_series/@ts/ts:pages2struct}\label{classes/time_series/@ts/ts:id416}
H1 line


\subsubsection{Syntax}
\label{classes/time_series/@ts/ts:id417}

\subsubsection{Inputs}
\label{classes/time_series/@ts/ts:id418}

\subsubsection{Outputs}
\label{classes/time_series/@ts/ts:id419}

\subsubsection{Description}
\label{classes/time_series/@ts/ts:id420}

\subsubsection{Examples}
\label{classes/time_series/@ts/ts:id421}
See also:


\bigskip\hrule{}\bigskip



\subsection{plot}
\label{classes/time_series/@ts/ts:plot}\label{classes/time_series/@ts/ts:id422}
H1 line


\subsubsection{Syntax}
\label{classes/time_series/@ts/ts:id423}

\subsubsection{Inputs}
\label{classes/time_series/@ts/ts:id424}

\subsubsection{Outputs}
\label{classes/time_series/@ts/ts:id425}

\subsubsection{Description}
\label{classes/time_series/@ts/ts:id426}

\subsubsection{Examples}
\label{classes/time_series/@ts/ts:id427}
See also:


\bigskip\hrule{}\bigskip



\subsection{plotyy}
\label{classes/time_series/@ts/ts:plotyy}\label{classes/time_series/@ts/ts:id428}
H1 line


\subsubsection{Syntax}
\label{classes/time_series/@ts/ts:id429}

\subsubsection{Inputs}
\label{classes/time_series/@ts/ts:id430}

\subsubsection{Outputs}
\label{classes/time_series/@ts/ts:id431}

\subsubsection{Description}
\label{classes/time_series/@ts/ts:id432}

\subsubsection{Examples}
\label{classes/time_series/@ts/ts:id433}
See also:


\bigskip\hrule{}\bigskip



\subsection{plus}
\label{classes/time_series/@ts/ts:id434}\label{classes/time_series/@ts/ts:plus}
H1 line


\subsubsection{Syntax}
\label{classes/time_series/@ts/ts:id435}

\subsubsection{Inputs}
\label{classes/time_series/@ts/ts:id436}

\subsubsection{Outputs}
\label{classes/time_series/@ts/ts:id437}

\subsubsection{Description}
\label{classes/time_series/@ts/ts:id438}

\subsubsection{Examples}
\label{classes/time_series/@ts/ts:id439}
See also:


\bigskip\hrule{}\bigskip



\subsection{power}
\label{classes/time_series/@ts/ts:id440}\label{classes/time_series/@ts/ts:power}
H1 line


\subsubsection{Syntax}
\label{classes/time_series/@ts/ts:id441}

\subsubsection{Inputs}
\label{classes/time_series/@ts/ts:id442}

\subsubsection{Outputs}
\label{classes/time_series/@ts/ts:id443}

\subsubsection{Description}
\label{classes/time_series/@ts/ts:id444}

\subsubsection{Examples}
\label{classes/time_series/@ts/ts:id445}
See also:


\bigskip\hrule{}\bigskip



\subsection{quantile}
\label{classes/time_series/@ts/ts:quantile}\label{classes/time_series/@ts/ts:id446}
H1 line


\subsubsection{Syntax}
\label{classes/time_series/@ts/ts:id447}

\subsubsection{Inputs}
\label{classes/time_series/@ts/ts:id448}

\subsubsection{Outputs}
\label{classes/time_series/@ts/ts:id449}

\subsubsection{Description}
\label{classes/time_series/@ts/ts:id450}

\subsubsection{Examples}
\label{classes/time_series/@ts/ts:id451}
See also:


\bigskip\hrule{}\bigskip



\subsection{rand}
\label{classes/time_series/@ts/ts:rand}\label{classes/time_series/@ts/ts:id452}
H1 line


\subsubsection{Syntax}
\label{classes/time_series/@ts/ts:id453}

\subsubsection{Inputs}
\label{classes/time_series/@ts/ts:id454}

\subsubsection{Outputs}
\label{classes/time_series/@ts/ts:id455}

\subsubsection{Description}
\label{classes/time_series/@ts/ts:id456}

\subsubsection{Examples}
\label{classes/time_series/@ts/ts:id457}
See also:


\bigskip\hrule{}\bigskip



\subsection{randn}
\label{classes/time_series/@ts/ts:randn}\label{classes/time_series/@ts/ts:id458}
H1 line


\subsubsection{Syntax}
\label{classes/time_series/@ts/ts:id459}

\subsubsection{Inputs}
\label{classes/time_series/@ts/ts:id460}

\subsubsection{Outputs}
\label{classes/time_series/@ts/ts:id461}

\subsubsection{Description}
\label{classes/time_series/@ts/ts:id462}

\subsubsection{Examples}
\label{classes/time_series/@ts/ts:id463}
See also:


\bigskip\hrule{}\bigskip



\subsection{range}
\label{classes/time_series/@ts/ts:range}\label{classes/time_series/@ts/ts:id464}
H1 line


\subsubsection{Syntax}
\label{classes/time_series/@ts/ts:id465}

\subsubsection{Inputs}
\label{classes/time_series/@ts/ts:id466}

\subsubsection{Outputs}
\label{classes/time_series/@ts/ts:id467}

\subsubsection{Description}
\label{classes/time_series/@ts/ts:id468}

\subsubsection{Examples}
\label{classes/time_series/@ts/ts:id469}
See also:


\bigskip\hrule{}\bigskip



\subsection{rdivide}
\label{classes/time_series/@ts/ts:id470}\label{classes/time_series/@ts/ts:rdivide}
H1 line


\subsubsection{Syntax}
\label{classes/time_series/@ts/ts:id471}

\subsubsection{Inputs}
\label{classes/time_series/@ts/ts:id472}

\subsubsection{Outputs}
\label{classes/time_series/@ts/ts:id473}

\subsubsection{Description}
\label{classes/time_series/@ts/ts:id474}

\subsubsection{Examples}
\label{classes/time_series/@ts/ts:id475}
See also:


\bigskip\hrule{}\bigskip



\subsection{regress}
\label{classes/time_series/@ts/ts:regress}\label{classes/time_series/@ts/ts:id476}
H1 line


\subsubsection{Syntax}
\label{classes/time_series/@ts/ts:id477}

\subsubsection{Inputs}
\label{classes/time_series/@ts/ts:id478}

\subsubsection{Outputs}
\label{classes/time_series/@ts/ts:id479}

\subsubsection{Description}
\label{classes/time_series/@ts/ts:id480}

\subsubsection{Examples}
\label{classes/time_series/@ts/ts:id481}
See also:


\bigskip\hrule{}\bigskip



\subsection{reset\_start\_date}
\label{classes/time_series/@ts/ts:id482}\label{classes/time_series/@ts/ts:reset-start-date}
H1 line


\subsubsection{Syntax}
\label{classes/time_series/@ts/ts:id483}

\subsubsection{Inputs}
\label{classes/time_series/@ts/ts:id484}

\subsubsection{Outputs}
\label{classes/time_series/@ts/ts:id485}

\subsubsection{Description}
\label{classes/time_series/@ts/ts:id486}

\subsubsection{Examples}
\label{classes/time_series/@ts/ts:id487}
See also:


\bigskip\hrule{}\bigskip



\subsection{rolling}
\label{classes/time_series/@ts/ts:rolling}\label{classes/time_series/@ts/ts:id488}
H1 line


\subsubsection{Syntax}
\label{classes/time_series/@ts/ts:id489}

\subsubsection{Inputs}
\label{classes/time_series/@ts/ts:id490}

\subsubsection{Outputs}
\label{classes/time_series/@ts/ts:id491}

\subsubsection{Description}
\label{classes/time_series/@ts/ts:id492}

\subsubsection{Examples}
\label{classes/time_series/@ts/ts:id493}
See also:


\bigskip\hrule{}\bigskip



\subsection{sin}
\label{classes/time_series/@ts/ts:sin}\label{classes/time_series/@ts/ts:id494}
H1 line


\subsubsection{Syntax}
\label{classes/time_series/@ts/ts:id495}

\subsubsection{Inputs}
\label{classes/time_series/@ts/ts:id496}

\subsubsection{Outputs}
\label{classes/time_series/@ts/ts:id497}

\subsubsection{Description}
\label{classes/time_series/@ts/ts:id498}

\subsubsection{Examples}
\label{classes/time_series/@ts/ts:id499}
See also:


\bigskip\hrule{}\bigskip



\subsection{sinh}
\label{classes/time_series/@ts/ts:id500}\label{classes/time_series/@ts/ts:sinh}
H1 line


\subsubsection{Syntax}
\label{classes/time_series/@ts/ts:id501}

\subsubsection{Inputs}
\label{classes/time_series/@ts/ts:id502}

\subsubsection{Outputs}
\label{classes/time_series/@ts/ts:id503}

\subsubsection{Description}
\label{classes/time_series/@ts/ts:id504}

\subsubsection{Examples}
\label{classes/time_series/@ts/ts:id505}
See also:


\bigskip\hrule{}\bigskip



\subsection{skewness}
\label{classes/time_series/@ts/ts:skewness}\label{classes/time_series/@ts/ts:id506}
H1 line


\subsubsection{Syntax}
\label{classes/time_series/@ts/ts:id507}

\subsubsection{Inputs}
\label{classes/time_series/@ts/ts:id508}

\subsubsection{Outputs}
\label{classes/time_series/@ts/ts:id509}

\subsubsection{Description}
\label{classes/time_series/@ts/ts:id510}

\subsubsection{Examples}
\label{classes/time_series/@ts/ts:id511}
See also:


\bigskip\hrule{}\bigskip



\subsection{sort}
\label{classes/time_series/@ts/ts:sort}\label{classes/time_series/@ts/ts:id512}
H1 line


\subsubsection{Syntax}
\label{classes/time_series/@ts/ts:id513}

\subsubsection{Inputs}
\label{classes/time_series/@ts/ts:id514}

\subsubsection{Outputs}
\label{classes/time_series/@ts/ts:id515}

\subsubsection{Description}
\label{classes/time_series/@ts/ts:id516}

\subsubsection{Examples}
\label{classes/time_series/@ts/ts:id517}
See also:


\bigskip\hrule{}\bigskip



\subsection{spectrum}
\label{classes/time_series/@ts/ts:spectrum}\label{classes/time_series/@ts/ts:id518}
H1 line


\subsubsection{Syntax}
\label{classes/time_series/@ts/ts:id519}

\subsubsection{Inputs}
\label{classes/time_series/@ts/ts:id520}

\subsubsection{Outputs}
\label{classes/time_series/@ts/ts:id521}

\subsubsection{Description}
\label{classes/time_series/@ts/ts:id522}

\subsubsection{Examples}
\label{classes/time_series/@ts/ts:id523}
See also:


\bigskip\hrule{}\bigskip



\subsection{std}
\label{classes/time_series/@ts/ts:std}\label{classes/time_series/@ts/ts:id524}
H1 line


\subsubsection{Syntax}
\label{classes/time_series/@ts/ts:id525}

\subsubsection{Inputs}
\label{classes/time_series/@ts/ts:id526}

\subsubsection{Outputs}
\label{classes/time_series/@ts/ts:id527}

\subsubsection{Description}
\label{classes/time_series/@ts/ts:id528}

\subsubsection{Examples}
\label{classes/time_series/@ts/ts:id529}
See also:


\bigskip\hrule{}\bigskip



\subsection{step\_dummy}
\label{classes/time_series/@ts/ts:step-dummy}\label{classes/time_series/@ts/ts:id530}
H1 line


\subsubsection{Syntax}
\label{classes/time_series/@ts/ts:id531}

\subsubsection{Inputs}
\label{classes/time_series/@ts/ts:id532}

\subsubsection{Outputs}
\label{classes/time_series/@ts/ts:id533}

\subsubsection{Description}
\label{classes/time_series/@ts/ts:id534}

\subsubsection{Examples}
\label{classes/time_series/@ts/ts:id535}
See also:


\bigskip\hrule{}\bigskip



\subsection{subsasgn}
\label{classes/time_series/@ts/ts:id536}\label{classes/time_series/@ts/ts:subsasgn}
H1 line


\subsubsection{Syntax}
\label{classes/time_series/@ts/ts:id537}

\subsubsection{Inputs}
\label{classes/time_series/@ts/ts:id538}

\subsubsection{Outputs}
\label{classes/time_series/@ts/ts:id539}

\subsubsection{Description}
\label{classes/time_series/@ts/ts:id540}

\subsubsection{Examples}
\label{classes/time_series/@ts/ts:id541}
See also:


\bigskip\hrule{}\bigskip



\subsection{subsref}
\label{classes/time_series/@ts/ts:id542}\label{classes/time_series/@ts/ts:subsref}
H1 line


\subsubsection{Syntax}
\label{classes/time_series/@ts/ts:id543}

\subsubsection{Inputs}
\label{classes/time_series/@ts/ts:id544}

\subsubsection{Outputs}
\label{classes/time_series/@ts/ts:id545}

\subsubsection{Description}
\label{classes/time_series/@ts/ts:id546}

\subsubsection{Examples}
\label{classes/time_series/@ts/ts:id547}
See also:


\bigskip\hrule{}\bigskip



\subsection{sum}
\label{classes/time_series/@ts/ts:id548}\label{classes/time_series/@ts/ts:sum}
H1 line


\subsubsection{Syntax}
\label{classes/time_series/@ts/ts:id549}

\subsubsection{Inputs}
\label{classes/time_series/@ts/ts:id550}

\subsubsection{Outputs}
\label{classes/time_series/@ts/ts:id551}

\subsubsection{Description}
\label{classes/time_series/@ts/ts:id552}

\subsubsection{Examples}
\label{classes/time_series/@ts/ts:id553}
See also:


\bigskip\hrule{}\bigskip



\subsection{tail}
\label{classes/time_series/@ts/ts:tail}\label{classes/time_series/@ts/ts:id554}
H1 line


\subsubsection{Syntax}
\label{classes/time_series/@ts/ts:id555}

\subsubsection{Inputs}
\label{classes/time_series/@ts/ts:id556}

\subsubsection{Outputs}
\label{classes/time_series/@ts/ts:id557}

\subsubsection{Description}
\label{classes/time_series/@ts/ts:id558}

\subsubsection{Examples}
\label{classes/time_series/@ts/ts:id559}
See also:


\bigskip\hrule{}\bigskip



\subsection{times}
\label{classes/time_series/@ts/ts:id560}\label{classes/time_series/@ts/ts:times}
H1 line


\subsubsection{Syntax}
\label{classes/time_series/@ts/ts:id561}

\subsubsection{Inputs}
\label{classes/time_series/@ts/ts:id562}

\subsubsection{Outputs}
\label{classes/time_series/@ts/ts:id563}

\subsubsection{Description}
\label{classes/time_series/@ts/ts:id564}

\subsubsection{Examples}
\label{classes/time_series/@ts/ts:id565}
See also:


\bigskip\hrule{}\bigskip



\subsection{transform}
\label{classes/time_series/@ts/ts:id566}\label{classes/time_series/@ts/ts:transform}
H1 line


\subsubsection{Syntax}
\label{classes/time_series/@ts/ts:id567}

\subsubsection{Inputs}
\label{classes/time_series/@ts/ts:id568}

\subsubsection{Outputs}
\label{classes/time_series/@ts/ts:id569}

\subsubsection{Description}
\label{classes/time_series/@ts/ts:id570}

\subsubsection{Examples}
\label{classes/time_series/@ts/ts:id571}
See also:


\bigskip\hrule{}\bigskip

\phantomsection\label{classes/time_series/@ts/ts:transpose}
\textbf{transpose}
\begin{quote}

-- no help found
\end{quote}


\bigskip\hrule{}\bigskip

\phantomsection\label{classes/time_series/@ts/ts:ts}
\textbf{ts}
\begin{quote}

-- no help found
\end{quote}


\bigskip\hrule{}\bigskip



\subsection{uminus}
\label{classes/time_series/@ts/ts:uminus}\label{classes/time_series/@ts/ts:id572}
H1 line


\subsubsection{Syntax}
\label{classes/time_series/@ts/ts:id573}

\subsubsection{Inputs}
\label{classes/time_series/@ts/ts:id574}

\subsubsection{Outputs}
\label{classes/time_series/@ts/ts:id575}

\subsubsection{Description}
\label{classes/time_series/@ts/ts:id576}

\subsubsection{Examples}
\label{classes/time_series/@ts/ts:id577}
See also:


\bigskip\hrule{}\bigskip



\subsection{values}
\label{classes/time_series/@ts/ts:id578}\label{classes/time_series/@ts/ts:values}
H1 line


\subsubsection{Syntax}
\label{classes/time_series/@ts/ts:id579}

\subsubsection{Inputs}
\label{classes/time_series/@ts/ts:id580}

\subsubsection{Outputs}
\label{classes/time_series/@ts/ts:id581}

\subsubsection{Description}
\label{classes/time_series/@ts/ts:id582}

\subsubsection{Examples}
\label{classes/time_series/@ts/ts:id583}
See also:


\bigskip\hrule{}\bigskip



\subsection{var}
\label{classes/time_series/@ts/ts:var}\label{classes/time_series/@ts/ts:id584}
H1 line


\subsubsection{Syntax}
\label{classes/time_series/@ts/ts:id585}

\subsubsection{Inputs}
\label{classes/time_series/@ts/ts:id586}

\subsubsection{Outputs}
\label{classes/time_series/@ts/ts:id587}

\subsubsection{Description}
\label{classes/time_series/@ts/ts:id588}

\subsubsection{Examples}
\label{classes/time_series/@ts/ts:id589}
See also:


\bigskip\hrule{}\bigskip



\subsection{zeros}
\label{classes/time_series/@ts/ts:zeros}\label{classes/time_series/@ts/ts:id590}
H1 line


\subsubsection{Syntax}
\label{classes/time_series/@ts/ts:id591}

\subsubsection{Inputs}
\label{classes/time_series/@ts/ts:id592}

\subsubsection{Outputs}
\label{classes/time_series/@ts/ts:id593}

\subsubsection{Description}
\label{classes/time_series/@ts/ts:id594}

\subsubsection{Examples}
\label{classes/time_series/@ts/ts:id595}
See also:


\chapter{Markov Chain Monte Carlo for Bayesian Estimation}
\label{mcmc::doc}\label{mcmc:markov-chain-monte-carlo-for-bayesian-estimation}

\section{Metropolis Hastings}
\label{mcmc:metropolis-hastings}

\section{Gibbs sampling}
\label{mcmc:gibbs-sampling}

\section{Marginal data density}
\label{mcmc:marginal-data-density}

\subsection{Laplace approximation}
\label{mcmc:laplace-approximation}

\subsection{Modified harmonic mean}
\label{mcmc:modified-harmonic-mean}

\subsection{Waggoner and Zha (2008)}
\label{mcmc:waggoner-and-zha-2008}

\subsection{Mueller}
\label{mcmc:mueller}

\subsection{Chib and Jeliazkov}
\label{mcmc:chib-and-jeliazkov}

\chapter{Derivative-free optimization}
\label{derivative_free_optimization:derivative-free-optimization}\label{derivative_free_optimization::doc}\begin{itemize}
\item {} 
differential evolution

\item {} 
bee algorithm

\item {} 
biogeography

\item {} 
studga

\item {} 
ants

\end{itemize}


\chapter{Monte Carlo Filtering}
\label{classes/utils/@mcf/mcf:monte-carlo-filtering}\label{classes/utils/@mcf/mcf::doc}

\section{methods}
\label{classes/utils/@mcf/mcf:methods}\begin{itemize}
\item {} 
{[} {\hyperref[classes/utils/@mcf/mcf:addlistener]{addlistener}} {]}(mcf/addlistener)

\item {} 
{[} {\hyperref[classes/utils/@mcf/mcf:cdf]{cdf}} {]}(mcf/cdf)

\item {} 
{[} {\hyperref[classes/utils/@mcf/mcf:cdf-plot]{cdf\_plot}} {]}(mcf/cdf\_plot)

\item {} 
{[} {\hyperref[classes/utils/@mcf/mcf:correlation-patterns-plot]{correlation\_patterns\_plot}} {]}(mcf/correlation\_patterns\_plot)

\item {} 
{[} {\hyperref[classes/utils/@mcf/mcf:delete]{delete}} {]}(mcf/delete)

\item {} 
{[} {\hyperref[classes/utils/@mcf/mcf:eq]{eq}} {]}(mcf/eq)

\item {} 
{[} {\hyperref[classes/utils/@mcf/mcf:findobj]{findobj}} {]}(mcf/findobj)

\item {} 
{[} {\hyperref[classes/utils/@mcf/mcf:findprop]{findprop}} {]}(mcf/findprop)

\item {} 
{[} {\hyperref[classes/utils/@mcf/mcf:ge]{ge}} {]}(mcf/ge)

\item {} 
{[} {\hyperref[classes/utils/@mcf/mcf:gt]{gt}} {]}(mcf/gt)

\item {} 
{[} {\hyperref[classes/utils/@mcf/mcf:isvalid]{isvalid}} {]}(mcf/isvalid)

\item {} 
{[} {\hyperref[classes/utils/@mcf/mcf:kolmogorov-smirnov-test]{kolmogorov\_smirnov\_test}} {]}(mcf/kolmogorov\_smirnov\_test)

\item {} 
{[} {\hyperref[classes/utils/@mcf/mcf:le]{le}} {]}(mcf/le)

\item {} 
{[} {\hyperref[classes/utils/@mcf/mcf:lt]{lt}} {]}(mcf/lt)

\item {} 
{[} {\hyperref[classes/utils/@mcf/mcf:mcf]{mcf}} {]}(mcf/mcf)

\item {} 
{[} {\hyperref[classes/utils/@mcf/mcf:ne]{ne}} {]}(mcf/ne)

\item {} 
{[} {\hyperref[classes/utils/@mcf/mcf:notify]{notify}} {]}(mcf/notify)

\item {} 
{[} {\hyperref[classes/utils/@mcf/mcf:scatter]{scatter}} {]}(mcf/scatter)

\end{itemize}


\section{properties}
\label{classes/utils/@mcf/mcf:properties}\begin{itemize}
\item {} 
{[}lb{]} -

\item {} 
{[}ub{]} -

\item {} 
{[}nsim{]} -

\item {} 
{[}procedure{]} -

\item {} 
{[}parameter\_names{]} -

\item {} 
{[}samples{]} -

\item {} 
{[}is\_behaved{]} -

\item {} 
{[}nparam{]} -

\item {} 
{[}is\_sampled{]} -

\item {} 
{[}check\_behavior{]} -

\item {} 
{[}number\_of\_outputs{]} -

\item {} 
{[}user\_outputs{]} -

\item {} 
{[}known\_procedures{]} -

\end{itemize}


\section{Synopsis and description on methods}
\label{classes/utils/@mcf/mcf:synopsis-and-description-on-methods}

\bigskip\hrule{}\bigskip

\phantomsection\label{classes/utils/@mcf/mcf:addlistener}\begin{quote}
\begin{description}
\item[{\textbf{ADDLISTENER}   Add listener for event.}] \leavevmode
el = ADDLISTENER(hSource, `Eventname', Callback) creates a listener
for the event named Eventname, the source of which is handle object
hSource.  If hSource is an array of source handles, the listener
responds to the named event on any handle in the array.  The Callback
is a function handle that is invoked when the event is triggered.

el = ADDLISTENER(hSource, PropName, `Eventname', Callback) adds a
listener for a property event.  Eventname must be one of the strings
`PreGet', `PostGet', `PreSet', and `PostSet'.  PropName must be either
a single property name or cell array of property names, or a single
meta.property or array of meta.property objects.  The properties must
belong to the class of hSource.  If hSource is scalar, PropName can
include dynamic properties.

For all forms, addlistener returns an event.listener.  To remove a
listener, delete the object returned by addlistener.  For example,
delete(el) calls the handle class delete method to remove the listener
and delete it from the workspace.

See also MCF, NOTIFY, DELETE, EVENT.LISTENER, META.PROPERTY, EVENTS,
DYNAMICPROPS

\end{description}
\end{quote}
\begin{description}
\item[{Help for mcf/addlistener is inherited from superclass HANDLE}] \leavevmode\begin{description}
\item[{Reference page in Help browser}] \leavevmode
doc mcf/addlistener

\end{description}

\end{description}


\bigskip\hrule{}\bigskip

\phantomsection\label{classes/utils/@mcf/mcf:cdf}
\textbf{cdf}
\begin{quote}

-- no help found
\end{quote}


\bigskip\hrule{}\bigskip

\phantomsection\label{classes/utils/@mcf/mcf:cdf-plot}
\textbf{cdf\_plot}
\begin{quote}

-- no help found
\end{quote}


\bigskip\hrule{}\bigskip

\phantomsection\label{classes/utils/@mcf/mcf:correlation-patterns-plot}
\textbf{correlation\_patterns\_plot}
\begin{quote}

-- no help found
\end{quote}


\bigskip\hrule{}\bigskip

\phantomsection\label{classes/utils/@mcf/mcf:delete}\begin{quote}
\begin{description}
\item[{\textbf{DELETE}   Delete a handle object.}] \leavevmode
The DELETE method deletes a handle object but does not clear the handle
from the workspace.  A deleted handle is no longer valid.

DELETE(H) deletes the handle object H, where H is a scalar handle.

See also MCF, MCF/ISVALID, CLEAR

\end{description}
\end{quote}
\begin{description}
\item[{Help for mcf/delete is inherited from superclass HANDLE}] \leavevmode\begin{description}
\item[{Reference page in Help browser}] \leavevmode
doc mcf/delete

\end{description}

\end{description}


\bigskip\hrule{}\bigskip



\subsection{eq}
\label{classes/utils/@mcf/mcf:eq}\label{classes/utils/@mcf/mcf:id1}\begin{description}
\item[{== (EQ)   Test handle equality.}] \leavevmode
Handles are equal if they are handles for the same object.

H1 == H2 performs element-wise comparisons between handle arrays H1 and
H2.  H1 and H2 must be of the same dimensions unless one is a scalar.
The result is a logical array of the same dimensions, where each
element is an element-wise equality result.

If one of H1 or H2 is scalar, scalar expansion is performed and the
result will match the dimensions of the array that is not scalar.

TF = EQ(H1, H2) stores the result in a logical array of the same
dimensions.

See also MCF, MCF/GE, MCF/GT, MCF/LE, MCF/LT, MCF/NE

\end{description}

Help for mcf/eq is inherited from superclass HANDLE


\bigskip\hrule{}\bigskip

\phantomsection\label{classes/utils/@mcf/mcf:findobj}\begin{quote}
\begin{description}
\item[{\textbf{FINDOBJ}   Find objects matching specified conditions.}] \leavevmode
The FINDOBJ method of the HANDLE class follows the same syntax as the
MATLAB FINDOBJ command, except that the first argument must be an array
of handles to objects.

HM = FINDOBJ(H, \textless{}conditions\textgreater{}) searches the handle object array H and
returns an array of handle objects matching the specified conditions.
Only the public members of the objects of H are considered when
evaluating the conditions.

See also FINDOBJ, MCF

\end{description}
\end{quote}
\begin{description}
\item[{Help for mcf/findobj is inherited from superclass HANDLE}] \leavevmode\begin{description}
\item[{Reference page in Help browser}] \leavevmode
doc mcf/findobj

\end{description}

\end{description}


\bigskip\hrule{}\bigskip

\phantomsection\label{classes/utils/@mcf/mcf:findprop}\begin{quote}
\begin{description}
\item[{\textbf{FINDPROP}   Find property of MATLAB handle object.}] \leavevmode
p = FINDPROP(H,'PROPNAME') finds and returns the META.PROPERTY object
associated with property name PROPNAME of scalar handle object H.
PROPNAME must be a string.  It can be the name of a property defined
by the class of H or a dynamic property added to scalar object H.

If no property named PROPNAME exists for object H, an empty
META.PROPERTY array is returned.

See also MCF, MCF/FINDOBJ, DYNAMICPROPS, META.PROPERTY

\end{description}
\end{quote}
\begin{description}
\item[{Help for mcf/findprop is inherited from superclass HANDLE}] \leavevmode\begin{description}
\item[{Reference page in Help browser}] \leavevmode
doc mcf/findprop

\end{description}

\end{description}


\bigskip\hrule{}\bigskip



\subsection{ge}
\label{classes/utils/@mcf/mcf:ge}\label{classes/utils/@mcf/mcf:id2}\begin{description}
\item[{\textgreater{}= (GE)   Greater than or equal relation for handles.}] \leavevmode
H1 \textgreater{}= H2 performs element-wise comparisons between handle arrays H1 and
H2.  H1 and H2 must be of the same dimensions unless one is a scalar.
The result is a logical array of the same dimensions, where each
element is an element-wise \textgreater{}= result.

If one of H1 or H2 is scalar, scalar expansion is performed and the
result will match the dimensions of the array that is not scalar.

TF = GE(H1, H2) stores the result in a logical array of the same
dimensions.

See also MCF, MCF/EQ, MCF/GT, MCF/LE, MCF/LT, MCF/NE

\end{description}

Help for mcf/ge is inherited from superclass HANDLE


\bigskip\hrule{}\bigskip



\subsection{gt}
\label{classes/utils/@mcf/mcf:gt}\label{classes/utils/@mcf/mcf:id3}\begin{description}
\item[{\textgreater{} (GT)   Greater than relation for handles.}] \leavevmode
H1 \textgreater{} H2 performs element-wise comparisons between handle arrays H1 and
H2.  H1 and H2 must be of the same dimensions unless one is a scalar.
The result is a logical array of the same dimensions, where each
element is an element-wise \textgreater{} result.

If one of H1 or H2 is scalar, scalar expansion is performed and the
result will match the dimensions of the array that is not scalar.

TF = GT(H1, H2) stores the result in a logical array of the same
dimensions.

See also MCF, MCF/EQ, MCF/GE, MCF/LE, MCF/LT, MCF/NE

\end{description}

Help for mcf/gt is inherited from superclass HANDLE


\bigskip\hrule{}\bigskip

\phantomsection\label{classes/utils/@mcf/mcf:isvalid}\begin{quote}
\begin{description}
\item[{\textbf{ISVALID}   Test handle validity.}] \leavevmode
TF = ISVALID(H) performs an element-wise check for validity on the
handle elements of H.  The result is a logical array of the same
dimensions as H, where each element is the element-wise validity
result.

A handle is invalid if it has been deleted or if it is an element
of a handle array and has not yet been initialized.

See also MCF, MCF/DELETE

\end{description}
\end{quote}
\begin{description}
\item[{Help for mcf/isvalid is inherited from superclass HANDLE}] \leavevmode\begin{description}
\item[{Reference page in Help browser}] \leavevmode
doc mcf/isvalid

\end{description}

\end{description}


\bigskip\hrule{}\bigskip



\subsection{kolmogorov\_smirnov\_test}
\label{classes/utils/@mcf/mcf:id4}\label{classes/utils/@mcf/mcf:kolmogorov-smirnov-test}
tests the equality of two distributions using their CDFs


\bigskip\hrule{}\bigskip



\subsection{le}
\label{classes/utils/@mcf/mcf:le}\label{classes/utils/@mcf/mcf:id5}\begin{description}
\item[{\textless{}= (LE)   Less than or equal relation for handles.}] \leavevmode
Handles are equal if they are handles for the same object.  All
comparisons use a number associated with each handle object.  Nothing
can be assumed about the result of a handle comparison except that the
repeated comparison of two handles in the same MATLAB session will
yield the same result.  The order of handle values is purely arbitrary
and has no connection to the state of the handle objects being
compared.

H1 \textless{}= H2 performs element-wise comparisons between handle arrays H1 and
H2.  H1 and H2 must be of the same dimensions unless one is a scalar.
The result is a logical array of the same dimensions, where each
element is an element-wise \textgreater{}= result.

If one of H1 or H2 is scalar, scalar expansion is performed and the
result will match the dimensions of the array that is not scalar.

TF = LE(H1, H2) stores the result in a logical array of the same
dimensions.

See also MCF, MCF/EQ, MCF/GE, MCF/GT, MCF/LT, MCF/NE

\end{description}

Help for mcf/le is inherited from superclass HANDLE


\bigskip\hrule{}\bigskip



\subsection{lt}
\label{classes/utils/@mcf/mcf:id6}\label{classes/utils/@mcf/mcf:lt}\begin{description}
\item[{\textless{} (LT)   Less than relation for handles.}] \leavevmode
H1 \textless{} H2 performs element-wise comparisons between handle arrays H1 and
H2.  H1 and H2 must be of the same dimensions unless one is a scalar.
The result is a logical array of the same dimensions, where each
element is an element-wise \textless{} result.

If one of H1 or H2 is scalar, scalar expansion is performed and the
result will match the dimensions of the array that is not scalar.

TF = LT(H1, H2) stores the result in a logical array of the same
dimensions.

See also MCF, MCF/EQ, MCF/GE, MCF/GT, MCF/LE, MCF/NE

\end{description}

Help for mcf/lt is inherited from superclass HANDLE


\bigskip\hrule{}\bigskip

\phantomsection\label{classes/utils/@mcf/mcf:mcf}
\textbf{mcf}
\begin{quote}

-- no help found
\end{quote}


\bigskip\hrule{}\bigskip



\subsection{ne}
\label{classes/utils/@mcf/mcf:id7}\label{classes/utils/@mcf/mcf:ne}\begin{description}
\item[{\textasciitilde{}= (NE)   Not equal relation for handles.}] \leavevmode
Handles are equal if they are handles for the same object and are
unequal otherwise.

H1 \textasciitilde{}= H2 performs element-wise comparisons between handle arrays H1
and H2.  H1 and H2 must be of the same dimensions unless one is a
scalar.  The result is a logical array of the same dimensions, where
each element is an element-wise equality result.

If one of H1 or H2 is scalar, scalar expansion is performed and the
result will match the dimensions of the array that is not scalar.

TF = NE(H1, H2) stores the result in a logical array of the same
dimensions.

See also MCF, MCF/EQ, MCF/GE, MCF/GT, MCF/LE, MCF/LT

\end{description}

Help for mcf/ne is inherited from superclass HANDLE


\bigskip\hrule{}\bigskip

\phantomsection\label{classes/utils/@mcf/mcf:notify}\begin{quote}
\begin{description}
\item[{\textbf{NOTIFY}   Notify listeners of event.}] \leavevmode
NOTIFY(H,'EVENTNAME') notifies listeners added to the event named
EVENTNAME on handle object array H that the event is taking place.
H is the array of handles to objects triggering the event, and
EVENTNAME must be a string.

NOTIFY(H,'EVENTNAME',DATA) provides a way of encapsulating information
about an event which can then be accessed by each registered listener.
DATA must belong to the EVENT.EVENTDATA class.

See also MCF, MCF/ADDLISTENER, EVENT.EVENTDATA, EVENTS

\end{description}
\end{quote}
\begin{description}
\item[{Help for mcf/notify is inherited from superclass HANDLE}] \leavevmode\begin{description}
\item[{Reference page in Help browser}] \leavevmode
doc mcf/notify

\end{description}

\end{description}


\bigskip\hrule{}\bigskip

\phantomsection\label{classes/utils/@mcf/mcf:scatter}
\textbf{scatter}
\begin{quote}

-- no help found
\end{quote}


\chapter{High dimensional model representation}
\label{classes/utils/@hdmr/hdmr::doc}\label{classes/utils/@hdmr/hdmr:high-dimensional-model-representation}

\section{methods}
\label{classes/utils/@hdmr/hdmr:methods}\begin{itemize}
\item {} 
{[} {\hyperref[classes/utils/@hdmr/hdmr:estimate]{estimate}} {]}(hdmr/estimate)

\item {} 
{[} {\hyperref[classes/utils/@hdmr/hdmr:first-order-effect]{first\_order\_effect}} {]}(hdmr/first\_order\_effect)

\item {} 
{[} {\hyperref[classes/utils/@hdmr/hdmr:hdmr]{hdmr}} {]}(hdmr/hdmr)

\item {} 
{[} {\hyperref[classes/utils/@hdmr/hdmr:metamodel]{metamodel}} {]}(hdmr/metamodel)

\item {} 
{[} {\hyperref[classes/utils/@hdmr/hdmr:plot-fit]{plot\_fit}} {]}(hdmr/plot\_fit)

\item {} 
{[} {\hyperref[classes/utils/@hdmr/hdmr:polynomial-evaluation]{polynomial\_evaluation}} {]}(hdmr/polynomial\_evaluation)

\item {} 
{[} {\hyperref[classes/utils/@hdmr/hdmr:polynomial-integration]{polynomial\_integration}} {]}(hdmr/polynomial\_integration)

\item {} 
{[} {\hyperref[classes/utils/@hdmr/hdmr:polynomial-multiplication]{polynomial\_multiplication}} {]}(hdmr/polynomial\_multiplication)

\end{itemize}


\section{properties}
\label{classes/utils/@hdmr/hdmr:properties}\begin{itemize}
\item {} 
{[}N{]} -

\item {} 
{[}Nobs{]} -

\item {} 
{[}n{]} -

\item {} 
{[}output\_nbr{]} -

\item {} 
{[}theta{]} -

\item {} 
{[}theta\_low{]} -

\item {} 
{[}theta\_high{]} -

\item {} 
{[}g{]} -

\item {} 
{[}x{]} -

\item {} 
{[}expansion\_order{]} -

\item {} 
{[}pol\_max\_order{]} -

\item {} 
{[}poly\_coefs{]} -

\item {} 
{[}Indices{]} -

\item {} 
{[}coefficients{]} -

\item {} 
{[}aggregate{]} -

\item {} 
{[}f0{]} -

\item {} 
{[}D{]} -

\item {} 
{[}sample\_percentage{]} -

\item {} 
{[}optimal{]} -

\item {} 
{[}param\_names{]} -

\end{itemize}


\section{Synopsis and description on methods}
\label{classes/utils/@hdmr/hdmr:synopsis-and-description-on-methods}

\bigskip\hrule{}\bigskip

\phantomsection\label{classes/utils/@hdmr/hdmr:estimate}
\textbf{estimate}
\begin{quote}

-- no help found
\end{quote}


\bigskip\hrule{}\bigskip

\phantomsection\label{classes/utils/@hdmr/hdmr:first-order-effect}
\textbf{first\_order\_effect}
\begin{quote}

-- no help found
\end{quote}


\bigskip\hrule{}\bigskip

\phantomsection\label{classes/utils/@hdmr/hdmr:hdmr}
\textbf{hdmr}
\begin{quote}

-- no help found
\end{quote}


\bigskip\hrule{}\bigskip

\phantomsection\label{classes/utils/@hdmr/hdmr:metamodel}
\textbf{metamodel}
\begin{quote}

-- no help found
\end{quote}


\bigskip\hrule{}\bigskip

\phantomsection\label{classes/utils/@hdmr/hdmr:plot-fit}
\textbf{plot\_fit}
\begin{quote}

-- no help found
\end{quote}


\bigskip\hrule{}\bigskip



\subsection{polynomial\_evaluation}
\label{classes/utils/@hdmr/hdmr:polynomial-evaluation}\label{classes/utils/@hdmr/hdmr:id1}
later on, the function that normalizes could come in here so that the
normalization is done according to the hdmr\_type of polynomial chosen.


\bigskip\hrule{}\bigskip



\subsection{polynomial\_integration}
\label{classes/utils/@hdmr/hdmr:id2}\label{classes/utils/@hdmr/hdmr:polynomial-integration}
polynomial is of the form a0+a1*x+...+ar*x\textasciicircum{}r
the integral is then a0*x+a1/2*x\textasciicircum{}2+...+ar/(r+1)*x\textasciicircum{}(r+1)


\bigskip\hrule{}\bigskip



\subsection{polynomial\_multiplication}
\label{classes/utils/@hdmr/hdmr:polynomial-multiplication}\label{classes/utils/@hdmr/hdmr:id3}
each polynomial is of the form a0+a1*x+...+ar*x\textasciicircum{}r


\chapter{Contributing to RISE}
\label{contributing:contributing-to-rise}\label{contributing::doc}

\section{contributing new code}
\label{contributing:contributing-new-code}

\section{contributing by helping maintain existing code}
\label{contributing:contributing-by-helping-maintain-existing-code}

\section{other ways to contribute}
\label{contributing:other-ways-to-contribute}

\section{recommended development setup}
\label{contributing:recommended-development-setup}

\section{RISE structure}
\label{contributing:rise-structure}

\section{useful links, FAQ, checklist}
\label{contributing:useful-links-faq-checklist}

\chapter{Acknowledgements}
\label{acknowledgements:acknowledgements}\label{acknowledgements::doc}
Many people have, oftentimes unknowingly, provided help in the form of reporting bugs, making suggestions, asking challenging questions, etc.
I would like to single out a few of them but the list is far from exhaustive:
\begin{itemize}
\item {} 
Dan Waggoner

\item {} 
Doug Laxton

\item {} 
Eric Leeper

\item {} 
Jesper Linde

\item {} 
Jim Nason

\item {} 
Kjetil Olsen

\item {} 
Kostas Theodoridis

\item {} 
Leif Brubakk

\item {} 
Marco Ratto

\item {} 
Michel Juillard

\item {} 
Pablo Winnant (dolo)

\item {} 
Pelin Ilbas

\item {} 
Raf Wouters

\item {} 
Tao Zha

\end{itemize}


\chapter{Bibliography}
\label{bibliography:bibliography}\label{bibliography::doc}

\chapter{Indices and tables}
\label{master_doc:indices-and-tables}\begin{itemize}
\item {} 
\emph{genindex}

\item {} 
\emph{modindex}

\item {} 
\emph{search}

\end{itemize}



\renewcommand{\indexname}{Index}
\printindex
\end{document}
